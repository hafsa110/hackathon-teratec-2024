\chapter{Getting Started with \TelApy{}  module}\label{ch:inp:outp}

\section{\TelApy{} module installation}

In order to be able to use \TelApy{} module, the \telemacsystem\ and all its
external libraries must be compiled in dynamic form. The explanation of dynamic
compilation is available on the \telemacsystem\ website in the wiki category
``installation notes''
(\url{http://wiki.opentelemac.org/doku.php?id=installation_notes_2_beta}).

Then after compiling the module, the use of \TelApy{} is presented and explained
in some notebooks documentation. In fact, the \TelApy{} module is provided with
some tutorial intended for people who want to run \telemac{2D} in an
interactive mode with the help of the \python\ programming language.

\section{How to run notebook documentation}

Html version of the notebooks (not executable) are available in release version
(starting from v8p2) in \verb!$HOMETEL/documentation/notebooks/index.html! (Run
it through a Internet browser).

In order to use notebooks, the user needs to install a notebook viewer such as
jupyter notebook. Notebook documents (or ``notebooks'', all lower case) are
documents which contain both computer code (e.g. \python) and rich text elements
(paragraph, equations, figures, links, etc\ldots). Notebook documents are both
human-readable documents containing the analysis description and the results
(figures, tables, etc\ldots) as well as executable documents which can be run to
perform data analysis.

First and foremost, the Jupyter Notebook is an interactive environment for
writing and running code. The notebook is capable of running code in a wide
range of languages. However, each notebook is associated with a single kernel.
This notebook is associated with the I\python\ kernel, therefore runs \python\
code. More details on the installation and use can be found in the Jupyter
website \url{http://jupyter.org}.

\section{Notebook examples in \TelApy{}}

As already mentioned, the \TelApy{} module is provided with some tutorial intended
for people who want to run \telemac{2D} (Or another module) in an interactive
mode with the help of the \python\ programming language.

In order to see and run the notebook examples, the user need to launch the
command ``jupyter notebook \$HOMETEL/notebooks/index.ipynb''. This will
launch a Jupyter server and a page should appear in your default internet
browser. This page in the index for all the notebooks of \telemacsystem{} you can find
the \TelApy{} ones in the \TelApy{} section.
