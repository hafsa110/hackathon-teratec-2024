\chapter{Developer manual}
\label{ch:hydrod:sim}

This section will describe firstly how the API is implemented in \fortran.  The
first section will describe the instance type used in the API\@. Same as before
we will be using TELEMAC{2D} in this section but it is the same behavior with
the other modules. This structure shows all variables accessible by the
\fortran API\@. Then, a section is dedicated to guidelines and advises to allow
a user to add access to a new variable from \telemac{2D} to the API\@.

Then the second part is focused on the Python guidelines and convention
of the \TelApy{} module.
%
\section{Instance}
%
As already explained in section~\ref{ch:TelApy_description}, in order to
control the data used by \telemac{2D} a \fortran Structure called "instance"
containing all global variables of \telemac{2D} is used. A part of this
structure is presented below.  In addition, some functions, described below,
are available in order to handle that structure. All the instance functions can
be found in the \fortran module \verb!m_instance_t2d! (here is small part of
the instance).

\begin{lstlisting}[language=Fortran]
type instance_t2d
   ! run position
   integer myposition
   ! list of all the variable for model
   type(bief_obj), pointer :: hbor
!
   type(bief_mesh), pointer :: mesh
!
   type(bief_obj), pointer :: lihbor
!
   integer,        pointer :: nit
   integer,        pointer :: lt
!
   type(bief_file), pointer :: t2d_files(:)
   integer :: maxlu_t2d
   integer :: maxkey
   integer, pointer :: t2dcli
!
   character(len=144), pointer :: coupling
!
   end type ! model_t2d
\end{lstlisting}

During the use of the API, two arrays are available in order to keep track the
used instances.
\begin{lstlisting}
INTEGER, PARAMETER :: MAX_INSTANCES=10
TYPE(INSTANCE_T2D), POINTER :: INSTANCE_LIST(:)
LOGICAL, ALLOCATABLE :: USED_INSTANCE(:)
\end{lstlisting}

Where \verb!INSTANCE_LIST! will contain all the instances used and
\verb!USED_INSTANCE! tells you if the id is used.

\subsection{Instance functions}

In addition to the instance definition, the API includes all routines needed to
manipulate it:

\begin{itemize}
\item \verb!CHECK_INSTANCE_T2D!. This function just checks that the id number
  is valid (between 1 and max\_instances).
\item \verb!CREATE_INSTANCE_T2D!. This function creates new instance and
  returns the id of that instance.
\item \verb!UPDATE_INSTANCE_T2D!. This function updates the link of the
  instance with the variables in declarations\_module.f. This is necessary as
  we are using pointers some are initialised later in the computation.
\item \verb!DELETE_INSTANCE_T2D!. This function deletes the instance and make
  the id available.
\end{itemize}
%
\subsection{How to add access to a new variable}
\label{var_add}
%
In order to add access to a new variable via the API, the following steps must
be done:

\begin{enumerate}
\item Get the name of the variable in \verb!declarations_telemac2d! and add ``,
  TARGET'' in its declaration.
\item Add that variable in the instance structure (file
  \verb!api_instance_t2d.f!).
\item Add the initialisation in \verb!update_instance_t2d!.
\item Add the variable in the get/set (do not forget the array ones) function
  that befits it.
\item Add the size of the variable in \verb!get_var_size_t2d_d!.
\item Add the type of the variable in \verb!get_var_type_t2d_d!.
\item Add the variable in \verb!SET_VAR_LIST_T2D_D!.
\item Increase the value of \verb!t2d_nb_var! in \verb!api_handle_var_t2d.f!.
\end{enumerate}

All those steps are handled through the script
\verb!scripts/add_api_variable.sh!. That takes as input a file describing the
variables to add (you can see an example in \verb!scripts/desc_file.csv!).

The input file contains the following information ';' separated:
\begin{itemize}
  \item Short name of the module: t2d, t3d, art, wac, sis...
  \item Api name of the variable: The name that will be given to the get/set
    (for example MODEL.NPOIN).
  \item Type of the variable:
    \begin{itemize}
      \item We found the four from the api (\verb!double!, \verb!boolean!,
        \verb!string! and \verb!integer!).
      \item \verb!bief_integer! and \verb!bief_double! when the \telemac{2D}
        variable is a bief obj containing an integer/double array.
      \item \verb!bloc_double! when the \telemac{2D} variable is a bloc (i.e.
        list of bief obj containing array of double).
    \end{itemize}
  \item Variable \fortran name: Name of the variable in the
    \verb!declarations_telemac2d.f!.
  \item Only for string contains size of the string 0 for the others.
  \item Number of dimension of the variable 0 for non array variable.
  \item readonly TRUE if the variable is cannot be set.
  \item get\_pos: Not used yet, set to NO\_POSITION.
  \item set\_pos: Not used yet, set to NO\_POSITION.
  \item description: Description of the variable this is the message that will
    be displayed by the api when looking at the list of variables.
\end{itemize}

Then to run the script just do:
\begin{lstlisting}
./add_api_variable.sh desc_file.csv
\end{lstlisting}

The last thing to do is to add ``, TARGET'' in the
declarations\_\textit{module}.f file for each of the added variable.

%
%------------------------------------------------------------------------------
\section{Coding conventions}
%------------------------------------------------------------------------------
%
\begin{WarningBlock}{Warning:}
\centering
Be careful, the node numbering is dependent of the convention code used.
When \fortran API are used the node numbering is considered from
$1$ to $npoin$ and in python is considered from $0$ to $(npoin-1)$
\end{WarningBlock}

\subsection{\fortran part of API}

The \fortran part of the \telemacsystem API is submitted to the main rules
presented in the developer guide.

\subsection{\TelApy{} module}

The Python part is developed with the python convention "PEP 8". The aim of
PEP 8 guidelines used in the \TelApy{} module is to improve the readability of
code and make it consistent across the wide spectrum of Python code. A style
guide is about consistency. Consistency with this style guide is important.
Consistency within a project is more important. Consistency within one module
or function is the most important.

The PEP 8 convention coding can be easily checked using the Pylint code
analyser \url{https://www.pylint.org}. Pylint is a tool that checks for
errors in Python code, tries to enforce a coding standard and looks for code
smells. It can also look for certain type errors, it can recommend suggestions
about how particular blocks can be refactored and can offer you details about
the code’s complexity. Pylint will display a number of messages as it analyzes
the code and it can also be used for displaying some statistics about the
number of warnings and errors found in different files. The messages are
classified under various categories such as errors and warnings.

%
%------------------------------------------------------------------------------
\section{Validation}
%------------------------------------------------------------------------------
%
An example of each module/coupling should be added in
\verb!examples/python3/TelApy_api/vnv_api.py! this example should do a double
run and do some get/set (see other file for example).

The option \verb!--api! can be used in \verb!validate_telemac.py!. This will do
the following for each \verb!vnv_*.py! :
\begin{itemize}
  \item Every time a study (vnv\_1) is added in the ``pre'' function it will
    add a new command (vnv\_1\_api).
  \item Then it will copy all the files in the study folder (vnv\_1) into the
    api folder (vnv\_1\_api).
  \item If some files are used for multiple keywords (for example in coupling
    or with the RESTART FILE being used as GEOMETRY FILE as well) which is not
    allowed with the api a copy will be made and the steering modified
    accordingly.
  \item The command used in the api study (vnv\_1\_api) will be ``mpirun -n x
    template.py module cas --double-run''. This will two run of the case using
    the \TelApy{} module. We add the option double-run which will run the api
    twice. This is to check that variable are properly initialised.
  \item Finally we add a binary-wise comparison of all the output file of the
    study (vnv\_1) versus the one of the api command (vnv\_1\_api).
\end{itemize}

All of those step are added this means that the normal validation will be done
and api ones. This takes a while as in the end each study will be run 3
times (normal, api, api).

