%%%%%%%%%%%%%%%%%%%%%%%%%%%%%%%%%%%%%%%%%
%  WAQTEL Documentation
%  Technical manual
%
%%%%%%%%%%%%%%%%%%%%%%%%%%%%%%%%%%%%%%%%%

%----------------------------------------------------------------------------------------
%	PACKAGES AND OTHER DOCUMENT CONFIGURATIONS
%----------------------------------------------------------------------------------------
\documentclass[Waqtel]{../../data/TelemacDoc} % Default font size and left-justified equations


\begin{document}

\let\cleardoublepage\clearpage
%----------------------------------------------------------------------------------------
%	TITLE PAGE
%----------------------------------------------------------------------------------------
\title{\waqtel}
\subtitle{Technical manual}
\version{\telmaversion}
\date{\today}
\maketitle
\clearpage

%----------------------------------------------------------------------------------------
%	AUTHORS PAGE
%----------------------------------------------------------------------------------------

%----------------------------------------------------------------------------------------
%	TABLE OF CONTENTS
%----------------------------------------------------------------------------------------


\pagestyle{empty} % No headers

\tableofcontents% Print the table of contents itself

%\cleardoublepage % Forces the first chapter to start on an odd page so it's on the right

\pagestyle{fancy} % Print headers again

\thispagestyle{empty}

\chapter*{Abstract}
This technical manual is mainly based on the translation of the TRACER principle note
``Outil de simulation 1-D MASCARET V7.1. Module de qualité d'eau TRACER. Note de principe''
written by Kamal El Kadi Abderrezzak and Marilyne Luck in 2012
\cite{elkadi_tracer_2012} (ref: EDF R\&D-LNHE H-P73-2011-01786-FR).
It also includes the heat atmosphere exchange subsection of the \telemac{3D} theory guide.

TRACER is the transport and water quality module of the 1D free surface code MASCARET.
TRACER simulates the evolution of several coupled tracers
without retroaction on the flow with respect to hydraulic conditions,
boundary conditions, external sources and biochemical interactions between tracers.\\

This note contains an explanation of the method retained for solving the advection-dispersion equation,
and its application to water quality modeling.

All the processes existing in TRACER were implemented in \telemac{2D} and \telemac{3D} through
a new module called \waqtel between v7.0 and v7.2.
The following water quality modules are described:
\begin{itemize}
\item O$_2$ (dissolved oxygen, organic and ammonia loads),
\item BIOMASS (phytoplankton biomass and nutrients),
\item EUTRO (dissolved oxygen, phytoplankton biomass, nutrients, organic and ammonia loads),
\item MICROPOL (micropollutants and suspended matter),
\item THERMIC (water temperature),
\item degradation law.
\end{itemize}
A reference to the library documentation of the water quality and aquatic ecology model
AED2 is also given.

\newpage

\chapter{Water quality models}
\label{waq_models}
For the sake of simplicity, the following operator $F(C)$ is defined:

\begin{equation}
  F(C) = \frac{\partial C}{\partial t} + \vec{U} \cdot \vec \nabla C
       - \nabla \cdot \left( k \vec \nabla C \right),
\end{equation}

with $C(x,y,z,t)$ is the tracer concentration,
$t$ is time, $(x,y,z)$ the coordinates,
$k$ the diffusion coefficient (m$^2$/s),
$\vec{U}$ the velocity vector (m/s).\\


The studied substances are advected and dispersed in the water mass.
The dispersion is due to the flow transport and to the flow turbulence.

The concentration of a substance (e.g. pollutant, oxygen) is also influenced by:
\begin{itemize}
\item punctual contributions, caused by releases (industrial, sewage treatment plants, etc.)
  called \emph{external sources},
\item the presence of other substances in the water mass, with which the tracer
  may react through biochemical transformations or the existence of forcings
  linked to its own concentration (e.g. the reaeration phenomenon for oxygen).
  These source terms for one tracer are called \emph{internal sources}
  (because internal to the water mass) and they characterize the water quality module
  (description of the interactions between the tracers).
\end{itemize}

Solving a water quality problem consists in solving a system of $N$ (number of tracers)
advection-dispersion equations (one equation per tracer) considering the presence of
external (e.g. releases) and internal sources.

A water quality model is characterized in WAQTEL by the coupled treatment of different tracers
and the description of the \emph{internal} source terms.\\

The internal source term for a given tracer $i$ can be written through the following form:
\begin{equation}
  S_{intern_i} = \lambda_i^0 + \sum_{j=1}^{N} \lambda_i^j C_j + \frac{\mu_i^0}{h}
               + \frac{\sum_{j=1}^{N}\mu_i^j C_j}{h}
\end{equation}

with $\lambda_i^0$ and $\mu_i^0$ are terms not depending on the tracer concentration $i$
and $h$ is the water depth.
The 1$^{\rm{st}}$ term represents the volumic internal sources
(e.g. chemical reactions) whereas the 2$^{\rm{nd}}$ term represents the surface internal sources
(e.g. deposition, re-suspension, evaporation).

Depending on the water quality module, matrices [$\lambda$] and [$\mu$]
containing the coefficients $\lambda_i^j$ and $\mu_i^j$ are written differently.
The internal source terms are treated
in an explicit way in the advection-diffusion equation,
because they depend on the concentration of other tracers that remain
unknown at the time step $n+1$.

If there are interactions between tracers or specific evolution laws of tracers,
a water quality module can be used to determine the internal sources of tracers
that can be involved in the transport equation.

Internal sources of tracers are computed in the water quality module \waqtel
at each time step, with respect to physical parameters and
the concentrations of different tracers.
Then they are given to \telemac{2D} or \telemac{3D} for computing
the tracer evolution (by advection and diffusion+dispersion) taking the
external source terms into account.\\

Several water quality modules are available in the \waqtel library:
\begin{itemize}
  \item O$_2$ module: simplified model of dissolved oxygen,
  \item BIOMASS module: phytoplankton biomass model,
  \item EUTRO module: eutrophication model (dissolved oxygen and algal biomass),
  \item MICROPOL module: evolution of heavy metals or radioelements,
    taking into account interactions with fine sediments (suspended matter).
    However, no changes of the bed geometry is considered,
  \item THERMIC module: evolution of water temperature under the influence
    of atmospheric fluxes,
  \item AED2 model: the water quality and aquatic ecology model,
  \item degradation law.
\end{itemize}

The different modules and their features are described in the following chapters
in which the internal source terms are also detailed.

Note that the structure chosen for \waqtel enables to easily add new
water quality modules, by implementing terms relating to internal sources.


\chapter{O$_2$ Module}

The O$_2$ module is a simple oxygen evolution module,
intermediate between the Streeter and Phelps
\cite{streeter_ohio_1925} model,
which is restricted to modeling reaeration and the global oxidizable load,
and a more complete module, such as EUTRO, simulating several oxidation processes
and explicitly computes the phytoplankton evolution and its influence on oxygen.
The O$_2$ model advantage is its simplicity
(eight parameters to calibrate, excluding parameterization of weirs).
Some important parameters are assumed constant, such as benthic demand or plant respiration.
The use of the model is consequently limited to phenomena of a few days duration,
such as reservoir emptying. Three tracers are involved:

\begin{itemize}
\item dissolved oxygen O$_2$ (mgO$_2$/l),
\item the organic load L (mgO$_2$/l),
\item the ammonia load NH4 (mgH$_4$/l).
\end{itemize}

These variables are advected and dispersed in the water mass accordingly
to the advection-dispersion equation, with external and internal sources.
Six factors influencing the concentration of dissolved oxygen are considered:

\begin{itemize}
\item four factors consuming oxygen: organic load, ammonia load,
  benthic demand and plant respiration,
\item two processes creating oxygen: photosynthesis and reaeration.
\end{itemize}

The terms dealing with internal sources of each considered tracer are explained
in the following sections.

\section{Dissolved oxygen}

\subsection{Benthic demand}

The benthic demand $BEN$ is provided by the user in gO$_2$/m$^2$/d.
It is adjusted according to the temperature $T$ as:

\begin{equation}
  BEN_T = BEN_{20^\circ \rm{C}} (1.065)^{T-20}.
\end{equation}

The following Table gives some typical values of benthic demand
at $T$ = 20$^\circ$C (i.e. $BEN_{20^\circ \rm{C}}$).\\

\begin{table}[H]
 			\centering
\begin{tabular}{p{3.0in}p{3.0in}}
\hline
%row no:1
\multicolumn{1}{|p{3.0in}}{Bottom type} & 
\multicolumn{1}{|p{3.0in}|}{Typical value of $BEN$ (gO$_2$/m$^2$/d) at 20$^{\circ}$C} \\
\hline
%\hhline{--}
%row no:2
\multicolumn{1}{|p{3.0in}}{Filamentous bacteria (10~g/m$^2$)} & 
\multicolumn{1}{|p{3.0in}|}{7} \\
\hline
%\hhline{--}
%row no:3
\multicolumn{1}{|p{3.0in}}{Mud from waste water, near to release} & 
\multicolumn{1}{|p{3.0in}|}{4} \\
\hline
%\hhline{--}
%row no:4
\multicolumn{1}{|p{3.0in}}{Mud from waste water, far from release } & 
\multicolumn{1}{|p{3.0in}|}{1.5} \\
\hline
%\hhline{--}
%row no:5
\multicolumn{1}{|p{3.0in}}{Estuarine silt} & 
\multicolumn{1}{|p{3.0in}|}{1.5} \\
\hline
%\hhline{--}
%row no:6
\multicolumn{1}{|p{3.0in}}{Sand} & 
\multicolumn{1}{|p{3.0in}|}{0.5} \\
\hline
%\hhline{--}
%row no:7
\multicolumn{1}{|p{3.0in}}{Mineral soil} & 
\multicolumn{1}{|p{3.0in}|}{0.007} \\
\hline
%\hhline{--}

\end{tabular}
\end{table}

\subsection{Plant respiration}

The plant respiration $R$ (in mgO$_2$/d/l) is provided by the user.

\subsection{Photosynthesis}

The photosynthesis $P$ (in mgO$_2$/d/l) depends on algae, water depth and light,
with order of magnitude between 0.3 mgO$_2$/l/d and 9 mgO$_2$/l/d
depending on the flow discharge \cite{streeter_ohio_1925}.
In the O$_2$ model, $P$ is provided by the user.

\subsection{Reaeration}

\subsubsection{Natural reaeration}

Reaeration is an oxygen supply through the free surface.
At macroscopic scale, it can be modeled as linearly proportional to ($C_s$ - [O$_2$]),
where $C_s$ is the oxygen concentration at saturation in the water (in mgO$_2$/l)
(Reminder: $C_s$ = 9 mg/l at 20$^{\circ}$C). Therefore:

\begin{equation}
  Reaeration = k_2 \left( C_s - [\rm{O}_2] \right).
\end{equation}

The coefficient $k_2$ (d$^{-1}$) is a parameter with
orders of magnitude indicated in the Table below
(from \cite{tchobanoglous_wq_1985}).\\

\begin{table}[H]
 			\centering
\begin{tabular}{p{3.0in}p{3.0in}}
\hline
%row no:1
\multicolumn{1}{|p{3.0in}}{Type of watercourse} & 
\multicolumn{1}{|p{3.0in}|}{Interval of $k_2$ (d$^{-1}$) at 20$^{\circ}$C} \\
\hline
%\hhline{--}
%row no:2
\multicolumn{1}{|p{3.0in}}{Small ponds and backwaters} & 
\multicolumn{1}{|p{3.0in}|}{0.10-0.23} \\
\hline
%\hhline{--}
%row no:3
\multicolumn{1}{|p{3.0in}}{Sluggish streams and large lakes} & 
\multicolumn{1}{|p{3.0in}|}{0.23-0.35} \\
\hline
%\hhline{--}
%row no:4
\multicolumn{1}{|p{3.0in}}{Large streams of low flow velocity} & 
\multicolumn{1}{|p{3.0in}|}{0.35-0.46} \\
\hline
%\hhline{--}
%row no:5
\multicolumn{1}{|p{3.0in}}{Large streams of normal flow velocity} & 
\multicolumn{1}{|p{3.0in}|}{0.46-0.69} \\
\hline
%\hhline{--}
%row no:6
\multicolumn{1}{|p{3.0in}}{Swift streams} & 
\multicolumn{1}{|p{3.0in}|}{0.69-1.15} \\
\hline
%\hhline{--}
%row no:7
\multicolumn{1}{|p{3.0in}}{Rapids and waterfalls} & 
\multicolumn{1}{|p{3.0in}|}{> 1.15} \\
\hline
%\hhline{--}

\end{tabular}
\end{table}

There are plenty of formulae calculating $k_2$,
which indicates the low level of our understanding of this process.
We can distinguish conceptual formulae,
valid only within limited conditions,
and semi-empirical and empirical formulae
that are valid where conditions are not too far from those of calibration.
Actually, there is little difference between the formulae when the water depth
is between 0.3~m and 3~m.
The O$_2$ model allows using four formulae cited in \cite{mccutcheon_wq_1989}:

\begin{equation}
  k_2 = 5.23 U h^{-1.67} \quad \rm{(Tennessee~Valley~Authority)},
\end{equation}

\begin{equation}
  k_2 = 5.33 U^{0.67} h^{-1.85} \quad \rm{(Owens~et~al.)},
\end{equation}

\begin{equation}
  k_2 = 0.746 U^{2.695} h^{-3.085} J^{-0.823} \quad \rm{(Churchill~et~al.)},
\end{equation}

\begin{equation}
  k_2 = 3.9 U^{0.5} h^{-1.5} \quad \rm{(O’Connor~and~Dobbins)},
\end{equation}

with $U$ is the magnitude of velocity (in m/s)
and $J$ is the energy slope (in m).

The O’Connor and Dobbins formula provides the best results for shallow rivers.
For deep and rapid rivers, Churchill et al.'s formula is preferable.
Figure \ref{validity_domain_reaeration} shows the areas of application for selected formulae
\cite{mccutcheon_wq_1989}:

\begin{figure}[H]
  \centering
  \includegraphics[scale=0.3]{graphics/validity_domain_reaeration_O2.png}
  \caption{Application areas of Owens et al., Churchill et al. and O'Connor
    and Dobbins formulae \cite{mccutcheon_wq_1989}.}
  \label{validity_domain_reaeration}
\end{figure}

One option allows the user fixing a value for $k_2$.
Another option allows choosing one of the four formulae described above.
The value of $k_2$, valid at 20$^{\circ}$C, is adjusted according to the temperature as:

\begin{equation}
  k_2 = (k_2)_{20^{\circ}\rm{C}} (1.0241)^{T-20}.
\end{equation}

The oxygen concentration at saturation in the water $C_s$ can be determined
according to the water temperature (at 20$^{\circ}$C, $C_s$ = 9 mgO$_2$/l).
This is a parameter that must be known and can be calculated
if connected to a temperature module such as THERMIC.
Elmore and Hayes (cited in \cite{mccutcheon_wq_1989}) proposed the following formula
(with $T$ in $^{\circ}$C) for calculating $C_s$:

\begin{equation}
  C_s = 14.652 - 0.41022 T + 0.007991 T^2 - 7.7774.10^{-5}T^3.
\end{equation}

More recent models include a correction for atmospheric pressure and, in estuaries, for salinity.
However, we consider that we are far from estuarial conditions and the variations in pressure
entail insignificant variations in dissolved oxygen compared to those that we need.
Montgomery et al.'s equation (cited in \cite{mccutcheon_wq_1989})
only deviates from standard formulae by $ \pm $  0.02 mgO$_2$/l
between 0 and 50$^{\circ}$C when there is negligible salinity.

\begin{equation}
  C_s = \frac{468}{31.6+T}.
\end{equation}

The model allows choosing either a fixed $C_s$ value given
by the user or by one of the two formulae described above.\\

\subsubsection{Reaeration due to weirs (not used at the moment in WAQTEL)}

A weir can provide between 1 and 3 mg/l of dissolved oxygen \cite{mccutcheon_wq_1989}.
The ratio $r$ is defined by the relationship $$r = \frac{C_s-C_u}{C_s-C_d},$$

with $C_u$ = oxygen concentration upstream of weir,
$C_d$ = oxygen concentration downstream of weir.
The knowledge of $C_u$ and $r$  enables calculating $C_d$.
$C_d$ is directly applied.
This term therefore is not treated as a source.
The rate $r$ can be determined empirically
\cite{mccutcheon_wq_1989}: e.g.,

\begin{equation}
  r = 1 + 0.5 a b \Delta h \quad \rm{(Gameson)},
\end{equation}

\begin{equation}
  r = 1+ 0.36 a b (1+0.046 T) \Delta h \quad \rm{(Gameson~et~al.)},
\end{equation}

\begin{equation}
  r = 1 + 0.69 \Delta h (1 - 0.11 \Delta h) (1 +0.046 T) \quad \rm{(Water~Research~Laboratory)},
\end{equation}

\begin{equation}
  r = 1 + 0.38 a b  \Delta h (1 - 0.11 \Delta h) (1 +0.046 T),
  \quad \rm{(Water~Research~Laboratory)}
\end{equation}

with $a$ = measure of water quality (= 0.65 for highly polluted streams,
= 1.8 for clear streams),
$b$ = characteristic parameter of the weir,
$\Delta h$ = water level difference between upstream and downstream of the weir.
The following Table gives $b$ values for different weirs \cite{mccutcheon_wq_1989}:\\

\begin{table}[H]
 			\centering
\begin{tabular}{p{3.0in}p{3.0in}}
\hline
%row no:1
\multicolumn{1}{|p{3.0in}}{Type of weir} & 
\multicolumn{1}{|p{3.0in}|}{$b$} \\
\hline
%\hhline{--}
%row no:2
\multicolumn{1}{|p{3.0in}}{flat broad-crested regular step} & 
\multicolumn{1}{|p{3.0in}|}{0.7} \\
\hline
%\hhline{--}
%row no:3
\multicolumn{1}{|p{3.0in}}{flat broad-crested irregular step} & 
\multicolumn{1}{|p{3.0in}|}{0.8} \\
\hline
%\hhline{--}
%row no:4
\multicolumn{1}{|p{3.0in}}{flat broad-crested vertical face} & 
\multicolumn{1}{|p{3.0in}|}{0.8} \\
\hline
%\hhline{--}
%row no:5
\multicolumn{1}{|p{3.0in}}{flat broad-crested straight slope face} & 
\multicolumn{1}{|p{3.0in}|}{0.9} \\
\hline
%\hhline{--}
%row no:6
\multicolumn{1}{|p{3.0in}}{flat broad-crested curved face} & 
\multicolumn{1}{|p{3.0in}|}{0.75} \\
\hline
%\hhline{--}
%row no:7
\multicolumn{1}{|p{3.0in}}{round broad-crested curved face} & 
\multicolumn{1}{|p{3.0in}|}{0.6} \\
\hline
%\hhline{--}
%row no:8
\multicolumn{1}{|p{3.0in}}{sharp-crested straight slope face} & 
\multicolumn{1}{|p{3.0in}|}{1.05} \\
\hline
%\hhline{--}
%row no:9
\multicolumn{1}{|p{3.0in}}{sharp-crested vertical face} & 
\multicolumn{1}{|p{3.0in}|}{0.8} \\
\hline
%\hhline{--}
%row no:10
\multicolumn{1}{|p{3.0in}}{sluice gates with submerged discharge} & 
\multicolumn{1}{|p{3.0in}|}{0.05} \\
\hline
%\hhline{--}

\end{tabular}
\end{table}

In the model, $r$ can be either a fixed value given by the user or
calculated by one of the 4 formulae previously described.\\

\section{Organic load}

The organic load $L$ (in mgO$_2$/l) is a variable evolving over time
from an initial condition according to a 1$^{\rm{st}}$ order law:

\begin{equation}
  F([L]) = -k_1 [L],
\end{equation}

where $k_1$ is the kinetic degradation constant of the organic load (d$^{-1}$).
It is a parameter of the model.
In the O$_2$ model, the organic load $L$ is considered to be an independent variable.

\section{Ammonia load}

Also consuming oxygen, the variable ammonia load NH$_4$ (in mgH$_4$/l) follows
a 1$^{\rm{st}}$ order decay law

\begin{equation}
  F([NH_4]) = -k_4 [NH_4],
\end{equation}

where $k_4$ the kinetic constant of nitrification (d$^{-1}$),
a parameter of the model.
In the O$_2$ model, the ammonia load is considered to be an independent variable.\\

\section{Solved equation in 2D}

The concentration of dissolved oxygen [O$_2$] (in mgO$_2$/l) changes
according to the effect of sources:

\begin{equation}
  F([O_2]) = k_2 (C_s - [O_2]) -k_1 [L] - k_4 [NH_4] + P - R - \frac{BEN_T}{h}.
\end{equation}

Using the terminology and notations of section \ref{waq_models}
and setting $C_1$ = [O$_2$], $C_2$ = [L] and $C_3$ = [NH$_4$],
the matrices [$ \lambda $] and [$ \mu $]
%containing the coefficients $\lambda_i^j$ and $ \mu_i^j$
are written as:\\

$$  [\lambda] = \frac{1}{86400}
  \begin{pmatrix}
    -k_2 & -k_1  & -k_4 \\
     0   & -k_1  & 0    \\
     0   &  0    & -k_4
  \end{pmatrix}
$$

$$
  [\mu] =
  \begin{pmatrix}
     0  & 0 & 0 \\
     0  & 0 & 0 \\
     0  & 0 & 0
  \end{pmatrix}
$$

(Note: Division by 86,400 scales down the time to one second).\\

The only non-zeros terms $ \lambda_1^0$ and $ \mu_1^0$ are:

\begin{equation}
  \lambda_1^0 = \frac{k_2 C_s + P - R}{86400},
\end{equation}

\begin{equation}
  \mu_1^0 = -\frac{BEN_T}{86400}.
\end{equation}

\section{Solved equation in 3D}

The concentration of dissolved oxygen [O$_2$] (in mgO$_2$/l) changes
according to the effect of sources:

\begin{equation}
  F([O_2]) = k_2 (C_s - [O_2]) -k_1 [L] - k_4 [NH_4] + P - R  - \frac{BEN_T}{\Delta z}.
\end{equation}

with $\Delta z$ half height of the bottom layer cells.

Using the terminology and notations of section \ref{waq_models}
and setting $C_1$ = [O$_2$], $C_2$ = [L] and $C_3$ = [NH$_4$],
the matrices [$ \lambda $] and [$ \mu $]
%containing the coefficients $\lambda_i^j$ and $ \mu_i^j$
are written as:\\

$$  [\lambda] = \frac{1}{86400}
  \begin{pmatrix}
    -k_2 & -k_1  & -k_4 \\
     0   & -k_1  & 0    \\
     0   &  0    & -k_4
  \end{pmatrix}
$$

$$
  [\mu] =
  \begin{pmatrix}
     0  & 0 & 0 \\
     0  & 0 & 0 \\
     0  & 0 & 0
  \end{pmatrix}
$$

(Note: Division by 86,400 scales down the time to one second).\\

The only non-zeros terms $ \lambda_1^0$ and $ \mu_1^0$ are:

\begin{equation}
  \lambda_1^0 = \frac{k_2 C_s + P - R - \frac{BEN_T}{\Delta z}}{86400},
\end{equation}

\begin{equation}
  \mu_1^0 = 0.
\end{equation}


\chapter{BIOMASS Module}

The BIOMASS module is a water quality module which allows the calculation of algal biomass.
It estimates the extent of vegetal colonization in terms of various parameters:
sunlight, water temperature, fertilization degree, water renewal ratio,
water turbidity and toxicity \cite{gosse_biomass_1983}.
The BIOMASS module is activated by setting \telkey{WATER QUALITY PROCESS} = 3.

It takes into account five tracers:

\begin{itemize}
\item phytoplankton biomass PHY,
\item the principal nutrients influencing its production (phosphorus, nitrogen)
  as well as the associated mineral forms, namely:

\begin{itemize}
\item dissolved mineral phosphorus assimilable by phytoplankton PO$_4$,
\item degradable phosphorus not assimilable by phytoplankton POR,
\item dissolved mineral nitrogen assimilable by phytoplankton NO$_3$,
\item degradable nitrogen not assimilable by phytoplankton NOR.
\end{itemize}
\end{itemize}

These variables are all expressed in mg/l except biomass that is expressed in $\mu$g(Chlorophyl a)/l.\\

The following sections explain the internal source terms.

For more details about the theory of the O2 module,
the reader can refer to the \waqtel technical manual.


\section{Processes represented}

The bottom and the processes that occur there are not modeled in the BIOMASS model.
Deposition is only represented by the deposition flux and,
once organic matter is deposited,
it no longer appears in the equations and can no longer be resuspended.
These deposition fluxes therefore correspond to a definitive loss of mass.


\section{Phytoplankton}

\subsection{Algal growth}

The algal growth rate $CP$ (d$^{-1}$) is given by:

\begin{equation*}
  CP = C_{max} RAY g_1 LNUT \alpha_1,
\end{equation*}

with $C_{max}$ = maximum algal growth rate at 20$^{\circ}$C;
one can set its value with the keyword \telkey{MAXIMUM ALGAL GROWTH RATE AT 20C} (default = 2).
$RAY$ represents the effect of sunlight on algal growth,
this dimensionless parameter ranges between 0 and 1.
$g_1 = T/20$ represents the effect of temperature on algal growth.
$LNUT$ represents the effects of phosphoric and nitric nutrients on algal growth.
$\alpha_1$ = water toxicity coefficient for algae ($\alpha_1$ = 1 in the absence of toxicity),
this last value can be chosen with the 1$^{\textrm{st}}$ value of the keyword
\telkey{ALGAL TOXICITY COEFFICIENTS} (default = 1).\\
$RAY$ is calculated by the Smith formula averaged over the vertical:

\begin{equation*}
  RAY = \frac{1}{k_e h} \log \left( \frac{I_0 + \sqrt{IK^2+I_0^2} }{ I_h + \sqrt{IK^2+I_h^2} }  \right),
\end{equation*}

where $k_e$ is the extinction coefficient of solar rays in water.
The formula to compute $k_e$ can be chosen with the keyword
\telkey{METHOD OF COMPUTATION OF RAY EXTINCTION COEFFICIENT} (default = 1):
\begin{itemize}
  \item 1: Atkins formula, it is calculated either by the Secchi depth $Z_s$
(keyword \telkey{SECCHI DEPTH}, default = 0.9~m, if it is a constant value),
    %$k_e$ = 1.7/$Z_s$,
  \item 2: the Moss relation: $k_e$ = $k_{pe}$+$ \beta $ [PHY] if $Z_s$ is unknown,
    where $k_{pe}$ is the coefficient of vegetal turbidity without phytoplankton
    provided with the keyword \telkey{VEGETAL TURBIDITY COEFFICIENT WITHOUT PHYTO}
    (default = 0~m$^{-1}$)
    and $\beta$ the Moss coefficient ($\beta \approx$ 0.015),
  \item 3: constant given by the user with the keyword
    \telkey{LIGHT EXTINCTION COEFFICIENT} (default = 0.2~m$^{-1}$).
\end{itemize}

$IK$ is a calibrating parameter
associated to the keyword \telkey{PARAMETER OF CALIBRATION OF SMITH FORMULA}
(default = 120~W/m$^2$) of an order of magnitude 100.
$I_0$ is the flux density of solar radiation on the surface
which can be set with the keyword \telkey{SUNSHINE FLUX DENSITY ON WATER SURFACE}
(default = 0~W/m$^2$)
and $I_h$ is the flux density of solar radiation at the bed bottom (W/m$^2$), calculated as:

\begin{equation*}
  I_h = I_0 \exp (-k_e h).
\end{equation*}

$LNUT$ is calculated by the formula:

\begin{equation*}
  LNUT = \min \left( \frac{[PO_4]}{KP+[PO_4]}, \frac{[NO_3]}{KN+[NO_3]} \right),
\end{equation*}

with $KP$ = phosphate half-saturation constant
set with the keyword \telkey{CONSTANT OF HALF-SATURATION WITH PHOSPHATE}
(default = 0.005 mgP/l),
and $KN$ = nitrate half-saturation constant
set with the keyword \telkey{CONSTANT OF HALF-SATURATION WITH NITROGEN}
(default = 0.03~mgN/l).\\

\subsection{Algal disappearance}

The algal disappearance rate $DP$ (d$^{-1}$) is given as:

\begin{equation*}
  DP = (RP+MP) g_2,
\end{equation*}

with $RP$ = algal biomass respiration rate at 20$^{\circ}$C
given by the keyword \telkey{RESPIRATION RATE OF ALGAL BIOMASS}
(default = 0.05~d$^{-1}$),
$MP$ = algal biomass disappearance rate at 20$^{\circ}$C (d$^{-1}$).
$g_2 = T/20$ represents the effect of temperature on algal disappearance.
$MP$ is given by the following relation:

\begin{equation*}
  MP = M_1 + M_2 [PHY] + \alpha_2,
\end{equation*}

with $M_1$ and $M_2$ = algal mortality coefficients at 20$^{\circ}$C
which can be set with the keyword
\telkey{COEFFICIENTS OF ALGAL MORTALITY AT 20C} (default = (0.1;0.003)),
$\alpha_2$ = water toxicity coefficient for algae,
this last value can be chosen with the 2$^{\textrm{nd}}$ value of the keyword
\telkey{ALGAL TOXICITY COEFFICIENTS} (default = 0).

\section{Nitric and phosphoric nutrients}

The following physical and biochemical parameters are used
to describe the processes influencing the evolution of nitric and phosphoric nutrients:

\begin{itemize}
\item \telkey{PROPORTION OF PHOSPHORUS WITHIN PHYTO CELLS}
  for the average proportion of phosphorus in the cells of living phytoplankton $fp$ (0.0025~mgP/$\mu$gChlA),
\item \telkey{PERCENTAGE OF PHOSPHORUS ASSIMILABLE IN DEAD PHYTO}
  for the proportion of directly assimilable phosphorus in dead phytoplankton $dtp$ (default = 0.5),
\item \telkey{RATE OF TRANSFORMATION OF POR TO PO4}
  for transformation rate of POR into PO$_4$ through bacterial mineralization $k_1$ (default = 0.03~d$^{-1}$),
\item \telkey{RATE OF TRANSFORMATION OF NOR TO NO3}
  for transformation rate of NOR into NO$_3$ through heterotrophic
  and autotrophic bacterial mineralization $k_2$ (default = 0~d$^{-1}$),
\item \telkey{PROPORTION OF NITROGEN WITHIN PHYTO CELLS}
  for the average proportion of directly assimilable nitrogen in living phytoplankton $fn$ (0.0035~mgN/$\mu$gChlA),
\item \telkey{PERCENTAGE OF NITROGEN ASSIMILABLE IN DEAD PHYTO}
  for the proportion of directly assimilable nitrogen in dead phytoplankton $dtn$ (default = 0.5),
\item $F_{POR}$: deposition flux of non-algal organic phosphorus (g/m$^2$/s).
  $F_{POR} = W_{POR} [POR]$,
  $W_{POR}$ is the sedimentation velocity of non-algal organic phosphorus
  given by the keyword \telkey{SEDIMENTATION VELOCITY OF ORGANIC PHOSPHORUS}
  (default = 0~m/s),
\item $F_{NOR}$: deposition flux of non-algal organic nitrogen (g/m$^2$/s).
  $F_{NOR} = W_{NOR} [NOR]$, $W_{NOR}$ is the sedimentation velocity of non-algal organic nitrogen
  given by the keyword \telkey{SEDIMENTATION VELOCITY OF NON ALGAL NITROGEN}
  (default = 0~m/s).
\end{itemize}

\begin{WarningBlock}{Warning:}
Since release 8.5 and as long as no good solution to implement the treatment of
sedimentation velocities (N and P) is found, this treatment has
been commented in the source code so that it is not possible to take them into
account \emph{in 3D}.
Thus keywords \telkey{SEDIMENTATION VELOCITY OF ORGANIC PHOSPHORUS} and
\telkey{SEDIMENTATION VELOCITY OF NON ALGAL NITROGEN} are not taken into account.
The behaviour is as they would be let to default value = 0~m/s.
\end{WarningBlock}


\chapter{EUTRO Module}

The EUTRO module describes the oxygenation of a river and
is not restricted to modeling reaeration and the global oxidizable load.
It takes into account the effect of planktonic photosynthesis,
models the nitric and phosphoric nutrients and their effect
on phytoplankton (\cite{gosse_doubs_1989} and \cite{gosse_doubs_1983}).\\

This module is a combination of the O$_2$ and BIOMASS modules,
except for a more precise treatment of some parameters,
taking into account the ammonia load in exchanges between nitrogen and phytoplankton,
and of phytoplankton in the calculation of photosynthesis.
More sophisticated than the O$_2$ module, the EUTRO module requires setting
the values of 28 parameters (excluding the parameterization of the weirs).\\

The EUTRO module involves 8 tracers:

\begin{itemize}
\item dissolved oxygen O$_2$,
\item phytoplankton biomass (which consumes oxygen through photosynthesis) PHY,
\item the main elements influencing their production
  (phosphorus, nitrogen, ammonia load, organic load)
  as well as the mineral forms associated with phosphorus and nitrogen:
\begin{itemize}
\item dissolved mineral phosphorus assimilable by phytoplankton PO$_4$,
\item degradable phosphorus non-assimilable by phytoplankton POR,
\item dissolved mineral nitrogen assimilable by phytoplankton NO$_3$,
\item ammonia load assimilable by phytoplankton (and consuming oxygen) NH$_4$,
\item degradable nitrogen non-assimilable by phytoplankton NOR,
\item organic load (consuming oxygen) L.
\end{itemize}
\end{itemize}


These variables are expressed in mg/l, except for biomass which is expressed in $\mu$g (Chlorophyll a)/l.\\

These substances act as tracers,
i.e. they are carried and dispersed in the water mass.
In addition, they react with each other through biochemical processes.

\section{Processes represented}

Figure \ref{ecosyst_scheme} presents the various phenomena modeled\ by the EUTRO model.\\

The following parts show the parameters used and detail internal sources for each of the 8 tracers studied.\\

As with the BIOMASS module, sediment transport and resulting bed changes
are not modeled in the EUTRO module
(only a deposition flux is taken into account and the quantities deposited no longer appear in the equations).\\

\begin{figure}[H]
  \centering
%  \includegraphics[scale=1.]{graphics/image38.png}
  \includegraphics[width=3.76in,height=4.01in]{graphics/image38.png}
  \caption{Schematic representation of the ecosystem modeled by the EUTRO module \cite{gosse_doubs_1989}.
    Figure taken from \cite{elkadi_tracer_2012}.}
  \label{ecosyst_scheme}
\end{figure}

\section{Phytoplankton}

\subsection{Algal growth}

The algal growth rate $CP$ (d$^{-1}$) is given by the equation:

\begin{equation}
  CP = C_{max} RAY g_1 LNUT \alpha_1.
\end{equation}

The parameters $C_{max}$, $RAY$, $g_1$ and $\alpha_1$ are defined in the same way as in the BIOMASS module.\\

The parameter representing the effects of phosphoric
and nitric nutrients on the algal growth $LNUT$ takes into account
the ammonia load assimilable by the phytoplankton NH$_4$ and is therefore defined by:

\begin{equation}
  LNUT = \min \left( \frac{[PO_4]}{KP+[PO_4]}, \frac{[NO_3]+[NH_4]}{KN+[NO_3]+[NH_4]} \right),
\end{equation}

with $KP$ half-saturation constant in phosphate (mg/l) (about 0.005 mgP/l),
and $KN$ half-saturation constant in nitrates (mg/l) (about 0.03 mgN/l).\\

Note: The influence of nutrients on phytoplankton growth PHY is only an intermediate limiting factor $LNUT$.
When [PO$_4$] and [NO$_3$] are high enough, $LNUT$ is close to 1
and phytoplankton evolution no longer depends on nutrients.
In this case, there is no need to model the cycles of phosphorus and
nitrogen for simulating the evolution of phytoplankton.

\subsection{Algal disappearance}

The algal disappearance rate $DP$ (d$^{-1}$) is given by the equation:

\begin{equation}
  DP = (RP + MP) g_2,
\end{equation}

where $RP$ and $MP$ are defined in the BIOMASS module.
The effect of temperature on algal disappearance
is represented by the function $g_2 = (1,050)^{T-20}$,
where $T$ is the water temperature ($^{\circ}$C) (valid for 5$^{\circ}$C < $T$ < 25$^{\circ}$C).

\section{Nitric and phosphoric nutrients}

The following physical and biochemical parameters are used to describe processes
influencing the evolution of nitric and phosphoric nutrients:

\begin{itemize}
\item $fp$: average proportion of phosphorus in the cells of living phytoplankton (mgP/$\mu$gChlA),
\item $dtp$: proportion of directly assimilable phosphorus in dead phytoplankton ($\%$),
\item $k_{320}$: transformation rate of POR into PO$_4$ through bacterial mineralization
  at 20$^{\circ}$C (d$^{-1}$),
\item $k_{620}$: transformation rate of NOR into NO$_3$ through heterotrophic and autotrophic
  bacterial mineralization at 20$^{\circ}$C (d$^{-1}$),
\item $fn$: average proportion of nitrogen in the cells of living phytoplankton (mgN/$\mu$gChlA),
\item $dtn$: proportion of directly assimilable nitrogen in dead phytoplankton ($\%$),
\item $n$: quantity of oxygen consumed by nitrification (mgO$_2$/mgNH$_4$),
\item $k_{520}$: kinetics of nitrification at 20$^{\circ}$C (d$^{-1}$),
\item $F_{POR}$: deposition flux of non-algal organic phosphorus (g/m$^2$/s) = W$_{POR}$.[POR],
  with W$_{POR}$ the sedimentation velocity of non-algal organic phosphorus (m/s),
\item $F_{NOR}$: deposition flux of non-algal organic nitrogen (g/m$^2$/s) = W$_{NOR}$.[NOR],
  with W$_{NOR}$ the sedimentation velocity of non-algal organic nitrogen (m/s),
\item $Rn$: proportion of nitrogen assimilated in the form of NH$_4$ = $\frac{[NH_4]}{[NH_4]+[NO_3]}$.
\end{itemize}

\section{Organic load}

The following physical and biochemical parameters are used to describe processes
influencing the evolution of the organic load ($L$):

\begin{itemize}
\item $k_{120}$: kinetic degradation constant for the organic load at 20$^{\circ}$C (d$^{-1}$),
\item $g_3$: effect of temperature on the degradation of organic load = $(1.047)^{T-20}$,
  where $T$ is the water temperature ($^{\circ}$C) (valid for 5$^{\circ}$C < $T$ < 25$^{\circ}$C),
\item $F_{LOR}$: deposition flux of the organic load (g/m$^2$/s) = $W_{LOR}$.[L],
  with $W_{LOR}$ the sedimentation velocity of the organic load (m/s).
\end{itemize}

\section{Dissolved oxygen}

The following physical and biochemical parameters are used to describe processes
influencing the dissolved oxygen balance (O$_2$):

\begin{itemize}
\item $f$: oxygen quantity produced by photosynthesis (mgO$_2$/$\mu$gChlA),
\item $BEN$: benthic oxygen demand (gO$_2$/m$^2$/d) (cf. O$_2$ model),
\item $k_2$: coefficient of water-atmosphere gaseous exchange,
  also called reaeration coefficient, at 20$^{\circ}$C (d$^{-1}$).
  It can be provided by the user or calculated using formulae in the literature (cf. O$_2$ module),
\item $g_4$: effect of temperature on natural reaeration = $(1.025)^{T-20}$,
  where $T$ is the water temperature ($^{\circ}$C) (valid for 5$^{\circ}$C < $T$ < 25$^{\circ}$C),
\item $C_s$: concentration of oxygen saturation in water (mgO$_2$/l).
  It can be determined from the water temperature (cf. O$_2$ module).
%\item $r$: relationship defining the influence of weirs on oxygen concentration (cf. O$_2$ module)
\end{itemize}

\section{Solved equations in 2D}

The EUTRO model equations are described below:\\

Tracer $\#$1: phytoplankton biomass

\begin{equation}
  F([PHY]) = (CP-DP) [PHY].
\end{equation}

Tracer $\#$2: assimilable mineral phosphorus

\begin{equation}
  F([PO_4]) = fp(dtp DP - CP) [PHY] + k_{320} g_2 [POR].
\end{equation}

Tracer $\#$3: non-assimilable phosphorus

\begin{equation}
  F([POR]) = fp(1-dtp) DP [PHY] - k_{320} g_2 [POR] - \frac{F_{POR}}{h}.
\end{equation}

Tracer $\#$4: assimilable mineral nitrogen

\begin{equation}
  F([NO_3]) = - fn (1- Rn) CP [PHY] + k_{520} g_2 [NH_4].
\end{equation}

Tracer $\#$5: non-assimilable nitrogen

\begin{equation}
  F([NOR]) = fn (1- dtn) DP [PHY] - k_{620} g_2 [NOR] - \frac{F_{NOR}}{h}.
\end{equation}

Tracer $\#$6: ammonia load

\begin{equation}
  F([NH_4]) = fn (dtn DP - Rn CP) [PHY] + k_{620} g_2 [NOR] - k_{520} g_2 [NH_4].
\end{equation}

Tracer $\#$7: organic load

\begin{equation}
  F([L]) = f.MP [PHY] - k_{120} g_3 [L] - \frac{F_{LOR}}{h}.
\end{equation}

Tracer $\#$8: dissolved oxygen

\begin{equation}
  F([O_2]) = f (CP - RP.g_1) [PHY] - n k_{520} g_2 [NH_4] - k_{120} g_3 [L] + k_2 g_4 (C_s - [O_2]) - \frac{BEN}{h}.
\end{equation}

Using the terminology and notations of section \ref{waq_models} and setting
$C_1$ = [PHY], $C_2$  = [PO$_4$], $C_3$ = [POR], $C_4$ = [NO$_3$], $C_5$ = [NOR],
$C_6$ = [NH$_4$], $C_7$ = [L], and $C_8$ = [O$_2$],
matrices (8 $\times$ 8) $[\lambda]$ and $[\mu]$
%$\lambda_i^j$ and $\mu_i^j$
are written as (only non-zero terms are mentioned):\\

$$  \lambda_i^j = \frac{1}{86400}
  \begin{pmatrix}
    CP-DP               & 0 &            0 & 0 & 0 & 0 & 0 & 0 \\
    fp (dtp DP -CP)     & 0 &  k_{320} g_2  & 0 & 0 & 0 & 0 & 0 \\
    fp (1-dtp) DP       & 0 & -k_{320} g_2  & 0 & 0 & 0 & 0 & 0 \\
   -fn (1 -Rn) CP       & 0 &        0 & 0 & 0 &  k_{520} g_2 & 0 & 0 \\
    fn (1-dtn) DP       & 0 &        0 & 0 & -k_{620} g_2 & 0 & 0 & 0 \\
    fn (dtn DP - Rn CP) & 0 &        0 & 0 &  k_{620} g_2 & -k_{520} g_2 & 0 & 0 \\
    f . MP              & 0 &        0 & 0 & 0 & 0 & -k_{120} g_3 & 0 \\
    f (CP - RP .  g_1 ) & 0 &        0 & 0 & 0 & -n.k_{520} g_2 & -k_{120} g_3 & - k_2 g_4
  \end{pmatrix}
$$

$$
  \mu_i^j =
  \begin{pmatrix}
   0 & 0 & 0 & 0 & 0 & 0 & 0 & 0 \\
   0 & 0 & 0 & 0 & 0 & 0 & 0 & 0 \\
   0 & 0 & -W_{POR} & 0 & 0 & 0 & 0 & 0 \\
   0 & 0 & 0 & 0 & 0 & 0 & 0 & 0 \\
   0 & 0 & 0 & 0 & -W_{NOR} & 0 & 0 & 0 \\
   0 & 0 & 0 & 0 & 0 & 0 & 0 & 0 \\
   0 & 0 & 0 & 0 & 0 & 0 & -W_{LOR} & 0 \\
   0 & 0 & 0 & 0 & 0 & 0 & 0 & 0
  \end{pmatrix}
$$

The only non-zero terms $\lambda_i^0$ and $\mu_i^0$ are:\\

$\lambda_8^0 = \frac{k_2 g_4 C_s}{86400}$ ; $\mu_8^0 = -\frac{BEN}{86400}$.

Divisions by 86,400 are performed to scale down time to one second.

\section{Solved equations in 3D}

The EUTRO model equations are described below:\\

Tracer $\#$1: phytoplankton biomass

\begin{equation}
  F([PHY]) = (CP-DP) [PHY].
\end{equation}

Tracer $\#$2: assimilable mineral phosphorus

\begin{equation}
  F([PO_4]) = fp(dtp DP - CP) [PHY] + k_{320} g_2 [POR].
\end{equation}

Tracer $\#$3: non-assimilable phosphorus

\begin{equation}
%  F([POR]) = fp(1-dtp) DP [PHY] - k_{320} g_2 [POR] - F_{POR}.
  F([POR]) = fp(1-dtp) DP [PHY] - k_{320} g_2 [POR].
\end{equation}

Tracer $\#$4: assimilable mineral nitrogen

\begin{equation}
  F([NO_3]) = - fn (1- Rn) CP [PHY] + k_{520} g_2 [NH_4].
\end{equation}

Tracer $\#$5: non-assimilable nitrogen

\begin{equation}
%  F([NOR]) = fn (1- dtn) DP [PHY] - k_{620} g_2 [NOR] - F_{NOR}.
  F([NOR]) = fn (1- dtn) DP [PHY] - k_{620} g_2 [NOR].
\end{equation}

Tracer $\#$6: ammonia load

\begin{equation}
  F([NH_4]) = fn (dtn DP - Rn CP) [PHY] + k_{620} g_2 [NOR] - k_{520} g_2 [NH_4].
\end{equation}

Tracer $\#$7: organic load

\begin{equation}
%  F([L]) = f.MP [PHY] - k_{120} g_3 [L] - F_{LOR}.
  F([L]) = f.MP [PHY] - k_{120} g_3 [L].
\end{equation}

Tracer $\#$8: dissolved oxygen

\begin{equation}
  F([O_2]) = f (CP - RP.g_1) [PHY] - n k_{520} g_2 [NH_4] - k_{120} g_3 [L] + k_2 g_4 (C_s - [O_2]) - \frac{BEN_T}{\Delta z}.
\end{equation}

with $\Delta z$ half height of the bottom layer cells.

Using the terminology and notations of section \ref{waq_models} and setting
$C_1$ = [PHY], $C_2$  = [PO$_4$], $C_3$ = [POR], $C_4$ = [NO$_3$], $C_5$ = [NOR],
$C_6$ = [NH$_4$], $C_7$ = [L], and $C_8$ = [O$_2$],
matrices (8 $\times$ 8) $[\lambda]$ and $[\mu]$
%$\lambda_i^j$ and $\mu_i^j$
are written as (only non-zero terms are mentioned):\\

$$  \lambda_i^j =
  \frac{
  \begin{pmatrix}
    CP-DP               & 0 &            0 & 0 & 0 & 0 & 0 & 0 \\
    fp (dtp DP -CP)     & 0 &  k_{320} g_2  & 0 & 0 & 0 & 0 & 0 \\
%    fp (1-dtp) DP       & 0 & -k_{320} g_2 - W_{POR} & 0 & 0 & 0 & 0 & 0 \\
    fp (1-dtp) DP       & 0 & -k_{320} g_2  & 0 & 0 & 0 & 0 & 0 \\
   -fn (1 -Rn) CP       & 0 &        0 & 0 & 0 &  k_{520} g_2 & 0 & 0 \\
%    fn (1-dtn) DP       & 0 &        0 & 0 & -k_{620} g_2 - W_{NOR} & 0 & 0 & 0 \\
    fn (1-dtn) DP       & 0 &        0 & 0 & -k_{620} g_2 & 0 & 0 & 0 \\
    fn (dtn DP - Rn CP) & 0 &        0 & 0 &  k_{620} g_2 & -k_{520} g_2 & 0 & 0 \\
%    f . MP              & 0 &        0 & 0 & 0 & 0 & -k_{120} g_3 - W_{LOR} & 0 \\
    f . MP              & 0 &        0 & 0 & 0 & 0 & -k_{120} g_3 & 0 \\
    f (CP - RP .  g_1 ) & 0 &        0 & 0 & 0 & -n.k_{520} g_2 & -k_{120} g_3 & - k_2 g_4
  \end{pmatrix}
  }{86400}
$$

$$
  \mu_i^j =
  \begin{pmatrix}
   0 & 0 & 0 & 0 & 0 & 0 & 0 & 0 \\
   0 & 0 & 0 & 0 & 0 & 0 & 0 & 0 \\
   0 & 0 & 0 & 0 & 0 & 0 & 0 & 0 \\
   0 & 0 & 0 & 0 & 0 & 0 & 0 & 0 \\
   0 & 0 & 0 & 0 & 0 & 0 & 0 & 0 \\
   0 & 0 & 0 & 0 & 0 & 0 & 0 & 0 \\
   0 & 0 & 0 & 0 & 0 & 0 & 0 & 0 \\
   0 & 0 & 0 & 0 & 0 & 0 & 0 & 0
  \end{pmatrix}
$$

The only non-zero terms $\lambda_i^0$ and $\mu_i^0$ are:\\

$\lambda_8^0 = \frac{k_2 g_4 C_s - BEN/\Delta z}{86400}$ ; $\mu_8^0 = 0$.

Divisions by 86,400 are performed to scale down time to one second.

\begin{WarningBlock}{Warning:}
Since release 8.5 and as long as no good solution to implement the treatment of
sedimentation velocities (N, P and organic load) is found, this treatment has
been commented in the source code so that it is not possible to take them into
account \emph{in 3D}.
\end{WarningBlock}


\chapter{MICROPOL Module}

The MICROPOL module simulates the evolution of a micropollutant (radioelement or heavy metal)
in the three compartments considered to be of major importance in a river ecosystem:
water, Suspended Particulate Matter (SPM) and bottom material.

It is activated by setting \telkey{WATER QUALITY PROCESS} = 7.\\

Each of these compartments represents an homogeneous class:
SPM and sediments represent the grain-size class of clay and silt
(cohesive fine sediments, of diameter about less than 20 to 25 $\mu$m),
likely to attach the majority of micropollutants.\\

Due to adsorption and desorption of micropollutants,
SPM is one of the first links in the chain of contamination.
SPM is carried and dispersed in the water mass
as a tracer and is also subject to the laws of sedimentary physics:
it settles in calm waters and produces bottom sediments,
and can be re-suspended by a high flow.
Deposits cannot move. They are treated as tracers that can be neither advected
nor dispersed by the water mass, but are likely to be re-suspended.\\

The model considers 5 tracers:

\begin{itemize}
\item suspended matter (SS),
\item bottom sediments (SF), neither advected nor dispersed,
\item dissolved form of micropollutant,
\item the fraction adsorbed by suspended particulate matter,
\item the fraction adsorbed by bottom sediments, neither advected nor dispersed.
\end{itemize}

\subsubsection{Notes, and limitations of the MICROPOL module}

\begin{itemize}
\item whether in suspension or deposited on the bottom, the matter is considered
  to be a passive tracer:
  in other words, it does not influence the flow (no feedback).
  This hypothesis involves that the deposits depth must be negligible compared
  to the water depth (the bed is assumed to be unmodified).
\item there is no direct adsorption/desorption of dissolved micropollutants
  on the deposited matter, only on the SPM
  (the model assumes a preponderance of water – SPM exchanges over direct water
  – bottom sediment exchanges).
  Bottom sediments only become radioactive by means of polluted SPM deposition. 
\end{itemize}

\section{Suspended matter}

The model describing the evolution of SPM and bottom sediments involved in MICROPOL
is a classic representation of the deposition laws and re-suspension
of cohesive SPM, that are the laws of Krone \cite{krone_flume_1962}
and Partheniades \cite{partheniades_erosion_deposition_1965}.\\

Both processes require the knowledge of characteristic constants:

\begin{itemize}
\item deposition occurs when bottom shear stress $\tau_b$,
  which varies according to the flow conditions, becomes lower than a threshold value $\tau_s$,
  known as the critical shear stress for sedimentation
  and which can be set with the keyword \telkey{SEDIMENTATION CRITICAL STRESS}
  (default = 5~Pa) .
  It is then assumed that the SPM settles at a constant velocity $w$
  (known as the settling velocity or velocity of sedimentation)
  with the keyword \telkey{SEDIMENT SETTLING VELOCITY}
  (default = 6.10$^{-6}$~m/s),
\item re-suspension occurs when a threshold $\tau_r$,
  known as the critical shear stress for re-suspension, is exceeded.
  It can be set with the keyword \telkey{CRITICAL STRESS OF RESUSPENSION}
  (default = 1,000~Pa).
  Its importance is weighted by a constant $e$, the rate of erosion characteristic
  of deposited SPM (also known as the Partheniades constant),
  which associated keyword is \telkey{EROSION RATE} (default = 0).
\end{itemize}

\section{Micropollutants}

The model representing the evolution of micropollutants assumes
that the transfers of micropollutants (radioelement, metal)
between the dissolved and particulate phases correspond to either
direct adsorption or ionic exchanges modeled by a reversible reaction,
of 1$^{\rm{st}}$ kinetic order.
%\cite{ciffroy_doubs_1995}.
In the case of direct adsorption, the reaction can be represented in the form of
a reversible reaction, controlled by adsorption ($k_1$ in l/g/s)
and desorption velocities ($k_{-1}$ in s$^{-1}$)
which last associated keyword is \telkey{CONSTANT OF DESORPTION KINETIC}
(default = 2.5 10$^{-7}$s$^{-1}$).
It leads to an equilibrium state, and then a distribution of micropollutants
between the dissolved and particulate phase described
by the distribution coefficient $K_d = \frac{k_1}{k_{-1}}$
(set with the keyword
\telkey{COEFFICIENT OF DISTRIBUTION}, default = 1,775~l/g).
Once adsorbed, the fixed micropollutants act like SPM (deposition, re-suspension)
and can also produce areas of polluted sediment.\\

The model includes an exponential decay law (radioactive decay type) of micropollutant
concentrations in each compartment of the modeled ecosystem,
through a constant written $L$
which can be set with the keyword
\telkey{EXPONENTIAL DESINTEGRATION CONSTANT} (default = 1.13 10$^{-7}~\textrm{s}^{-1}$).

\subsection{Two-step reversible model}

Two successive-step reversible model is activated by setting the keyword \telkey{KINETIC EXCHANGE MODEL} = 2.\\

This allows the model to consider two additional tracers:

\begin{itemize}
\item $C_{ss2}$: concentration of micropollutants adsorbed by SPM ``specific sites'',
\item $C_{ff2}$: concentration of micropollutants adsorbed by bottom sediments
  ``specific sites''.
\end{itemize}

Ionic exchanges are then modelled with one supplementary slower step, associated with
keywords \telkey{CONSTANT OF DESORPTION KINETIC 2} (default = 2.5 10$^{-9}$~s$^{-1}$)
and \telkey{COEFFICIENT OF DISTRIBUTION 2}, default = 1,775~g/g).
Because of the slower exchange dynamics in the second reversible step, the default
value of \telkey{CONSTANT OF DESORPTION KINETIC 2} is significantly smaller than the
one of \telkey{CONSTANT OF DESORPTION KINETIC}.


\chapter{The THERMIC module}
\label{subs:therm:mod}
For a majority of water quality processes, the interaction with atmosphere is a key parameter.
The THERMIC module is activated by setting \telkey{WATER QUALITY PROCESS} = 11.
The neighboring conditions are taken into account through a meteorological file
like the one described in section \ref{subs:meteo:file}.
It is important to underline that the data contained in this file
can vary depending on the considered case.
The subroutine \telfile{meteo.f} can be edited by the user to customize it to his specific model.

Before version 7.0, heat exchange between water and atmosphere could have been
done with a linearised formula of the balance of heat exchange fluxes at the
free surface in \telemac{3D}. An example of an exchange with a constant atmosphere temperature
and a constant sea salinity was given as standard (as comments) through a
direct programming in the \telfile{BORD3D} subroutine.\\

A much more elaborated model has been introduced in \waqtel for 2D and 3D.

The evolution of temperature of water is tightly linked to heat fluxes through the free surface.
These fluxes (in W/m$^2$) are of 5 natures:

\begin{itemize}
\item solar radiation or sun ray flux $RS$,
\item atmospheric radiation flux $RA$,
\item water radiation or free surface radiation flux $RE$,
\item latent heat or heat flux due to advection $CV$,
\item sensitive heat of conductive origin or heat flux due to evaporation $CE$.
\end{itemize}

The final balance of (surface) source terms is given by:
\[S_{surf}=RS+RA-RE-CV-CE\]
We will give a brief description for each of these terms, for more details see \cite{El-Kadi2012}.
This surface source term is treated explicitly in \telemac{2D},
the following term is added in the explicit source term of advection-diffusion equation
of tracer$\frac{S_{surf}}{\rho C_pH}$.\\

There is a distinction done in 3D compared to 2D:
whereas the long wave radiation (atmospheric radiation $RA$) is absorbed in
the first centimetres of the water column, the short wave radiation (solar
radiation $RS$) penetrates the water column. Evaporation is calculated in 3D.

The choice of the heat exchange model can be done with the keyword
\telkey{ATMOSPHERE-WATER EXCHANGE MODEL} in the \waqtel steering file
(default value = 0: no exchange
model). Value 1 will use with the linearised formula at the free surface,
whereas value 2 will use with the model with complete balance.

These calculations require additional meteo data which may vary in time
in an expected format, rather defined
in the \telkey{ASCII ATMOSPHERIC DATA FILE} of the \telemac{3D} steering file,
see the example "heat\_exchange".

Since release v8.2 the format of the \telkey{ASCII ATMOSPHERIC DATA FILE} is
flexible with respect to the order of columns (but with the mandatory convention
for data names with the shortnames listed in \ref{subs:meteo:file}).

If not filling the necessary variables
(in 3D: wind velocities, air temperature, atmospheric pressure, cloud cover,
rainfall, relative humidity;
 in 2D: wind velocities, air temperature, atmospheric pressure, cloud cover,
rainfall, solar radiation, saturated vapour pressure),
the missing variables are considered constant along the whole computation and
their values are set by associated keywords values
(\telkey{WIND VELOCITY ALONG X}, \telkey{WIND VELOCITY ALONG Y}, or
\telkey{SPEED AND DIRECTION OF WIND} for wind velocity,
\telkey{VALUE OF ATMOSPHERIC PRESSURE} for atmospheric pressure,
\telkey{RAIN OR EVAPORATION IN MM PER DAY} for rainfall,
\telkey{AIR TEMPERATURE}, \telkey{CLOUD COVER}, \telkey{RELATIVE HUMIDITY},
\telkey{SOLAR RADIATION}, \telkey{VAPOROUS PRESSURE}).

%The format may be changed but the user has to change the
%implementation of the reading and the interpolation of the meteorological data.

When using the complete module, evaporation is calculated by \telemac{3D}, but
the user has to provide rainfall data with units homogeneous with length over
time.

The main developments of this module are implemented in the module
\telfile{EXCHANGE\_WITH\_ATMOSPHERE} in 3D and in the \telfile{CALCS2D\_THERMIC} in 2D.


\section{Sun ray flux RS}

Sun ray flux is simply provided in the \telkey{ASCII ATMOSPHERIC DATA FILE} in 2D.
In a majority of cases, when no measurements are available,
this flux is estimated using the method of Perrin \& Brichambaut (\cite{El-Kadi2012}),
which uses the cloud cover of the sky that varies during the day (function of time).
So far, this flux is considered constant in space.

Since release 8.2, solar radiation or sun ray can be either read in the 
\telkey{ASCII ATMOSPHERIC DATA FILE} or computed by \waqtel in 3D.
To read it in the meteo file, the keyword
\telkey{SOLAR RADIATION READ IN METEO FILE} is to be activated (default = NO,
i.e. it is computed by \waqtel in 3D).
Moreover, an additional column is to be written with the shortname RAY3
as headline of the column.

For more real cases, the user is invited to use the ``heat exchange'' module
(in folder sources/telemac3d).
A sun ray flux varying in space, common between \telemac{2D} and \telemac{3D} will be implemented in next releases.

In 3D, examples of solar radiation penetration in the water $RS$ are given in the
\telfile{CALCS3D\_THERMICV} subroutine. Two laws are suggested: the first one
uses the \emph{in situ} measurements of Secchi length and is
recommended if available; the second one uses two exponential laws that may be
difficult to calibrate and require an estimation of the type of water from
turbidity.
An example of the a double exponential law is commented.
A similar law as Atkins formula with Secchi length (see BIOMASS module) can also be
used with light extinction coefficient directly given with the keyword
\telkey{LIGHT EXTINCTION COEFFICIENT} as soon as
\telkey{METHOD OF COMPUTATION OF RAY EXTINCTION COEFFICIENT} is set to 3.

The type of sky related to the luminosity of the site has to be chosen
with respect to the considered area, with the \waqtel keyword
\telkey{LIGHTNESS OF THE SKY} (1: very pure sky, 2: mean pure sky which is default
or 3: industrial zone sky) in 3D.


\section{Atmospheric radiation RA}

The atmospheric radiation $RA$ is estimated with meteorological
data collected at the ground level.
It takes into account energy exchanges with the ground, water
(and energy) exchanges with the underground, etc.\\

In 2D, $RA$ is estimated mainly by the air temperature, like:
\begin{equation*}
RA=e_{air}\sigma\left(T_{air}+273.15 \right)^4\left(1+k\left(\frac{c}{8}\right)^2 \right),
\end{equation*}
where:
\begin{itemize}
\item $e_{air}$ is a calibrating coefficient given by the keyword
  \telkey{COEFFICIENTS FOR CALIBRATING ATMOSPHERIC RADIATION} (default = 0.97),
\item $\sigma$ is the constant of Stefan-Boltzmann (= 5.67.10$^{-8}$ Wm$^{-2}$K$^{-4}$),
\item $T_{air}$ is air temperature given in the \telkey{ASCII ATMOSPHERIC DATA FILE},
\item $c$ = cloudiness (octas), given in the atmospheric data file
(WARNING: in \khione, it is given in tenths),
\item $k$ is the coefficient that represents the nature and elevation of clouds,
it has a mean value of 0.2 and can be changed with the keyword
\telkey{COEFFICIENT OF CLOUDING RATE}.
To simplify calculations, an average value of $k$ = 0.2 is usually taken in 2D
(but default value = 0.17 like in 3D since release 8.2,
old default value was 0.2 until release 8.1).
However, it varies like indicated in Table \ref{tab:kcloud}.
\end{itemize}

\begin{table}
  \centering
  \begin{tabular}{|l|c|}
     \hline
     Type of cloud & $k$ \\
     \hline \hline
     Cirrus & 0.04 \\
     Cirro-stratus & 0.08 \\
     Altocumulus & 0.17 \\
     Altostratus & 0.2 \\
     Cumulus & 0.2 \\
     Stratus & 0.24\\
     \hline
   \end{tabular}
  \caption{Values of $k$ depending on cloud type}\label{tab:kcloud}
\end{table}

In 3D, clouds and albedo at
the free surface determine the atmospheric radiation $RA$ penetrating the water:
\begin{equation*}
RA = (1-alb_{lw}) e_{air}\sigma(T_{air}+273.15)^{4}(1+k . \left( \frac{C}{8} \right)^{2}),
\end{equation*}
where:
\begin{itemize}
\item $alb_{lw}$ = 0.03 is the water albedo for long radiative waves
  (common value used in the literature \cite{imerito_dyresm_2007},
  \cite{henderson-sellers_energy_balance_1986}),
  (1-$alb_{lw}$) is equal to the calibrating coefficient given by the keyword
  \telkey{COEFFICIENTS FOR CALIBRATING ATMOSPHERIC RADIATION} (default = 0.97,
  hence $alb_{lw}$ = 0.03 as default value),
%\item $T_{air}$ ($^{\circ}$C) is the air temperature,
\item $e_{air}$ is the air emissivity (= $0.937.10^{-5}(T_{air}+273.15)^{2}$
if using Swinbank formula, default option),
\item $\sigma= 5.67.10^{-8}~\mathrm{{W.m^{-2}.K^{-4}}}$ is Stefan-Boltzmann's constant,
\item $C$ is the nebulosity (octas). Some meteorological services such as
M\'{e}t\'{e}o France provide this data in octas, it needs to be converted
into tenths, hence the division by 8 in the formula,
(WARNING: in \khione, it is given in tenths),
\item $k$ (dimensionless) is a parameter characterising the type of
cloud. In practise, it is difficult to know the type of cloud during the
period of simulation and a mean value of 0.17 is often used \cite{tva_heat_1972},
\cite{imerito_dyresm_2007}.
This is the new default value for the keyword
\telkey{COEFFICIENT OF CLOUDING RATE} since release 8.2.
Other choices are possible
%hard-coded in 3D but other choices are possible
(see the table \ref{tab:kcloud}).
%and can be changed in the module
%\telfile{EXCHANGE\_WITH\_ATMOSPHERE}.
\end{itemize}

When coupling \waqtel with \telemac{3D}, the \telkey{FORMULA OF ATMOSPHERIC RADIATION}
can be changed:
\begin{itemize}
\item 1: Idso and Jackson (1969),
\item 2: Swinbank (1963) which is the default formula,
\item 3: Brutsaert (1975),
\item 4: Yajima Tono Dam (2014).
\end{itemize}

The formulae in 2D and 3D are almost the same with few differences.


\section{Free surface radiation RE}

The available water is assumed to behave like a grey body.
Radiation generated by this grey body through the free surface is given by:
\begin{equation*}
RE = e_{water}\sigma\left(T_{water}+273.15 \right)^4,
\end{equation*}
where:
\begin{itemize}
\item $T_{water}$ is the mean water temperature in ${}^\circ$C.
$T_{water}$ is given by the keyword \telkey{WATER TEMPERATURE} (default 7${}^\circ$C),
\item $e_{water}$ can be seen as a calibration coefficient which depends on the nature
of the site and obstacles around it.
This coefficient is given with
\telkey{COEFFICIENTS FOR CALIBRATING SURFACE WATER RADIATION} (default 0.97).
For instance, for a narrow river with lots of trees on its banks,
$e_{water}$ is around 0.97, for large rivers or lakes it is about 0.92.
\end{itemize}


\section{Advection heat flux CV}

This flux (also called sensitive heat flux) is estimated empirically:
\begin{equation*}
CV=\rho_{air}C_{p_{air}}f(V)\left(T_{water}-T_{air} \right),
\end{equation*}
where:
\begin{itemize}
  \item $\rho_{air}$ is the air density given by
${\rho }_{air}=\ \frac{100\ P_{atm}}{\left(T_{air}+273.15\right)287}$
where $P_{atm}$ is the atmospheric pressure,
introduced in the \telkey{ASCII ATMOSPHERIC DATA FILE} or using the keyword
\telkey{VALUE OF ATMOSPHERIC PRESSURE} (default 100,000~Pa),
this is a keyword of \telemac{2D} and \telemac{3D},
\item $C_{p_{air}}$ is the air specific heat (J/kg${}^\circ$C)
  given by \telkey{AIR SPECIFIC HEAT} (default 1,005),
\item $f(V)$ is a function of the wind velocity $V$:
  \begin{itemize}
  \item in 2D: $f(V) = a+bV$,
  \item in 3D: $f(V) = b(1+V)$ for wind velocity at 2~m high
    or $f(V) = a+bV$ for wind velocity at 10~m high,
    depending on user's choice,
  \end{itemize}
\item $V$ is the wind velocity (m/s),
\item $a$, $b$ are empirical coefficients to be calibrated in 2D and in 3D (only $a$ if one single calibrating coefficient).
Their values are very close to 0.0025, but they can be changed
using \telkey{COEFFICIENTS OF AERATION FORMULA} (default = (0.002, 0.0012)).
\end{itemize}

Because the site of a study may not be equipped with local wind measurements
and these kinds of data are available at a different location, possibly far
from the studied site, a wind function is used.
In 3D, this can be a linear function with
a single coefficient of calibration $b:f(U_{2}) = b(1+U_{2}$) where $U_{2}$ is
the wind velocity at 2~m high.

To get the wind velocity at 2~m high from classical wind data at 10~m high, a
roughness length of ${z}_{0}~=~0.0002$~m has been chosen in the code,
that leads to $U_{2} \approx 0.85 U_{10}$. This value
of 0.85 (or the roughness length) may be changed by the user if needed.\\

In 3D, except for the coefficient to model the penetration of solar radiation in the
water column,
and if there is only one single calibrating coefficient,
the parameter $b$ that appears in the wind function is the
single calibration parameter of this module. Its value is given by the keyword
\telkey{COEFFICIENT TO CALIBRATE THE ATMOSPHERE-WATER EXCHANGE MODEL}
in the \waqtel steering file (default
value = 0.0025 but recommended values are between 0.0017 and 0.0035). This
keyword is both used for the linearised formula at the free surface and the
model with complete balance (values 1 and 2 for the keyword
\telkey{ATMOSPHERE-WATER EXCHANGE MODEL} in the \waqtel steering file).

Since release 8.2, the user can define a wind function depending on 2 coefficients
by filling in the \telkey{COEFFICIENTS OF AERATION FORMULA}
(not letting the default values).
These 2 coefficients are then taken into account rather the 1 single coefficient
if using \telkey{ATMOSPHERE-WATER EXCHANGE MODEL} = 2.
Contrary to the wind function with one single coefficient to calibrate where
wind velocity is taken at 2~m high, the wind function with 2 coefficients
uses wind velocity taken at 10~m high.


\section{Evaporation heat flux CE}

The evaporative heat flux $CE$ (also called latent heat flux)
is given by the following empirical formula:
\begin{equation*}
CE = L(T_{water})\rho_{air}f(V) \left(H^{sat}-H \right)
\end{equation*}
where:
\begin{itemize}
\item $L(T_{water})$ = 2,500,900 - 2,365.$T_{water}$ is the vaporization latent heat (J/Kg),
\item $f(V)$ is a function of the wind velocity $V$:
  \begin{itemize}
  \item in 2D: $f(V) = a+bV$,
  \item in 3D: $f(V) = b(1+V)$ for wind velocity at 2~m high
    or $f(V) = a+bV$ for wind velocity at 10~m high,
    depending on user's choice,
  \end{itemize}
\item $V$ is the wind velocity (m/s),
\item $H^{sat}=\frac{0.622P^{sat}_{vap}}{P_{atm}-0.378P^{sat}_{vap}}$
is the air specific moisture (humidity) at saturation (kg/kg),
\item $H = \frac{0.622P_{vap}}{P_{atm}-0.378P_{vap}}$ is the air specific humidity (kg/kg),
\item $P_{vap}$ is the partial pressure of water vapour in the air (hPa)
which is given in the \telkey{ASCII ATMOSPHERIC DATA FILE},
\item $P^{sat}_{vap}$ is the partial pressure of water vapour at saturation (hPa) which is estimated with:
\begin{equation*}
P^{sat}_{vap} = 6.11 \exp \left(\frac{17.27T_{water}}{T_{water}+237.3} \right).
\end{equation*}
\end{itemize}

When $H{}^{sat} < H$, the atmospheric radiation $RA$ is corrected by multiplying it with 1.8.



\chapter{AED2 Module}

See the AED2 model technical manual (water quality and aquatic ecology model)
available on the AED2 website:

http://aed.see.uwa.edu.au/research/models/AED/downloads/AED\_ScienceManual\_v4\_draft.pdf


\chapter{Degradation law}

%WAQTEL simulates the evolution of a tracer $C$ over time from an initial condition
%according to a degradation law that is assumed to be of 1$^{\rm{st}}$ order (i.e. a tracer decrease):

%\begin{equation}
%  F([C]) = -k_1 [C],
%\end{equation}

%where $k_1$ is the constant of tracer kinetic degradation $C$ (d$^{-1}$),
%to be specified by the user.

\waqtel can simulate usual laws for bacterial degradation with
$T_{90}$ coefficient(s):
time(s) required for 90\% of the initial bacterial population to disappear
or also described as the time for bacterial or viral concentration to decrease
by one log unit (hence the 2.3 coefficient below).
It is expressed in hours.
%mortality rate(s) of 90~\% when 90~\% of bacteria die.

In other words, it simulates the evolution of tracer(s) $C$ over time from
initial condition(s) according to a degradation law assumed to be of
1$^{\rm{st}}$ order (i.e. a tracer decrease) with constant(s) of tracer kinetic
degradation equal to $\frac{2.3}{T_{90}}$:

\begin{equation}
  F([C]) = -\frac{2.3}{T_{90}} [C],
\end{equation}

with $T_{90}$ coefficient(s) described above, in hours.
%where $T_{90}$ the time when 90~\% of bacteria die, in hours.
\\

%\begin{WarningBlock}{Note:}
%  This is not
\waqtel can also simulate the evolution of tracer(s) $C$ over time
from initial condition(s)
according to a degradation law that is assumed to be of 1$^{\rm{st}}$ order
(i.e. a tracer decrease):

\begin{equation}
  F([C]) = -k_1 [C],
\end{equation}

where $k_1$ is the constant (or one of the constants) of tracer kinetic
degradation $C$ (it can be given in h$^{-1}$ or d$^{-1}$).
%\end{WarningBlock}


\chapter{Conclusion}

WAQTEL simulates the transport of several tracers in a river or the sea
(by resolution of the advection-diffusion equation) possibly coupled
(\textit{via} source terms of the equation).
\waqtel offers a structure that allows programming further water quality modules.\\

This technical manual first shows the method of resolving the convection-dispersion equation
and its application to water quality.\\

The water quality modules available in the WAQTEL tool library are described,
namely:

\begin{itemize}
\item O2: a simplified module for dissolved oxygen,
\item BIOMASS: a module for phytoplankton biomass,
\item EUTRO: a module for river eutrophication (dissolved oxygen and algal biomass),
\item MICROPOL: a module for heavy metals or radioelements,
  taking into account their interaction with fine sediments (suspended particulate matter),
\item THERMIC: a module for water temperature evolution under the influence of atmospheric fluxes,
\item AED2: the water quality and aquatic ecology model,
\item a degradation law.
\end{itemize}

%In the future, the following tasks are expected to be addressed:
%\begin{itemize}
%\item enabling simulation of tracer transport in a hydraulic network that involves floodways,
%\item enabling a distinction to be made between minor riverbeds and major riverbeds
%  in studies of water quality,
%\item finally, taking account of the presence of dead zones (dead/storage zones).
%  This may be useful for river applications during periods of low water.
%\end{itemize}

%===========================================================================
% Bibliography
%===========================================================================

%\addcontentsline{toc}{section}{References}
\bibliographystyle{plainnat}
%\bibliography{latex/waqtel_theory_guide}

\bibliography{../../data/biblio}

%\printbibliography
\end{document}
