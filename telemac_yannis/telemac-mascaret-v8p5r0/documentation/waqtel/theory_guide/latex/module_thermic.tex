\chapter{THERMIC Module}

The THERMIC module computes the water mass temperature through the conservation of energy.
%by calculating the heat flux balance applied to it.
Only atmospheric exchanges through the free surface are taken into account,
the heat exchanges with banks or bottom are neglected.\\

\section{Origin of the tracer formulation: the equation of conservation of energy}

The energy conservation equation is written as:

\begin{equation}
  \label{energy_cons_equation}
  \frac{\partial E}{\partial t} + \vec{U} \cdot \vec \nabla E
  - \nabla \cdot \left( \lambda \vec \nabla T \right),
  = S_{vol} + \frac{S_{surf}}{h},
\end{equation}

with $E$ = internal energy (J), $\lambda$ = thermal conductivity (W.m$^{-1}$.$^{\circ}$C$^{-1}$),
$T$ = water temperature.\\

The volumic sources $S_{vol}$
(inputs linked to tributaries or load releases, inputs through the Joule effect, etc.) are
implemented by the user. The surface sources $S_{surf}$,
which are the exchange fluxes through the free surface,
are calculated by the THERMIC module.\\

The conservation energy is rearranged as a function of temperature $T$,
considering that the internal energy $E$ is:

\begin{equation}
  E = \rho C_p T,
\end{equation}

where $C_p$ is water specific heat (J/kg$^{\circ}$C)
and $\rho$ (kg.m$^{-3}$) is the water density.
Omitting $S_{vol}$, equation \ref{energy_cons_equation} reads as:

\begin{equation}
  \frac{\partial T}{\partial t} + \vec{U} \cdot \vec \nabla T
  - \nabla \cdot \left( \alpha \vec \nabla T \right)
  = \frac{S_{surf}}{\rho C_p h},
\end{equation}

where $\alpha = \frac{\lambda}{\rho C_p}$ is the thermic diffusivity coefficient
(m$^2$/s).\\

If turbulence is taken into account, the coefficient of turbulent diffusion
(or dispersion) $k$ is added to the diffusivity coefficient.
However, it can be negected as the value is usually low.
The final equation is therefore:\\

\begin{equation}
  \label{final_equation_t}
  F(T) = \frac{\partial T}{\partial t} + \vec{U} \cdot \vec \nabla T
  - \nabla \cdot \left( k \vec \nabla T \right)
  = \frac{S_{surf}}{\rho C_p h}.
\end{equation}

Equation \ref{final_equation_t} shows that the calculation of temperature $T$
is carried out in the same way
as for all other tracers, and comes down to determining the right-hand side
of the equation, i.e. surface source terms.\\

\section{A first simplified model of heat exchanges with the atmosphere in 3D}

Two models computing water-atmosphere heat exchange are available
in 3D: a linearised formula of the balance of heat exchange fluxes at
the free surface that firstly described in this section
and a model that computes the
complete balance of exchange fluxes (which gather
common formulae in 2D and 3D).\\

In the first model, the thermal power liberated into the atmosphere per
surface unit, denoted as $\Phi$, is assumed to be proportional to $(T-T_{air})$
where $T$ is the water temperature on the surface and $T_{air}$ that of
air. $\Phi=A(T-T_{air})$, where $A$ is the exchange coefficient in
W/m$^{2}$/${^{\circ}}$C.

The heat flux leaving the liquid domain is written as:
\begin{equation}
\Phi=-\rho C_{p\,}k\,\nabla T.\vec{n}=-\rho
C_{p\,}k\dfrac{\partial T}{\partial z},
\end{equation}
with $C_{p}=4180\,$J/kg/${^{\circ}}$C. Recall that $k$ (m$^2$s$^{-1}$) is the
molecular heat diffusion coefficient in water, $\vec{n}$ the normal vector to the surface.

By equating the two formulations, one deduces the boundary conditions:%
\begin{equation}
k\dfrac{\partial T}{\partial z}=-\dfrac{A}{\rho C_{p}}\left(
T-T_{air}\right).
\end{equation}

The coefficient $A$ should include phenomena such as radiation, air convection
in contact with water and latent heat produced by water evaporation. Sweers
\cite{sweers_wind_1976} expresses the coefficient $A$, in W/m$^{2}$/${^{\circ}}$C,
according to the water temperature $T$ and wind velocity $V$ measured
at the point under consideration (in m/s):

\begin{equation}
\label{sweers}A=(4.48+0.049T)+2021.5\,b\,(1+V)\,(1.12+0.018\,T+0.00158\,T^{2})
\end{equation}

The parameter $b$ varies depending on the location. Its average value on the
shores of the English Channel and of the Atlantic ocean is 0.0025. It is higher in the
Mediterranean sea, where it reaches approximately 0.0035.\\

A much more elaborated model is also implemented both in 2D and 3D.
This module calculates the complete balance of exchanged fluxes involved
and is described in the following section.

\section{Heat fluxes affecting the energy balance}

Surface sources are fluxes through the free surface.\\

There are 5 fluxes, expressed in W/m$^2$:

\begin{itemize}
\item the solar radiation flux $RS$,
\item the atmospheric radiation flux $RA$,
\item the radiation flux from water body $RE$,
\item the heat flux driven by convection $CV$,
\item the heat flux driven by evaporation $CE$.
\end{itemize}

Every flux is detailed in
\cite{gilbert_num_comp_nat_river_1986}. The result in 2D is:

\begin{equation}
  S_{surf} = RS + RA - RE - CV - CE.
\end{equation}

Whereas the long wave radiation (atmospheric radiation $RA$) is absorbed in
the first centimetres of the water column, the short wave radiation (solar
radiation $RS$) penetrates the water column. Evaporation is calculated in 3D.
Heat exchanges with the atmosphere are taken into account at two levels \textbf{in 3D}:
\begin{itemize}
\item the complete balance of exchange fluxes at the free surface is
calculated at the surface boundary conditions for temperature:
\begin{equation}
k \left.  \dfrac{\partial T}{\partial z}\right|_{z=\eta} =
\dfrac{RA-RE-CE-CV}{\rho C_p},
\end{equation}
\item the penetration of solar radiation into the water column is taken into
account in the source term of the advection-diffusion equation of the
temperature:
\begin{equation}
S=\dfrac{1}{\rho C_p}\dfrac{\partial Q(z,RS)}{\partial z},
\end{equation}
\end{itemize}
%with $K$ ($\mathrm{{m}^{2}.{s}^{-1}}$) the molecular diffusivity coefficient of
%temperature, $\rho_{water}$ ($\mathrm{{kg}.{m}^{-3}}$) is the water density,
%$C_{p}$ = 4180 $\mathrm{{J}.{kg}^{-1}.^{\circ}{C}^{-1}}$ is the specific heat
%of water at a constant pressure,
where the function $Q(z, RS)$ is the residual solar radiation at the elevation $z$.

\subsection{Solar radiation RS}

The solar radiation $RS$ can be provided to the THERMIC module
in the atmospheric data file in 2D.
If there is no measurement, global solar radiation on a water body
can be calculated by Perrin de Brichambaut's method,
depending on the cloudiness of the sky, the date and the hour
\cite{klein_ray_solaire_1979}.\\

In 3D, the solar flux penetrating the water $RS$ is calculated with:
\begin{itemize}
\item the solar radiation reaching the surface with a clear sky. It depends on
time and the location of the site. Perrin de Brichambaut's method
\cite{perrin_ray_solaire_1963}, \cite{perrin_res_solaires_1975} is used,
\item cloud cover: a corrective term depending on the nebulosity is applied
(Berliand's formulae \cite{berliand_cloud_1952}, \cite{berliand_radiatssi_1960}),
\item albedo that enables to compute the effective part of solar radiation
penetrating the water.
\end{itemize}
Solar radiation $RS$ is written as:
\begin{equation}
RS = AA.\sin(ang)^{BB}(1-0.65C^{2})(1-Alb),
\end{equation}
where:

\begin{itemize}
\item $AA$ ($\mathrm{{W}.m^{-2}}$) and $BB$ (dimensionless) are coefficients
related to luminosity and sky colour. They have to be chosen with respect to the
considered area. Three possibilities are suggested (see the table \ref{tab_sky_type}%
) but mean values are used by default in 3D,
\item $ang$ is the angular height of the sun (rad), which depends on the latitude and
longitude of the location and changes with day and hour,
%\item $C$ is the nebulosity (tenths). Some meteorological services like
%M\'{e}t\'{e}o France provide these data in octas and then need to be converted
%in tenths,
\item $Alb$ water albedo for short waves that may vary every month
\cite{payne_albedo_1972}.
\end{itemize}

\begin{table}[ptbh]
\caption{Proposed values of the coefficients $AA$ and $BB$ for the calculation of solar
radiation related to luminosity and sky colour }%
\label{tab_sky_type}%
\centering
\begin{tabular}
[c]{|c|c|c|}\hline
Type of sky & $AA$ (W.m$^{-2}$) & $BB$\\\hline
Very pure sky (type 1) & 1130 & 1.15\\\hline
Mean pure sky (type 2) & 1080 & 1.22\\\hline
Industrial area (type 3) & 995 & 1.25\\\hline
\end{tabular}
\end{table}

The penetration of solar radiation into the water column is taken into account
in the advection-diffusion equation of temperature by introducing a source
term. It depends on the water turbidity and is linked to dissolved and
suspended particles (mineral or organic). Two laws describing the solar
radiation penetration are suggested:
\begin{itemize}
\item the first one consists of two exponential laws that may be difficult to
calibrate and require an estimation of the degree of water turbidity,
\item the second one uses the \emph{in situ} measurements of Secchi length and
is therefore recommended.
\end{itemize}
In the first case, the source term reads:
\begin{equation}
Q(z,RS) = RS.\left(  R \exp\left(  -\dfrac{z_{s}-z}{\zeta_{1}} \right)  + (1-R)
\exp\left(  -\dfrac{z_{s}-z}{\zeta_{2}} \right)  \right)  ,
\end{equation}
with $RS$ ($\mathrm{{W}.m^{-2}}$) is the solar radiation penetrating the
water, $z_{s}$ (m) the free surface elevation, $\zeta_{1}$ and $\zeta_{2}$ (m)
are attenuation lengths and $R$ is a dimensionless coefficient.
%This equation means the penetration of solar radiation in the water column
%with a law of double exponentials.
This equation takes into account a selective
absorption of the solar spectrum by water, separating the radiation quickly
attenuated in the water column and the one penetrating deeply.\\

In the second case, the source term is written from Beer-Lambert's law:
\begin{equation}
Q(z,RS) = RS. \exp\left(  -\dfrac{1.7(z_{s}-z)}{Z_{s}} \right)
\end{equation}
with $RS$ ($\mathrm{{W}.m^{-2}}$) is also the solar radiation penetration the
water, $z_{s}$ (m) the free surface elevation, $Z_{s}$ (m) is the Secchi
length provided by \emph{in situ} measurements.

The coefficients $\zeta_{1}$, $\zeta_{2}$ and $R$ are characteristics of
optical properties of water adjusted to the studied water body.
Examples of combination are suggested in Table \ref{jerlov}. This
classification has been done for coastal and marine flows and is widely used
in numerical modelling.

\begin{table}[ptbh]
\caption{Selected values of $R$, $\zeta_{1}$, $\zeta_{2}$
  according to the water turbidity (\cite{paulson_irradiance_1977},
  \cite{jerlov_optical_1968}, \cite{kraus_atmos_1972})}%
\label{jerlov}%
\centering
\begin{tabular}
[c]{|c|c|c|c|}\hline
Type of water & $R$ & $\zeta_{1}$ (m) & $\zeta_{2}$ (m)\\\hline
Very clear (Kraus) & 0.40 & 5 & 40\\\hline
Type I (Jerlov) & 0.58 & 0.35 & 23\\\hline
Type IA (Jerlov) & 0.62 & 0.6 & 20\\\hline
Type IB (Jerlov) & 0.67 & 1.0 & 17\\\hline
Type II (Jerlov) & 0.77 & 1.5 & 14\\\hline
Very turbid - Type III (Jerlov) & 0.78 & 1.4 & 7.9\\\hline
\end{tabular}
\end{table}

The formulae require additional data (wind magnitude and direction,
air temperature, atmospheric pressure, relative humidity, nebulosity and rainfall).
The study site may not be equipped for local wind measurements,
while data are available at a different location (even far from the studied
site). A wind function is then used, which is a linear function with a single
coefficient of calibration $b$:
\begin{equation}
f(V_2) = b(1+V_2)
\end{equation}
with $V_2$ that is the wind velocity at 2~m high. As for the linearised
formula \eqref{sweers}, the parameter $b$ varies depending on the location. Its
average value on the shores of the English Channel and of the Atlantic ocean is 0.0025. It
is higher in the Mediterranean sea, reaching approximately 0.0035 and can
be around 0.0017 \cite{salencon_lac_1997}.

Wind data are often provided at 10 m high whereas latent and sensible fluxes
need wind velocities at 2~m high. Velocities at 2~m high need to be calculated
from wind velocities at 10~m high $V_{10}$. A logarithmic wind velocity vertical profile is
considered, with a roughness length of $z_{0} =
0.0002~\mathrm{{m}}$. This leads to $V_{2} = 0.85 V_{10}$. This value
of 0.85 (or the roughness length) may be changed if needed.
\begin{equation}
V_{2} = \dfrac{V_{10}\log\left(\dfrac{2}{z_{0}}\right)}{\log\left(\dfrac{10}{z_{0}}\right)}%
\end{equation}


\subsection{Atmospheric radiation RA}

The atmosphere emits a long wave radiation corresponding to the reemission of a
part of direct solar energy. It emits like a black body.

In 2D, the atmospheric radiation $RA$ is estimated through meteorological data collected at ground level.
Some processes of energy exchange (exchanges of heat with the bottom,
and of water with the water table, biodegradation reactions,
industrial or urban effluents) are unknown or poorly known.
They considerably depend on temperature.
In the 2D calculation, they are combined with atmospheric radiation
through the calibration coefficient $e_{air}$. The formula in 2D is:

\begin{equation}
  RA = e_{air} \sigma (T_{air} + 273.15 )^4 \left( 1+k\left( \frac{c}{8} \right) ^2 \right),
\end{equation}

with:
\begin{itemize}
  \item $e_{air}$ = calibration coefficient of atmospheric radiation (1$^{\rm{st}}$ parameter of the model),
  \item $\sigma = $ Stefan-Boltzman constant, $\sigma = 5.67.10^{-8}$ Wm$^{-2}$K$^{-4}$,
  \item $T_{air}$ = air temperature ($^{\circ}$C),
  \item $c$ = cloudiness (octas), given in the atmospheric data file
(WARNING: in \khione, it is given in tenths),
  \item $k$ = coefficient of cloud type, which depends on the type of clouds and their height
    (see Table \ref{tab_cloud_type}).
\end{itemize}
To simplify calculations, an average value of $k$ = 0.2 is usually taken in 2D.\\

\begin{table}[H]
\caption{Coefficient $k$ for the calculation of atmospheric radiation}
\label{tab_cloud_type}
\centering
\begin{tabular}{p{0.98in}p{0.39in}}
\hline
%row no:1
\multicolumn{1}{|p{0.98in}}{Type of cloud} & 
\multicolumn{1}{|p{0.39in}|}{$k$} \\
\hline
%\hhline{--}
%row no:2
\multicolumn{1}{|p{0.98in}}{Cirrus} & 
\multicolumn{1}{|p{0.39in}|}{0.04} \\
\hline
%\hhline{--}
%row no:3
\multicolumn{1}{|p{0.98in}}{Cirro-Stratus} & 
\multicolumn{1}{|p{0.39in}|}{0.08} \\
\hline
%\hhline{--}
%row no:4
\multicolumn{1}{|p{0.98in}}{Altocumulus} & 
\multicolumn{1}{|p{0.39in}|}{0.17} \\
\hline
%\hhline{--}
%row no:5
\multicolumn{1}{|p{0.98in}}{Altostratus} & 
\multicolumn{1}{|p{0.39in}|}{0.2} \\
\hline
%\hhline{--}
%row no:6
\multicolumn{1}{|p{0.98in}}{Cumulus} & 
\multicolumn{1}{|p{0.39in}|}{0.2} \\
\hline
%\hhline{--}
%row no:7
\multicolumn{1}{|p{0.98in}}{Stratus} & 
\multicolumn{1}{|p{0.39in}|}{0.24} \\
\hline
%\hhline{--}

\end{tabular}
\end{table}

In 3D, clouds and albedo at
the free surface determine the atmospheric radiation $RA$ penetrating the water:
\begin{equation}
RA = (1-alb_{lw}) e_{air}\sigma(T_{air}+273.15)^{4}(1+k . \left( \frac{C}{8} \right)^{2}),
\end{equation}
where:
\begin{itemize}
\item $alb_{lw}$ = 0.03 is the water albedo for long radiative waves
  (common value used in the literature \cite{imerito_dyresm_2007},
  \cite{henderson-sellers_energy_balance_1986}),
  (1-$alb_{lw}$) is a calibrating coefficient of atmospheric radiation,
%\item $T_{air}$ ($^{\circ}$C) is the air temperature,
\item $e_{air}$ is the air emissivity (= $0.937.10^{-5}(T_{air}+273.15)^{2}$
if using Swinbank formula, default option),
\item $\sigma= 5.67.10^{-8}~\mathrm{{W.m^{-2}.K^{-4}}}$ is Stefan-Boltzmann's constant,
\item $C$ is the nebulosity (octas). Some meteorological services such as
M\'{e}t\'{e}o France provide this data in octas, it needs to be converted
into tenths, hence the division by 8 in the formula
(WARNING: in \khione, it is given in tenths),
\item $k$ (dimensionless) is a parameter characterising the type of
cloud. In practise, it is difficult to know the type of cloud during the
period of simulation and a mean value of 0.17 is often used \cite{tva_heat_1972},
\cite{imerito_dyresm_2007} but other choices are possible
(see the table \ref{tab_cloud_type}).
\end{itemize}

When coupling \waqtel with \telemac{3D}, the formula to compute air emissivity
$e_{air}$ times the factor depending on cloud cover $C$ can be chosen:
\begin{itemize}
\item 1: Idso and Jackson (1969),
\item 2: Swinbank (1963),
\item 3: Brutsaert (1975),
\item 4: Yajima Tono Dam (2014).
\end{itemize}

The formulae in 2D and 3D are almost the same with few differences.

%\begin{table}[ptbh]
%\caption{Coefficient $k$ characterising the type of cloud }%
%\label{tab_cloud_type}%
%\centering
%\begin{tabular}
%[c]{|c|c|c|c|c|c|c|}\hline
%Type of & Cirrus & Cirro & Alto & Alto & Cumulus & Stratus\\
%cloud &  & Stratus & Cumulus & Stratus &  & \\\hline
%$k$ & 0.04 & 0.08 & 0.17 & 0.20 & 0.20 & 0.24\\\hline
%\end{tabular}
%\end{table}

\subsection{Radiation emitted from a water body RE}

One can assume, as a good approximation, that a water body behaves like a grey body.
The expression of radiation emitted by the water body $RE$
(also called water radiation) is then:\\

\begin{equation}
  RE = e_{water} \sigma (T_{water} + 273.15)^4,
\end{equation}

with $T_{water}$ = water temperature ($^{\circ}$C)
taken at the surface in 3D,
$e_{water}$ = water emissivity which can be seen as a
calibration coefficient of radiation from a water body,
which depends on the location and obstacles surrounding the water body.
For a narrow river bordered by trees, $e_{water}$ could be around 0.97
(which is the default value)
and for a widely uncovered area, $e_{water}$ could be close to 0.92.

\subsection{Convection heat flux CV}

This flux (also called sensitive heat flux)
is estimated by the following empirical formula:

\begin{equation}
  CV = \rho_{air} C_{p_{air}} f(V) (T_{water} - T_{air}),
\end{equation}

with:
\begin{itemize}
  \item $\rho_{air}$ = air density (kg/m$^3$),
    given by $\rho_{air} = \frac{100 P_{atm}}{(T_{air} + 273.15).287}$,
    where $P_{atm}$ is the atmospheric pressure in hPa,
  \item $C_{p_{air}}$ = air thermic capacity (J/kg$^{\circ}$C),
  \item $f(V)$ is a function of the wind velocity $V$:
    \begin{itemize}
    \item in 2D: $f(V) = a+bV$,
    \item in 3D: $f(V) = b(1+V)$ for wind velocity at 2~m high
      or $f(V) = a+bV$ for wind velocity at 10~m high,
      depending on user's choice,
    \end{itemize}
  \item $V$ = wind velocity (m/s),
  \item $a, b$ = empirical coefficients
    to be calibrated (2 in 2D and 3D, only $a$ in 3D if one single calibrating coefficient).
    Their values are very close, around 0.0025.
\end{itemize}

\subsection{Evaporative heat flux CE}

The evaporative heat flux $CE$ (also called latent heat flux)
due to evaporation is given by the empirical formula:

\begin{equation}
  CE = L(T_{water}) \rho_{air} f(V) (H^{sat} - H),
\end{equation}

with:
\begin{itemize}
  \item $L(T_{water}) = 2500900 - 2365 T_{water}$ = latent heat of evaporation (J/kg)
    taken at the surface in 3D,
  \item $f(V)$ is a function of the wind velocity $V$:
    \begin{itemize}
      \item in 2D: $f(V) = a+bV$,
      \item in 3D: $f(V) = b(1+V)$ for wind velocity at 2~m high
        or $f(V) = a+bV$ for wind velocity at 10~m high,
        depending on user's choice,
    \end{itemize}
  \item $V$ = wind velocity (m/s),
  \item $H^{sat} = \frac{0.622 P_{vap}^{sat}}{P_{atm}-0.378 P_{vap}^{sat}}$
    = the saturated air specific humidity (kg/kg),
    in 3D at the surface temperature,
  \item $H = \frac{0.622 P_{vap}}{P_{atm}-0.378 P_{vap}}$
    = the air specific humidity (kg/kg),
  \item $P_{vap}$ the partial pressure of air water vapor (hPa)
    provided in the atmospheric data file in 2D
    and calculated with the following formula in 3D:
    $P_{vap} = \frac{H_{rel}}{100} P_{air}$,
  \item  $P_{vap}^{sat}$ the partial pressure of saturation water vapor (hPa)
    calculated by Magnus-Tetens's formula:
\end{itemize}

\begin{equation}
  P_{sat}^{vap} = 6.11 \exp \left( \frac{17.27 T_{water}}{T_{water} + 237.3} \right)
  = \exp \left( 2.3026 \left( \frac{7.5 T_{water}}{T_{water} + 237.3} + 0.7858 \right) \right).
\end{equation}

When $H_{sat} < H$, the atmospheric radiation $RA$ is corrected
by multiplying the preceding expression by 1.8 in 2D.

The surface evaporation flow rate is then calculated in 3D by:
\begin{equation}
Deb = \frac{CE}{\rho_{water}L(T_{water})},
\end{equation}

\section{Solved equation}

The global surface source reads as:

\begin{equation}
  S_{surf} = RS + RA - RE - CV - CE.
\end{equation}

%The normal formulation in TRACER, in which source terms are written as linear combinations of tracer concentrations present (cf. Part 2), cannot therefore be carried out here.\\

In the calculations, the global surface source $S_{surf}$ is treated explicitly
and the $(\lambda_i^j, \mu_i^j)$ are all equal to zero.

The following equation is explicitly solved:

\begin{equation}
  F(T) = \frac{S_{surf}}{\rho C_p h}.
\end{equation}
