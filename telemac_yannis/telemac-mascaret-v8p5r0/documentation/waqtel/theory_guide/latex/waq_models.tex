\chapter{Water quality models}
\label{waq_models}
For the sake of simplicity, the following operator $F(C)$ is defined:

\begin{equation}
  F(C) = \frac{\partial C}{\partial t} + \vec{U} \cdot \vec \nabla C
       - \nabla \cdot \left( k \vec \nabla C \right),
\end{equation}

with $C(x,y,z,t)$ is the tracer concentration,
$t$ is time, $(x,y,z)$ the coordinates,
$k$ the diffusion coefficient (m$^2$/s),
$\vec{U}$ the velocity vector (m/s).\\


The studied substances are advected and dispersed in the water mass.
The dispersion is due to the flow transport and to the flow turbulence.

The concentration of a substance (e.g. pollutant, oxygen) is also influenced by:
\begin{itemize}
\item punctual contributions, caused by releases (industrial, sewage treatment plants, etc.)
  called \emph{external sources},
\item the presence of other substances in the water mass, with which the tracer
  may react through biochemical transformations or the existence of forcings
  linked to its own concentration (e.g. the reaeration phenomenon for oxygen).
  These source terms for one tracer are called \emph{internal sources}
  (because internal to the water mass) and they characterize the water quality module
  (description of the interactions between the tracers).
\end{itemize}

Solving a water quality problem consists in solving a system of $N$ (number of tracers)
advection-dispersion equations (one equation per tracer) considering the presence of
external (e.g. releases) and internal sources.

A water quality model is characterized in WAQTEL by the coupled treatment of different tracers
and the description of the \emph{internal} source terms.\\

The internal source term for a given tracer $i$ can be written through the following form:
\begin{equation}
  S_{intern_i} = \lambda_i^0 + \sum_{j=1}^{N} \lambda_i^j C_j + \frac{\mu_i^0}{h}
               + \frac{\sum_{j=1}^{N}\mu_i^j C_j}{h}
\end{equation}

with $\lambda_i^0$ and $\mu_i^0$ are terms not depending on the tracer concentration $i$
and $h$ is the water depth.
The 1$^{\rm{st}}$ term represents the volumic internal sources
(e.g. chemical reactions) whereas the 2$^{\rm{nd}}$ term represents the surface internal sources
(e.g. deposition, re-suspension, evaporation).

Depending on the water quality module, matrices [$\lambda$] and [$\mu$]
containing the coefficients $\lambda_i^j$ and $\mu_i^j$ are written differently.
The internal source terms are treated
in an explicit way in the advection-diffusion equation,
because they depend on the concentration of other tracers that remain
unknown at the time step $n+1$.

If there are interactions between tracers or specific evolution laws of tracers,
a water quality module can be used to determine the internal sources of tracers
that can be involved in the transport equation.

Internal sources of tracers are computed in the water quality module \waqtel
at each time step, with respect to physical parameters and
the concentrations of different tracers.
Then they are given to \telemac{2D} or \telemac{3D} for computing
the tracer evolution (by advection and diffusion+dispersion) taking the
external source terms into account.\\

Several water quality modules are available in the \waqtel library:
\begin{itemize}
  \item O$_2$ module: simplified model of dissolved oxygen,
  \item BIOMASS module: phytoplankton biomass model,
  \item EUTRO module: eutrophication model (dissolved oxygen and algal biomass),
  \item MICROPOL module: evolution of heavy metals or radioelements,
    taking into account interactions with fine sediments (suspended matter).
    However, no changes of the bed geometry is considered,
  \item THERMIC module: evolution of water temperature under the influence
    of atmospheric fluxes,
  \item AED2 model: the water quality and aquatic ecology model,
  \item degradation law.
\end{itemize}

The different modules and their features are described in the following chapters
in which the internal source terms are also detailed.

Note that the structure chosen for \waqtel enables to easily add new
water quality modules, by implementing terms relating to internal sources.
