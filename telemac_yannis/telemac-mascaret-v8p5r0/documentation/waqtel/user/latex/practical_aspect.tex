\chapter{Practical aspects}

To activate the water quality module \waqtel, the integer keyword
\telkey{WATER QUALITY PROCESS} must be set to a value different from 1 (default = 1)
in the \telemac{2D} or \telemac{3D} \telkey{STEERING FILE}.
The possible choices are:
\begin{itemize}
\item 0: all available processes,
\item 1: nothing (default value),
\item 2: O$_2$ module,
\item 3: BIOMASS module,
\item 5: EUTRO module,
\item 7: MICROPOL module,
\item 11: THERMIC module,
\item 13: AED2 model,
\item 17: degradation law.
\end{itemize}

Several modules can be combined by giving the multiplication of the process choices,
e.g. 55 = 5 $\times$ 11 activates EUTRO and THERMIC modules.
It is noted that AED2 should be used on its own, for the time being,
without possible combination with other processes.\\

In \waqtel, a new dictionary is created and fully dedicated to water quality applications.
The description of this new dictionary will be the subject of a separate manual
(see the \waqtel reference manual).
%\telemac{2D} or \telemac{3D} will not read automatically this dictionary,
%unless the keyword \telkey{WAQ DICTIONARY} is used.
%This keyword is the path to dictionary.
To introduce WAQ parameters, a separate steering file is necessary.
It is read with the use of the keyword \telkey{STEERING FILE}
and its name has to be declared in the \telemac{2D} or \telemac{3D}
\telkey{STEERING FILE}.
In the next sections which are dedicated to water quality,
all the keywords to be introduced to the water quality steering file
will be written in \telkey{THIS FONT}.

Depending on the \waqtel module, \telemac{2D} or \telemac{3D}
automatically increase the number of tracers of the model.
In fact, the total number of tracers (variable \telfile{NTRAC})
is increased by 3 (for O$_2$ process), by 5 (for BIOMASS and MICROPOL processes)
or by 8 (for EUTRO process).
%For the THERMIC module, the number of tracers is increased only
%if no temperature is present in the set of tracers already existing in the model.

Some general parameters can also be chosen for any water quality process.
For instance, user can give a title for the study by using \telkey{WAQ CASE TITLE}.
He/she can introduce water density (\telkey{WATER DENSITY}).
%viscosity of water (\telkey{KINEMATIC WATER VISCOSITY}), read but not used at the moment


\section{The meteo file or ASCII ATMOSPHERIC DATA FILE}
\label{subs:meteo:file}
Most of the water quality processes are tightly linked to
meteorological data on the domain.
Besides the specific treatment for wind, atmospheric pressure
and rainfall described in the section 6.3 of the \telemac{2D} User Manual
or in the section 5.5 of the \telemac{3D} User Manual,
water quality needs different meteorological data such as
nebulosity, air temperature, evaporation pressure etc.\\

Since release 8.2 and the creation of the new \telfile{METEO\_TELEMAC} module
based on the already existing \telfile{METEO\_KHIONE} module, it is possible to
manage meteo data in the \telkey{ASCII ATMOSPHERIC DATA FILE}.
Moreover, this module enables to use flexible format for this file as long as
the names of the meteo data are included in a specific list.
Therefore, the order of columns is flexible.\\

Here is a list of the available variables for this module and the shortnames to
be used in the headline of the \telkey{ASCII ATMOSPHERIC DATA FILE}:
\begin{itemize}
\item WINDS and WINDD: wind speed + wind direction (in m/s) or,
\item WINDX and WINDY: wind velocity components along $x$ and $y$ (in m/s),
\item TAIR: air temperature (in $^{\circ}$C),
\item PATM: atmospheric pressure (in mbar or hPa),
\item CLDC: cloud cover or nebulosity (in octas for \waqtel or tenths for
\khione),
\item RAINI or RAINC: rainfall (last letter I or C depending on if it is
an interpolated variable as other usual variables
or if it is  given as cumulated variable),
\item RAY3: solar radiation (in W/m$^2$),
\item HREL: relative humidity (in \%),
\item PVAP: saturated vapor pressure (in hPa),
\item TDEW: dewpoint temperature (in $^{\circ}$C),
\item VISBI: visibility (in m),
\item SNOW: snow (in m),
\end{itemize}

Hereafter we give examples of \telkey{ASCII ATMOSPHERIC DATA FILE}
when coupling with \telemac{2D} (see Table \ref{tab:meteo2d})
or with \telemac{3D} (see Table \ref{tab:meteo3d}).\\

The user can change the order of columns.
If additional parameters are needed, it can be done
by changing the \telfile{METEO\_TELEMAC} available in the
sources/utils/bief folder.\\

%The user can edit this file and add new parameters.
%However, in this case, he must edit subroutine \telfile{meteo.f} consequently.

Since release 8.2, a time reference can be given:
If a \#REFDATE with a date + hour in YYYY-MM-DD HH:MM:SS
in year, month, day, hour, minute, second format is written in ASCII files
related to time,
the date+hour will be added to the times in these ASCII files
when using \telemac{2D} and \telemac{3D}.
For SERAFIN or MED binary files, the date in these files will be used.

Such feature is used in the \telkey{ASCII ATMOSPHERIC DATA FILE} of the
heat\_exchange example of \waqtel (see Table \ref{tab:meteo3d}).

%TELEMAC-2D/TELEMAC-3D: Time reference in all Telemac2d/3d time related files:
%You can add #REFDATE in the ascii files and the date will be added to the times
%in the files when used in Telemac2d/3d. For SERAFIN/MED it will use the date in
%the file.

\begin{table}[h]
    \centering
%  \begin{tabular}{|l|c|c|c|c|c|c|c|c|}
  \begin{tabular}{rrrrrrrrr}
%     \hline \hline
     % after \\: \hline or \cline{col1-col2} \cline{col3-col4} ...
%     Time & $T_{air}$ & $P_{vap}$ & Wind speed & Wind dir. & Nebulo. & Ray. & $P_{atm}$ & Rain \\
     T & TAIR & PVAP & WINDS & WINDD & CLDC & RAY3 & PATM & RAINI \\
%     \hline \hline
%     s & $^\circ$C & hPa & m/s & $^\circ$ & Octa & W/m$^2$ & mbar & mm \\
     s & degC & hPa & m/s & degree & octa & W/m2 & mbar & mm \\
%     \hline \hline
     0 & 20 & 10 & 0.5 & 70 & 5 & 160 & 1012.7 & 0 \\
     3600 & 20 & 10 & 0.5 & 70 & 5 & 160 & 1012.7 & 0 \\
     7200 & 20 & 10 & 0.5 & 70 & 5 & 160 & 1012.7 & 0 \\
     %     \hline
   \end{tabular}
  \caption{Example of meteo data file for THERMIC module in 2D}\label{tab:meteo2d}
\end{table}

\begin{table}[h]
\#REFDATE 2006-10-01 00:00:00\\
  \centering
  \begin{tabular}{rrrrrrrrr}
   T     & WINDS & WINDD & TAIR & PATM   & HREL & CLDC & RAINI     \\
    sec  & m/s   & deg   & degC & mbar   & \%   & octa & mm3/s/mm2 \\
       0 & 6     & 140   & 22.1 & 1012.5 & 84   & 7    & 0.000E+00 \\
   10800 & 5     & 140   & 22   & 1011.8 & 83   & 7    & 0.000E+00 \\
   21600 & 5     & 130   & 21.6 & 1012.2 & 80   & 7    & 0.000E+00 \\
   \end{tabular}
  \caption{Example of meteo data file for heat exchange module in 3D}\label{tab:meteo3d}
\end{table}
