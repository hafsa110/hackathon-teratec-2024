\chapter{Theoretical aspects}

\waqtel offers the use of 7 water quality (WAQ) processes.
These processes generate source terms that are added to the advection-diffusion equation
resolved in \telemac{2D} or \telemac{3D}.
These processes are the following:

\begin{itemize}
\item O$_2$ module: which gives the evolution of oxygen O$_2$ in the flow
  and accounts for the interaction with the organic load and ammoniacal load.
  This module is simple since it does not take into consideration
  all the complexity of biological phenomena linked to the production,
  the elimination and the transport of oxygen.
  For more details about this process, the reader is invited
  to the following manual and references therein (\cite{El-Kadi2012})
  or the \waqtel technical manual,

\item BIOMASS module: it allows the computation of the algal biomass.
  It estimates the vegetal colonization as a function of several parameters
  such as sunshine, water temperature, ratio of renewing of water etc.
  This module introduces and uses 5 tracers:

\begin{enumerate}
\item phytoplanktonic biomass (PHY),

\item dissolved mineral phosphorus (PO$_4$),

\item degradable phosphorus assimilated by phytoplankton (POR),

\item dissolved mineral nitrogen assimilated by phytoplankton (NO$_3$),

\item degradable nitrogen assimilated by phytoplankton (NOR),
\end{enumerate}

\item EUTRO module: this module describes the oxygenation of a river.
  It is much more complex than the O$_2$ module since it takes into account
  vegetal photosynthesis and nutrients and their interactions with phytoplankton.
  This module introduces 8 tracers:
\begin{enumerate}
\item phosphorus assimilated by phytoplankton (POR),

\item dissolved oxygen (O$_2$),

\item phytoplanktonic biomass (PHY),

\item dissolved mineral phosphorus (PO$_4$),

\item dissolved mineral nitrogen assimilated by phytoplankton (NO$_3$),

\item degradable nitrogen assimilated by phytoplankton (NOR),

\item ammoniacal load (NH$_4$),

\item  organic load (L).
\end{enumerate}

These tracers are in mg/l, except biomass which is given in $\mu$g.

\item MICROPOL module: this module gives the evolution of micro-pollutants
  (radio-elements or heavy metals) in the main locations in river flows
  i.e. water, suspended load and bed sediments.
  This module introduces 5 tracers:
\begin{enumerate}
\item suspended sediments (SS),

\item bed sediments (BS), which are considered fix (not advected neither dispersed),

\item micro-pollutant species in dissolved form,

\item part adsorbed by suspended sediments,

\item part adsorbed by bed sediments,
\end{enumerate}

\item THERMIC module: this module computes the evolution of water temperature
  as a function of heat exchange balance with atmosphere.
  Only the exchanges with atmosphere are considered, those with lateral boundaries
  and with the bed are neglected or have to be given in the boundary conditions file,
  
\item the Aquatic Ecodynamics library (AED2):
  this library is fully developed by an Australian consortium,
  see website for more information http://aed.see.uwa.edu.au/research/models/AED/ 

\item degradation law: it enables to model the evolution of one or several tracer(s)
over time from an initial condition according to a degradation law
that is assumed to be 1$^{\rm{st}}$ order (i.e. a tracer decrease).

\end{itemize}
