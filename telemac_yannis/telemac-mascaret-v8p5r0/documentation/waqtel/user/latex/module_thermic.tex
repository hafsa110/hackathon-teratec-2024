\chapter{The THERMIC module}
\label{subs:therm:mod}
For a majority of water quality processes, the interaction with atmosphere is a key parameter.
The THERMIC module is activated by setting \telkey{WATER QUALITY PROCESS} = 11.
The neighboring conditions are taken into account through a meteorological file
like the one described in section \ref{subs:meteo:file}.
It is important to underline that the data contained in this file
can vary depending on the considered case.
The subroutine \telfile{meteo.f} can be edited by the user to customize it to his specific model.

Before version 7.0, heat exchange between water and atmosphere could have been
done with a linearised formula of the balance of heat exchange fluxes at the
free surface in \telemac{3D}. An example of an exchange with a constant atmosphere temperature
and a constant sea salinity was given as standard (as comments) through a
direct programming in the \telfile{BORD3D} subroutine.\\

A much more elaborated model has been introduced in \waqtel for 2D and 3D.

The evolution of temperature of water is tightly linked to heat fluxes through the free surface.
These fluxes (in W/m$^2$) are of 5 natures:

\begin{itemize}
\item solar radiation or sun ray flux $RS$,
\item atmospheric radiation flux $RA$,
\item water radiation or free surface radiation flux $RE$,
\item latent heat or heat flux due to advection $CV$,
\item sensitive heat of conductive origin or heat flux due to evaporation $CE$.
\end{itemize}

The final balance of (surface) source terms is given by:
\[S_{surf}=RS+RA-RE-CV-CE\]
We will give a brief description for each of these terms, for more details see \cite{El-Kadi2012}.
This surface source term is treated explicitly in \telemac{2D},
the following term is added in the explicit source term of advection-diffusion equation
of tracer$\frac{S_{surf}}{\rho C_pH}$.\\

There is a distinction done in 3D compared to 2D:
whereas the long wave radiation (atmospheric radiation $RA$) is absorbed in
the first centimetres of the water column, the short wave radiation (solar
radiation $RS$) penetrates the water column. Evaporation is calculated in 3D.

The choice of the heat exchange model can be done with the keyword
\telkey{ATMOSPHERE-WATER EXCHANGE MODEL} in the \waqtel steering file
(default value = 0: no exchange
model). Value 1 will use with the linearised formula at the free surface,
whereas value 2 will use with the model with complete balance.

These calculations require additional meteo data which may vary in time
in an expected format, rather defined
in the \telkey{ASCII ATMOSPHERIC DATA FILE} of the \telemac{3D} steering file,
see the example "heat\_exchange".

Since release v8.2 the format of the \telkey{ASCII ATMOSPHERIC DATA FILE} is
flexible with respect to the order of columns (but with the mandatory convention
for data names with the shortnames listed in \ref{subs:meteo:file}).

If not filling the necessary variables
(in 3D: wind velocities, air temperature, atmospheric pressure, cloud cover,
rainfall, relative humidity;
 in 2D: wind velocities, air temperature, atmospheric pressure, cloud cover,
rainfall, solar radiation, saturated vapour pressure),
the missing variables are considered constant along the whole computation and
their values are set by associated keywords values
(\telkey{WIND VELOCITY ALONG X}, \telkey{WIND VELOCITY ALONG Y}, or
\telkey{SPEED AND DIRECTION OF WIND} for wind velocity,
\telkey{VALUE OF ATMOSPHERIC PRESSURE} for atmospheric pressure,
\telkey{RAIN OR EVAPORATION IN MM PER DAY} for rainfall,
\telkey{AIR TEMPERATURE}, \telkey{CLOUD COVER}, \telkey{RELATIVE HUMIDITY},
\telkey{SOLAR RADIATION}, \telkey{VAPOROUS PRESSURE}).

%The format may be changed but the user has to change the
%implementation of the reading and the interpolation of the meteorological data.

When using the complete module, evaporation is calculated by \telemac{3D}, but
the user has to provide rainfall data with units homogeneous with length over
time.

The main developments of this module are implemented in the module
\telfile{EXCHANGE\_WITH\_ATMOSPHERE} in 3D and in the \telfile{CALCS2D\_THERMIC} in 2D.


\section{Sun ray flux RS}

Sun ray flux is simply provided in the \telkey{ASCII ATMOSPHERIC DATA FILE} in 2D.
In a majority of cases, when no measurements are available,
this flux is estimated using the method of Perrin \& Brichambaut (\cite{El-Kadi2012}),
which uses the cloud cover of the sky that varies during the day (function of time).
So far, this flux is considered constant in space.

Since release 8.2, solar radiation or sun ray can be either read in the 
\telkey{ASCII ATMOSPHERIC DATA FILE} or computed by \waqtel in 3D.
To read it in the meteo file, the keyword
\telkey{SOLAR RADIATION READ IN METEO FILE} is to be activated (default = NO,
i.e. it is computed by \waqtel in 3D).
Moreover, an additional column is to be written with the shortname RAY3
as headline of the column.

For more real cases, the user is invited to use the ``heat exchange'' module
(in folder sources/telemac3d).
A sun ray flux varying in space, common between \telemac{2D} and \telemac{3D} will be implemented in next releases.

In 3D, examples of solar radiation penetration in the water $RS$ are given in the
\telfile{CALCS3D\_THERMICV} subroutine. Two laws are suggested: the first one
uses the \emph{in situ} measurements of Secchi length and is
recommended if available; the second one uses two exponential laws that may be
difficult to calibrate and require an estimation of the type of water from
turbidity.
An example of the a double exponential law is commented.
A similar law as Atkins formula with Secchi length (see BIOMASS module) can also be
used with light extinction coefficient directly given with the keyword
\telkey{LIGHT EXTINCTION COEFFICIENT} as soon as
\telkey{METHOD OF COMPUTATION OF RAY EXTINCTION COEFFICIENT} is set to 3.

The type of sky related to the luminosity of the site has to be chosen
with respect to the considered area, with the \waqtel keyword
\telkey{LIGHTNESS OF THE SKY} (1: very pure sky, 2: mean pure sky which is default
or 3: industrial zone sky) in 3D.


\section{Atmospheric radiation RA}

The atmospheric radiation $RA$ is estimated with meteorological
data collected at the ground level.
It takes into account energy exchanges with the ground, water
(and energy) exchanges with the underground, etc.\\

In 2D, $RA$ is estimated mainly by the air temperature, like:
\begin{equation*}
RA=e_{air}\sigma\left(T_{air}+273.15 \right)^4\left(1+k\left(\frac{c}{8}\right)^2 \right),
\end{equation*}
where:
\begin{itemize}
\item $e_{air}$ is a calibrating coefficient given by the keyword
  \telkey{COEFFICIENTS FOR CALIBRATING ATMOSPHERIC RADIATION} (default = 0.97),
\item $\sigma$ is the constant of Stefan-Boltzmann (= 5.67.10$^{-8}$ Wm$^{-2}$K$^{-4}$),
\item $T_{air}$ is air temperature given in the \telkey{ASCII ATMOSPHERIC DATA FILE},
\item $c$ = cloudiness (octas), given in the atmospheric data file
(WARNING: in \khione, it is given in tenths),
\item $k$ is the coefficient that represents the nature and elevation of clouds,
it has a mean value of 0.2 and can be changed with the keyword
\telkey{COEFFICIENT OF CLOUDING RATE}.
To simplify calculations, an average value of $k$ = 0.2 is usually taken in 2D
(but default value = 0.17 like in 3D since release 8.2,
old default value was 0.2 until release 8.1).
However, it varies like indicated in Table \ref{tab:kcloud}.
\end{itemize}

\begin{table}
  \centering
  \begin{tabular}{|l|c|}
     \hline
     Type of cloud & $k$ \\
     \hline \hline
     Cirrus & 0.04 \\
     Cirro-stratus & 0.08 \\
     Altocumulus & 0.17 \\
     Altostratus & 0.2 \\
     Cumulus & 0.2 \\
     Stratus & 0.24\\
     \hline
   \end{tabular}
  \caption{Values of $k$ depending on cloud type}\label{tab:kcloud}
\end{table}

In 3D, clouds and albedo at
the free surface determine the atmospheric radiation $RA$ penetrating the water:
\begin{equation*}
RA = (1-alb_{lw}) e_{air}\sigma(T_{air}+273.15)^{4}(1+k . \left( \frac{C}{8} \right)^{2}),
\end{equation*}
where:
\begin{itemize}
\item $alb_{lw}$ = 0.03 is the water albedo for long radiative waves
  (common value used in the literature \cite{imerito_dyresm_2007},
  \cite{henderson-sellers_energy_balance_1986}),
  (1-$alb_{lw}$) is equal to the calibrating coefficient given by the keyword
  \telkey{COEFFICIENTS FOR CALIBRATING ATMOSPHERIC RADIATION} (default = 0.97,
  hence $alb_{lw}$ = 0.03 as default value),
%\item $T_{air}$ ($^{\circ}$C) is the air temperature,
\item $e_{air}$ is the air emissivity (= $0.937.10^{-5}(T_{air}+273.15)^{2}$
if using Swinbank formula, default option),
\item $\sigma= 5.67.10^{-8}~\mathrm{{W.m^{-2}.K^{-4}}}$ is Stefan-Boltzmann's constant,
\item $C$ is the nebulosity (octas). Some meteorological services such as
M\'{e}t\'{e}o France provide this data in octas, it needs to be converted
into tenths, hence the division by 8 in the formula,
(WARNING: in \khione, it is given in tenths),
\item $k$ (dimensionless) is a parameter characterising the type of
cloud. In practise, it is difficult to know the type of cloud during the
period of simulation and a mean value of 0.17 is often used \cite{tva_heat_1972},
\cite{imerito_dyresm_2007}.
This is the new default value for the keyword
\telkey{COEFFICIENT OF CLOUDING RATE} since release 8.2.
Other choices are possible
%hard-coded in 3D but other choices are possible
(see the table \ref{tab:kcloud}).
%and can be changed in the module
%\telfile{EXCHANGE\_WITH\_ATMOSPHERE}.
\end{itemize}

When coupling \waqtel with \telemac{3D}, the \telkey{FORMULA OF ATMOSPHERIC RADIATION}
can be changed:
\begin{itemize}
\item 1: Idso and Jackson (1969),
\item 2: Swinbank (1963) which is the default formula,
\item 3: Brutsaert (1975),
\item 4: Yajima Tono Dam (2014).
\end{itemize}

The formulae in 2D and 3D are almost the same with few differences.


\section{Free surface radiation RE}

The available water is assumed to behave like a grey body.
Radiation generated by this grey body through the free surface is given by:
\begin{equation*}
RE = e_{water}\sigma\left(T_{water}+273.15 \right)^4,
\end{equation*}
where:
\begin{itemize}
\item $T_{water}$ is the mean water temperature in ${}^\circ$C.
$T_{water}$ is given by the keyword \telkey{WATER TEMPERATURE} (default 7${}^\circ$C),
\item $e_{water}$ can be seen as a calibration coefficient which depends on the nature
of the site and obstacles around it.
This coefficient is given with
\telkey{COEFFICIENTS FOR CALIBRATING SURFACE WATER RADIATION} (default 0.97).
For instance, for a narrow river with lots of trees on its banks,
$e_{water}$ is around 0.97, for large rivers or lakes it is about 0.92.
\end{itemize}


\section{Advection heat flux CV}

This flux (also called sensitive heat flux) is estimated empirically:
\begin{equation*}
CV=\rho_{air}C_{p_{air}}f(V)\left(T_{water}-T_{air} \right),
\end{equation*}
where:
\begin{itemize}
  \item $\rho_{air}$ is the air density given by
${\rho }_{air}=\ \frac{100\ P_{atm}}{\left(T_{air}+273.15\right)287}$
where $P_{atm}$ is the atmospheric pressure,
introduced in the \telkey{ASCII ATMOSPHERIC DATA FILE} or using the keyword
\telkey{VALUE OF ATMOSPHERIC PRESSURE} (default 100,000~Pa),
this is a keyword of \telemac{2D} and \telemac{3D},
\item $C_{p_{air}}$ is the air specific heat (J/kg${}^\circ$C)
  given by \telkey{AIR SPECIFIC HEAT} (default 1,005),
\item $f(V)$ is a function of the wind velocity $V$:
  \begin{itemize}
  \item in 2D: $f(V) = a+bV$,
  \item in 3D: $f(V) = b(1+V)$ for wind velocity at 2~m high
    or $f(V) = a+bV$ for wind velocity at 10~m high,
    depending on user's choice,
  \end{itemize}
\item $V$ is the wind velocity (m/s),
\item $a$, $b$ are empirical coefficients to be calibrated in 2D and in 3D (only $a$ if one single calibrating coefficient).
Their values are very close to 0.0025, but they can be changed
using \telkey{COEFFICIENTS OF AERATION FORMULA} (default = (0.002, 0.0012)).
\end{itemize}

Because the site of a study may not be equipped with local wind measurements
and these kinds of data are available at a different location, possibly far
from the studied site, a wind function is used.
In 3D, this can be a linear function with
a single coefficient of calibration $b:f(U_{2}) = b(1+U_{2}$) where $U_{2}$ is
the wind velocity at 2~m high.

To get the wind velocity at 2~m high from classical wind data at 10~m high, a
roughness length of ${z}_{0}~=~0.0002$~m has been chosen in the code,
that leads to $U_{2} \approx 0.85 U_{10}$. This value
of 0.85 (or the roughness length) may be changed by the user if needed.\\

In 3D, except for the coefficient to model the penetration of solar radiation in the
water column,
and if there is only one single calibrating coefficient,
the parameter $b$ that appears in the wind function is the
single calibration parameter of this module. Its value is given by the keyword
\telkey{COEFFICIENT TO CALIBRATE THE ATMOSPHERE-WATER EXCHANGE MODEL}
in the \waqtel steering file (default
value = 0.0025 but recommended values are between 0.0017 and 0.0035). This
keyword is both used for the linearised formula at the free surface and the
model with complete balance (values 1 and 2 for the keyword
\telkey{ATMOSPHERE-WATER EXCHANGE MODEL} in the \waqtel steering file).

Since release 8.2, the user can define a wind function depending on 2 coefficients
by filling in the \telkey{COEFFICIENTS OF AERATION FORMULA}
(not letting the default values).
These 2 coefficients are then taken into account rather the 1 single coefficient
if using \telkey{ATMOSPHERE-WATER EXCHANGE MODEL} = 2.
Contrary to the wind function with one single coefficient to calibrate where
wind velocity is taken at 2~m high, the wind function with 2 coefficients
uses wind velocity taken at 10~m high.


\section{Evaporation heat flux CE}

The evaporative heat flux $CE$ (also called latent heat flux)
is given by the following empirical formula:
\begin{equation*}
CE = L(T_{water})\rho_{air}f(V) \left(H^{sat}-H \right)
\end{equation*}
where:
\begin{itemize}
\item $L(T_{water})$ = 2,500,900 - 2,365.$T_{water}$ is the vaporization latent heat (J/Kg),
\item $f(V)$ is a function of the wind velocity $V$:
  \begin{itemize}
  \item in 2D: $f(V) = a+bV$,
  \item in 3D: $f(V) = b(1+V)$ for wind velocity at 2~m high
    or $f(V) = a+bV$ for wind velocity at 10~m high,
    depending on user's choice,
  \end{itemize}
\item $V$ is the wind velocity (m/s),
\item $H^{sat}=\frac{0.622P^{sat}_{vap}}{P_{atm}-0.378P^{sat}_{vap}}$
is the air specific moisture (humidity) at saturation (kg/kg),
\item $H = \frac{0.622P_{vap}}{P_{atm}-0.378P_{vap}}$ is the air specific humidity (kg/kg),
\item $P_{vap}$ is the partial pressure of water vapour in the air (hPa)
which is given in the \telkey{ASCII ATMOSPHERIC DATA FILE},
\item $P^{sat}_{vap}$ is the partial pressure of water vapour at saturation (hPa) which is estimated with:
\begin{equation*}
P^{sat}_{vap} = 6.11 \exp \left(\frac{17.27T_{water}}{T_{water}+237.3} \right).
\end{equation*}
\end{itemize}

When $H{}^{sat} < H$, the atmospheric radiation $RA$ is corrected by multiplying it with 1.8.
