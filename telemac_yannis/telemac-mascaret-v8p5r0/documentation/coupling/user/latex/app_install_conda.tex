\chapter{Installation of the conda
  environment}\label{ann:conda_install}

If conda is not yet installed on your machine, please refer to
\url{https://www.anaconda.com/}

For a lighter installation, we suggest the use of miniconda3
[\url{https://docs.conda.io/en/latest/miniconda.html}].
\newline

You need to create a complete environment under python3 (the default
python version for miniconda3), but you should avoid to install
\texttt{mpi4py} from conda itself ({\em cf.} section
\ref{ann:mpi4py}). You have to provide a name for the dedicated
environment: in the following instructions we arbitrarily chose \texttt{cpl\_py3}.
\begin{verbatim}
xxx>source deactivate

xxx>conda update -n base conda

xxx>conda create -n cpl_py3 -c conda-forge \
>python openturns=1.10 matplotlib numpy pandas scipy scikit-learn \
>pathos jsonschema paramiko sphinx sphinx_rtd_theme pytest \
>pytest-runner mock ffmpeg shapely cython

xxx>source activate cpl_py3
\end{verbatim}


\section{Installation of mpi4py}\label{ann:mpi4py}
It is mandatory that the python MPI interface \texttt{mpi4py} is
compiled against the same version of MPI used for the coupled
codes. For this reason we should not install the precompiled version
provided by conda, but we have to compile and install it via
\texttt{pip}.
\newline

Please, check that the commands \texttt{mpicc} and \texttt{mpirun}
found by default in your environment (as determined by the
\texttt{PATH} variable) are the same that you used to compile Open
TELEMAC-MASCARET.

In our case we check that
\begin{verbatim}
xxx>which mpicc
/softs/local/openmpi400_gcc731/bin/mpicc

xxx>mpicc --version
gcc (GCC) 7.3.1 20180130 (Red Hat 7.3.1-2)
\end{verbatim}
and
\begin{verbatim}
xxx>which mpirun
/softs/local/openmpi400_gcc731/bin/mpirun

xxx>mpirun --version
mpirun (Open MPI) 4.0.0
\end{verbatim}

Once the paths are correctly set, you can proceed to the installation of
\texttt{mpi4py}.
Notice that the installation is normally driven by
\begin{verbatim}
xxx>pip install mpi4py
\end{verbatim}
but that a specific feature of OpenMPI 4 requires a maintenance
version, until the official mpi4py distro isn't updated
\begin{verbatim}
xxx>pip install --user https://bitbucket.org/mpi4py/mpi4py/get/maint.zip
\end{verbatim}

