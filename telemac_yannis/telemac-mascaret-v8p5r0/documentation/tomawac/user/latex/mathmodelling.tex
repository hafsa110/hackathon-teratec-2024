
\chapter{Mathematical modelling procedures used by \tomawac}
\label{chapter4}
\section{ Scope of sea state modelling}
The directional spectrum of wave action density, as defined in paragraph
\ref{se:seastate}, is considered as a function of five variables:
\[N(\vec{x},\vec{k},t)=N(x,y,k_{x} ,k_{y} ,t)\]
using,
as discretization variables:

 \begin{enumerate}
 \item the position vector $\vec{x} = (x, y)$ for spatial location in a
   Cartesian coordinate system

 \item the wave number vector $\vec{k} = (k_x, k_y) = (k.sin \theta, k.cos
   \theta)$ for directional spectrum discretization, $\theta$ denoting the
   wave propagation direction (direction in which the waves travel).

 \item the time t.
\end{enumerate}

 Under the hypotheses made on the wave representation (see in paragraph
 \ref{se:seastate}) as well as on the model application domain and the
 modelled physical processes (see in paragraph \ref{se:physicalprocesse}),
 an equation of evolution of the directional spectrum of wave action can be
 written in the following form (see in \cite{Willebrand1975}
 \cite{Phillips1977} \cite{Bretherton1969} for a detailed demonstration
 of the way that equation is arranged):
\begin{equation} \label{GrindEQ__4_1_}
  \frac{\partial N}{\partial t} +\frac{\partial (\dot{x}N)}{\partial x} +
  \frac{\partial (\dot{y}N)}{\partial y} +\frac{\partial (\dot{k}_{x} N)}
       {\partial k_{x} } +\frac{\partial (\dot{k}_{y} N)}{\partial k_{y} } =
       Q(k_{x} ,k_{y} ,x,y,t)
\end{equation}
The equation expresses that, in the general case of waves propagating in a
non-homogeneous, unsteady environment (currents and/or sea levels varying in
time and space), the wave action is preserved to within the source and sink
terms (designated by the term Q).

 The following notation is also used in \eqref{GrindEQ__4_1_}:
 \[\dot{g}=\frac{dg}{dt} =\frac{\partial g}{\partial t} +\frac{\partial x}
 {\partial t} \frac{\partial g}{\partial x} +\frac{\partial y} {\partial t}
 \frac{\partial g}{\partial y} \]
 In that form (conservative writing in the form of a flux), equation
 \eqref{GrindEQ__4_1_} can be transposed to other coordinate systems and, for
 instance, $(k,\theta), (f_a,\theta)$ or else $(f_r,\theta)$ can be used for
 the discretization of directional spectrum \cite{Komen1994} \cite{Tolman1991}

Working in $(x, y, k_x, k_y)$, however, makes it possible to remain in the
canonical coordinate system and to write, for the propagation equations (also
named Hamilton's equations):

\bequ
\label{eq:hamilton}
\left\{
\barr{lll}
\dsp \dot{x}=\frac{\partial \Omega }{\partial k_{x} } &\mbox{ and }&\dsp \dot{y}
=\frac{\partial \Omega }{\partial k_{y} } \\[12pt]
\dsp \dot{k}_{x} =-\frac{\partial \Omega }{\partial x}& \mbox{ and }&\dsp
\dot{k}_{y} =-\frac{\partial \Omega }{\partial y}
\earr
\right.
\eequ  

wherein W results from the Doppler relation applied to the wave dispersion
relation for the general case with current:
\begin{equation} \label{GrindEQ__4_3_}
\Omega (\vec{k},\vec{x},t)=\omega =\sigma +\vec{k}.\vec{U}
\end{equation}
wherein:  w is the absolute angular frequency observed in a fixed coordinate
system.fa = w/(2p) is named absolute frequency.

 $\vec{U}$ denotes the current velocity (depth-integrated).

$\sigma$ denotes the intrinsic or relative angular frequency, which is
observed in a coordinate system moving at the velocity $\vec{U}$. It is given
by the dispersion relation in the zero-current case:
\bequ
\label{eq:reldispersion}    
\sigma^2 = g.k.tanh(k.d) 
\eequ
$ f_r = \sigma/(2\pi)$ is named intrinsic or relative wave frequency. $d$
denotes the water height.

Through the Hamilton's equations (\ref{eq:hamilton}), it can be demonstrated
that we have:

\bequ 
\label{eq:equ4.5.a}
\frac{\partial \dot{x}}{\partial x} +\frac{\partial \dot{y}}{\partial y} +
\frac{\partial \dot{k}_{x} }{\partial k_{x} } +\frac{\partial \dot{k}_{y} }
     {\partial k_{y} } =0
\eequ

or  $div(\vec{V}) = 0$   when defining: $\vec{V}=(\dot{x},\dot{y},\dot{k}_{x} ,
\dot{k}_{y} )$

The evolution equation \eqref{GrindEQ__4_1_} can then alternatively be written
in the following form (the so-called transport form):
\bequ
\label{eq:transform}
\barr{l}
\dsp
\frac{\partial N}{\partial t} +\dot{x}\frac{\partial N}{\partial x} +\dot{y}
\frac{\partial N}{\partial y} +\dot{k}_x \frac{\partial N}{\partial k_x }
+\dot{k}_y \frac{\partial N}{\partial k_y } =Q(k_x ,k_y ,x,y,t)\\[6pt]
\dsp
\frac{\partial N}{\partial t} +\vec{V}.grad_{\vec{x},\vec{k}} (N)=Q
\earr 
\eequ

The transfer rates are given by the linear wave theory \cite{Chaloin1989}
\cite{Komen1994} \cite{Mei1983} \cite{Tolman1991}
\bequ
\label{eq:transfer}
\barr{l}
\dsp
\dot{x}=C_g \frac{k_x }{k} +U_x \\[6pt]
\dsp
\dot{y}=C_g \frac{k_y }{k} +U_y \\[6pt]
\dsp
\dot{k}_x =-\frac{\partial \sigma }{\partial d} \frac{\partial d}{\partial x}
-\vec{k}.\frac{\partial \vec{U}}{\partial x} \\[6pt]
\dsp
\dot{k}_y =-\frac{\partial \sigma }{\partial d} \frac{\partial d}{\partial y}
-\vec{k}.\frac{\partial \vec{U}}{\partial y}
\earr 
\eequ
$C_g$ is the relative (or intrinsic) group velocity of waves, i.e. as is
observed in a coordinate system moving at the velocity of the current:

\bequ
\label{eq:cg4_8}
C_g =\frac{\partial \sigma }{\partial k} =n\frac{\sigma }{k} \mbox{ with }
n=\frac{1}{2} \left(1+\frac{2kd}{\sinh (2kd)} \right)
\eequ

The relative (or intrinsic) phase velocity C of waves is also introduced:
$C=\frac{\sigma }{k} $

The sea state spectral modelling will then consist of solving the evolution
equations \eqref{GrindEQ__4_1_} or (\ref{eq:transform}), using the kinematic
equations (\ref{eq:transfer}).

The transport equation formulation (\ref{eq:transform}) has been adopted in
\tomawac, since it is closely related to other equations applied in hydraulics,
which have already been treated at the LNHE and for which methods and a
know-how have been developed long ago.

As regards the discretization variables being used in \tomawac, we have
already mentioned in paragraph \ref{se:selecting} that:

 \begin{itemize}
 \item spatial discretization can be based either on a Cartesian coordinate
   system in (x, y) or on a spherical coordinate system at the Earth's surface
   in $(\lambda, \phi) =$ (longitude, latitude).
 \item Discretization of angular spectrum uses the pair $(f_r, \theta) =$
   (relative frequency ; propagation direction).
\end{itemize}

 The following conventions are adopted for writing the equations:

 \begin{itemize}
 \item the x-axis (in the Cartesian coordinate system) or the l-axis of
   longitudes (in the spherical coordinate system) is assumed to be horizontal,
   directed to the right, whereas the y-axis (in the Cartesian coordinate
   system) or the j-axis of latitudes (in the spherical coordinate system) is
   assumed to be vertical, upwardly directed. Then, in spherical coordinates,
   the vertical axis points at the north, whereas the horizontal axis points
   to the East.

 \item In either case, the wave propagation directions q are defined with
   respect to the vertical axis in the clockwise direction.
\end{itemize}

 These conventions are illustrated below in Figure \ref{fig:defloc} Those
 equations that correspond to the two spatial discretizations options are
 developed in the next paragraphs.

\begin{figure}[H]%
\begin{center}
\includegraphics*[width=3.25in,  keepaspectratio=true]{graphics/defloc}
\caption{definition of location conventions as used in \tomawac}
\label{fig:defloc}
\end{center}
\end{figure}

\section{Equations solved}
\subsection{Equations solved in a Cartesian spatial coordinate system}
By switching the variable from (x, y, $k_x$, $k_y$) to (x, y, $f_r$, q), it can
be shown that the following relation exists for the directional spectrum of wave
action as expressed in both coordinate systems:
\begin{equation} \label{GrindEQ__4_9_}
  N(x,y,k_x ,k_y ,t)=\frac{CC_g }{2\pi \sigma } \tilde{N}(x,y,f_r,\theta,t)
  =\tilde{B}.\tilde{F}(x,y,f_r,\theta,t)
\end{equation}

\bequ
\label{eq:GrindEQ__4_10}
\mbox{putting: } \tilde{B}=\frac{C\; C_g }{2\pi \sigma ^{2} }
=\frac{C_g }{\left(2\pi \right)^{2} k\; f_r }
\eequ
The evolution equation (\ref{eq:transform}) is then written as:
\begin{equation} \label{GrindEQ__4_11_}
  \frac{\partial (\tilde{B}\tilde{F})}{\partial t} +\dot{x}\frac{\partial
    (\tilde{B}\tilde{F})}{\partial x} +\dot{y}\frac{\partial (\tilde{B}
    \tilde{F})}{\partial y} +\dot{\theta }\frac{\partial (\tilde{B}
    \tilde{F})}{\partial \theta } +\dot{f}_{r} \frac{\partial (\tilde{B}
    \tilde{F})}{\partial f_{r} } =\tilde{B}.\tilde{Q}(x,y,\theta ,f_{r} ,t)
\end{equation}
with the following transfer rates, as computed from the linear wave theory:
\bequ
\label{eq:equ4_12}
\barr{l}
\dsp \dot{x}=C_g .\sin \theta +U_x \\[6pt]
\dsp \dot{y}=C_g .\cos \theta +U_y\\[6pt] 
\dsp \dot{\theta }=-\frac{1}{k} \frac{\partial \sigma }{\partial d}
\tilde{G}_n (d)-\frac{\vec{k}}{k} .\tilde{G}_n (\vec{U})\\[6pt]
\dsp \dot{f}_{r} =\frac{1}{2\pi } \left[\frac{\partial \sigma }{\partial d}
  {\kern 1pt} \left(\frac{\partial d}{\partial t} +\vec{U}.\vec{\nabla }d\right)
  -C_g\; \, \vec{k}.\tilde{G}_t (\vec{U})\right]
\earr
\eequ
The operators $\tilde{G}_n $ and $\tilde{G}_t $ refer to the computation of a
function gradient in directions that are respectively normal and tangential to
the characteristic curve with the direction q:
\bequ
\label{eq:equ4_13}
\barr{l}
\dsp \tilde{G}_{n} (g)=\vec{n}.\vec{\nabla }g=\cos \theta \, \frac{\partial g}
     {\partial x} \; -\sin \theta \, \frac{\partial g}{\partial y} \\[6pt]
\dsp \tilde{G}_{t} (g)=\vec{t}.\vec{\nabla }g=\sin \theta \, \frac{\partial g}
          {\partial x} \; +\cos \theta \, \frac{\partial g}{\partial y}
\earr
\eequ

Besides, using the dispersion relation \ref{eq:reldispersion}, it can be
demonstrated that:
\begin{equation} \label{GrindEQ__4_14_}
\frac{\partial \sigma }{\partial d} =\frac{\sigma k}{\sinh (2kd)}
\end{equation}
The spatial transfer rates $\dot{x}$ and $\dot{y}$ (equations \ref{eq:equ4_12})
model the spatial wave propagation and the shoaling. The directional transfer
rate $\dot{\theta }$ models the refraction-induced change of wave propagation
direction. Refraction is generated by the spatial variations of those
properties of the environment in which the waves propagate and can result
either from a bathymetric variation (first term) or from current gradients
(second term). The relative frequency transfer rate $\dot{f}_r $ models the
relative frequency changes resulting from sea level variations both in space
and time and/or from current variations in space.

It is noteworthy that this last term is zero in the case of zero-current and
of no variation of sea level in time: the advection equation is then reduced
to a three-dimensional equation.

Lastly, as regards the source terms, it should be mentioned that changing the
coordinate system and using the factor $\tilde{B}$ allows to switch from the
term Q to a term $\tilde{Q}$ that is directly expressed in terms of the
directional variance spectrum with a variance $\tilde{F}(f_r, \theta)$. The
content of that term is explained in paragraph \ref{se:sourceterm}

\subsubsection{ Consideration of diffraction}

If diffraction is taken into account, a new wave number \textit{K} is
computed, whose modulus is the sum of the wave number \textit{k} calculated
by the linear wave theory plus a term that takes into account the effects of
the wave diffraction. This modified wave number can be used to represent the
effect of diffraction in a phase-averaged spectral model, as proposed by
Holthujisen (2003).

 Two different mathematical formulations of diffraction are considered:

\begin{itemize}
\item  the equation of Berkhoff~(1972), also called Mild-Slope Equation (MSE),
  which gives the velocity potential of a monochromatic and unidirectional
  wave propagating over a slowly-varying bathymetry and describes the combined
  effects of diffraction and refraction;

\item  a modified form of the MSE, the Revised Mild Slope Equation (RMSE)
  proposed by Porter~(2003).
\end{itemize}

In the presence of diffraction the effective wave number \textit{K} is defined
as:
\begin{equation} \label{eq:GrindEQ__4_15_}
\left|{\rm K} \right|^{2} =k^{2} (1+\delta )
\end{equation}
where $\delta$ is a diffraction parameter defined as

when referring to the \textbf{Mild-Slope Equation} \cite{Berkhoff1972}
\bequ
\label{eq:GrindEQ__4_16_}
 \delta _{MSE} =\frac{\nabla .\left(CC_{g} \nabla a\right)}{k^{2} CC_{g} a} 
\eequ       
     
when referring to the \textbf{Revised Mild-Slope Equation} \cite{Porter2003}
(with $A=ak\sqrt{CC_g}$)
\bequ
\label{eq:GrindEQ__4_17_}
\delta _{RMSE} =\frac{\nabla .\left(k^{-2} \nabla A\right)}{A} 
\eequ

In the presence of diffraction, the energy propagation speed in geographic
space $C_{gd}$ is \cite{Holthuijsen2003}:
\begin{equation} \label{GrindEQ__4_18_}
C_{gd} =\frac{K}{k} C_g
\end{equation}
Under the hypothesis of zero current and of no varying water levels (i.e.
$\dot{f}_r $=0) and inserting (\ref{eq:GrindEQ__4_15_}) in (\ref{eq:equ4_12}),
the transfer rates in presence of diffraction are:
\begin{equation} \label{GrindEQ__4_19_}
\dot{x}=\sqrt{(1+\delta )} C_g .\sin \theta
\end{equation}
\begin{equation} \label{GrindEQ__4_20_}
\dot{y}=\sqrt{(1+\delta )} C_g .\cos \theta
\end{equation}
\begin{equation} \label{GrindEQ__4_21_}
  \dot{\theta }=-\frac{(1+\delta )^{1/2} }{k} \frac{\partial \sigma }{\partial d}
  \tilde{G}_n (d)-\frac{C_g }{2(1+\delta )^{1/2} } \tilde{G}_n (\delta )
\end{equation}
The above terms (\ref{GrindEQ__4_19_}) to (\ref{GrindEQ__4_21_}) have been
implemented in \tomawac, in both the MSE and RMSE formulations, to represent
the effect due to diffraction.


\subsection{ Equations solved in a spherical spatial coordinate system}

By switching the variables from $(x, y, k_x, k_y)$ to $(\lambda, \phi, f_r,
\theta)$, it can be shown that the following relation exists for the
directional spectrum of wave action as expressed in both coordinate systems:
\begin{equation} \label{GrindEQ__4_22_}
  N(x,y,k_x ,k_y ,t)=\frac{CC_g }{2\pi \sigma R^{2} \cos \phi } \hat{N}(\lambda ,
  \phi ,f_{r} ,\theta ,t)=\hat{B}.\hat{F}(\lambda ,\phi ,f_r ,\theta ,t)
\end{equation}

\bequ
\label{eq:GrindEQ__4_23}
\mbox{putting: }\hat{B}=\frac{C\; C_{g} }{2\pi \sigma ^{2} R^{2} \cos \phi }
=\frac{C_g }{\left(2\pi \right)^2 k\; f_r R^{2} \cos \phi }  
\eequ

R denotes the Earth's radius (R = 6400 km) and, once more, $\lambda$ and
$\phi$ are respectively the longitude and the latitude of the point being
considered.

 The evolution equation (\ref{eq:transform}) is then written as:
\begin{equation} \label{GrindEQ__4_24_}
  \frac{\partial (\hat{B}\hat{F})}{\partial t} +\dot{\lambda }\frac{\partial
    (\hat{B}\hat{F})}{\partial \lambda } +\dot{\phi }\frac{\partial
    (\hat{B}\hat{F})}{\partial \phi } +\dot{\theta }\frac{\partial
    (\hat{B}\hat{F})}{\partial \theta } +\dot{f}_r \frac{\partial
    (\hat{B}\hat{F})}{\partial f_r} =\hat{B}.\hat{Q}(\lambda,\phi,\theta,f_r,t)
\end{equation}
with the following transfer rates:
\bequ
\label{eq:equ4_25}
\barr{l}
\dsp \dot{\lambda }=\frac{1}{R\cos \phi } \left(C_g .\sin \theta +U_{\lambda }
\right) \\[6pt]
\dsp \dot{\phi}=\frac{1}{R} \left(C_g .\cos \theta +U_{\phi } \right)\\[6pt]
\dsp \dot{\theta }=\frac{1}{R} \left[C_{g} \; \sin \theta \; \tan \phi \;
  -\frac{1}{k} \; \frac{\partial \sigma }{\partial d} \hat{G}_n (d)
  -\frac{\vec{k}}{k} .\hat{G}_n (\vec{U})\right]\\[6pt]
\dsp \dot{f}_r =\frac{1}{2\pi R} \left[\frac{\partial \sigma }{\partial d}
  \left(\frac{\partial d}{\partial t} +\frac{U_{\lambda } }{\cos \phi }
  \frac{\partial d}{\partial \lambda } +U_{\phi } \frac{\partial d}
       {\partial \phi } \right)-C_g\quad \vec{k}.\hat{G}_{t} (\vec{U})\right]
\earr
\eequ

As in the previous case, the operators $\hat{G}_n $ and $\hat{G}_t $ refer to
the computation of a function gradient in directions that are respectively
normal and tangential to the characteristic curve with the direction q:
\bequ
\label{eq:equ4_26}
\barr{l}
\dsp \hat{G}_n (g)=\frac{\cos \theta }{\cos \phi } \; \frac{\partial g}
     {\partial \lambda } \; -\sin \theta \; \frac{\partial g}{\partial \phi }
     \\[6pt]
\dsp \hat{G}_t (g)=\frac{\sin \theta }{\cos \phi } \; \frac{\partial g}
     {\partial \lambda } \; +\cos \theta \; \frac{\partial g}{\partial \phi } 
\earr
\eequ

As previously, the spatial transfer rates $\dot{\lambda }$ and $\dot{\phi }$
(equations \ref{eq:equ4_25}) model the wave propagation in space and the
shoaling. In that coordinate system, the directional transfer rate
$\dot{\theta }$ has an additional term (the first term) compared to the case
in Cartesian coordinates. That term results from the propagation in spherical
coordinates, in such a way that waves are located with respect to the North
change during the propagation over a large circle arc at the Earth's surface
\cite{Wamdi1988} \cite{Komen1994}. Both second and third terms $\dot{\theta }$
model the refraction caused respectively by bathymetry and currents. The
relative frequency transfer rate $\dot{f}_{r} $  models the changes of relative
frequency resulting from variations of the sea level or of the current in both
space and time. It is noteworthy that this last term is zero in the case of
zero current and of no variation of the sea level in time: the advection
equation is then reduced to a three-dimensional equation.


\section{ \tomawac source and sink terms}
\label{se:sourceterm}

\subsection{ Generals}

The source and sink terms that compose $\tilde{Q}$ and $\hat{Q}$ in the
right-hand members of evolution equations \eqref{GrindEQ__4_11_} and
\eqref{GrindEQ__4_24_} of directional spectrum of wave action gather the
contributions from the physical processes listed in paragraph 3.3. for
the application domain of \tomawac:
\bequ
\label{eq:semimp}
Q = Q_{in} + Q_{ds} + Q_{nl} + Q_{bf} + Q_{br} + Q_{tr} + Q_{ds,cur} + Q_{veg} +
Q_{porous}
\eequ
 wherein 
\begin{itemize}
\item $Q_{in}$: wind-driven wave generation
\item $Q_{ds}$: whitecapping-induced energy dissipation
\item $Q_{nl}$: non-linear quadruplet interactions
\item $Q_{bf}$: bottom friction-induced energy dissipation
\item $Q_{br}$: bathymetric breaking-induced energy dissipation
\item $Q_{tr}$: non-linear triad interactions
\item $Q_{ds,cur}$: enhanced breaking dissipation of waves on a current
\item $Q_{veg}$: dissipation due to vegetation.
\item $Q_{porous}$: dissipation due to porous media.
\end{itemize}

These source and sink terms are numerically modelled and parameterized as
detailed in the next paragraphs. For most of these processes, several models
or formulations are proposed and available in \tomawac.


\subsection{ Wind input (term $Q_{in}$)}
\label{WIND_INPUT}

Three wind generation models are available in \tomawac. The model to be
activated is selected through the keyword \textit{WIND GENERATION} in the
steering file, which can take four values, namely:

\begin{itemize}
\item  0 no wind input \textit{(default value)}
\item  1 Janssen's model \cite{Janssen1989} \cite{Janssen1991} (WAM cycle 4)
  % (see in paragraph \ref{parag4.3.2.1}).
\item  2 Snyder \textit{et al. }model \cite{Snyder1981}
  %(see in paragraph \ref{parag4.3.2.2}).
\item  3 Yan's model \cite{Yan1987}
  % (see in paragraph \ref{parag4.3.2.3})
\end{itemize}

Beside those exponential growth-type wind generation models, a linear growth
model is also available in \tomawac, which has been proposed by Cavaleri \&
Malanotte-Rizzoli \cite{Cavaleri1981} (see last paragraph in this section
% \ref{parag4.3.2.4}.
The model can be activated through the keyword \textit{LINEAR WAVE GROWTH},
and can be used together with one of the three above mentioned models. Its
main feature is that it permits to start a wave simulation from a nil wave
spectrum (whereas the three above mentioned models need some initial energy
level for the wave spectrum to grow under wind action).


  \subsubsection{Option 1 for wind input: Janssen's model}
\label{parag4.3.2.1}
With that option, the model implemented for the wind input term is based upon
the Janssen's works \cite{Janssen1989} \cite{Janssen1991}; Janssen proposed a
quasi-linear theory for modelling the ocean/atmosphere interactions. The linear
growth term is ignored and only an exponential energy growth is taken into
account, following Miles' results \cite{Miles1957}.

A quasi-linear source term is obtained as a function of the directional
variance spectrum:
\begin{equation} \label{GrindEQ__4_27_}
  Q_{in} =\sigma .\varepsilon .\beta .\left(\left[\frac{u_{*} }{C} +z_{\alpha }
    \right]\max \left[\cos (\theta -\theta _{w} );0\right]\right)^{2} F
\end{equation}
with the following notations:

$ \epsilon = \rho_{air}/\rho_{water }$is the ratio of air and water specific
gravities ($ \epsilon = 1.25\hspace{0.1cm}10^{-3}$).

 $C = \sigma/k$ is the wave phase velocity

 $\theta_w$ is the local wind direction (direction in which it blows)

$u_*$ is the friction velocity, being linked to the surface stress $\tau_s$ by
the
following relation:
\begin{equation} \label{GrindEQ__4_28_}
u_{*} =\sqrt{\frac{\tau _{s} }{\rho _{air} } }
\end{equation}

 $z_\alpha$ is a constant allowing to offset the growth curve.

The operator 'max' in the source term expression limits the wave generation
for the propagation directions included within a $\pm 90^\circ$ angular sector
with respect to the local wind direction qw. For the wave directions making an
angle in excess of 90${}^\circ$ with respect to the wind direction $\theta_w$,
the wind input term is void.

In the Janssen's model \cite{Janssen1991}, the Miles' parameter $\beta$ is a
function of two non-dimensional variables:

\begin{itemize}
\item  the wave age:  $\dsp A=\frac{u_{*} }{C} $
\item  the wind profile parameter: $\dsp \Omega =\frac{g.z_{0} }{u_{*}^{2} } $
\end{itemize}
\bequ
\label{GrindEQ__4_29_}
\mbox{ It is written as }\beta =\frac{\beta _{m} }{\kappa _{}^{2} } \mu \ln ^{4}
\mu 
\eequ
 where $\kappa$ is the Von Karman's constant

 $\beta_m$ denotes a coefficient set to 1.2 by Janssen \cite{Janssen1991}

 $z_0$ denotes the roughness length

 $\mu$ denotes the non-dimensional critical height:
\begin{equation} 
\label{GrindEQ__4_30_}
\barr{ll}
\mu &\dsp =\min \left[\frac{g.z_{0} }{C^{2} } \exp \left(\frac{\kappa }{
    \left[\frac{u_{*} }{C} +z_{\alpha } \right]\cos (\theta -\theta _{w} )}
  \right);1\right] \\[12pt]
&\dsp =\min \left[\Omega .A^{2} \exp \left(\frac{\kappa }{\left[A+z_{\alpha }
      \right]\cos (\theta -\theta _{w} )} \right);1\right]
\earr
\end{equation}
The Janssen's model \cite {Janssen1989} \cite{Janssen1991} is characterized by
the method it uses for computing u* and z0. The surface stress $\tau_s$ is a
function depending, on the one hand, on the wind velocity $U_{10}$ at 10 m and,
on the other hand, on the sea state roughness through the wave stress $\tau_w$.
It is obtained by solving the following system of equations:

\bequ
\label{eq:defu10}
U_{10} =\frac{u_* }{\kappa } \ln \left(\frac{10+z_0 +\tilde{z}_0 }{z_0 }
\right)\approx \frac{u_* }{\kappa } \ln \left(\frac{10}{z_0 } \right)
\eequ

\bequ
\label{eq:defz0}
z_0 =\frac{\tilde{z}_0 }{\sqrt{1-\tau _w /\tau _s } } \mbox{ avec }
\tilde{z}_0 =\alpha \frac{u_*^{2} }{g}\mbox{  et  }u_*
=\sqrt{\frac{\tau _{s} }{\rho _{air} } }
\eequ

The solution of the system of equations through a Newton-Raphson's iterative
method yields the surface stress $\tau_s$, the friction velocity $u_*$ and the
roughness length $z_0$.

The initial value of friction velocity $u_*$ being used in the iterative
algorithm is obtained considering a constant drag coefficient:

$u_*^ =\sqrt{C_{D} } U_{10} $ where: $C_D = 1.2875\hspace{.1cm} 10^{-3}$ by
default.

The wave stress $\tau_w$ itself is computed from the variance spectrum F
(via the source term Qin) using the following relation:
\begin{equation} \label{GrindEQ__4_32_}
  \tau _{w} =\left|\iint \;  \rho _{water} \; \sigma \; Q_{in} (f_{r} ,\theta )\;
  \left(\cos \theta ,\sin \theta \right)df_{r} \; d\theta \right|
\end{equation}
That integral is numerically computed over the discretized portion of the
spectrum and a parametrization, based upon a decrement of variance in $f^{-n}$,
is used for the high frequencies portion of the spectrum.

In fact, that source term has eight parameters, namely:

\begin{itemize}
\item  coefficient $\beta_m$ (corresponding to the keyword \textit{WIND
  GENERATION COEFFICIENT }in the steering file). Its default value is taken
  as 1.2, in accordance with the Janssen's proposal \cite{Janssen1991} and
  the value adopted in the model WAM-Cycle 4.
\item  air specific gravity $\rho_{air}$ (corresponding to the keyword
  \textit{AIR DENSITY} in the steering file. Its default value is taken as
  1.225 kg/m${}^{3}$.
\item  water specific gravity ${\rho}_{water}$ (corresponding to the keyword
  \textit{WATER DENSITY} in the steering file). Its default value is taken as
  $1,000 kg/m^{3}$.
\item  constant $\alpha$ (corresponding to the keyword \textit{CHARNOCK
  CONSTANT} in the steering file). Its default value is taken as 0.01, in
  accordance with the Janssen's proposal and the standard value adopted in the
  model WAM-Cycle 4.
\item  constant $\kappa$ (corresponding to the keyword \textit{VON KARMAN
  CONSTANT} in the steering file). Its default value is taken as 0.41, i.e.
  the typical value.
\item  initial drag coefficient $C_D$ (corresponding to the keyword
  \textit{WIND DRAG COEFFICIENT} in the steering file). This drag coefficient
  is provided for initializing the iterative computation of friction velocity
  $u_*$. Its default value is taken as $1.2875\hspace{.1cm} 10^{-3}$.
\item  offset constant $z_\alpha$ (corresponding to the keyword \textit{SHIFT
  GROWING CURVE DUE TO WIND} in the steering file). Its default value is taken
  as 0.011, in accordance with the value adopted in the model WAM-Cycle 4.
\item  elevation at which the wind is recorded (corresponding to the keyword
  \textit{WIND MEASUREMENTS LEVEL} in the steering file). Its default value
  is taken as 10 m: its corresponds to the typical value and to the value
  being adopted in the above explanations.
\end{itemize}


\subsubsection{Option 2 for wind input: Snyder et al. model}
\label{parag4.3.2.2}

In that option, the model implemented for the wind input term is based upon
the works conducted by Snyder \textit{et al.} \cite{Snyder1981}, as amended by
Komen \textit{et al.} \cite{Komen1984} to take into account the friction
velocity u* instead of the wind velocity at 5 m. It corresponds to the
formulation being used in the cycle 3 release of WAM model. The formulation
is simpler than the Janssen's theory which Option 1 is based upon (see in
preceding paragraph):

As in Option 1, the linear growth term is ignored and only an exponential
energy growth is taken into account, following the Miles' results
\cite{Miles1957}:
\bequ
\label{eq:GrindEQ__4_33_}
Q_{in} =\beta F \mbox{    where:  }\beta =\max \left[0\; ;\; 0.25
  \frac{\rho _{air} }{\rho _{water} } \left(28\frac{u_{*} }{C} \cos
  (\theta -\theta _{w} )-1\right)\right]\sigma 
\eequ

The shear velocity value $u_*$ used is obtained considering a drag coefficient
linearly depending on the wind velocity:
\bequ
\barr{lll}
u_{*}^{} =\sqrt{C_{D} } U_{10} \mbox{ where }: & C_D = 6.5
\hspace{0.1cm} 10^{-5}  + 8\hspace{0.1cm} 10^{-4} U_{10}&
\mbox{ if }U_{10} > 7.5 m/s.\\[6pt]
& C_D = 1.2875\hspace{0.1cm}10^{-3} & \mbox{ if }U_{10} < 7.5 m/s.
\earr
\eequ
That source term only uses two parameters, namely:

\begin{itemize}
\item  air density $\rho_{air}$ (corresponding to the keyword \textit{AIR
  DENSITY} in the steering file. Its default value is taken as 1.225 $kg/m^3$.
\item  water density $\rho_{water}$ (corresponding to the keyword \textit{WATER
  DENSITY} in the steering file). Its default value is taken as 1,000 $kg/m^3$.
\end{itemize}


\subsubsection{Option 3 for wind input: Yan's model}
\label{parag4.3.2.3}

The Yan's model \cite{Yan1987} consists of a combination of $u_*/C$ and
$(u_*/C)^2$ terms. It is valid over a wide range of frequencies and wind speeds:
\bequ
\label{eq:GrindEQ__4_34_}
\barr{l}
\dsp Q_{in} =\beta F \\[12 pt]
\dsp  \mbox{ with }\beta =\left[D\left(\frac{u_* }{C} \right)^{2} \cos
  (\theta -\theta _w )+E\frac{u_* }{C} \cos (\theta -\theta _w )+F\cos
  (\theta -\theta _w )+H\right]\sigma 
\earr
\eequ

To select this model, the keyword \textit{WIND GENERATION} must be set to 3 in
the steering file.

 This source term makes use of four parameters. The default values of those parameters correspond to the coefficients proposed by Westhuysen \cite{Westhuys2007}.

\begin{itemize}
\item  The coefficient D, corresponding to the keyword \textit{YAN GENERATION
  COEFFICIENT D}, has a default value of $4.0\hspace{0.1cm} 10^{-2}$;
\item  The coefficient E, corresponding to the keyword \textit{YAN GENERATION
  COEFFICIENT E}, has a default value of $5.52\hspace{0.1cm} 10^{-3}$;
\item  The coefficient F, corresponding to the keyword \textit{YAN GENERATION
  COEFFICIENT F}, has a default value of $5.2\hspace{0.1cm} 10^{-5}$;
\item  The coefficient H, corresponding to the keyword \textit{YAN GENERATION
  COEFFICIENT H}, has a default value of $-3.02\hspace{0.1cm} 10^{-4}$.
\end{itemize}

\subsubsection{Linear wave growth: Cavaleri and Malanotte-Rizzoli model}
\label{parag4.3.2.4}

The linear growth mechanism described by Phillips \cite{Phillips1957},
\cite{Phillips1958} is useful to initialise the wave growth. If this term is
neglected, it is necessary to set a non-zero sea-state as initial condition
in order to enable the wave energy spectrum to grow.

The term that has been implemented in \tomawac is the linear wave growth term
of Cavaleri \& Malanotte-Rizzoli \cite{Cavaleri1981}, as formulated by Tolman
\cite{Tolman1992}:
\begin{equation} \label{GrindEQ__4_35_}
  Q_{in} (f,\theta )=\alpha (f,\theta )=1,5.10^{-3} g^{-2} \left[u_{*}
    \max (0,\cos \left(\theta -\theta _{w} \right)\right]^{4} \exp
  \left[-\left(\frac{f}{f_{PM} } \right)^{-4} \right]
\end{equation}
where $u_*$ is the friction wind velocity, ${\theta}_{w}$ the wind direction
and ${f}_{PM}$ is a peak frequency called Pierson-Moskowitz frequency
\cite{Pierson1964}, defined as:
\begin{equation} \label{GrindEQ__4_36_}
f_{PM} =\frac{1}{2\pi } \frac{g}{28.u_{*} }
\end{equation}
To select this model, the keyword \textit{LINEAR WAVE GROWTH} must be set to
1 in the steering file. This model does not require any input parameter.

\subsection{ Whitecapping-induced dissipations (term $Q_{ds}$)}
\label{WHITECAPPING}
Two models are available in \tomawac. The whitecapping or the free surface
slope-induced breaking is activated through the keyword \textit{WHITE CAPPING
  DISSIPATION} in the steering file; the keyword can take three values, namely:

\begin{itemize}
\item  0 no whitecapping-induced dissipation \textit{(default value)}
\item  1 Komen \textit{et al.} \cite{Komen1984} and Janssen's
  \cite{Janssen1991} dissipation model%see \ref{parag4.3.3.1}.
\item  2 Westhuysen \textit{et al.} dissipation model \cite{Westhuys2007}.
  %see \ref{parag4.3.3.2}
\end{itemize}

For a more detailed description of the issues related to the whitecapping
dissipation modelling and of the recent advances in this field, reference can
be made to \cite{Wise2007}.

\subsubsection{  Option 1 for whitecapping: Komen and Janssen dissipation model}
\label{parag4.3.3.1}

In deep water, that term is written as follows in \tomawac:
\begin{equation} \label{GrindEQ__4_37_}
  Q_{ds} =-\frac{1}{g^{4} } C_{dis} \; \bar{\sigma }^{9} \; m_{0}^{2} \;
  \left(\delta \left(\frac{\sigma }{\bar{\sigma }} \right)^{2} +(1-\delta )
  \left(\frac{\sigma }{\bar{\sigma }} \right)^{4} \right)\; F
\end{equation}
With a finite water height, \tomawac uses the following formulation:
\begin{equation} \label{GrindEQ__4_38_}
  Q_{ds} =-C_{dis} \; \bar{\sigma }\; \bar{k}^{4} \; m_{0}^{2} \; \left(\delta
  \frac{k}{\bar{k}} +(1-\delta )\left(\frac{k}{\bar{k}} \right)^{2} \right)\; F
\end{equation}
$m_0$ denotes the total variance, $\bar{\sigma}$denotes the average intrinsic
frequency and $\bar{k}$ denotes the average wave number; they are respectively
computed as followings:
\bequ
\label{eq:defm0}
m_0 =\int _{f_r =0}^{\infty }\int _{\theta =0}^{2\pi }  F(f_r ,\theta )df_r d\theta 
\eequ
\bequ
\label{eq:defsigma}
\bar{\sigma }=\left(\frac{1}{m_0 } \int _{f_r =0}^{\infty }\int _{\theta =0}^{2\pi }
\frac{1}{\sigma }  F(f_{r} ,\theta )df_{r} d\theta \right)^{-1}
\eequ
\bequ
\label{eq:defk}
\bar{k}=\left(\frac{1}{m_{0} } \int _{f_{r} =0}^{\infty }\int _{\theta =0}^{2\pi }
\frac{1}{\sqrt{k} }  F(f_{r} ,\theta )df_{r} d\theta \right)^{-2}
\eequ


The formulas for computing the average frequency and the average wave number
are derived from those in use in WAM-cycle 4 \cite{Komen1994}. These averages
are not directly weighted by the variance spectrum, since it was found, when
WAM-cycle 3 \cite {Wamdi1988} was being developed, that the above expressions
yielded more stable results than the conventional weighted averages. Lastly,
it should be pointed out that in \tomawac, the above average quantities are
computed not only on the discretized portion of the variance spectrum, but
also analytically on the high frequency portion (up to + 8) considering a
decreasing variance in $f^{-n}$.

That source term has two parameters:

\begin{itemize}
\item  constant $C_{dis}$ (corresponding to the keyword \textit{WHITE CAPPING
  DISSIPATION COEFFICIENT }in the steering file). Its default value is taken
  as 4.5, in accordance with the proposal made by Komen \textit{et al.}
  \cite{Komen1984} and the standard value adopted in the model WAM-Cycle 4.
\item  weighting parameter $\delta$ (corresponding to the keyword \textit{WHITE
  CAPPING WEIGHTING COEFFICIENT} in the steering file). Its default value is
  taken as the 0.5 average value.
\end{itemize}

\subsubsection{Option 2 for whitecapping: Westhuysen dissipation model}
\label{parag4.3.3.2}
The Westhuysen dissipation model \cite{Westhuys2007} is based on a
saturation-based model formulation, which defines the $Q_{ds}$ term as
depending on the saturation threshold $B_r$.

 The expression proposed by Westhuysen is:
\begin{equation} \label{GrindEQ__4_40_}
  Q_{ds} =-C_{dis,break} \; \; \left(\frac{B(k)}{B_r } \right)^{p_{0} /2} \; g^{1/2}
  k^{1/2} F(f,\theta )
\end{equation}

$$\mbox{where }\dsp B(k)=\frac{1}{2\pi } \int _{0}^{2\pi }C_{g} k^{3} F(f,\theta )
d\theta  =C_{g} k^{3} \frac{E(f)}{2\pi } $$

$$\mbox{ and }p_{0} \left(\frac{u_* }{C} \right)=3+\tanh \left[w
  \left(\frac{u_* }{C} -0,1\right)\right]$$

 The variable \textit{w} is set equal to 25.

 This model is implemented in \tomawac in its most recent version, as
 formulated by Westhuysen \cite{Westhuys2008}, which combines the terms of
 Komen \cite{Komen1984} ($Q_{ds}^{K}$) with that of Westhuysen
 \cite{Westhuys2007} as follows:
\begin{equation} \label{GrindEQ__4_41_}
Q_{ds} =f_{br} (f).Q_{ds} ^{W} +\left(1-f_{br} (f)\right).Q_{ds} ^K
\end{equation}
with $f_{br} =\frac{1}{2} +\frac{1}{2} \tanh \left\{10\left[
  \left(\frac{B(k)}{B_{r} } \right)^{1/2} -1\right]\right\}$

This model is selected by setting the keyword \textit{WHITE CAPPING
  DISSIPATION} to 2 in the steering file.\textbf{}

This source term makes use of 4 parameters. Their default values correspond
to the coefficients proposed by Westhuysen  \cite{Westhuys2008}.

\begin{itemize}
\item  The coefficient ${C}_{dis,break}$, corresponding to the keyword
  \textit{WESTHUYSEN  DISSIPATION COEFFICIENT}, has a default value of
  $5.0\; 10^{-5}$,
\item  The coefficient ${B}_{r}$, corresponding to the keyword
  \textit{SATURATION THRESHOLD FOR THE DISSIPATION}, has a default value of
  $1.75\; 10^{-3}$,
\item  The coefficient ${C}_{dis,non-break}$, corresponding to the keyword
  \textit{WESTHUYSEN WHITE CAP\-PING DISSIPATION}, has a default value of 3.29,
\item  The coefficient $\delta$, corresponding to the keyword \textit{WESTHUYSEN
  WEIGHTING COEFFICIENT}, has a default value of 0.0.
\end{itemize}


\subsection{ Bottom friction-induced dissipations (term $Q_{bf}$)}
\label{BOTTOMFRICTION}
A single model is available in \tomawac. The bottom friction-induced
dissipation is activated through the keyword \textit{BOTTOM FRICTION
  DISSIPATION} in the steering file; the keyword can take two values, namely:

 \begin{itemize}
\item 0 no bottom friction-induced dissipation \textit{(default value)}

\item 1 expression obtained during the JONSWAP campaign (Hasselmann
  \textit{et al.} \cite{Hasselmann1973}) and taken up by Bouws and Komen
  \cite{Bouws1983}.
\end{itemize}

 The model used for the bottom friction-induced energy losses is an empirical
 expression globally representing the various contributions from the
 wave-bottom interaction (percolation, friction...):
\begin{equation} \label{GrindEQ__4_42_}
Q_{bf} =-\Gamma \left(\frac{\sigma }{g.\sinh \left(k.d\right)} \right)^{2} F
\end{equation}
That (linear) expression is programmed in \tomawac using the following
alternative form, which involves the dispersion relation:
\begin{equation} \label{GrindEQ__4_43_}
Q_{bf} =-\Gamma \frac{2k}{g.\sinh \left(2.k.d\right)} F
\end{equation}
That source term has a single parameter:

\begin{itemize}
\item  constant G (corresponding to the keyword \textit{BOTTOM FRICTION
  COEFFICIENT} in the steering file). Its default value is taken as
  0.038 m2.s-3, in accordance with what had been obtained during the JONSWAP
  campaign \cite{Hasselmann1973} and with the standard value being used in
  the model WAM-Cycle 4.
\end{itemize}

\subsection{ Bathymetric breaking-induced dissipations (term $Q_{br}$)}

In \tomawac, four parametric formulas are proposed for reproducing the
effects of the bathymetric breaking-induced energy dissipation on the spectrum.
The bathymetric breaking-induced dissipation is activated through the keyword
\textit{DEPTH-INDUCED BREAKING DISSIPATION} in the steering file; the keyword
can take five values:

\begin{itemize}
\item  No breaking-induced dissipation \textit{(default value)}
\item  Battjes and Janssen's model \cite{Battjes1978}% see \ref{parag4.3.5.1}
\item  Thornton and Guza's model \cite{Thornton1983}% see \ref{parag4.3.5.2}
\item  Roelvink's model \cite{Roelvink1993}% see \ref{parag4.3.5.3}
\item  Izumiya and Horikawa's model \cite{Izumiya1984}% see \ref{parag4.3.5.4}
\end{itemize}

The first three models are parametric spectral models developed for studying
the random waves, whereas the fourth one is a turbulence model initially
developed for studying the regular waves.

The general principle of the parametric spectral models consists in developing
an expression for the total dissipation of wave energy by combining a rate of
breaker-induced dissipation with a breaking probability.

Whatever model is adopted, the directional spectrum version of the bathymetric
breaking-induced dissipation term is based on the assumption that breaking does
not affect the energy frequency and direction distributions.


\subsubsection{Battjes and Janssen's model (1978)}
\label{parag4.3.5.1}
The Battjes and Janssen's breaking model \cite{Battjes1978} is based on the
analogy with a hydraulic jump. Besides, it assumes that all the breaking waves
have a height $H_m$, which is of the same order of magnitude as the water depth.
The total energy dissipation term $D_{br}$ is expressed as follows
\begin{equation} \label{GrindEQ__4_44_}
D_{br} =-\frac{\alpha Q_b f_c H_m^{2} }{4}
\end{equation}
where $H_m$ denotes the maximum wave height being compatible with the water
depth, $Q_b$ is the fraction of breaking wave, $f_c$ is a characteristic wave
frequency and a is a numerical constant of order 1.

$H_m$ can be computed either through the relation:
\begin{equation} \label{GrindEQ__4_45_}
H_m =\gamma _2 d
\end{equation}
or through a relation derived from the Miche's criterion
\begin{equation} \label{GrindEQ__4_46_}
  H_{m} =\frac{\gamma _1 }{k_c } \tanh \left(\frac{\gamma _2 k_c d}{\gamma _1 }
  \right)
\end{equation}
where $k_c$ is linked to $f_c$ by the linear wave dispersion relation.

$Q_b$ is estimated, according to the Battjes and Janssen's theory, as a
solution of the implicit equation:
\begin{equation} \label{GrindEQ__4_47_}
\frac{1-Q_b }{\ln Q_b } =-\frac{H_{m0}^{2} }{2H_m^{2} }
\end{equation}
In \tomawac, that equation can be solved either in a dichotomous way or
through explicit approximations as proposed by Dingemans \cite{Dingemans1983}.
The latter are expressed as follows when putting:
\[b=\frac{H_{m0} }{\sqrt{2} H_{m} } \]

 - \underbar{version 1}: 
\bequ
\barr{ll}
\dsp Q_b =0   &\mbox{ if }b < C_b, (C_b = 0.5) \\[6pt]
\dsp Q_b =\left(\frac{b-C_b }{1-C_b } \right)^{2}  &\mbox{ if } b \geq C_b
\earr
\eequ

 - \underbar{version 2}: 
\bequ
\barr{ll}
 \dsp q_0 =\left(2b-1\right)^{2}   &\mbox{ if } 0.5 < b < 1\\[6pt]
 \dsp q_0 =0  &\mbox{ if } b \leq 0.5\\[6pt]
 \dsp q_1 =q_0 -b^2 \frac{q_0 -e^{\left[\left(q_0 -1\right)/b^2 \right]} }
      {b^2 -e^{\left[\left(q_{0} -1\right)/b^2 \right]} } \\[6pt]
 \dsp    Q_b =0    &\mbox{ if } b \leq C_b, (C_b = 0.3)\\[6pt]
\dsp  Q_b =q_1    &\mbox{ if } C_b < b < 0.9\\[6pt]
 \dsp Q_b =q_0    &\mbox{ if } 0.9 \leq b \leq 1.0\\[6pt]
\earr
\eequ

 - \underbar{version 3}:  
\bequ
Q_b =2.4*b^7 
\eequ

The directional spectrum version of the sink term is based on the assumption
that breaking does not modify the frequency and directional distribution of
energy. The source term $Q_{br} $ is then written as:
\begin{equation} \label{GrindEQ__4_48_}
  Q_{br} \left(f,\theta \right)=-\frac{\alpha Q_{b} f_{c} H_{m}^{2} }{4}
  \frac{F\left(f,\theta \right)}{m_{0} }
\end{equation}
Three constants can be modified using keywords:

\begin{itemize}
\item  constant $\alpha$ corresponds to the keyword \textit{DEPTH-INDUCED
  BREAKING 1 (BJ) COEFFICIENT ALPHA} in the steering file. Its default value
  is taken as 1, in accordance with the value as recommended by Battjes and
  Janssen \cite{Battjes1978}
\item  constant g${}_{1}$ corresponds to the keyword \textit{DEPTH-INDUCED
  BREAKING 1 (BJ) COEFFICIENT GAMMA1} in the steering file. Its default value
  is taken as 0.88, in accordance with the value as recommended by Battjes and
  Janssen \cite{Battjes1978}
\item  constant g${}_{2}$ corresponds to the keyword \textit{DEPTH-INDUCED
  BREAKING 1 (BJ) COEFFICIENT GAMMA2} in the steering file. Its default value
  is taken as 0.8, in accordance to the value as recommended by Battjes and
  Janssen \cite{Battjes1978}
\end{itemize}

 The following keywords are for selecting the model options:

 \begin{itemize}
 \item The characteristic wave frequency is selected through the keyword
   \textit{DEPTH-INDUCED BREAKING 1 (BJ) CHARACTERISTIC FREQUENCY}. Six values
   are possible:
\begin{enumerate}
\item average frequency: $\bar{f}=\frac{\bar{\sigma }}{2\pi } $  (refer to
  equation (4.36.b))
\item average frequency: $f_{01} $, computed from the spectrum moments $m_0$
  and $m_1$ \textit{(default value)}
\item average frequency: $f_{02} $, computed from the spectrum moments $m_0$
  and $m_2$
 \item discrete peak frequency: $f_p$
 \item peak frequency computed through the Read's method to order 5: $f_{R5}$
 \item peak frequency computed through the Read's method to order 8: $f_{R8}$
\end{enumerate}

\item The computation mode for breaking probability $Q_b $ (exact
  computation or utilization of an explicit approximation as proposed by
  Dingemans \cite{Dingemans1983}) is selected through the keyword
  \textit{DEPTH-INDUCED BREAKING 1 (BJ) QB COMPUTATION METHOD}. By default,
  version 2 of the explicit formulations as proposed by Dingemans is used
  (see above). For applications, it is recommended \textbf{not} to modify the
  value of that keyword.

\item The computation mode for the maximum height compatible with the local
  water depth, $H_m$, is selected through the keyword \textit{DEPTH-INDUCED
    BREAKING 1 (BJ) HM COMPUTATION METHOD}. Two values are possible:
\begin{enumerate}
 \item Relation: $H_{m} =\gamma_2 d$  \textit{(default value)}
 \item Miches' relation (see in (\ref{GrindEQ__4_46_}) above)
\end{enumerate}
\end{itemize}

\subsubsection{Thornton and Guza's model (1983)}
\label{parag4.3.5.2}
The Thornton and Guza's breaking model \cite{Thornton1983} is based on the
analogy with a hydraulic jump and on two types of breaking wave height
distribution. The energy sink term is written according to the breaking wave
height distribution being retained:

\bequ
\label{eq:function1} 
\barr{l}
\mbox{function 1: }Q_{br1} \left(f,\theta \right)=-48\sqrt{\pi } B^{3} f_c
\frac{\left(2m_0 \right)^{5/2} }{H_m^{4} d} \; F\left(f,\theta \right)
\earr
\eequ
\bequ
\barr{l}
\label{eq:function2} 
\mbox{function 2: }Q_{br2} \left(f,\theta \right)=-12\sqrt{\pi } B^3 f_c
\frac{\left(2m_0 \right)^{3/2} }{H_m^{2} d} \left[1-\left(1+
  \left(\frac{8m_0 }{H_m^{2} } \right)\right)^{-5/2} \right]\; F\left(f,\theta
\right)
\earr
\eequ 
$f_c $ is the characteristic wave frequency (average frequency, $f_{01}$,
$f_{02}$ or peak frequency) and B is a parameter ranging from 0.8 to 1.5 (its
default value in \tomawac is $B=1.0$). The maximum wave height compatible with
the water depth, $H_m$, is governed by the parameter $\gamma$ ($H_m =\gamma d$).

The breaking model as proposed by Thornton and Guza can then be parameterized
by the user via the following 4 keywords:

 \begin{itemize}
 \item \textit{DEPTH-INDUCED BREAKING 2 (TG) WEIGHTING FUNCTION} Two values are
   possible:
 \begin{enumerate}
 \item weighting function 1 (see in \ref{eq:function1})
 \item weighting function 2 (see in \ref{eq:function2})\textit{ (default value)}
\end{enumerate}
 \item \textit{DEPTH-INDUCED BREAKING 2 (TG) CHARACTERISTIC FREQUENCY} Six values are possible:
 \begin{enumerate}
 \item average frequency: $\bar{f}=\frac{\bar{\sigma }}{2\pi } $  (refer to equation (4.36.b))
 \item average frequency: $f_{01} $, computed from the spectrum moments $m_0$ et$ m_1$
 \item average frequency: $f_{02} $, computed from the spectrum moments $m_0$ et $m_2$
 \item discrete peak frequency: $f_p$
 \item peak frequency computed through the Read's method to order 5: $f_{R5}$ \textit{(default value)}
 \item peak frequency computed through the Read's method to order 8: $f_{R8}$
\end{enumerate}
\item \textit{ DEPTH-INDUCED BREAKING 2 (TG) COEFFICIENT B} corresponding to
  the B variable. Its default value in the model is taken as 1.
\item \textit{DEPTH-INDUCED BREAKING 2 (TG) COEFFICIENT GAMMA} corresponding
  to the $\gamma$ variable. Its default value in the model is taken as 0.42.
\end{itemize}


\subsubsection{Roelvink's model (1993)}
\label{parag4.3.5.3}
The Roelvink's breaking model \cite{Roelvink1993} is based on the analogy
with a hydraulic jump and on two types of wave height distribution in the
breaking zone (Weibull or Rayleigh). The energy sink term is written according
to the wave height distribution in the breaking zone:

\begin{itemize}
\item  \underbar{Weibull's distribution:}
\begin{equation} \label{GrindEQ__4_50_}
\barr{ll}
\dsp Q_{br1} \left(f,\theta \right)=&\dsp -\alpha f_c mA\, \sqrt{\frac{2}{m_0 }}
F\left(f,\theta \right)\int _0^{\infty }\left(\frac{H}{\sqrt{8m_0} } \right)^{2m+1}
\\[12pt]
&\dsp \exp \left[-A\left(\frac{H}{\sqrt{8m_{0} } } \right)^{2m} \right].
\left[1-\exp \left(-\left(\frac{H}{\gamma d} \right)\right)^{N} \right] \, dH
\earr
\end{equation}
\end{itemize}

\begin{equation}
\label{eq:defweibull}
A=\left[\Gamma \left(1+\frac{1}{m} \right)\right]^m\mbox{ with } m=1+0.7\tan^2
\left( \frac{\pi}{2}\frac{1}{\gamma_2}\frac{\sqrt{8m_0}}{d} \right)
\eequ
The coefficient $\gamma_2 $ is usually set to 0.65.

\begin{itemize}
\item  \underbar{Rayleigh's distribution:}
\begin{equation} \label{GrindEQ__4_52_}
\barr{ll}
\dsp Q_{br2} \left(f,\theta \right)=&\dsp -\alpha f_{c} \sqrt{\frac{2}{m_{0} } }
F\left(f,\theta \right)\; \int_0^{\infty }\left(\frac{H}{\sqrt{8m_0 } } \right)^3
\\[12pt]
&\dsp \exp \left[-\left(\frac{H}{\sqrt{8m_0 } } \right)^2 \right]\,
\left[1-\exp \left(-\left(\frac{H}{\gamma d} \right)\right)^N \right]\, dH
\earr
\end{equation}
\end{itemize}
$f_c $ denotes the characteristic wave frequency (average frequency, $f_{01} $,
$f_{02} $ or peak frequency), a is a numerical constant of order 1, $\gamma$ is
the proportional control factor between the allowable wave height and the
water depth (by default, $\gamma =0.54$) and N is an exponent in the wake
breaking weighting function (typically N=10).

Thus, the Roelvink's breaking model can be parameterized by the user via the
following 5 keywords:

\begin{itemize}
\item  \textit{DEPTH-INDUCED BREAKING 3 (RO) COEFFICIENT ALPHA}, corresponding
  to the a  variable. Its default value in the model is taken as 1.0.

\item  \textit{DEPTH-INDUCED BREAKING 3 (RO) COEFFICIENT GAMMA},
  corresponding to the $\gamma$ variable. Its default value in the model is
  taken as 0.54.

\item  \textit{DEPTH-INDUCED BREAKING 3 (RO) COEFFICIENT GAMMA2},
  corresponding to the $\gamma_2$ variable. Its default value in the model is
  taken as 0.65.

\item  \textit{DEPTH-INDUCED BREAKING 3 (RO) WAVE HEIGHT DISTRIBUTION}
  provided for retaining either a Weibull distribution \eqref{GrindEQ__4_43_}
  if the (default) value of the parameter is 1 or a Rayleigh distribution
  \eqref{GrindEQ__4_45_} if the parameter value is 2.

\item  \textit{DEPTH-INDUCED BREAKING 3 (RO) EXPONENT WEIGHTING FUNCTION},
  corresponding to the N variable. Its default value in the model is 10.

\item  \textit{DEPTH-INDUCED BREAKING 3 (RO) CHARACTERISTIC FREQUENCY} Six
  values are possible:\textit{}
\begin{enumerate}
\item average frequency: $\bar{f}=\frac{\bar{\sigma }}{2\pi } $  (refer to
  equation (4.36.b))
\item average frequency: $f_{01} $, as computed from the spectrum moments
  $m_0$ and $m_1$
\item average frequency: $f_{02} $, as computed from the spectrum moments
  $m_0$ and $m_2$
 \item discrete peak frequency: $f_p$
 \item peak frequency as computed through the Read's method to order 5:
   $f_{R5}$ \textit{(default value)}
 \item peak frequency as computed through the Read's method to order 8:
   $f_{R8}$
\end{enumerate}
\end{itemize}


\subsubsection{Izumiya and Horikawa's turbulence model (1984)}
\label{parag4.3.5.4}
Izumiya and Horikawa \cite{Izumiya1984} sought an estimate of the dissipation
by breaking-induced turbulence in the case of regular waves. From the Reynolds'
equations and only considering a one-dimensional condition, they obtained an
expression of the breaking-induced dissipation of wave energy in the following
form:
\begin{equation} \label{GrindEQ__4_53_}
  \frac{d}{dx} \left(EC_{g} \right)=-\alpha \frac{E^{3/2} }{\rho ^{1/2} d^{3/2} }
  \left(\frac{2C_{g} }{c} -1\right)^{1/2}
\end{equation}
$E$ denotes the total wave energy, $C_g $ and c are respectively group and
phase velocities associated to the characteristic wave frequency $f_c $
(average frequency $f_{01} $, $f_{02} $ or peak frequency), $\alpha $ is a
parameter governing the magnitude of the energy dissipation to be determined.
For any profile, a shoal may induce the wave reforming. In order to take that
process into account, Izumiya and Horikawa express the factor$\alpha $ in
terms of a deviation from a steady state:
\[\alpha =\beta_0 \left(M_*^2 -M_{*S}^2 \right)^{1/2} \]
where $M_*$ is a dimensionless quantity in the form:
$M_*^2 =\frac{C_g}{c} .\frac{E}{\rho gd^2 }$

From laboratory data, Izumiya and Horikawa set $M_{*S}^2 $ to $9\;10^{-3}$
and $\beta_0 $ to 1.8.

Assuming that the breaking does not affect the frequency and direction
distribution of energy, the dissipation term is lastly written as:
\begin{equation} \label{GrindEQ__4_54_}
  Q_{br} \left(f,\theta \right)=-\beta_0 \left(\frac{C_g}{c} .\frac{m_0}{d^2 }
  -M_{*S}^2 \right)^{1/2} \frac{g^{1/2} m_0^{1/2} }{d^{3/2} } \left(\frac{2C_g}{c}
  -1\right)^{1/2} F\left(f,\theta \right)
\end{equation}
Thus, the breaking model as proposed by Izumiya and Horikawa can be
parameterized by the user through the three following keywords:

\begin{itemize}
\item  \textit{DEPTH-INDUCED BREAKING 4 (IH) COEFFICIENT BETA0},
  corresponding to the $\beta _{0} $ variable. The default value in the model is
  1.8.
\item  \textit{DEPTH-INDUCED BREAKING 4 (IH) COEFFICIENT M2STAR},
  corresponding to the $M_{*S}^{2} $ variable. The default value in the model is
  0.009.
\item \textit{DEPTH-INDUCED BREAKING 4 (IH) CHARACTERISTIC FREQUENCY} Six
  values are possible:
\begin{enumerate}
\item average frequency: $\bar{f}=\frac{\bar{\sigma }}{2\pi } $  (refer to
  equation (4.29.b))
\item average frequency: $f_{01} $, as computed from the spectrum moments
  $m_0$ and $m_1$
\item average frequency: $f_{02} $, as computed from the spectrum moments
  $m_0$ and $m_2$
 \item discrete peak frequency: $f_p$
 \item peak frequency as computed through the Read's method to order 5:
   $f_{R5}$ \textit{(default value)}
 \item peak frequency as computed through the Read's method to order 8:
   $f_{R8}$
\end{enumerate}
\end{itemize}


\subsection{ Non-linear quadruplet interactions (term Qnl)}
\label{Quadruplet}
Three non-linear quadruplet interactions models are available in \tomawac.
The non-linear quadruplet interactions are activated through the keyword
\textit{NON-LINEAR TRANSFERS BETWEEN FREQUENCIES} in the steering file; the
keyword can take four values, namely:

\begin{itemize}
\item  0 no non-linear quadruplet interaction \textit{(default value)}
\item  1 DIA method (Discrete Interaction Approximation) of Hasselmann
  \textit{et al.} \cite{Hasselmann1985_1} \cite{Hasselmann1985_2} which is a
  discrete parameterization of the exact computation operator as proposed by
  Hasselmann \cite{Hasselmann1962_1} \cite{Hasselmann1962_2}
  % see \ref{parag4.3.6.1}
\item  2 MDIA method (Multiple DIA) as proposed by Tolman \cite{Tolman2004}
  % see \ref{parag4.3.6.2}
\item  3 Quasi exact GQM method (Gaussian Quadrature Method) as introduced by
  Lavrenov \cite{Lavrenov2001} and implemented by Gagnaire-Renou et al.
  \cite{Gagnaire2010}% see \ref{parag4.3.6.3}.
\end{itemize}


\subsubsection{Option 1 for non-linear quadruplet interactions: DIA method}
\label{parag4.3.6.1}

The method and its implementation in \tomawac have been the subject of a
specific report \cite{Benoit1997} which the reader is invited to refer to for
further information. The major teachings of the DIA method are summarized below.

The exact expression of the deep water interactions term as set by Hasselmann
\cite{Hasselmann1962_1} \cite{Hasselmann1962_2}, expressed herein for
convenience as a function of the wave number vector, is analogous to a
Boltzmann integral:
\[Q_{nl}^{exact} =\iiint  \sigma_4 .G.\delta \left(\vec{k}_1 +\vec{k}_2 -
\vec{k}_3 -\vec{k}_4 \right)\delta \left(\sigma_1 +\sigma_2 -\sigma_3 -\sigma_4
\right)\]
\begin{equation} \label{GrindEQ__4_55_}
  \left[\frac{F(\vec{k}_1 )}{\sigma_1} \frac{F(\vec{k}_2 )}{\sigma_2}
    \left(\frac{F(\vec{k}_3 )}{\sigma_3} +\frac{F(\vec{k}_4 )}{\sigma_4 }
    \right)-\frac{F(\vec{k}_3 )}{\sigma_3} \frac{F(\vec{k}_4 )}{\sigma_4}
    \left(\frac{F(\vec{k}_1)}{\sigma_1 } +\frac{F(\vec{k}_2 )}{\sigma_2}
    \right)\right]d\vec{k}_1 d\vec{k}_2 d\vec{k}_3
\end{equation}
The energy exchanges, in that integral (\textit{a }priori rather uneasily computable), take place between quadruplets meeting the resonance conditions:
\bequ
\label{eq:resonancenl}
\sigma_1 +\sigma_2 =\sigma_3 +\sigma_4 \mbox{ and }\vec{k}_1 +\vec{k}_2
=\vec{k}_3 +\vec{k}_4 
\eequ

as evidenced by the two Dirac functions $\delta$ in the integral.

G denotes the value of the coupling term for the resonant quadruplet
interactions$\left(\vec{k}_1 ,\vec{k}_2 ,\vec{k}_3 ,\vec{k}_4 \right)$.
Establishing and computing its expression is also an awkward task. Hasselmann
\cite{Hasselmann1962_1} proposed a computation mode that was also taken up and
given a more concise form by such other authors as Webb \cite{webb1978}.

The exact computation of the above Boltzmann integral is too complex and
time-consuming for such a sea state operational model as \tomawac (see e.g.
\cite{Hasselmann1985_1}). That is why, starting from the experiment as
developed in WAM \cite{Wamdi1988} \cite{Komen1994}, \tomawac uses the DIA
(Discrete Interaction Approximation) approximate computation method as proposed
by Hasselmann \textit{et al.} \cite{Hasselmann1985_1}. Whereas the exact
computation requires the summation of the contributions from a great number of
quadruplets, the approximate computation implements only a small number of
quadruplet configurations which are all similar.

That standard interaction configuration is defined as follows:

\begin{itemize}
\item  two of the wave numbers are alike: $\vec{k}_1 =\vec{k}_2 =\vec{k}$,
  which also involves that the two related frequencies are identical:
  $\sigma_1 = \sigma_2 = \sigma$

\item  the other two frequencies s3 and s4 are defined by:
\end{itemize}

  $\sigma_3 = (1+\lambda) \sigma = \sigma^+$

  $\sigma_4 = (1-\lambda) \sigma = \sigma^-$

Through the value $\lambda = 0.25$, a good correlation with the exact
computation of the integral ( see\cite{Hasselmann1985_1}) could be achieved.
That value is used in the model WAM \cite{Wamdi1988} \cite{Komen1994} and is
taken up in \tomawac.

\begin{itemize}
\item  since the wave vectors $\vec{k}_3 =\vec{k}^+ $ and $\vec{k}_4 =\vec{k}^-$
  should observe the resonance condition, it can be shown they are featured
  by angles $\theta_3=11.5^\circ$ and $\theta_4=-33.6^\circ$ with respect to the
  common direction of $\vec{k}_1 =\vec{k}_2 =\vec{k}$ (refer to
  \cite{Hasselmann1985_1}).

\item  Furthermore, the mirror image is taken into account by considering the
  vectors as symmetrical with respect to the direction of
  $\vec{k}_1 =\vec{k}_2 =\vec{k}$.
\end{itemize}

The standard interaction configuration (in full line) and its mirror image
(in dotted line) are shown schematically in Figure \ref{fig:dia}.

\begin{figure}[H]%
\begin{center}
\includegraphics*[width=4.46in]{graphics/fig42}
\caption{Schematic standard interaction configuration for the DIA method}
\label{fig:dia}
\end{center}
\end{figure}

With this standard configuration, the non-linear interaction term for all
four resonant wave numbers is written as \cite{Hasselmann1985_1}:
\begin{equation} \label{GrindEQ__4_57_}
  \left[\begin{array}{c} Q_{nl}  \\ Q_{nl}^{-}  \\ Q_{nl}^{+} \end{array}\right]
  =\left[\begin{array}{c} -2 \\ 1 \\ 1 \end{array}\right]\; \Pi \; g^{-4} \;
  f_r^{11} \; \left(F^2 \left(\frac{F^+ }{\left(1+\lambda \right)^4 } +
  \frac{F^-}{\left(1-\lambda \right)^{4} } \right)-
  \frac{2FF^+ F^- }{\left(1-\lambda ^{2} \right)^{4} } \right)
\end{equation}
With such a computation method, the vector $\vec{k}$ scans all the
discretization nodes of the directional spectrum mesh. The number of
configurations being considered is then twice as large as the number of points
in that mesh. In relation to the full computation, the 5-dimensional space
(three integration dimensions and two dimensions for $\vec{k}_4$) of all the
possible resonant quadruples is reduced to a 2-dimensional space.

In a finite water depth, from exact computations of the Boltzmann integral,
Herterich and Hasselmann \cite{Herterich1980} suggested to make a deep water
computation based on the previous method, then to multiply it by a coefficient
R representing the effect of the finite water height:
\begin{equation} \label{GrindEQ__4_58_}
Q_{nl} (d)=R.Q_{nl} (d=\infty )
\end{equation}
Coefficient R is a function of the normalized water height $\bar{k}.d$ and is
expressed as follows:
\bequ
\label{eq:defcoefR}
R(\chi )=1+\frac{5.5}{\chi } (1-\frac{5}{6} \chi )\exp \left(-\frac{5}{4}
\chi \right) \mbox{ where } \chi =\frac{3}{4} \bar{k}.d
\eequ
The average wave number $\bar{k}$ was defined in the previous paragraph
(see \ref{eq:defk}).

In its authors' opinion, that relation remains valid as long as
$\bar{k}.d > 1$. It is used as such in \tomawac for the finite water depth
computations.

That source-term has a single parameter:

\begin{itemize}
\item  constant l (corresponding to the keyword \textit{STANDARD CONFIGURATION
  PARAMETER} of the steering file). Its default value is taken as 0.25, in
  accordance with the proposal made by Hasselmann \textit{et al.}
  \cite{Hasselmann1985_1} and with the standard value in the model WAM-Cycle 4.
\end{itemize}


\subsubsection{Option 2 for non-linear quadruplet interactions: MDIA method}
\label{parag4.3.6.2}
The MDIA method (multiple DIA) is an extension of the DIA algorithm. We use
here the version proposed by Tolman \cite{Tolman2004}

This method can give very reasonable results in simple situations, but in
case of unsteady or rapidly changing sea-state conditions (e.g. in the
situation of abrupt changes of wind direction) it results in significant
qualitative and quantitative differences when compared with exact methods
\cite{Benoit2005}.

The MDIA method consists of using a quadruplet depending on 2 parameters,
$\lambda$ and $\mu$, defined as follows:
\begin{equation} \label{GrindEQ__4_60_}
\vec{k}_0 +\vec{k}_1 =\vec{k}_2 +\vec{k}_3 =2\vec{k}
\end{equation}

\begin{equation} \label{eq:defMDIA}
\barr{lr}
\sigma_0 =(1+\mu )\sigma, & \sigma_1 =(1-\mu )\sigma  \\[12pt]
\sigma_2 =(1+\lambda )\sigma, &\sigma_3 =(1-\lambda )\sigma 
\earr
\end{equation}


The $\lambda$ and $\mu$ values proposed by Tolman that allow to best estimate
the $Q_{nl4}$ source term in the case of 4 interacting quadruplets are shown in
Figure \ref{fig:tolman}


\begin{figure}[H]%
\begin{center}
 \includegraphics*[width=4.46in]{graphics/tabtolman}
 \caption{Values of the parameters $\lambda$ and $\mu$ proposed by Tolman to
   best estimate quadruplet interactions with the MDIA method, in the case of
   4 interacting quadruplets \cite{Tolman2004}.}
\label{fig:tolman}
\end{center}
\end{figure}

To select this model in \tomawac the keyword \textit{NON-LINEAR TRANSFERS
  BETWEEN FREQUENCIES} must be set to 2 in the steering file.

This model does not require any other parameter: the values of the $\lambda$
and $\mu$ parameters are set as constants in the code. However they can be
modified when considering a larger number of interacting quadruplets.


\subsubsection{Option 3 for non-linear quadruplets interactions: GQM method}
\label{parag4.3.6.3}

The Gaussian Quadrature Method (GQM) is based on the use of Gaussian
quadratures for the different numerical integrations arising in evaluating
Equation \ref{GrindEQ__4_55_}. This technique, proposed by Lavrenov
\cite{Lavrenov2001}, has been developed and optimised to adequate results
regarding both precision and CPU time \cite{Benoit2005}, \cite{Gagnaire2009},
\cite{Gagnaire2010}

Several steps are needed to transform Equation \ref{GrindEQ__4_55_} into an
expression that can be integrated via Gaussian quadrature method. They can be
summarized as follows (for a detailed description reference can be made to
\cite{Gagnaire2009}):

\begin{enumerate}
\item  Elimination of the Dirac function on the wave numbers
  $\delta \left(\vec{k}_0 +\vec{k}_1 -\vec{k}_2 -\vec{k}_3 \right)$, by imposing
  $\vec{k}_3 =\vec{k}_0 +\vec{k}_1 -\vec{k}_2 $ (see the resonance condition,
  Equation \ref{eq:resonancenl}). Equation \ref{GrindEQ__4_55_} is therefore
  reduced to an integral with 4 dimensions, including a single Dirac function
  on the frequency $\sigma$.

\item  Variable change, to work with $(\sigma,\theta)$ instead of $\vec{k}$,
  and reformulation of the equation in terms of variance density (F) instead
  of wave action density (N).

\item  Integration over ${\theta}_2$ and elimination of the Dirac function on
  the frequency. A 3-dimension integral is obtained, without any Dirac function.

\item  Final expression of the non-linear transfer term: the variables
  $\sigma_3$, $\theta_2$ and $\theta_3$, are expressed as functions of
  $\sigma_1$, $\theta_1$ and $\sigma_2$, and of $\sigma_0$ and $\theta_0$, which
  $Q_{nl4}$ term depends on. The variables $\sigma_a, k_a \mbox{ et } \epsilon_a$
  defined by $\sigma_a = \sigma_0 + \sigma_1$, $\vec{k}_a =\vec{k}_0 +\vec{k}_1 $
  and $\epsilon_a=2gk_a/\sigma_{a}^{2}$, and depending only on ($\sigma_0$ and
  $\theta_0$) and ($\sigma_1$ and $\theta_1$), are used as well.
\begin{equation} \label{GrindEQ__4_62_}
  Q_{nl4} =\int _{\sigma_1 =0}^{+\infty }\; \int _{\theta_1 =0}^{2\pi }\;
  \int _{\sigma_2 =0}^{\sigma_a /2}   2\frac{\sigma_a^{4} \; T}{\sigma_1 \sigma_2
    \sigma_3 } \; \frac{F_2 F_3 (F_0 \sigma_1^4 +F_1 \sigma_0^4 )-F_0 F_1
    (F_2 \sigma_3^4 +F_3 \sigma_2^4 )}{\sqrt{\tilde{B}_0 \left(\varepsilon_a ,
      s_2 \right)\; \tilde{B}_1 \left(\varepsilon_a ,s_2 \right)\;
      \tilde{B}_2 \left(\varepsilon_a ,s_2 \right)} } \;
  d\sigma_1 d\theta_1 d\sigma_2
\end{equation}
\end{enumerate}
where $s_2$ is defined as $s_2= \sigma_2/ \sigma_a$.

Equation \ref{GrindEQ__4_62_} is then integrated using different quadrature
methods:

\begin{itemize}
\item  Gauss-Legendre or Gauss-Chebyshev quadratures are used for the
  integration over $\sigma_2$, depending on the $\epsilon_a$ values determining
  number and type of singularities.

\item  Gauss-Chebyshev quadratures are used for the integration over $\theta_1$.

\item  The integration over $\sigma_1$ is realized using the trapezoidal rule.
\end{itemize}

 Three different GQM method resolutions have been tested:

\begin{itemize}
\item  A "fine" resolution, considered as the exact calculation of the
  non-linear transfer term, as no improvement in the results is noticed when
  further increasing the method resolution.

\item  An "intermediate" resolution.

\item  A "coarse" resolution, whose parameters are given as default values in
  \tomawac, which represents the best compromise between accuracy of the
  solution and CPU time.
\end{itemize}

The configurations that do not effect significantly the global computation of
$Q_{nl4}$ are neglected. This configuration selection allows to reduce the CPU
time. The threshold values set as default in \tomawac reduce the number of
configuration:

\begin{itemize}
\item  by 21\% in the "fine" resolution case

\item  by 34\% in the "intermediate" resolution case

\item  by 64\% in the "coarse" resolution case
\end{itemize}

The GQM method is selected by setting to 3 the keyword
\textit{NON-LINEAR TRANSFERS BETWEEN FREQUENCIES} in the steering file.

This method makes use of 6 parameters. The default values of those parameters
correspond to the "coarse" resolution case:

\begin{itemize}
\item  The three keywords \textit{SETTING FOR INTEGRATION ON OMEGA1},
  \textit{SETTING FOR INTEGRATION ON OMEGA2} and \textit{SETTING FOR
    INTEGRATION ON THETA1} determine the number of integration points over the
  three variables $\sigma_1$, $\theta_1$ and $\sigma_2$ and their default values
  are respectively 3, 3 and 6 ("coarse" resolution). The values 1, 4, 8 and 2,
  8, 12 correspond respectively to the "intermediate" and "fine" resolution
  cases.

\item  The three keywords \textit{THRESHOLD0 FOR CONFIGURATIONS ELIMINATION},
  \textit{THRESHOLD1 FOR CONFIGURATIONS ELIMINATION} and \textit{THRESHOLD2
    FOR CONFIGURATIONS ELIMINATION} affect the percentage of discarded
  configurations. Their default values are respectively 0, $10^{10}$ and 0.15.
  For the "intermediate" and "fine" resolution cases, the first two values are
  the same, and the threshold2 values is equal respectively to 0.01 and 0.001.
\end{itemize}

\subsection{Non-linear transfers between triads (Qtr term)}
Two non-linear triads interactions model are available in \tomawac. The triad
model is activated through the keyword \textit{TRIAD INTERACTIONS} in the
steering file. The keyword can take three values
\begin{itemize}
\item 0 No triad interaction \textit{(default value)}
\item 1 LTA model interaction, a model proposed by Eldeberky
  \cite{Eldeberky1995} see \ref{parag4.3.7.1}.
\item 2 SPB model proposed by Becq \cite{Becq1998} see \ref{parag4.3.7.2}.
\end{itemize}

\subsubsection{LTA (Lumped Triad Approximation) model}
\label{parag4.3.7.1}
A parametric model allowing to take into account the non-linear triad
interactions in the averaged-phase models has been proposed by Eldeberky and
Battjes \cite{Eldeberky1995}. The LTA model is a parametric approach that is
based on the Madsen and Sorensen's deterministic spectral model
\cite{Madsen1993}. Simplifying hypotheses are introduced for reducing the
computation cost. Thus, a parametric formulation is given for the biphase as
a function of the Ursell's number and the model is restricted to the
self-interactions.

 The source term is written as:
\begin{equation} \label{GrindEQ__4_63_}
%\left.
\begin{array}{l} 
Q_{LTA} (f,\theta )=Q_{LTA}^+ (f,\theta )+Q_{LTA}^- (f,\theta ) \\[12pt] 
Q_{LTA}^+ (f,\theta )=\alpha_{LTA} \, c\; C_g \; g^2 \frac{R_{f/2,f/2}^2(f/2,f/2)}
{S_f^2} \sin \left|\, \beta _{f/2,f/2} \right|\; \; \left[F^{2} (f/2,\theta )-
  2F(f,\theta )F(f/2,\theta )\right] \\[12pt] 
Q_{LTA}^- (f,\theta )=-2\; Q_{LTA}^+ (2f,\theta )
 \end{array}
%\right\}
\end{equation}
$\alpha _{LTA} $ is the model adjustment coefficient; c and $C_g$ denote the
phase and group velocities, respectively.

R is the self-interaction coefficient: 
\bequ
\label{eq:selfinter}
\dsp R_{f,f} =\left(2k\right)^2 \, \left[\frac{1}{2}
  +\frac{\left(2\pi f\right)^2 }{gd\, k^2 } \right]
\eequ


S is given by the relation: 
\bequ
\label{eq:sf}
\dsp S_f =\, -2\; k\, \left(gd+2 B\, gd^3 k^2 -(B+1/3)\, d^2
\left(2\pi f\right)^2 \right)
\eequ

The biphase $\beta $ is given by the relation: 
\bequ
\label{eq:biphase}
\dsp \beta \left(f,f\right) = -\frac{\pi }{2} +\frac{\pi }{2}
\tanh \left(\frac{0.2}{Ur} \right)
\eequ

where $Ur$ denotes the Ursell's number: 
\bequ
\label{eq:Ursell}
\dsp Ur=\frac{g}{8\pi ^2 \sqrt{2} } \frac{H_{m0} \, T_m^{2} }{d^{2} } 
\eequ

with $H_{m0} $ being the significant spectral height and $T_m$ being the
average wave time.

$Q_{LTA}^{\pm } $ denotes the negative and positive contributions of the
self-interactions. Since $Q_{LTA}^+$ denotes the positive contributions to the
first upper harmonic, it should be positive. The negative values of $Q_{LTA}^+$
are replaced by the zero value. In the numerical integration of the energy
equation, the source term for the triad interactions is generally only computed
for frequencies that are lower than $R_{fm}f_m$ (Ris \cite{Ris1997}) where
$R_{fm} =2.5$.

 Two constants can me modified through keywords:

\begin{itemize}
\item  constant $\alpha _{LTA} $ corresponding to the keyword \textit{TRIADS
  1 (LTA) COEFFICIENT ALPHA}. Its default value is $\alpha _{LTA} =0.5$
\item  constant $R_{fm} $ corresponding to the keyword \textit{TRIADS 1 (LTA)
  COEFFICIENT RFMLTA}. Its default value is $R_{fm} =2.5$
\end{itemize}


\subsubsection{SPB model}
\label{parag4.3.7.2}
The SPB model was developed by Becq [Becq,~1998] from the extended Boussinesq
equations as proposed by Madsen and Sorensen \cite{Madsen1992}. The model is for
simulating the effects induced by the collinear and non-collinear interactions
of spectral components. The source term is written as:
\begin{equation} \label{GrindEQ__4_68_}
\begin{array}{l} 
  \dsp Q\left(f,\theta \right)=\frac{B'\, g}{2S_{1,f} } \int_0^f\int_0^{2\pi}
  \int_0^f\int_0^{2\pi}df_1 df_2  d\theta_1 d\theta_2 T_{f_1 ,f_2 }
  \delta \left(\theta_{\vec{k}} -\theta_{\vec{k}_1 +\vec{k}_2 } \right)
  \delta \left(f-f_1 -f_2 \right) \\[12pt]
  \dsp+\frac{B'\, g}{S_{1,f} } \int _0^\infty \int_0^{2\pi }\int_0^\infty\int_0^{2\pi }
  df_1 df_2     d\theta_1 d\theta_2 T_{-f_2 ,f_1 }
  \delta \left(\theta _{\vec{k}_1 } -\theta _{\vec{k}+\vec{k}_2 } \right)
  \delta \left(f_1 -f-f_2 \right) \end{array}
\end{equation}
with: 
\bequ \label{eq:tf1f2}
T_{f_1 ,f_2 } =\frac{gK}{K^2 +\Delta k^2 } \; R_{f_1 ,f_2 } \; \left[-
  \frac{R_{-f_2 ,f} }{\, S_{2,f_1 } k_1 } F F_{2}
  -\frac{R_{-f_1 ,f} }{\, S_{2,f_2 } k_2 } F F_{1}
  +\frac{R_{f_1 ,f_2 } }{\, S_{2,f} k} F_1 F_2 \right]
\eequ

\bequ \label{eq:Bprime}
B'=\frac{Cg}{2\pi k} 
\eequ

\bequ \label{eq:rf1f2}
R_{f1,f2} =\left(k_1 +k_2 \right)^2 \, \left[\frac{1}{2} +
  \frac{\left(2\pi \right)^2 f_1 f_2 }{gd\, k_1 k_2 } \right]
\eequ

\bequ \label{eq:Sf2}% (idem {eq:sf}
S_f  = -2 k\left(gd+2\, B\, gd^3 k^2 -(B+1/3)\, d^2 \left(2\pi f\right)^2 \right)
\eequ

$F$ denotes the variance spectrum in terms of frequencies and directions,
$F\left(f,\theta \right)$. $T_{f_1 ,f_2 } $ and $T_{-f_2 ,f_1 } $ respectively
correspond to the sum and difference interactions. $K$ is the model adjustment
parameter.

Since the model was designed for taking into account the energy transfers for
all the possible triad configurations within the spectrum, the computation
times are very long. In order to shorten these computation times, the
interactions can be restricted over a range of spectral components that are
included within a given angular sector. Thus, directional limits can be
user-prescribed.

Three constants can be modified through keywords:

\begin{itemize}
\item  constant $K$ corresponding to the keyword \textit{'TRIADS 2 (SPB)
  COEFFICIENT K}. Its default value is $K=0.34$

\item  the lower and upper directional markers corresponding to the keywords
  \textit{TRIADS 2 (SPB) LOWER DIRECTIONAL BOUNDARY} and \textit{TRIADS 2
    (SPB) UPPER DIRECTIONAL BOUNDARY}. Their respective default values are 0
  and 360.
\end{itemize}


\subsection{Wave blocking effects (upper limit for spectrum or dissipation
  $Q_{ds,cur}$)}

When water waves meet a strong adverse current, with a velocity that
approaches the wave group velocity, waves are blocked. Two options can be
considered in \tomawac to take into account wave blocking effects.

\underbar{Option 1} : consider an equilibrium range spectrum (in the presence
of ambient flow) applied as an upper limit for the spectrum (Hedges et al.
\cite{Hedges1985})

\underbar{Option 2} : add a dissipative term on the right-hand side of the
action balance equation \cite{Westhuys2012}

The inclusion of wave-blocking effects is activated through the keyword
\textit{DISSIPATION BY STRONG CURRENT} in the steering file; the keyword can
take three values, namely:

\begin{itemize}
\item 0 No wave-blocking effects (default value)
\item 1 Upper limit for spectrum Hedges et al \cite{Hedges1985} see
  \ref{parag4.3.8.1}.
\item 2 Westhuysen \textit{et al.} \cite{Westhuys2012} enhanced dissipation of
  waves on a strong current see\ref{parag4.3.8.2}.
\end{itemize}


\subsubsection{option 1 for wave blocking: upper limit for the spectrum}
\label{parag4.3.8.1}
An upper limit is imposed to the spectrum, using a Phillips
\cite{Phillips1977} shape, i.e.

$$ 
\mbox{ If } 
E(f)>E\max =\frac{\alpha g^{2} }{\left(2\pi \right)^{4} f^{5} } \mbox{ then } 
F(f,\theta )=\frac{E\max }{E(f)} F(f,\theta )
$$
with $\alpha$ the Phillips's constant in the Pierson-Moskowitz spectrum,
equal to = 0.0081. 

This model is selected by setting the keyword \textit{DISSIPATION BY STRONG C
  URRENT }to 1 in the steering file.\textbf{}

\subsubsection{option 2 for wave blocking: Westhuysen formulation }
\label{parag4.3.8.2}

 The expression proposed by Westhuysen is:
 \[\dsp Q_{ds,cur} =-C_{ds,cur} \max \left(\frac{\mathop{f_r }\limits^{\bullet} }{f},
 0\right)\; \; \left(\frac{B(k)}{B_{r} } \right)^{p_{0} /2} \; F(f,\theta )\]
 where $$B(k)=\frac{1}{2\pi } \int_0^{2\pi }C_g k^3 F(f,\theta )d\theta
 =C_g k^3 \frac{E(f)}{2\pi } $$

 and $$p_0 \left(\frac{u_* }{C} \right)=3+\tanh
 \left[w\left(\frac{u_*}{C} -0,1\right)\right]$$.

The variable \textit{w} is set equal to 25.

This model is selected by setting the keyword \textit{DISSIPATION BY STRONG
  CURRENT} to 2 in the steering file.

This source term makes use of 2 parameters. Their default values correspond to
the coefficients proposed by Westhuysen \cite{Westhuys2012}:

\begin{itemize}
\item  The coefficient $C_{ds,cur}$, corresponding to the keyword DISSIPATION
  COEFFICIENT FOR STRONG CURRENT has a default value of 0.65,

\item  The coefficient ${B}_{r}$, corresponding to the keyword
  \textit{SATURATION THRESHOLD FOR THE DISSIPATION}, has a default value of
  $1.75 10^{-3}$.
\end{itemize}


\subsection{ Dissipation due to vegetation.  }
When the ratio between vegetation height and water depth is important,
vegetation can imply some dissipation. Some methods exist to modelize this
dissipation. The method we chose is based on the formulation proposed by Suzuki
et al. \cite{Suzuki2011}
\[Q_{veg} =-\sqrt{\frac{2}{\pi } } \tilde{C}_{D} b_{v} N_{v}
\left(\frac{\bar{k}}{\bar{\sigma }} \right)^{3} \frac{\sinh ^{3}
  (\bar{k}\alpha h)+3\sinh (\bar{k}\alpha h)}{3\bar{k}\cosh ^{3}
  (\bar{k}h)} \sqrt{m_{0} } F(\sigma ,\theta )\]
$m_0$ denotes the total variance, $\bar{\sigma }$ denotes the average
intrinsic frequency and $\bar{k}$ denotes the average wave number; they are
respectively computed as followings:
\bequ \label{eq:defmObis }
m_{0} =\int _{f_{r} =0}^{\infty }\int _{\theta =0}^{2\pi }\;  \;  F(f_{r} ,\theta )\;
df_{r} \; d\theta 
\eequ

\bequ \label{eq:defsigmabis }
\bar{\sigma }=\left(\frac{1}{m_{0} } \int_{f_r =0}^{\infty }\int _{\theta =0}^{2\pi }\; 
\frac{1}{\sigma } \;  F(f_{r} ,\theta )\; df_{r} \; d\theta \right)^{-1}
\eequ
\bequ \label{eq:defkbis}
\bar{k}=\left(\frac{1}{m_{0} } \int _{f_{r} =0}^{\infty }\int _{\theta =0}^{2\pi }\; 
\frac{1}{\sqrt{k} } \;  F(f_{r} ,\theta )\; df_{r} \; d\theta \right)^{-2} 
\eequ
Where $\tilde{C}_D$ is the vegetation drag coefficient ({\it BULK DRAG
  COEFFICIENT}), $b_v[m]$  is the stem diameter of cylinder (plant) ({\it STEM
    DIAMETER}), $N_v$ is the number of plants per square meter ({\it NUMBER
    OF PLANTS M2}) and $\alpha $is the ratio between vegetation height ({\it
    VEGETATION HEIGHT}) and water depth.

  The vegetation is activated through the keyword {\it VEGETATION TAKEN INTO
    ACCOUNT}

  The area can be defines through a list of polygons using a files. 
{\it FILE WITH DEFINITION OF POLYGONS}

\subsection{ Dissipation due to porous media.}

The propagation of wave over a porous medium implies some dissipation. The
dispersion equation for waves that propagate over a dissipative medium derived
by \cite{Losada1996} was expressed in general form by \cite{Mendez2004_2}
\bequ
\label{eq:disipdipersion}
\Gamma-x\tanh(\beta x) - \Phi\tanh(\alpha x) [ x - \Gamma \tanh (\beta x) ] = 0
\eequ
where $x=Kh$ a dimensionless complex wave number, $\Gamma=\sigma^2h/g$ the
relative wave frequency, $\alpha h $ the thickness of the porous layer, $h$
the water depth, $g$ the acceleration due to gravity, $\Phi=\tau/(s-if)$ the
dissipation factor, $s$ the inertial coefficient $s=1+C_a(1-\tau)/\tau$, with
$\tau$ the porosity rate and $C_a$ the damping mass coefficient, $f$ the
linear friction coefficient and $\beta=1-\alpha$.  If $\Gamma$, $\Phi$ and
$\alpha$ are known, it can be shown that the solution of equation
\ref{eq:disipdipersion} gives the sink term of the equations used by
\tomawac under the form.
\bequ
\label{eq:sinkporous}
Q_{porous} = -2 K_{i}\tilde{C}_g F(f,\theta )
\eequ
where $K_{i}$ is the imaginary part of the complex solution of the equation
\ref{eq:disipdipersion} and $\tilde{C}_g$ is the group velocity.

Solving \ref{eq:disipdipersion} is not so easy but in \cite{Chang2006}, the
authors proposed a method based on taylor developpements that solves it. That
is what is implemented in \tomawac

An approximation is done with the calculation of the group velocity in that
case, we calculate it considering that the depth is $\beta h$. which is in
between the velocity for a depth of $h$ and the real velocity of a porous
media which would be difficult to calculate.

The area can be defines through a list of polygons using a files.
{\it FILE WITH DEFINITION OF POLYGONS}

\section{ Surface rollers}

\subsection{Theoretical background}
When waves break, a part of the wave energy is transferred to surface rollers.
These propagate shoreward, thus causing a delay between the point, where the
waves begin to break and the point, where wave setup and longshore currents
develop. In additional the surface rollers transfer mass towards the coast, thus
influencing the return current that occurs. Finally, they can influence
the stirring up of sediment. In the current implementation, only the effect of
surface rollers on the longshore current is taken into account.

The surface rollers can be switched on by setting \textit{ SURFACE ROLLERS} to
\textit{TRUE}. In the implementation, we follow the work by
\cite{roelvink2010xbeach}. According to this reference, the surface roller
energy $E_r$  per unit of volume, integrated over the water column is given by:
\bequ
\label{eq:surfrol}
E_r=\frac{1}{2} A \frac{\overline{U_{roller}^2+W_{roller}^2 }}{L}
\eequ

Here, $L$ is the wave length, $A$ is cross-sectional area of the surface
roller, and $U_{roller}$ and $W_{roller}$ the velocity components in the horizontal
and vertical direction respectively. The evolution of the surface roller
energy is given by the following differential equation:

\bequ
\label{eq:surfrolbal}
\frac{\partial E_r}{\partial t} +\frac{\partial c_x E_r}{\partial x}+
\frac{\partial c_y E_r}{\partial y} = D_w -D_r
\eequ
Here $c_x$ and $c_y$ are the x and y components of the phase velocity of the
waves. Using this formulation, effects of wave current interaction are not taken
into account. $D_w$ is the energy dissipation due to breaking of the waves, and
$D_r$ is the energy dissipation of the surface rollers, which is parametrized
as:
\bequ
\label{eq:surfroldis}
D_r = 2\frac{\beta_s}{\beta_2}\frac{g}{c}E_r
\eequ

Here,  $\beta_s$ and  $\beta_2$  are calibration parameters, and $g$ is the
gravitational acceleration. The parameters are set with the keywords
\textit{BETA S SURFACE ROLLERS} and \textit{BETA 2 SURFACE ROLLERS}.
The effect of the surface rollers of the current comes from its effect on the
radiation  stresses, which are given by:
\bequ
\label{eq:surfrolrsxx}
S_{xx}=S_{xx,waves}+E_r \cos^2 \theta
\eequ
\bequ
\label{eq:surfrolrsxy}
S_{xy}=S_{xy,waves}+E_r \cos \theta \sin \theta
\eequ
\bequ
\label{eq:surfrolrsyy}
S_{yy}=S_{yy,waves}+E_r \sin^2 \theta
\eequ
Here, $S_{xx}$, $S_{xy}$ and $S_{yy}$ are the different components of the
radiation stress tensor, $S_{xx,waves}$, $S_{xy,waves}$ and $S_{yy,waves}$ the
radiation stress components from the surface waves (as calculated in \tomawac),
and $\theta$ the mean wave direction.

In \tomawac, the surface roller energy $E_r$, breaking wave dissipation $D_w$,
and roller wave dissipation $D_r$ can all be exported to an output file by
adding respectively \textit {SRE}, \textit {DBR}, and \textit {DSR} to the
keyword \textit{VARIABLES FOR 2D GRAPHIC PRINTOUTS}.

\subsection{Implementation}
The surface roller energy balance is calculated after each time step in
\tomawac. It is possible to perform multiple substeps for the calculation of
the surface roller energy balance by setting the keyword \textit{NUMBER OF
  SURFACE ROLLER TIME STEPS}. The surface roller energy balance is calculated
using a fractional step method. The advection step is solved first using
the method of characteristics. The velocity field is determined every time
step from the peak period, water depth and mean wave direction calculated by
\tomawac (for the moment ignoring the effect of wave-current interaction).
After the advection step, the source and sink terms are applied. Here, $D_r$
is calculated using an implicit numerical discretisation while $D_w$ uses an
explicit discretisation. $D_w$ is exported directly from the calculation
of the depth induced breaking source term in \tomawac.

