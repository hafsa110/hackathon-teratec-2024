\chapter{Defining friction by domains}
\label{tel2d:app5}
When a complex definition of the friction has to be used for a computation, this
option can be chosen, which divides the domain in sub-domains (domains of friction)
where different parameters of friction can be defined and easily modified.
The procedure is triggered by the keyword \telkey{FRICTION DATA} = YES
(default = NO) and the data are contained in a file \telkey{FRICTION DATA FILE}.

The user has to:

\begin{itemize}
\item  define the domains of friction in the mesh,

\item  define the parameters of friction for each domain of friction,

\item  add the corresponding keywords in the steering file of \telemac{2D}
in order to use this option.
\end{itemize}

\textbf{I -- Friction domains}

In order to make a computation with variable coefficients of friction,
the user has to describe, in the computational domain, the zones where the
friction parameters will be the same.
For that, a friction ID (code number), which represents a friction domain, has to be given to
each node.
The nodes with the same friction ID will use the same friction parameters.

This allocation is done thanks to the \telfile{FRICTION\_USER} user subroutine.
All nodes can be defined ''manually'' in this subroutine, or this subroutine can
be used in order to read a file where the link between nodes and friction IDs is
already generated (for example with the Janet software from the SmileConsult).
This file is called \telkey{ZONES FILE} and will be partitioned in case of
parallelism.

\textbf{II -- Friction parameters}

The frictions parameters of each friction domain are defined in the
\telkey{FRICTION DATA FILE}.
In this file we find, for each friction ID of friction domain:

\begin{itemize}
\item  a friction law for the bottom and their parameters,

\item  a friction law for the vegetation and the parameters for vegetation (only if the option is used).
\end{itemize}

Example of friction data file:



\begin{tabular}{|p{0.4in}|p{0.5in}|p{0.3in}|p{0.5in}|p{0.6in}|p{0.5in}|p{0.5in}|p{0.6in}|p{0.6in}|} \hline
*Zone & Bottom &  &  & Vegetation &  &  &   \\ \hline
*no & TypeBo & Rbo & MdefBo & TypeBo & Par1 & Par2 & Par3 ... - Par15  \\ \hline
From 4 to 6 & NFRO &  &  & NULL & &  &   \\ \hline
20 & NIKU & 0.10 &  & BAPT & 1.0 & 0.04 & 10.0  \\ \hline
27 & COWH & 0.13 & 0.02 & LIND & 1.0 &  5.0 & \\ \hline
END &  &  &  &  &  &  &   \\ \hline
\end{tabular}


\begin{itemize}
\item The first column defines the friction ID of the friction domain.
Here, there are 3 lines with the friction IDs: 4 to 6, 20, 27,

\item The columns from 2 to 4 are used in order to define the bottom law:
the name of the law used (\telkey{NFRO}, \telkey{NIKU} or \telkey{COWH} for this
example, see below for the name of the laws),
the roughness parameter used and the Manning's default value
(used only with the Colebrook-White law).
If the friction parameter (when there is no friction) or the Manning's default
are useless, nothing has to be written in the column,

\item The column 5 describes the law of the vegetation friction:
in the example it is \telkey{NULL}, \telkey{BAPT} or \telkey{LIND}. 
The columns 
6 to max 20 should contain the parameters dedicated to the chosen vegetation 
law in case of no vegetation (\telkey{NULL}) the columns should be empty. 


\item The last line of the file must have only the word END, (or FIN or ENDE).
\end{itemize}

In order to add a comment in the \telkey{FRICTION DATA FILE},
the line must begin with a star ''*''.

Link between the bottom friction laws implemented and their names in the friction data file:

\begin{tabular}{|p{1.0in}|p{0.5in}|p{0.8in}|p{1.0in}|} \hline
Law &  Number & Name for data file & Parameters used \\ \hline
No Friction & 0 & NOFR & No parameter \\ \hline
Haaland & 1 & HAAL & Roughness coefficient \\ \hline
Ch\'{e}zy & 2 & CHEZ & Roughness coefficient \\ \hline
Strickler & 3 & STRI & Roughness coefficient \\ \hline
Manning & 4 & MANN & Roughness coefficient \\ \hline
Nikuradse & 5 & NIKU & Roughness coefficient \\ \hline
Colebrook-White & 7 & COWH & Roughness coefficient\newline Manning coefficient \\ \hline
\end{tabular}



Link between the vegetation friction laws implemented and their names in the friction data file:

\begin{tabular}{|p{1.0in}|p{0.5in}|p{0.8in}|p{1.0in}|} \hline
 Law &  Number & Name for data file & Parameters used \\ \hline
No Vegetation Friction & 0 & NULL & No parameter \\ \hline
Jaervelae & 1 & JAER & Cdx, LAI, Uref, Vogel, Hp \\ \hline
Lindner \& Pasche& 2 & LIND & D, sp \\ \hline
Whittaker et al & 3 & WHIT & Cd0, Ap0, EI, Vogel, sp, Hp \\ \hline
Baptist et al & 4 & BAPT & Cd, mD, Hp \\ \hline
Huthoff et al & 5 & HUTH & Cd, mD, Hp, sp \\ \hline
Van Velzen et al & 6 & VANV & Cd, mD, Hp \\ \hline
Luhar \& Nepf & 7 & LUHE & Cd, Cv, a, Hp \\ \hline
Vastila \& Jaervelae & 8 & VAST & Cdf, LAI, Ureff, Vogelf, CDs,  SAI, Vogels, Urefs, Hp \\ \hline
Folke et al & 1 & HYBR & Cdx, LAI, Uref, Vogel, Hp \\ \hline
\end{tabular}

Explanation of the used parameters: 

\begin{tabular}{p{0.5in}p{5.0in}} 
Cd: & Vegetation bulk drag coefficient (Baptist)\\
Cdx: & Vegetation drag coefficient specie specific (Jaervelae)\\
Cd0: & Vegetation initial drag coefficient (WHittaker)\\
Cdf: & Foliage drag coefficient \\
Cds: & Stem drag coefficient \\
Lai: & Leaf area index \\
SAI: & Stem area index \\
Uref: & Reference velocity (lowest velocity used to determine the Vogel exponent \\
Ureff: & Foliage reference velocity \\
Urefs: & Stem reference velocity \\
Vogel: & Vogel exponent \\
Vogelf: & Foliage Vogel exponent \\
Vogels: & Stem Vogel exponent \\
Hp: & Plant height \\
D: & Vegetation diameter \\
sp: & Vegetation spacing  \\
NOTE: Huthoff et al defines sp as spacing between two trees (bark to bark)\\
Ap0: & Initial projected area \\
EI: & Flexural rigidity \\
mD: & m: vegetation density (1/sp**2) D: vegetation diameter 0.5*LAI/Hp (Finnigan 2000) \\
Cv: & Friction coefficient on top of the vegetation layer, for emerged case Cv=0\\
a: & Frontal area of vegetation per volume (=mD) \\
\end{tabular}




\textbf{III -- Steering file}

In order to use a friction computation by domains, the next keywords have to be
added:

For the friction data file:

\telkey{FRICTION DATA} = YES.

\telkey{FRICTION DATA FILE} = `name of the file where friction is given'.

For the vegetation friction (if used):

\telkey{VEGETATION FRICTION} = YES.

By default, 10 zones are allocated, this number can be changed with the keyword:

\telkey{MAXIMUM NUMBER OF FRICTION DOMAINS} = 80.

Link between nodes and friction IDs of friction domains is achieved with:

\telkey{ZONES FILE} = `name of the file'
or \newline
the friction IDs can be given as variable
\telkey{FRIC\_ID} in the geometry file.

\textbf{IV -- Advanced options}

If some friction domains with identical parameters have to be defined,
it is possible to define them only with one line thanks to the keyword:
from... to... (it is also possible to use French de... a...
or German von... bis...).

The first friction ID of the domains and the last friction ID of the domains
have to be set.
All domains of friction with a friction ID between these two values will be
allocated with the same parameters, except:
\begin{itemize}
\item If a friction domain is defined in two different groups,
the priority is given to the last group defined,

\item A single friction domain has ever the priority on a group even
if a group with this domain is defined afterwards,

\item If a single friction domain is defined twice, the priority is given to the
last definition.
\end{itemize}

\textbf{V -- Programming}

A new module, \telfile{FRICTION\_DEF}, has been created in order to save the
data read in the friction file.
This module is built on the structure of the \telfile{BIEF} objects.
The domain of friction ''I'' is used as follows:
\begin{lstlisting}[language=TelFortran]
 TYPE(FRICTION_DEF) :: TEST_FRICTION

 TEST_FRICTION%ADR(I)%P
\end{lstlisting}
 The components of the structure are:



\begin{tabular}{|p{3.in}|p{2.5in}|} \hline
%\telfile{TEST\_FRICTION\%ADR(I)\%P\%GNUM(1)} & 1$^{\rm{st}}$ friction ID of the friction domains \\ \hline
%\telfile{TEST\_FRICTION\%ADR(I)\%P\%GNUM(2)} & Last friction ID of the friction domains \\ \hline
%\telfile{TEST\_FRICTION\%ADR(I)\%P\%RTYPE} & Friction law used for the bottom \\ \hline
%\telfile{TEST\_FRICTION\%ADR(I)\%P\%RCOEF} & Roughness parameters for the bottom \\ \hline
%\telfile{TEST\_FRICTION\%ADR(I)\%P\%NDEF} & Default Manning for the bottom \\ \hline
%\telfile{TEST\_FRICTION\%ADR(I)\%P\%VCOEF(1 ... 15)} & 1 to max. 15  parameters for the friction laws for vegetation \\ \hline
\end{tabular}


\telfile{TEST\_FRICTION\%ADR(I)\%P\%GNUM(1)} and \telfile{TEST\_FRICTION\%ADR(I)\%P\%GNUM(2)}
have the same value if a single friction domain is defined.

%\telfile{TEST\_FRICTION\%ADR(I)\%P\%RTYPE} is \telfile{KFROT} when there is only one domain.

%\telfile{TEST\_FRICTION\%ADR(I)\%P\%RCOEF} is \telfile{CHESTR} when there is only one domain.



The link between \telemac{2D} and the computation of the friction is done
with the \telfile{FRICTION\_CHOICE} subroutine.
It is used in order to initialize the variables for the option
\telkey{FRICTION DATA} at the beginning of the program and/or in order to call
the right friction subroutine for the computation at each iteration.

\textbf{\underbar{Initializing:}}

During the initialization, the parameters of the friction domains are saved
thanks to the \telfile{FRICTION\_READ} subroutine and the friction ID of each nodes
are saved thanks to \telfile{FRICTION\_USER} in the array \telfile{KFROPT\%I}.
With the subroutine \telfile{FRICTION\_INIT}, the friction IDs for all nodes are
checked and the arrays \telfile{CHESTR\%R} and \telfile{NKFROT\%I}
(\telfile{KFROT} for each node) are built.
\telfile{KFROT} is used in order to know if all friction parameters are null or
not.
This information is used during the computation.

\textbf{\underbar{Computing:}}

For the optimization, the computation of the friction coefficient is done
in the \telfile{FRICTION\_CALC} subroutine for each node thanks to the loop
\telfile{I = N\_START, N\_END}.
When the option \telkey{FRICTION DATA} is not used, \telfile{N\_START} and
\telfile{N\_END} are initialized to 1 and \telfile{NPOIN} in the subroutine
\telfile{FRICTION\_UNIF}.
Else, they take the same value and the loop on the node is done in the
\telfile{FRICTION\_ZONES} subroutine (the parameters used for each node can be
different).

The choice of \telkey{VEGETATION FRICTION} is only possible together with
 \telkey{FRICTION DATA}. Then one of the  routines 
\telfile{FRICTION\_BAPTIST}, \telfile{FRICTION\_HUTHOFF}, \telfile{FRICTION\_JAERVELAE}, 
\telfile{FRICTION\_LINDNER}, \telfile{FRICTION\_LUHARNEPF}, 
\telfile{FRICTION\_VANVELZEN}, 
\telfile{FRICTION\_VASTILA}, \telfile{FRICTION\_WHITTAKER},
\telfile{FRICTION\_HYBRID}
calculates the additional roughness due to vegetation. 

\textbf{VI -- Accuracy}

When the option \telkey{FRICTION DATA} is not used, \telfile{CHESTR} can be read
in the \telkey{GEOMETRY FILE}.
The values stored in this file can be in single precision if 
\telkey{GEOMETRY FILE FORMAT} is not set to SERAFIND.
However \telfile{CHESTR} is defined in double precision, then,
the \telfile{CHESTR} value is not exactly the right value.

With the option \telkey{FRICTION DATA}, \telfile{CHESTR} is set
thanks to the \telkey{FRICTION DATA FILE} where the value of each domains
are stored in double precision.

Then when a comparison is done between both methods, the difference may come
from the difference between single and double precision.
