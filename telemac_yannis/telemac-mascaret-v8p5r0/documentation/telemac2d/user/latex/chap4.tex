\chapter{Hydrodynamic simulation}
\label{ch:hydrod:sim}

\section{Prescribing initial conditions}

The purpose of the initial conditions is to describe the state of the model
at the start of the simulation.

In the case of a continued computation, this state is provided by the last time
step of the results file of the previous computation.
The tables of variables that are essential for continuing the computation
must therefore be stored in a file used for this purpose.
This case is described in section \ref{subs:cont:comput}.

In other cases, the initial state must be defined by the user.
In simple cases, this can be done using keywords, or by programming in more
complex ones.
It is also possible to define the initial state using FUDAA-PREPRO.

% Option OUTPUT OF INITIAL CONDITIONS deleted in V7P1
% If the user wants to store the initial state in the results file,
%the keyword \telkey{OUTPUT OF INITIAL CONDITIONS} must be activated (default value).

\subsection{Prescribing using keywords}

In all cases, the kind of the initial conditions is set by the keyword
\telkey{INITIAL CONDITIONS}, except if it has been defined using
FUDAA-PREPRO INITIAL CONDITIONS.
The keyword \telkey{INITIAL CONDITIONS} may have any of the following six values:

\begin{itemize}
\item 'ZERO ELEVATION': This initializes the free surface elevation at 0
(default value).
The initial water depths are therefore calculated from the bottom elevation,

\item 'CONSTANT ELEVATION': This initializes the free surface elevation
at the value supplied by the keyword \telkey{INITIAL ELEVATION}
(default value = 0.).
The initial water depths are then calculated by subtracting
the bottom elevation from the free surface elevation.
In areas where the bottom elevation is higher than the initial elevation,
the initial water depth is zero,

\item 'ZERO DEPTH': All water depths are initialized with a zero value
(free surface same as bottom).
In other words, the entire domain is dry at the start of the computation,

\item 'CONSTANT DEPTH': This initializes the water depths at the value
supplied by the keyword \telkey{INITIAL DEPTH} (default value = 0.),

\item `TPXO SATELLITE ALTIMETRY': The initial conditions are set using
information provided by the OSU harmonic constants database (TPXO for instance)
or HAMTIDE model
in the case of the use of this database for the imposition of maritime boundary
conditions (see subsection \ref{subs:tidal:harm:datab}),

\item 'PARTICULAR' or 'SPECIAL': The initial conditions are defined in the
\telfile{USER\_CONDIN\_H} subroutine (see section \ref{subs:presc:CONDIN}).
This solution must be used whenever the initial conditions of the model
do not correspond to one of the five cases above.
\end{itemize}


\subsection{Prescribing particular initial conditions with user subroutines}
% USER\_CONDIN... subroutines
\label{subs:presc:CONDIN}

The \telfile{USER\_CONDIN} subroutine must be programmed whenever the keyword
\telkey{INITIAL CONDITIONS} has the value 'PARTICULAR' or 'SPECIAL'.

The \telfile{CONDIN} subroutine initializes successively the velocities,
the water depth, the tracer and the viscosity.
The part of the subroutine concerning the initialization of the water depth
is divided into two zones.
The first corresponds to the processing of simple initial conditions
(defined by keywords)
and the second regards the processing of particular initial conditions
with the \telfile{USER\_CONDIN\_H} subroutine.

By default, the standard version of the \telfile{USER\_CONDIN\_H} subroutine
stops the computation if the keyword \telkey{INITIAL CONDITIONS} is set
at 'PARTICULAR' or 'SPECIAL' without the subroutine being actually modified.

The user is entirely free to fill this subroutine.
For example, he can re-read information in a formatted or binary file
using the keywords \telkey{FORMATTED DATA FILE} or \telkey{BINARY DATA FILE}
offered by \telemac{2D}.

When the \telfile{USER\_CONDIN\_H} subroutine is being used,
it may be interesting to check that the variables are correctly initialised.
To do this, it is simply a question of assigning the name of the variables
to be checked to the keyword \telkey{VARIABLES TO BE PRINTED},
and starting the computation with a zero number of time steps.
The user then obtains the value of the variables required
at each point of the mesh in the listing printout.

The user may also change initial velocities with the help of the
\telfile{USER\_CONDIN\_UV} subroutine in addition to or without
any treatment for initial water depth.

\subsection{Continuing a computation}
\label{subs:cont:comput}
\telemac{2D} enables the user to resume a computation taking a time step
of a previous computation on the same mesh as initial state.
It is thus possible to modify the computation data, such as, for example,
the time step, some boundary conditions or the turbulence model,
or to start the computation once a steady state has been reached.

By default, \telemac{2D} reads the last time step of the previous computation
result file.
Using the keyword \telkey{RECORD NUMBER FOR RESTART} allows specifying
the number of the iteration to be read.

Note that FUDAA-PREPRO is using this possibility:
when defining the initial conditions, information is actually written
in a pseudo continuation file.

In this case, it is essential that the continuation file contains
all the information required by \telemac{2D}, i.e. the velocities $U$ and $V$,
the water depth and the bottom elevations.
However, in some cases, the software is capable of recomputing some
variables from others provided
(for example the depth of water from the free surface and the bottom elevation).

If some variables are missing from the continuation file,
they are then fixed automatically at zero.
However, it is possible, in this case, to provide initial values in a standard
way (e.g. using a keyword).
A frequent application is to use the result of a hydrodynamic computation
to compute the transport of a tracer.
The continuation file does not normally contain any result for the tracer.
However, it is possible to provide the initial value for this by using
the keyword \telkey{INITIAL VALUES OF TRACERS}.

In order to use the continuation file, it is necessary to enter
two keywords in the steering file:

\begin{itemize}
\item The keyword \telkey{COMPUTATION CONTINUED} must have the value YES
(default value = NO),

\item The keyword \telkey{PREVIOUS COMPUTATION FILE} must provide
the name of the file that will supply the initial state.
\end{itemize}

N.B.: the mesh for which the results are computed must be exactly the same
as the one to be used in continuing the computation.

If necessary, the keyword \telkey{PREVIOUS COMPUTATION FILE FORMAT}
can be used to select a specific format.
For example, in order to increase the accuracy of the initial state,
it is possible to use double precision SERAFIN format ('SERAFIND') or
MED format ('MED').
Obviously, this configuration is possible only if the previous computation
was correctly configured in terms of results file format.
\\

Resuming the computation usually leads to small differences in results
compared to the same calculation without interruption.
In \telemac{2D}, the differences are mainly due to the fact that
the increment of water depth $\Delta H$ is not known correctly at
the first time step if \telkey{INITIAL GUESS FOR H} = 1 (default value) or 2,
because this operation requires information from the
previous time step.
If \telkey{INITIAL GUESS FOR U} = 2 (default value = 1), the increments of
the 2 components of velocity $\Delta U$ and $\Delta V$ are also the key issue.
To correct this, the user has a specific recovery procedure
to improve the accuracy of calculations, using double precision format SERAFIN
files (or MED format):

\begin{itemize}
\item In the first computation, the keyword \telkey{RESTART MODE} is set to
YES (default = NO),
which generates a specific file containing the full information at one or a
few time step(s) of the simulation.
The name of this file is given by the keyword \telkey{RESTART FILE}.

\item In the second computation, this specific file must be used as
\telkey{PREVIOUS COMPUTATION FILE} specifying the \telkey{PREVIOUS COMPUTATION
FILE FORMAT} is 'SERAFIND' (SERAFIN double precision).
If the restart should be done from the last time step saved in the
\telkey{RESTART FILE}, the keyword \telkey{RECORD NUMBER FOR RESTART} is to be
let to default value (= -1), otherwise it should be changed.
\end{itemize}

In the first computation, there are 2 keywords which can enable to tune the
resuming of the computation:
\begin{itemize}
\item If wanting to generate a \telkey{RESTART FILE} at a specific number of
time step different from the last one, the keyword
\telkey{RECORD NUMBER IN RESTART FILE} is to be used
(default = -1 means the \telkey{RESTART FILE} is only written at the last
time step or periodically at the period \telkey{RESTART FILE PRINTOUT PERIOD}),
\item If wanting to generate a \telkey{RESTART FILE} periodically to secure a
file to resume computation in case of crash e.g., the keyword
\telkey{RESTART FILE PRINTOUT PERIOD} defines the printout period in number of
time steps.
Default = 0 means no periodic writing and variables are only written at the last
time step of the computation or at the time step number
\telkey{RECORD NUMBER IN RESTART FILE} if not equal to -1).
\end{itemize}

However, it has to be mentioned that even if it is not advisable, the creation
of specific restart file can be done not only SERAFIND format, but also with
any other available format in the TELEMAC system, especially in single
precision. In this case, the keyword \telkey{RESTART FILE FORMAT} (by default
set at 'SERAFIND') must be set to the proper value.
\\

When continuing a computation, it is necessary to specify the value of
the start time of the second computation.
By default, the initial time of the second computation is equal to the value
of the time step in the previous computation file used for continuation.
This can be modified using the keyword \telkey{INITIAL TIME SET TO ZERO}
(default = NO)
if the user wants to reset the time value (possibly with respect to a basic
value set in the preceding calculation.
See Chapter \ref{ch:gen:par:def:comp}).

When resuming a computation, if wanting to find maximum elevation and its
associated time or maximum velocity and its associated time (with MAXZ, TMXZ,
MAXC, TMXV for \telkey{VARIABLES FOR GRAPHIC PRINTOUTS}), the user can set
the keyword \telkey{USE MAXIMUM VALUES FROM PREVIOUS COMPUTATION FILE} with
2 options:
\begin{itemize}
\item YES (= default): compute these maximum values from the
\telkey{PREVIOUS COMPUTATION FILE} and current computation,
\item NO: compute these maximum values only from the current computation
(not taking into account the \telkey{PREVIOUS COMPUTATION FILE}).
\end{itemize}

At the beginning of a simulation, the launcher creates a temporary directory
where all input files are copied.
This is also the case for the previous computation file which can be quite huge.
In this situation and to avoid copying too large a file,
it is recommended to extract the time step used for the continuation
(the only one used by \telemac{2D}).


\section{Prescribing boundary conditions}
\label{sec:presc:bc}

The maximum number of boundaries is set to 30 by default but it can be changed
by the user with the keyword \telkey{MAXIMUM NUMBER OF BOUNDARIES}.
This avoids changing the previously hardcoded values (until release 7.0),
which required recompiling the whole package.

\subsection{Possible choices}

Boundary conditions are given for each of the boundary points.
They concern the main variables of \telemac{2D} or the values deduced from them:
water depth, the two components of velocity (or flowrate) and the tracer.
The boundary conditions of functions $k$ and $\epsilon$ in the turbulence model
are determined by \telemac{2D} and are thus not required from the user.
Turbulence specialists may want to change the boundary conditions of $k$
and $\epsilon$ in \telfile{KEPSCL} subroutine.

The various types of boundary conditions may be combined to prescribe boundary
conditions of any type (inflow or outflow of liquid in a supercritical or
subcritical regime, open sea, wall, etc.).
However, some combinations are not physical.

Some boundary conditions apply to segments, such as friction at the walls,
no flux condition or incident wave conditions.
However, wall definition is ambiguous if boundary conditions are to be defined
by points.
The following convention is used in such cases to determine the nature of a
segment located between two points of different type.
A liquid segment is one between two points of liquid type.
In a similar way, when a condition is being prescribed for a segment,
the point must be configured at the start of the segment.

The way in which a boundary condition is prescribed depends on the spatial
and temporal variations in the condition.
Five types of condition may be distinguished:

\begin{itemize}
\item The condition is constant at the boundary and constant in time.
The simplest solution is then to prescribe the condition by means of a keyword
in the steering file,

\item The condition is constant at the boundary and variable in time.
It will then be prescribed by programming the subroutines \telfile{USER\_Q},
\telfile{USER\_SL} and \telfile{USER\_VIT} (and \telfile{TR}
if a tracer is used) or by the open boundaries file,

\item The condition is variable in space and constant in time.
It will then be prescribed via the \telkey{BOUNDARY CONDITIONS FILE}.
In some cases, the velocity profile can be specified using the keyword
\telkey{VELOCITY PROFILES} (see section \ref{subs:presc:vel:prof}),

\item The condition is variable in time and space.
Direct programming via the \telfile{USER\_BORD} subroutine is then necessary,

\item The boundary condition type is variable in time.
Direct programming in the \telfile{PROPIN\_TELEMAC2D} subroutine is then
necessary (see \ref{sec:chang:type:bc:propin}).
\end{itemize}

The type of boundary condition, if constant in time, is read from
\telkey{BOUNDARY CONDITIONS FILE}.
In contrast, the prescribed value (if one exists) may be given at four different
levels, namely (in the order in which they are processed during the computation)
the boundary conditions file (not frequently used), the steering file,
the open boundaries file and the FORTRAN file (programming of subroutines
\telfile{USER\_Q}, \telfile{USER\_SL}, \telfile{USER\_VIT}, \telfile{TR}
or \telfile{USER\_BORD}).

Boundary types may be connected in any way along a contour.
However, two liquid boundaries must be separated by at least a solid segment
(for example, there cannot be an open boundary with a prescribed depth directly
followed by an open boundary with a prescribed velocity).
Moreover, another limitation is that a boundary must consist of at least two
points (a minimum of four points is strongly advised).


\subsection{Description of various types of boundary conditions}
\label{subs:desc:bc}
The type of boundary condition at a given point is provided in the
boundary conditions file in the form of four integers named
\telfile{LIHBOR}, \telfile{LIUBOR}, \telfile{LIVBOR} and \telfile{LITBOR},
which may have any value from 0 to 6.

The possible choices are as follows:

\begin{itemize}
\item Depth conditions:

\begin{itemize}

\item Open boundary with prescribed depth: \telfile{LIHBOR} = 5,

\item Open boundary with free depth: \telfile{LIHBOR} = 4,

\item Closed boundary (wall): \telfile{LIHBOR} = 2,
\end{itemize}

\item Flowrate or velocity condition:

\begin{itemize}

\item Open boundary with prescribed flowrate: \telfile{LIUBOR}/\telfile{LIVBOR}
= 5,

\item Open boundary with prescribed velocity: \telfile{LIUBOR}/\telfile{LIVBOR}
= 6,

\item Open boundary with free velocity: \telfile{LIUBOR}/\telfile{LIVBOR} = 4,

\item Closed boundary with slip or friction: \telfile{LIUBOR}/\telfile{LIVBOR}
= 2,

\item Closed boundary with one or two null velocity components: \telfile{LIUBOR}
and/or \telfile{LIVBOR} = 0,
\end{itemize}

\item Tracer conditions:

\begin{itemize}

\item Open boundary with prescribed tracer: \telfile{LITBOR} = 5,

\item Open boundary with free tracer: \telfile{LITBOR} = 4,

\item Closed boundary (wall): \telfile{LITBOR} = 2.
\end{itemize}

\end{itemize}
 \underbar{Remarks:}

\begin{itemize}
\item It is possible to change the type of boundary condition within an open
boundary.
In that case, a new open boundary will be detected in the output control listing,

\item The type of boundary condition during the simulation may be modified with
the \telfile{PROPIN\_TELEMAC2D} subroutine (see \ref{sec:chang:type:bc:propin}).
\end{itemize}


\subsection{The boundary conditions file}
\label{sub:bc:file}
The file is normally supplied by most of the mesh generators compatible
with \tel, or \stbtel
%FUDAA-PREPRO, Blue Kenue or \stbtel,
but may be created and modified using FUDAA-PREPRO or a text editor.
Each line of this file is dedicated to one point of the mesh boundary.
The numbering of the boundary points is the same as that of the lines of the
file.
It describes first of all the contour of the domain in a trigonometric direction,
and then the islands in the opposite direction.

This file specifies a numbering of the boundaries.
This numbering is very important because it is used when prescribing values.

The following values are given for each point
(see also the section dedicated to parallel processing for some specific aspects):

\telfile{LIHBOR, LIUBOR, LIVBOR, HBOR, UBOR, VBOR, AUBOR, LITBOR, TBOR, ATBOR, BTBOR, N, K}.

\begin{itemize}
\item \telfile{LIHBOR, LIUBOR, LIVBOR} and \telfile{LITBOR} are
the boundary type integers for each of the variables.
They are described in section \ref{subs:desc:bc},

\item \telfile{HBOR} (real) represents the prescribed depth if \telfile{LIHBOR}
= 5,

\item \telfile{UBOR} (real) represents the prescribed velocity $U$
if \telfile{LIUBOR} = 6,

\item \telfile{VBOR} (real) represents the prescribed velocity $V$
if \telfile{LIVBOR} = 6,

\item \telfile{AUBOR} represents the friction coefficient at the boundary
if \telfile{LIUBOR} or \telfile{LIVBOR} = 2.
\end{itemize}
The friction law is then written as follows:
\[\nu _{t} \frac{dU}{dn} {\kern 1pt} {\kern 1pt} {\kern 1pt} ={\kern 1pt} {\kern 1pt} {\kern 1pt} AUBOR{\kern 1pt} {\kern 1pt} {\kern 1pt} {\kern 1pt} \times{\kern 1pt} {\kern 1pt} U{\kern 1pt} {\kern 1pt} {\kern 1pt} {\kern 1pt} {\kern 1pt} {\kern 1pt} {\kern 1pt} {\kern 1pt} {\kern 1pt} {\kern 1pt} {\kern 1pt} {\kern 1pt} {\kern 1pt} {\rm ~~~~and/or~~~~ } {\kern 1pt} {\kern 1pt} {\kern 1pt} {\kern 1pt} {\kern 1pt} {\kern 1pt} {\kern 1pt} \nu _{t} {\kern 1pt} {\kern 1pt} \frac{dV}{dn} {\kern 1pt} {\kern 1pt} {\kern 1pt} ={\kern 1pt} {\kern 1pt} {\kern 1pt} AUBOR{\kern 1pt} {\kern 1pt} {\kern 1pt} \times {\kern 1pt} {\kern 1pt} {\kern 1pt} V\]
The \telfile{AUBOR} coefficient applies to the segment included between
the boundary point considered and the following point
(in the counter clockwise direction for the outside outline
and in the clockwise direction for the islands).
The default value is \telfile{AUBOR} = 0.
Friction corresponds to a negative value.
With the $k$-$\epsilon$ model, the value of \telfile{AUBOR} is computed by
\telemac{2D} and the indications in the boundary conditions file are then
ignored.

\begin{itemize}
\item \telfile{TBOR} (real) represents the prescribed value of the tracer when
\telfile{LITBOR} = 5.

\item \telfile{ATBOR} and \telfile{BTBOR} represent the coefficients of the
flux law which is written as:
\end{itemize}
\[\nu _{t} \frac{dT}{dn} {\kern 1pt} {\kern 1pt} {\kern 1pt} {\kern 1pt} {\kern 1pt} ={\kern 1pt} {\kern 1pt} {\kern 1pt} {\kern 1pt} {\kern 1pt} ATBOR{\kern 1pt} {\kern 1pt} {\kern 1pt} {\kern 1pt} \times {\kern 1pt} {\kern 1pt} T{\kern 1pt} {\kern 1pt} {\kern 1pt} {\kern 1pt} {\kern 1pt} {\kern 1pt} {\kern 1pt} +{\kern 1pt} {\kern 1pt} {\kern 1pt} {\kern 1pt} {\kern 1pt} BTBOR\]
The \telfile{ATBOR} and \telfile{BTBOR} coefficients apply to the segment
between the boundary point considered and the next point
(in the counter clockwise direction for the outside outline and in the
clockwise direction for the islands).

\begin{itemize}
\item \telfile{N} represents the global number of boundary points.

\item \telfile{K} represents initially the point number in the boundary point
numbering.
But this number can also represent a node colour modified manually by the user
(it can be any integer).
This number, called \telfile{BOUNDARY\_COLOUR}, can be used in parallelism
to simplify implementation of specific cases.
Without any manual modification, this variable represents the global boundary
node number.
For example a test like:

\telfile{IF (I.EQ.144) THEN} can be replaced by
\telfile{IF(BOUNDARY\_COLOUR\%I(I).EQ.144) THEN}
which is compatible with parallel mode.
Be careful not to modify the last column of the boundary conditions file
that contains this \telfile{BOUNDARY\_COLOUR} table,
when using tidal harmonic constants databases (cf. \cite{Pham2012}).
\end{itemize}

\subsection{Prescribing values through keywords}
\label{subs:val:key}
In most simple cases, boundary conditions are prescribed using keywords.
However, if the values to be prescribed vary in time, it is necessary to program
the appropriate functions or use the open boundaries file
(see \ref{subs:val:funct:bf}).

The keywords used for prescribing boundary conditions are the following:

\begin{itemize}
\item \telkey{PRESCRIBED ELEVATIONS}:
This is used to define the elevation of an open boundary with prescribed
elevation (free surface).
It is an array that can contain up to \telfile{MAXFRO}
(set to 300 and can be changed by user) real numbers for managing up to
\telfile{MAXFRO} boundaries of this type.
The values defined by that keyword overwrite the depth values read
from the \telkey{BOUNDARY CONDITIONS FILE}.
\end{itemize}

N.B.: the value given here is the free surface level, whereas the value given
in the \telkey{BOUNDARY CONDITIONS FILE} is the water depth.

\begin{itemize}
\item \telkey{PRESCRIBED FLOWRATES}:
This is used to set the flowrate value of an open boundary with prescribed
flowrate.
It is an array that contains up to \telfile{MAXFRO} real numbers for managing up
to \telfile{MAXFRO} boundaries of this type.
A positive value corresponds to an inflow into the domain, whereas a negative
value corresponds to an outflow.
The values provided with this keyword overwrite the flowrate values read
from the \telkey{BOUNDARY CONDITIONS FILE}.
In this case, the technique used by \telemac{2D} to compute the velocity profile
is that described in section \ref{subs:presc:vel:prof}.

\item \telkey{PRESCRIBED VELOCITIES}:
This is used to set the velocity value of an open boundary with prescribed
velocity.
The scalar value provided is the intensity of the velocity perpendicular
to the wall.
A positive value corresponds to an inflow into the domain.
It is an array that contains up to \telfile{MAXFRO} real numbers for managing up
to \telfile{MAXFRO} boundaries of this type.
The values provided with this keyword overwrite the values read from the
\telkey{BOUNDARY CONDITIONS FILE}.
\end{itemize}

Some simple rules must also be complied with:

\begin{itemize}
\item There must of course be agreement between the type of boundary specified
in the \telkey{BOUNDARY CONDITIONS FILE} and the keywords of the steering file
(do not use the keyword \telkey{PRESCRIBED FLOWRATES} if there are no boundary
points with the \telfile{LIUBOR} and \telfile{LIVBOR} values set at 5),

\item If a boundary type is defined in the \telkey{BOUNDARY CONDITIONS FILE},
the corresponding keyword must be defined in the steering file,

\item The keywords \telkey{PRESCRIBED {\dots}}, if present,
supersede the data read in the \telkey{BOUNDARY CONDITIONS FILE},

\item For each keyword, the number of specified values must be equal
to the total number of open boundaries.
If a boundary does not correspond to the specified keyword, the value will be
ignored (for example, the user can specify 0.0 in all cases).
In the examples in the introductory manual, the first boundary (downstream)
is with prescribed elevation, and the second one (upstream) is with prescribed
flowrate.
It is therefore necessary to specify in the steering file:
\end{itemize}
\begin{lstlisting}[language=bash]
   PRESCRIBED ELEVATIONS = 265.0 ; 0.0
   PRESCRIBED FLOWRATES = 0.0 ; 500.0
\end{lstlisting}

\subsection{Prescribing values by programming subroutines or using the open boundaries file}
\label{subs:val:funct:bf}
Values that vary in time but are constant along the open boundary in question
are prescribed by using the open boundaries file or by programming a particular
subroutine, which may be:

\begin{itemize}
\item Subroutine \telfile{USER\_VIT} to prescribe a velocity,

\item Subroutine \telfile{USER\_Q} to prescribe a flowrate,

\item Subroutine \telfile{USER\_SL} to prescribe an elevation,

\item Function \telfile{TR} to prescribe a tracer concentration
(see Chapter \ref{ch:tra:trans})
\end{itemize}

Subroutines \telfile{USER\_Q}, \telfile{USER\_VIT} and \telfile{USER\_SL} are
programmed in the same way.
In each case, the user has the time, the boundary rank (for determining,
for example, whether the first or second boundary with a prescribed flowrate
is being processed),
the global number of the boundary point (useful in case of parallel computing)
and in the case of \telfile{USER\_Q}, information on the depth of water
at the previous time step.
By default the functions prescribe the value read from
the \telkey{BOUNDARY CONDITIONS FILE} or supplied by keywords.

For example, the body of subroutine \telfile{USER\_Q} for prescribing a flowrate
ramp lasting 1,000~seconds and reaching a value of 400 m$^3$/s could take
a form similar to:
\begin{lstlisting}[language=TelFortran]
 IF (AT.LT.1000.D0) THEN
    Q = 400.D0 * AT/1000.D0
 ELSE
    Q = 400.D0
 ENDIF
\end{lstlisting}
Using the liquid boundaries file is an alternative to programming the subroutines
mentioned above.
This is an ASCII file edited by the user, the name of which is given with the
keyword \telkey{LIQUID BOUNDARIES FILE}.
This file has the following format:

\begin{itemize}
\item A line beginning with the sign \# is a line of comments,

\item It must contain a line beginning with \telfile{T} (\telfile{T} meaning
time) to identify the value provided in this file.
Identification is by a mnemonic identical to the name of the variables:
\telfile{Q} for flow rate, \telfile{SL} for water level, \telfile{VIT}
for velocity (giving the magnitude) and \telfile{TR} for tracer.
An integer between brackets specifies the rank of the boundary in question.
In the case of tracers, the identification, uses a 2-index mnemonic
\telfile{TR}(b,t) with b providing the rank of the boundary and t the number of
the tracer.
This line is followed by another indicating the unit of the variables.

\item The values to be prescribed are provided by a succession of lines
that must have a format consistent with the identification line.
The time value must increase, and the last time value provided must be the same
as or greater than the corresponding value at the last time step of the
simulation.
If not, the calculation will stop.
\end{itemize}

When \telemac{2D} reads this file, it makes a linear interpolation in order to
calculate the value to be prescribed at a particular time step.
The value actually prescribed by the code is printed in the control printout.

An example of an open boundaries file is given below:
\begin{lstlisting}[language=bash]
# Example of open boundaries file
# 2 boundaries managed
#
T  Q(1) SL(2)
s  m3/s m
0.  0. 135.0
25. 15. 135.2
100. 20. 136.
500. 20. 136.
\end{lstlisting}
\begin{WarningBlock}{Note:}
Up to release 7.0, it is necessary to have the corresponding keywords
\telkey{PRESCRIBED {\dots}} to trigger the use of the liquid boundaries file.
\end{WarningBlock}

Since release 8.2, a time reference can be given:
If a \#REFDATE with a date + hour in YYYY-MM-DD HH:MM:SS
in year, month, day, hour, minute, second format is written in the file (after first \# line)
the date+hour will be added to the times in these ASCII files.

\subsection{Stage-discharge curves}
\label{subs:stage:dis:curve}
It is possible to manage a boundary where the prescribed value of the elevation
is a function of the local discharge (and vice versa).
This is particularly useful for river application.

First, it is necessary to define which boundary will use this type of condition
using the keyword \telkey{STAGE-DISCHARGE CURVES} which requires one integer per
liquid boundary.
This integer can be:

\begin{itemize}
\item 0: no stage-discharge curve (default value),

\item 1: elevation as function of discharge.
In that case, the boundary has to be defined as a prescribed elevation boundary,

\item 2: discharge as function elevation.
In that case, the boundary has to be defined as a prescribed discharge boundary.
\end{itemize}

The keyword \telkey{STAGE-DISCHARGE CURVES FILE} supplies the name of the ASCII
file containing the curves.
One example is shown hereafter:
\begin{lstlisting}[language=bash]
#
#  STAGE-DISCHARGE CURVE BOUNDARY 1
#
Q(1)     Z(1)
m3/s      m
61.       0.
62.       0.1
63.       0.2
#
#  STAGE-DISCHARGE CURVE BOUNDARY 2
#
Z(2)     Q(2)
m      m3/s
10.       1.
20.       2.
30.       3.
40.       4.
50.       5.
\end{lstlisting}
The order of curves is not important.
The columns order may be swapped like in the example for boundary 2.
Lines beginning with \# are comments.
Lines with units are mandatory but units are not checked so far.
The number of points given is free and is not necessarily the same for different
curves.
\\

N.B.: at initial conditions the discharge at exits may be null.
The initial elevation must correspond to what is in the stage-discharge curve,
otherwise a sudden variation will be imposed.
To avoid extreme situations the curve may be limited to a range of discharges.
In the example above for boundary 1, discharges below 61 m$^3$/s will all give
an elevation of 0.~m, discharges above 63 m$^3$/s will give an elevation of
0.2~m.
\\

When using value 1 for the keyword \telkey{STAGE-DISCHARGE CURVES}, the relation
between elevation and discharge may not be exactly the expected values.
Indeed, there is a delay (relaxation) to avoid triggering resonances and
high oscillations by means of a relaxation coefficient.
This coefficient allows to smooth high gradients of the prescribed values.
If set to 1., the elevation is instantaneously prescribed corresponding to the
stage-discharge curve, but this may lead to instabilities.
Setting a value between 0. and 1. is a compromise between the goal of the
stage-discharge curve and possible instabilities.

Since release 8.3, the relaxation coefficient can be changed with the keyword
\telkey{STAGE-DISCHARGE CURVES RELAXATION COEFFICIENT} (default value = 0.02).
This keyword substitutes the old hard-coded value of 0.02 at the end of function
%Before, a value of 0.02 was hard-coded at the end of function
\telfile{STA\_DIS\_CUR} (this value could have been changed manually).


\subsection{Prescribing complex values}
\label{subs:pres:compl:val}
If the values to be prescribed vary in both time and space, it is necessary
to program the \telfile{USER\_BORD} subroutine as this enables values
to be prescribed on a node-by-node basis.

This subroutine describes all the open boundaries (loop on \telfile{NPTFR}).
For each boundary point, it determines the type of boundary in order
to prescribe the appropriate value (velocity, elevation or flowrate).
However, there is little sense in programming \telfile{USER\_BORD} to prescribe
a flowrate, as this value is usually known for the entire boundary and not for
each segment of it.

In the case of a prescribed flowrate boundary located between two solid
boundaries with no velocities, the velocities on the angle points are cancelled.

N.B.: The \telfile{USER\_BORD} subroutine also enables the tracer limit values
to be prescribed (see \ref{sec:tr:prescr:bc}).


\subsection{Prescribing velocity profiles}
\label{subs:presc:vel:prof}
In the case of a flowrate or velocity conditions, the user can specify the
velocity profile computed by \telemac{2D}, using the keyword
\telkey{VELOCITY PROFILES}.
The user must supply one value for each open boundary.
The following options are available:

\begin{itemize}
\item 1: The velocity vector is normal to the boundary
(default value for all boundaries).
In the case of a prescribed flowrate, the value of the vector is set to 1
and then multiplied by a constant in order to obtain the desired flowrate
(given by the keyword \telkey{PRESCRIBED FLOWRATES} or by the subroutine
\telfile{USER\_Q}).
In the case of a prescribed velocity, the value used for the velocity norm
is provided by the keyword \telkey{PRESCRIBED VELOCITIES}
or by the subroutine \telfile{USER\_VIT}.
In any case, the velocity profile is constant along the boundary.
It is the default configuration,

\item 2: The values $U$ and $V$ are read from the
\telkey{BOUNDARY CONDITIONS FILE}
(\telfile{UBOR} and \telfile{VBOR} values).
In the case of a prescribed flowrate, these values are multiplied by a constant
in order to reach the prescribed flowrate,

\item 3: The velocity vector is imposed normal to the boundary.
Its value is read from the \telkey{BOUNDARY CONDITIONS FILE} (\telfile{UBOR}
value).
In the case of a prescribed flowrate this value is then multiplied by a constant
in order to obtain the appropriate flowrate,

\item 4: The velocity vector is normal to the boundary and its norm
is proportional to the square root of the water depth.
This option is valid only for prescribed flowrate.

\item 5: The velocity vector is normal to the boundary and its norm
is proportional to the square root of the virtual water depth computed
from the lower point of the free surface at the boundary.
\end{itemize}

In the case of a flow normal to a closed boundary, it is not recommended to have
velocities perpendicular to the solid segments
(as shown in the Figure \ref{bad:prescr:vel:prof}),
\begin{figure}
\includegraphics*[width=5.24in, height=2.04in, keepaspectratio=false]{./graphics/bad_profile.png}
\caption{Bad prescription of velocity profile.}
\label{bad:prescr:vel:prof}
\end{figure}

because the finite element interpolation will generate a non-zero flow
though a solid segment.
In this case, it is better to cancel the velocities on the first and last points
of the boundary, as shown on Figure \ref{good:presc:vel:prof}.

\begin{figure}
\includegraphics*[width=5.24in, height=2.04in, keepaspectratio=false]{./graphics/good_profile.png}
\caption{Good prescription of velocity profile.}
\label{good:presc:vel:prof}
\end{figure}


\subsection{Thompson boundary conditions}

In some cases, not all the necessary information concerning the boundary
conditions is available.
This is usual for coastal domains where only the values of the sea level
on several points are known.
This kind of model is referred to as an ''under-constrained'' model.

To solve this problem, the Thompson method uses the theory of characteristics
to calculate the missing values.
For example, \telemac{2D} will compute the velocity at the boundary in the case
of a prescribed elevation.

This method can also be used for ''over-constrained'' models.
In this case, the user specifies too much information at the boundary.
If the velocity information and the level information are not consistent,
the Thompson technique computes new values that will comply
with the theory of characteristics.

For this, the user can use the keyword \telkey{OPTION FOR LIQUID BOUNDARIES},
which offers two values (the user must specify one value for each open boundary):

\begin{itemize}
\item 1: strong setting (default value for all boundaries),

\item 2: Thompson method.
\end{itemize}

Taking a simplified view, it may be said that, in the case of the first option,
the values are ``imposed'', in the case of the second option, the values are
``suggested''.

\begin{WarningBlock}{Note:}
\begin{enumerate}
\item This keyword should be given for ALL liquid boundaries like given in the
following example: \newline
\telkey{OPTION FOR LIQUID BOUNDARIES}= 2;1;2;2 \newline
which means that you are asking \telemac{2D} to apply Thompson boundary
condition to boundaries number 1,3 and 4;
unlike boundary number 2, where \telemac{2D} will use a strong prescription.
\item This option will trigger the computation of characteristics' trajectories
in order to get information from inside the domain.
\end{enumerate}
\end{WarningBlock}

\subsection{Soft boundary conditions}

Jetting flows can occur on liquid boundaries where the elevation is prescribed.
To prevent this, a so called soft boundary can be used since release 8.5.
On a soft boundary, the prescribed level is modified by a multiple of the speed
normal to the boundary or the speed squared. \newline

The keyword to set soft boundaries is \telkey{OPTION FOR SOFT BOUNDARIES}. The user
should give one integer for each open boundary:

\begin{itemize}
\item 0: Not a soft boundary (default value for all boundaries),

\item 1: Method 1. Proportional to speed,

\item 2: Method 2. Proportional to speed squared.
\end{itemize}

Each soft boundary needs a coefficient.
These coefficients are specified by the keyword
\telkey{COEFFICIENT FOR SOFT BOUNDARIES} (default = 0.). \newline

\subsection{Elements Masking}

\telemac{2D} offers the possibility of masking some elements.
This means, for example, that islands can be created in an existing mesh.
The boundaries created in this way are processed as solid walls with a slip
condition.

This option is activated with the logical keyword
\telkey{ELEMENTS MASKED BY USER} (default value: NO).
In this case, the user must indicate the number of elements masked
by programming the \telfile{USER\_MASKOB} subroutine.
This manages an array of real values \telfile{MASKEL}, the size of which
is equal to the number of points and in which each value can be 0.D0
for a masked element and 1.D0 a normal one.

N.B.: This option is not compatible with perfectly conservative advection schemes.


\subsection{Definition of types of boundary condition when preparing the mesh}

When using Blue Kenue, the boundary condition type is prescribed during the last
step of mesh generation.

When using the other mesh generators, it is generally possible to define
the type of boundary condition during the mesh generation session, by prescribing
a colour code.
Each colour code corresponds to a particular type of boundary
(wall, open boundary with prescribed velocity, etc.).
The table showing colour codes of some meshers and corresponding types of
boundary is given in Appendix \ref{tel2d:app5}.


\subsection{Tidal harmonic constituents databases}
\label{subs:tidal:harm:datab}

\subparagraph{General parameters}

To prescribe the boundary conditions of a coastal boundary subject to tidal
evolution, it is generally necessary to have the information characterizing
this phenomenon (harmonic constants).
One of the most common cases is to use the information provided by large scale
models.

5 databases of harmonic constants are interfaced with \telemac{2D}:

\begin{itemize}
\item the JMJ database resulting from the LNHE Atlantic coast TELEMAC model by
Jean-Marc JANIN \cite{Janin1992},

\item the global TPXO database and its regional and local variants from the OSU
(Oregon State University) \cite{Egbert2022},

\item the HAMTIDE global model \cite{Taguchi2014} since release 8.5,

\item the regional North-East Atlantic atlas (NEA) \cite{Pairaud2008,Pairaud2010}
and the global atlas FES
(e.g. FES2004 or FES2014 \cite{Lyard2021}) coming from the works of LEGOS
(Laboratoire d'Etudes en G\'{e}ophysique et Oc\'{e}anographie Spatiales),

\item the PREVIMER atlases \cite{Pineau2013}.
\end{itemize}

However it is important to note that, in the current release of the code,
the latter 2 databases are not completely interfaced with \telemac{2D}
and their use is recommended only for advanced users.

The keyword \telkey{OPTION FOR TIDAL BOUNDARY CONDITIONS} activates the use of
one of the available database when set to a value different from 0
(the default value 0 means that this function is not activated).
Since release 7.1, this keyword is an array of integers separated by semicolons
(one per liquid boundary) so that the user can describe whether tidal boundary
conditions should be computed or not (e.g. a weir) on a liquid boundary.
When this keyword is activated, every tidal
boundary is treated using the prescribed algorithms for the boundaries with
prescribed water depths or velocities, with the same option for tidal boundary
conditions (the values not equal to 0 have to be the same),
except the boundaries with prescribed flowrate.

The database used is specified using the keyword \telkey{TIDAL DATA BASE}
which can take the values:

\begin{itemize}
\item 1: JMJ,

\item 2: TPXO or HAMTIDE,

\item 3: Miscellaneous (LEGOS-NEA, FES20XX, PREVIMER...).
\end{itemize}

If using HAMTIDE model, the keyword
\telkey{VELOCITIES IN BINARY DATABASE 2 FOR TIDE} has to be set to YES
(default = NO, e.g. if using OSU tidal solution like TPXO), see below.
\\

Depending on the database used, some keywords have to be specified:

\begin{itemize}
\item if using the JMJ database, the name of the database (typically bdd\_jmj)
is given by the keyword \telkey{ASCII DATABASE FOR TIDE} and the corresponding
mesh file is specified using the keyword \telkey{TIDAL MODEL FILE},

\item if using TPXO database or HAMTIDE model, the name of the water level
database is given
by the keyword \telkey{BINARY DATABASE 1 FOR TIDE} (for example h\_tpxo7.2)
and the name of the velocity database is given by the keyword
\telkey{BINARY DATABASE 2 FOR TIDE} (for example u\_tpxo7.2).
Moreover, it is possible to activate an interpolation algorithm of minor
constituents from data read in the database using the logical keyword
\telkey{MINOR CONSTITUENTS INFERENCE}, activation not done by default.
If using HAMTIDE model, as input data are velocity components rather than
transports (water depth times velocity components), the keyword
\telkey{VELOCITIES IN BINARY DATABASE 2 FOR TIDE} has to be set to YES
(default = NO).
\end{itemize}

The keyword \telkey{OPTION FOR TIDAL BOUNDARY CONDITIONS} allows specifying
the type of tide to prescribe.
Default value 0 means no prescribed tide or that the tide is not treated by
standard algorithms.
Value 1 corresponds to prescribing a real tide considering the time calibration
given by the keywords \telkey{ORIGINAL DATE OF TIME} (YYYY~; MM~; DD format) and
\telkey{ORIGINAL HOUR OF TIME} (HH~; MM~; SS format).
Other options are the following, available for every tidal database (JMJ,
TPXO-type from OSU, HAMTIDE, LEGOS-NEA, FES, PREVIMER \ldots).
They are called “schematic tide” for values from 2 to 6:

\begin{itemize}
\item 2: exceptional spring tide (French tidal coefficient
 approximately equal 110),

\item 3: mean spring tide (French tidal coefficient approximately
 equal 95),

\item 4: mean tide (French tidal coefficient approximately equal 70),

\item 5: mean neap tide (French tidal coefficient approximately
equal to 45),

\item 6: exceptional neap tide (French tidal coefficient
approximately equal to 30),

\item 7: real tide (before 2010 methodology, only available
with JMJ).
\end{itemize}

In the case of options 2 to 6 (schematic tides), the boundary conditions are
imposed so that the reference tide is approximately respected.
In order to shift the phases of the waves of the tidal constituents so that
the computation starts close to a High Water, two keywords are available.
If using a TPXO-type tidal database from Oregon State University
or HAMTIDE model, the keyword
\telkey{GLOBAL NUMBER OF THE POINT TO CALIBRATE HIGH WATER} has to be filled
with the global number of the point (between 1 and the number of boundary
nodes in the 2D mesh) with respect to which the phases are shifted
to start with a high water (mandatory, otherwise the computation stops).
This point has to be a maritime boundary node.
If using one of the other tidal databases (JMJ, NEA/FES, PREVIMER) the keyword
\telkey{LOCAL NUMBER OF THE POINT TO CALIBRATE HIGH WATER} should be filled in
with the local number between 1 and the number of tidal boundary points of the
\telkey{HARMONIC CONSTANTS FILE}; If not filled in (default value = 0), a value
is then automatically calculated.
However, it is usually necessary to wait for the second or third modelled tide
in order to overcome the transitional phase of start-up of the model.
It is also necessary to warn the user that the French tidal coefficients shown
are approximate.

During a simulation, data contained in the tidal database are interpolated
on boundary points.
When using the JMJ database, this spatial interpolation can be time consuming
if the number of boundary points is important, and is not yet available in case
of parallel computing.
It is therefore possible to generate a file containing harmonic constituents
specific to the model treated.
The principle is at a first step, to perform a calculation on a single time step
whose only goal is to extract the necessary information and to generate a file
containing for each boundary point of the model, the harmonic decomposition of
the tidal signal.
Subsequent calculations directly use that specific file rather than directly
addressing to the global database.
The harmonic constants specific file is specified using the keyword
\telkey{HARMONIC CONSTANTS FILE}, this file is an output file in the first
calculation, and an input file in subsequent calculations.
\\

If using tidal solutions coming from OSU (e.g. TPXO), to get velocity
components, it is necessary to divide the transports terms
(water depth times velocity components) by water depth,
contrary to HAMTIDE model which directly provides velocity components.
A minimum value of water depth to get them is taken to avoid divisions by 0.
Since release 8.2, it is possible to change this old hard coded value of 0.1~m,
both for boundary conditions and initial conditions, with the keywords
\telkey{MINIMUM DEPTH TO COMPUTE TIDAL VELOCITIES BOUNDARY CONDITIONS} and
\telkey{MINIMUM DEPTH TO COMPUTE TIDAL VELOCITIES INITIAL CONDITIONS}.
Both default values are 0.1~m.
Moreover, for initial conditions, if water depth is below
\telkey{MINIMUM DEPTH TO COMPUTE TIDAL VELOCITIES INITIAL CONDITIONS},
the velocity components are set to 0.
These 2 keywords enable to decrease artificially too high velocities
in particular at open boundaries with shallows or if the tidal solutions
have shallows at the same location.
If this effect of the latter keyword is not sufficient,
the user can initialise the computation with no velocities with
\telkey{INITIAL VELOCITIES COMPUTED BY TPXO} = NO (default = YES, i.e. the
tidal velocities are computed by algorithms to use OSU tidal solutions).


\subparagraph{Horizontal spatial calibration}

In order to perform the spatial interpolation of the tidal data,
it is imperative to provide to \telemac{2D} information on the spatial
positioning of the mesh model relative to the grid of the tidal database.
To do this, the user has two keywords:

The first keyword specifies the geographic system used to establish the
coordinates of the 2D mesh of \telemac{2D}.
This keyword \telkey{GEOGRAPHIC SYSTEM}, which has no default value,
may take the following values:

\begin{itemize}
\item 0: User Defined,

\item 1: WGS84 longitude/latitude in real degrees,

\item 2: WGS84 UTM North,

\item 3: WGS84 UTM South,

\item 4: Lambert,

\item 5: Mercator projection.
\end{itemize}

The second keyword is used to specify the area of the geographic system used
to establish the coordinates of the 2D mesh of \telemac{2D}.
This keyword \telkey{ZONE NUMBER IN GEOGRAPHIC SYSTEM} which has no default
value, may take the following values:

\begin{itemize}
\item 1: Lambert 1 North,

\item 2: Lambert 2 Center,

\item 3: Lambert 3 South,

\item 4: Lambert 4 Corsica,

\item 22: Lambert 2 extended,

\item 93: Lambert 93,

\item \telkey{X}: UTM zone value of the WGS84 (\telkey{X}
is the number of the zone).
\end{itemize}

If using the Lambert 93 projection, the user has to copy the file provided
in the tide examples of \telemac{2D} called gr3df97a.txt which is used
for the conversion in the Lambert 93 projection.
The keyword \telkey{LAMBERT 93 CONVERSION FILE} has to indicate the path
and the name of the gr3df97a.txt file.
\\

Since release 8.2, it is possible to use $x$ and $y$ origin coordinates stored
in the geometry file to decrease the number of digits of coordinates
when modelling tide, e.g. when using UTM or Lambert projections.
The two numbers are stored in the \telfile{I\_ORIGIN} and \telfile{J\_ORIGIN}
variables reachable with the help of \telfile{GET\_MESH\_ORIG} subroutine.
Caution: these two numbers are integers, not floats as the preliminary structure
available in the SERAFIN format expected this type for these 2 variables.
The tidal computations automatically take into account this offset for
UTM + Lambert projections when generating the \telkey{HARMONIC CONSTANTS FILE}
for JMJ database or when interpolating harmonic constants to compute boundary
or initial conditions for solutions coming from OSU (e.g. TPXO) or HAMTIDE.
In particular for the operations to locate the nodes correctly (a simple
translation) but \telemac{2D} still continues to compute other steps in a local
coordinate system.

\subparagraph{Calibration of the information}

The transfer of information between a large scale model and the boundaries of
a more local model generally requires calibration.

To do this, the user has three keywords:

\begin{itemize}
\item the keyword \telkey{COEFFICIENT TO CALIBRATE SEA LEVEL} (default real
value 0.) allows to calibrate the mean tide level (the harmonic decomposition
of information provided by the various databases are used to generate the tidal
signal oscillating around mean tide level).
The calibration of the mean tide level must obviously be made depending on the
altimetric reference used in the model,

\item the keyword \telkey{COEFFICIENT TO CALIBRATE TIDAL RANGE} (default real
value 1.) allows to specify a calibration coefficient applied on the amplitude
of the tidal wave.
This coefficient is applied to the amplitude of the overall signal, and not on
the amplitude of each of the elementary waves,

\item the keyword \telkey{COEFFICIENT TO CALIBRATE TIDAL VELOCITIES}
(default real value 999,999.0) allows to specify the coefficient applied on
velocities.
The default value (999,999.0) means that the square root of the value specified
by the keyword \telkey{COEFFICIENT TO CALIBRATE TIDAL RANGE} tidal is used.
\end{itemize}

For more information, the reader may refer to the methodological guide
for tide simulation with version 6.2 \cite{Pham2012}.

\subparagraph{Inverted barometer effect}

Atmospheric pressure can be taken into account at tidal boundaries by using
inverted barometer method (adding a head loss computed from the difference of
pressures divided by water density and gravity acceleration)
when modelling storm surges e.g. (with wind and atmospheric pressure fields).
From release 8.5, the keyword \telkey{ATMOSPHERIC PRESSURE AT TIDAL BOUNDARIES}
is to be activated (default = NO) to do so.
