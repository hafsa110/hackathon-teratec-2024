% Numerical parameters found in suspension examples :
% COEFFICIENT DE DIFFUSION DES VASES = 0
% CONVECTION DES TRACEURS POUR COURLIS = vrai
% OPTION DE CONVECTION POUR LES TRACEURS POUR COURLIS = 4
% ORDRE DU SCHEMA DE CONVECTION VOLUMES FINIS POUR COURLIS = 3
% PARAMETRE W DU SCHEMA DE CONVECTION VOLUMES FINIS POUR COURLIS = 0
% LIMITEUR DE PENTE DU SCHEMA VOLUMES FINIS POUR COURLIS = vrai

\section{Vertical discretization and clipping parameters} \label{planim}
The vertical discretization step is set in the \xcas.

To limit the calls to the vertical discretization process, two options are proposed in \courlis :
\begin{itemize}
	\item \telkey{CLIP ABSOLU SUR L'EVOLUTION} or \telkey{ABSOLUTE CLIP EVOLUTION} : The vertical discretization process is done every time bottom variations are higher than the clipping value in meters. For example, \telkey{ABSOLUTE CLIP EVOLUTION = 0.05} means the vertical discretization process will take place every time bottom variations are higher than 5 centimenters. It is by default equal to \telkey{1e-5} m.
	\item The second option suggests a clipping depending on the water depth. If the \telkey{CLIPPING OPTION} is set to \telkey{TRUE} (by default, \telkey{CLIPPING OPTION = FALSE}), the vertical discretization process is called every time bottom variation is more than a given percentage (between 0 and 1) specified in \telkey{CLIP EVOLUTION}. For example, \telkey{CLIP EVOLUTION = 0.05} means the vertical discretization process will happen every time bottom variation represents more than 5\% of the water depth.
\end{itemize}
For now, no option has been shown to be more efficient than the other.
The clipping value can be adapted for each simulation and a compromise between errors generated by this clipping and computational time should be found.

To start, a clipping value of 1\% is recommended.

\section{Local slope option} \label{local_slope_opt}

The energy slope value used in bedload transport laws can be computed from the head values along the reach (computed by \mascaret) or from the Strickler formula (Equation \ref{eq:strickler}) using the mean water depth, the section geometry, the liquid discharge and the Strickler coefficient.

Historically, the first method was used in \courlis. Yet, in some cases, this option led to high bed instabilities caused by the hydraulics computed by \mascaret, sometimes locally perturbed.

That is why the option \telkey{PENTE LOCALE} was developped.

\begin{WarningBlock}{Recommendation}
	It is highly recommended to set \telkey{PENTE LOCALE = TRUE}.
\end{WarningBlock}

\section{Coupling parameters}\label{coupling_param}

The coupling parameters described in \S \ref{coupling} $n$ and $p$ correspond to the keywords \telkey{NOMBRE D'ITERATIONS HYDRAULIQUE} for the hydraulics number of iterations $n$ and \telkey{NOMBRE D'ITERATIONS SEDIMENTO} for the sediment transport number of iterations $p$.

\section{Uncentered scheme for SARAP kernel}\label{uncentered}
When using the permanent kernel of \mascaret, SARAP, an option is available in the \xcas to use an uncentered scheme on the energy slope (see Figure \ref{fig:xcas_decentrement}).

Traditionnally, the energy slope term $J$ in the shallow water equations (Equation \ref{eq:shallow}) was approximated by upstream and downstream values.
This option replaces this scheme to consider only the upstream value as described below :
\bequ
	    J = \frac{2}{\frac{1}{J_{downstream}}+\frac{1}{J_{upstream}}} \rightarrow J = J_{upstream}
\eequ

The use of this uncentered scheme with \Cbedload has shown good results and seems to be less mesh-sensitive.

\section{Printouts}\label{printouts}

The outputs files names (see \S \ref{outputs} for description of these files) are given with the following keywords :

\begin{itemize}
	\item \telkey{FICHIER LISTING COURLIS} : The listing file name, for example \telfile{results.listingcourlis} ;
	\item \telkey{FICHIER RESULTATS PROFIL EN LONG} : The longitudinal profile results file name, for example \telfile{results.plong} ;
	\item \telkey{FICHIER RESULTATS PROFIL EN TRAVERS} : The cross-sections results file name, for example \telfile{results.ptravers} ;
\end{itemize}

For the values printed in the listing, different printouts are available :

\begin{itemize}
	\item \telkey{IMPRESSION DES PARAMETRES SEDIMENTAIRES} : Printouts of sediment characteristics
	\item \telkey{IMPRESSION DES INTERFACES SEDIMENTAIRES} : Printouts of sediment layers interfaces
	\item \telkey{IMPRESSION DES PARAMETRES DE COUPLAGE} : Printouts of coupling parameters
	\item \telkey{IMPRESSION DES CONC INITIALES POUR COURLIS} : Printouts of initial concentrations
	\item \telkey{IMPRESSION DES LOIS DE CONCENTRATION} : Printouts of boundary conditions laws
	\item \telkey{IMPRESSION DES APPORTS SEDIMENTAIRES} : Printouts of sediment inflows
\end{itemize}

An important parameter to limit computational time and the size of the results files is the printout step. For \mascaret, printouts periods are given in the \xcas. For \courlis, several keywords are given to limit printouts during calculation :

\begin{itemize}
	\item \telkey{PAS D'IMPRESSION COURLIS} : Period in number of sediment timesteps for \courlis printouts in the \telfile{listing file}
	\item \telkey{PAS DE STOCKAGE POUR LE PROFIL EN LONG} : Period in number of sediment timesteps for \courlis printouts in the \telfile{plong file}
	\item \telkey{PAS DE STOCKAGE POUR LE PROFIL EN TRAVERS} : Period in number of sediment timesteps for \courlis printouts in the \telfile{ptravers file}
\end{itemize}

The boolean \telkey{DEBUG CHARRIAGE} permits to print bedload values in the listing for each section at each iteration to detect inconsistencies.
