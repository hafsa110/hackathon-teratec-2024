\section{First run}

From your working folder, in the same way as for hydraulics simulations with \mascaret, the user should call the \mascaret script:
\lstset{language=bash,
        basicstyle=\scriptsize\ttfamily}

\begin{lstlisting}[escapechar=|]
[bash:] mascaret.py <your_xcas_file.xcas>
\end{lstlisting}

An example of a \Cbedload simulation on a Debian 10 machine is given below :
\lstset{language=TelemacCas}
\begin{lstlisting}
Loading Options and Configurations
~~~~~~~~~~~~~~~~~~~~~~~~~~~~~~~~~~~~~~~~~~~~~~~~~~~~~~~~~~~~~~~~~~~~~~~~

(*@\color{codegreen} < COURLIS VERSION >@*)

Running your CAS file(s) for:
~~~~~~~~~~~~~~~~~~~~~~~~~~~~~~~~~~~~~~~~~~~~~~~~~~~~~~~~~~~~~~~~~~~~~~~~



~~~~~~~~~~~~~~~~~~~~~~~~~~~~~~~~~~~~~~~~~~~~~~~~~~~~~~~~~~~~~~~~~~~~~~~~

S10.gfortran.dyn: 
    

    +> Gfortran compiler 8.3.0 with open_mpi for a Debian 10

    +> root:    (*@\color{codegreen} <your root folder>@*)
    +> module: ad / api / artemis / bief
               damocles  / gaia  / gretel  / hermes
               identify_liq_bnd  / khione  / mascaret  / nestor
               parallel  / partel  / postel3d  / sisyphe
               special  / stbtel  / telemac2d  / telemac3d
               tomawac / waqtel


~~~~~~~~~~~~~~~~~~~~~~~~~~~~~~~~~~~~~~~~~~~~~~~~~~~~~~~~~~~~~~~~~~~~~~~~


... processing the steering file

... checking parallelisation

... handling temporary directories
~+> Creating FichierCas.txt
                                                                               
... checking the executable


Running your simulation(s) :
~~~~~~~~~~~~~~~~~~~~~~~~~~~~~~~~~~~~~~~~~~~~~~~~~~~~~~~~~~~~~~~~~~~~~~~~



In (*@\color{codegreen} <your simulation folder>@*):
mpirun -np 1 (*@\color{codegreen} <your root folder>@*)/builds/S10.gfortran.dyn/bin/mascaret


 Fichiercas : FichierCas.txt

 TELEMAC-MASCARET V8P2R0beta == Copyright (C) 2000-2020 EDF-CEREMA ==

 Data File : (*@\color{codegreen} <your xcas file.xcas>@*)

 Clip Evolution = (*@\color{codegreen} <non absolute clipping value>@*)

 Pente locale : T

 Parametres utilises seulement pour les formules de transport de Recking et de Lefort
 D84 = (*@\color{codegreen} < >@*)
 Diametre moyen = (*@\color{codegreen} < >@*)

 Parametre utilise seulement pour la formule de transport de Lefort
 D16 = (*@\color{codegreen} < >@*)

 Bedload option : T  (*@\color{codegreen} Bedload simulation@*)

 Suspension option : F

 sediment slide option : F

 planim clipping option : F

 absolute clipping = (*@\color{codegreen} <absolute clipping value>@*)

 bedload transport law is (*@\color{codegreen} <transport law>@*)
 Study name : (*@\color{codegreen} <your study name>@*)

 Hydraulic statistics (prior solve phase)
    ------ Geometric parameters ------
    Number of reach(es)       =            1
    Number of cross-sections  =           (*@\color{codegreen} <nb sections>@*)
    Number of open boundaries =            2
    Number of junction(s)     =            0
    Number of inflow(s)       =            0
    Number of lateral weir(s) =            0
    Number of storage area(s) =            0
    Number of link(s)         =            0
    Number of dam(s)/weir(s)  =            0
    ------ Numerical parameters ------
    Number of 1D nodes (mesh) =           (*@\color{codegreen} <nb sections>@*)
    Computation Kernel        =            1
    Initial time step         =      	(*@\color{codegreen} <dt>@*)
    Variable time step?       =            0
    Simulation time           =      	(*@\color{codegreen} <t max>@*)

 (*@\color{PantoneRed}OptionCourlis T@*)
 Start the Simulation...
\end{lstlisting}

\section{Calculation continuation}

wip


\begin{WarningBlock}{Warning}
	When updating your version of \courlis, make sure to delete the precedent \telfile{dictionnary file} before running your simulation again.
\end{WarningBlock}

