\chapter{Other configurations}

\section{Modification of bottom topography (\telfile{USER\_T3D\_CORFON})}

Bottom topography may be introduced at various levels, as stated
in section \ref{sec:topo}.

\telemac{3D} offers the possibility of modifying the bottom topography at the
beginning of a computation using the \telfile{USER\_T3D\_CORFON} subroutine.
This is called up once at the beginning of the computation and enables the value
of variable \telfile{ZF} to be modified at each point of the mesh.
To do this, a number of variables such as the point coordinates, the element
surface value, connectivity table, etc. are made available to the user.

By default, the \telfile{T3D\_CORFON} subroutine (which calls the
\telfile{USER\_T3D\_CORFON} subroutine) carries out a number
of bottom smoothings equal to \telfile{LISFON},
i.e. equal to the number specified by the keyword
\telkey{NUMBER OF BOTTOM SMOOTHINGS}
for which the default value is 0 (no smoothing).
The call to bottom smoothings can be done after or before the call to
\telfile{USER\_T3D\_CORFON} in the \telfile{T3D\_CORFON} subroutine with the
keyword \telkey{BOTTOM SMOOTHINGS AFTER USER MODIFICATIONS}.
By default, potential bottom smoothings are done after potential modifications
of the bottom in \telfile{USER\_T3D\_CORFON} subroutine (default = YES).

The \telfile{T3D\_CORFON} subroutine is not called up
if a computation is continued.
Neither is \telfile{USER\_T3D\_CORFON} subroutine.
This avoids having to carry out several bottom smoothings or modifications
of the bottom topography during the computation.


\section{Modifying coordinates (\telfile{USER\_CORRXY})}

\telemac{3D} also offers the possibility of modifying the mesh point coordinates
at the beginning of a computation.
This means, for example, that it is possible to change the scale
(from that of a reduced-scale model to that of the real object),
rotate or translate the object.

The modification is done in the \telfile{USER\_CORRXY} subroutine
(\bief library), which is called up at the beginning of the computation.
This subroutine is empty by default and gives an example of programming
a change of scale and origin, within commented statements.

It is also possible to specify the coordinates of the origin point of the mesh.
This is done using the keyword \telkey{ORIGIN COORDINATES}
which specify 2 integers (default = (0;0)).
These 2 integers will be transmitted to the results file in the SERAFIN format,
for a use by post-processors for superimposition of results with digital maps
(coordinates in meshes may be reduced to avoid large real numbers).
These 2 integers may also be used in subroutines under the names
\telfile{I\_ORIG} and \telfile{J\_ORIG}.
Otherwise they do not have a use yet.


\section{Spherical coordinates (\telfile{LATITU})}
\label{sec:spher:coord:LATI}
If a simulation is performed over a large domain, \telemac{3D} offers
the possibility of running the computation with spherical coordinates.

This option is activated when the keyword \telkey{SPHERICAL COORDINATES}
is set to YES (default value = NO).
In this case, \telemac{3D} calls a subroutine named \telfile{LATITU}
through the subroutine \telfile{INBIEF} at the beginning of the computation.
This calculates a set of tables depending on the latitude of each point.
To do this, it uses the Cartesian coordinates of each point provided
in the geometry file, and the latitude of origin point of the mesh
provided by the user in the steering file with the keyword
\telkey{LATITUDE OF ORIGIN POINT} (default value = 0 degrees).

The spatial projection type used for the mesh is then specified with the
keyword \telkey{SPATIAL PROJECTION TYPE}. That can take the following values:
\begin{itemize}
\item 1 : Lambert Cartesian not geo-referenced,

\item 2 : Mercator (default value),

\item 3 : Latitude/longitude (in degrees).
\end{itemize}

In this case of option 3, the coordinates of the mesh nodes should be
expressed with latitude and longitude in degrees. \telemac{3D} then converts with
the information with the help of the Mercator's projection.
It is important to notice here that, if option \telkey{SPHERICAL COORDINATES}
= YES, \telkey{SPATIAL PROJECTION TYPE} has to be 2 or 3.

The \telfile{LATITU} subroutine (\bief library) may be modified by the user to introduce
any other latitude-dependent computation.


\section{Adding new variables}
\label{sec:privarray}
A standard feature of \telemac{3D} is the storage of some computed variables.
In some cases, the user may wish to compute other variables and store them
in the results file (the number of variables is currently limited to four).

Since \telemac{3D} uses 2D and 3D variables, the treatments linked to these
variables may differ and call three subroutines:

\begin{itemize}
\item  \telfile{NOMVAR\_2D\_IN\_3D}: to manage 2D variables names,

\item  \telfile{NOMVAR\_TELEMAC3D}: to manage 3D variables names,

\item  \telfile{USER\_PRERES\_TELEMAC3D}: to compute new variables (2D and 3D).
\end{itemize}

\telemac{3D} has a numbering system in which, for example, the array containing
the Froude number has the number 7. The new variables created by the user may
have the numbers 25, 26, 27 and 28 (for 3D variables) and 27, 28, 29 and 30
(for 2D variables).

In the same way, each variable is identified by a letter in the keywords
\telkey{VARIABLES FOR 2D GRAPHIC PRINTOUTS} and
\telkey{VARIABLES FOR 3D GRAPHIC PRINTOUTS}.
The new variables are identified by the strings \telfile{PRIVE1, PRIVE2, PRIVE3}
and \telfile{PRIVE4} for 2D variables and \telfile{P1, P2, P3} and \telfile{P4}
for 3D variables.

At the end of the \telfile{NOMVAR\_TELEMAC3D} or \telfile{NOMVAR\_2D\_IN\_3D}
subroutines, it is
possible to change the abbreviations (mnemonics) used for the keywords
\telkey{VARIABLES FOR 2D GRAPHIC PRINTOUTS} and \telkey{VARIABLES FOR 3D GRAPHIC
PRINTOUTS}. Sequences of 8 letters may be used.
Consequently, the variables must be separated by spaces, commas or
semicolons in the keywords, e.g.:

\begin{lstlisting}[language=TelemacCas]
VARIABLES FOR 2D GRAPHIC PRINTOUTS : 'U, V, H, B'
\end{lstlisting}

In the software data structure, these four variables correspond to the tables
\telfile{PRIVE\%ADR(1)\%P\%R(X), PRIVE\%ADR(2)\%P\%R(X), PRIVE\%ADR(3)\%P\%R(X)}
and \telfile{PRIVE\%ADR(4)\%P\%R(X)} (in which \telfile{X} is the number of
nodes in the mesh).
These may be used in several places in the programming, like all \tel variables.
For example, they may be used in the subroutines \telfile{USER\_CORRXY},
\telfile{USER\_CORSTR}, \telfile{USER\_BORD3D}, etc.
If a \telfile{PRIVE} table is used to program a case, it is essential
to check the value of the keyword \telkey{NUMBER OF PRIVATE ARRAYS}.
This value fixes the number of tables used (0, 1, 2, 3 or 4)
and then determines the amount of memory space required.
The user can also access the tables via the aliases
\telfile{PRIVE1, PRIVE2, PRIVE3} and \telfile{PRIVE4}.

An example of programming using the second \telfile{PRIVE} table is given below.
It is initialised with the value 10.

\begin{lstlisting}[language=TelFortran]
DO I=1,NPOIN2
  PRIVE%ADR(2)%P%R(I) = 10.D0
ENDDO
\end{lstlisting}

New variables are programmed in two stages:

\begin{itemize}
\item Firstly, it is necessary to define the name of these new variables
by filling in the \telfile{NOMVAR\_TELEMAC3D} (or \telfile{NOMVAR\_2D\_IN\_3D})
subroutine.
This consists of two equivalent structures, one for English and the other
for French.
Each structure defines the name of the variables in the results file
that is to be generated and then the name of the variables to be read
from the previous computation if this is a restart.
This subroutine may also be modified when, for example, a file generated with
the English version of \telemac{3D} is to be continued with the French version.
In this case, the \telfile{TEXTPR} table of the French part of the subroutine
must contain the English names of the variables,

\item Secondly, it is necessary to modify the \telfile{USER\_PRERES\_TELEMAC3D}
subroutine in order to introduce the computation of the new variable(s).
The variables \telfile{LEO, SORG2D} and \telfile{SORG3D} are also used to
determine whether the variable is to be printed in the printout file
or in the results file at the time step in question.
\end{itemize}

User arrays can be handled to store extra variables in 2D with the help of
two keywords to define the number and the name of the extra variables
in the 2D private arrays: \telkey{NUMBER OF 2D PRIVATE ARRAYS} (up to 4,
default value = 0) and \telkey{NAMES OF 2D PRIVATE VARIABLES}.
It is the names of the user arrays \telfile{PRIVE\%ADR(1)\%P, PRIVE\%ADR(2)\%P}
\ldots up to 4, that will be seen in the results files.
The great advantage is that these variables will be read if present in the
\telkey{GEOMETRY FILE}.

\section{Array modification or initialization}

When programming \telemac{3D} subroutines, it is sometimes necessary to
initialize a table or memory space to a particular value.
To do that, the \bief library furnishes a subroutine called \telfile{FILPOL}
that lets the user modify or initialize tables in particular mesh areas.

A call of the type \telfile{CALL FILPOL (F, C, XSOM, YSOM, NSOM, MESH)}
fills table \telfile{F} with the \telfile{C} value in the convex polygon
defined by \telfile{NSOM} nodes (coordinates \telfile{XSOM, YSOM}).
The variable \telfile{MESH} is needed for the \telfile{FILPOL} subroutine
but has no meaning for the user.


\section{Validating a computation (\telfile{BIEF\_VALIDA})}

The structure of the \telemac{3D} software offers an entry point for validating
a computation, in the form of a subroutine named \telfile{BIEF\_VALIDA},
which has to be filled by the user in accordance with each particular case.
Validation may be carried out either with respect to a reference file
(which is therefore a file of results from the same computation that is taken
as reference, the name of which is supplied by the keyword
\telkey{REFERENCE FILE}), or with respect to an analytical solution that must
then be programmed entirely by the user.

When using a reference file, the keyword \telkey{REFERENCE FILE FORMAT}
specifies the format of this binary file ('SERAFIN ' by default).

The \telfile{BIEF\_VALIDA} subroutine is called at each time step
when the keyword \telkey{VALIDATION} has the value YES,
enabling a comparison to be done with the validation solution at each time step.
By default, the \telfile{BIEF\_VALIDA} subroutine only does a comparison
with the last time step.
The results of this comparison are given in the output listing.


\section{Coupling}
\label{sec:coupling}

The principle of coupling two (or in theory more) simulation modules involves
running the two calculations simultaneously and exchanging the various results
at each time step.
For example, the following principle is used to couple a hydrodynamic module
and a sediment transport module:

\begin{itemize}
\item The two codes perform the calculation at the initial instant
with the same information (in particular the mesh and bottom topography),

\item The hydrodynamic code runs a time step and calculates the water depth
and velocity components.
It provides this information to the sediment transport code,

\item The sediment transport code uses this information to run the solid
transport calculation over a time step and thus calculates a change in the
bottom,

\item The new bottom value is then taken into account by the hydrodynamic module
at the next time step, and so on.
\end{itemize}

Several modules can be coupled in the current release of the code:
the sediment transport modules \gaia
the sea state computational module \tomawac
with 4 options (see below to see the descriptions of the options)
and the water quality module \waqtel (and even DELWAQ).
The time step used for the two calculations is not necessarily the same and is
managed automatically by the coupling algorithms
and the keyword \telkey{COUPLING PERIOD FOR TOMAWAC} with the default value 1
(coupling at every iteration).

This feature requires two keywords.
The keyword \telkey{COUPLING WITH} indicates which simulation code is to be
coupled with \telemac{3D}.
The values of this keyword can be:

\begin{itemize}
\item \telkey{COUPLING WITH} = `GAIA' for coupling with the \gaia module,

\item \telkey{COUPLING WITH} = `TOMAWAC' for coupling with the \tomawac module,
forces are constant along the vertical (like with \telemac{2d}),

\item \telkey{COUPLING WITH} = `TOMAWACT3D' for coupling with the \tomawac
module but forces induced by waves are 3D,

\item \telkey{COUPLING WITH} = `TOMAWAC2' for coupling with the \tomawac
module, forces are constant along the vertical, but using the feature tel2tom,
the meshes and the domains of \telemac{3d} and \tomawac can be different,

\item \telkey{COUPLING WITH} = `TOMAWACT3D2' for coupling with the \tomawac
module, forces induced by waves are 3D, but using the feature tel2tom,
the meshes and the domains of \telemac{3d} and \tomawac can be different,

\item \telkey{COUPLING WITH} = `WAQTEL' for coupling with the \waqtel module.
\end{itemize}

If wanting to couple with 2 modules or more, the differents modules are to
be written in the \telkey{COUPLING WITH} separated with semicolon.

Depending on the module(s) used, the keywords \telkey{GAIA STEERING FILE},
\telkey{TOMAWAC STEERING FILE} and \telkey{WAQTEL STEERING FILE} indicate the
names of the steering files of the coupled modules.\\

If coupling with the water quality module \waqtel, the integer keyword
\telkey{WATER QUALITY PROCESS} must be set to a value different from 1
(default = 1) in the \telemac{3D} steering file.
The possible choices are:
\begin{itemize}
\item 0: all available processes,
\item 1: nothing (default value),
\item 2: O$_2$ module,
\item 3: BIOMASS module,
\item 5: EUTRO module,
\item 7: MICROPOL module,
\item 11: THERMIC module,
\item 13: AED2 model,
\item 17: degradation law.
\end{itemize}

Several modules can be combined by giving the multiplication of the process
choices, e.g. 55 = 5 $\times$ 11 activates EUTRO and THERMIC modules.
It is noted that AED2 should be used on its own, for the time being,
without possible combination with other processes.
Please refer to the \waqtel documentation for additional informations for
\waqtel.\\

If coupling with the \tomawac in a 3D way (the keyword \telkey{COUPLING WITH}
including TOMAWACT3D or TOMAWACT3D2), the keyword
\telkey{BOTTOM FRICTION DUE TO WAVES}
enables to take into account the momentum lost by waves due to
bottom friction (default = NO).
A fine mesh around the bottom is needed to be accurate.\\

The use of the feature tel2tom to have different meshes with \telemac{3D} and
\tomawac requires to calculate the weights of interpolation between the two
meshes before the simulation.
Those weights can be obtained and put in the meshes through the use of the
command:
\begin{lstlisting}[language=bash]
run_telfile.py tel2tom
\end{lstlisting}
An example of use of this command is done in the notebook
\$HOMETEL/pretel/tel2tom.ipynb
\\

The keyword \telkey{COUPLING WITH} is also used if the computation
has to generate the appropriate files necessary to run a water quality
simulation with DELWAQ.
In that case, it is necessary to specify \telkey{COUPLING WITH} = 'DELWAQ'.
Please refer to Appendix \ref{sec:delwaq} for all informations concerning
communications with DELWAQ.

\section{Checking the mesh (\telfile{CHECKMESH})}

The \telfile{CHECKMESH} subroutine of the \bief library is available to look for
errors in the mesh, e.g. superimposed points \ldots
The keyword \telkey{CHECKING THE MESH} (default value = NO) should be activated
to YES to call this subroutine.
