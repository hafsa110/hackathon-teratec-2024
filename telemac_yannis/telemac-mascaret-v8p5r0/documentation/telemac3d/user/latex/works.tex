\chapter{Construction works modelling}
\label{ch:constr:wm}

\section{Culverts}
\label{sec:culverts}
The keyword \telkey{NUMBER OF CULVERTS} (default value = 0) specifies
the number of culverts to be treated as source terms.
Culverts are described as couples of points between which flow may occur,
as a function of the respective water level at these points.
They must be described as sources in the domain.
%Since release 6.2 of \telemac{3D}, it is no longer
It is not necessary to describe each culvert
inflow and outflow as a source point like in \telemac{2D} before release 6.2.

There are two options to treat culverts in \telemac{3D}.
The choice can be done with \telkey{OPTION FOR CULVERTS}
(default value = 1).
For more information about this choice, the reader is invited to refer to the
\telemac{3D} theory guide.\\

Information about culvert characteristics is stored in the
\telkey{CULVERTS DATA FILE}.

The following file gives an example of a culvert:
\begin{lstlisting}[language=bash]
Relaxation, Number of culverts
0.2 1
I1  I2  CE1 CE2 CS1 CS2 LRG  HAUT1 CLP LBUS Z1  Z2  CV  C56 CV5 C5  CT  HAUT2 FRIC LENGTH CIRC D1  D2 A1 A2  AA
199 640 0.5 0.5 10  1.0 2.52 2.52  0   0.2  0.3 0.1 0.0 0.0 0.0 0.0 0.0 2.52  0.0  0.0    1    90. 0. 0. 90. 0

\end{lstlisting}

The relaxation coefficient is initially used to prescribe the discharge
in the culvert on a progressive basis in order to avoid the formation of an eddy.
Relaxation, at time $T$, between result computed at time $T$ and result
computed at previous time step.
A relaxation coefficient of 0.2 means that 20\% of time $T$ result is mixed
with 80\% of the previous result.
I1 and I2 are the numbers of each end of the culvert in the global point
numbering system.

The culvert discharge is calculated based on the formulae given
in the \telemac{3D} theory guide:
CE1 and CE2 are the head loss coefficients of 1 and 2 when they are operating as
inlets.
CS1 and CS2 are the head loss coefficients of 1 and 2 when they are operating as
outlets.
LRG is the width of the culvert.
HAUT1 and HAUT2 are the heights of the construction work (in meters)
at the inlet and outlet.
The flow direction is also imposed through the keyword CLP:\\
CLP = 0, flow is allowed in both directions,\\
CLP = 1, flow is only allowed from section 1 to section 2,\\
CLP = 2, flow is only allowed from section 2 to section 1,\\
CLP = 3, no flow allowed.\\
LBUS is the linear head loss in the culvert, generally equal to
$\lambda {\kern 1pt} {\kern 1pt} {\kern 1pt} {\kern 1pt} {\kern 1pt} \frac{L}{D} $
where $L$ is the length of the pipe, $D$ its diameter and $l$ the friction
coefficient.
Z1 and Z2 are the levels of the inlet and outlet.
CV refers to the loss coefficient due to the presence of a valve and
C56 is the constant used to differentiate flow types 5 and 6 in the formulation
by Bodhaine.
C5 and CV5 represent correction coefficients to C1 and to CV coefficients 
due to the occurrence of the type 5 flow in the Bodhaine formulation.
CT is the loss coefficient due to the presence of trash screens.
FRIC is the Manning Strikler coefficient.
LENGTH is the length of the culvert, and the culvert's shape can be specified
through the parameter CIRC (equal to 1 in 
case of a circular section, 0 for a rectangular section).
A1 and A2 are the angles with respect to the $x$ axis. 
D1 and D2 are the angles that the pipe makes with respect to the bottom,
in degrees.
For a vertical intake, the angle with the bottom will therefore be 90$^\circ$.
They are used to account for the current direction at source or sink point.
AA is a parameter which allows the user to choose whether A1 and A2 are
automatically computed by \telemac{3D}
or whether the data file values are used to set these angles:
AA=1 -- automatic angle; AA=0 -- user-set angle.


%Following lines come from the \telfile{BUSE} subroutine comments.
%The choice can be done with the keyword \telkey{OPTION FOR CULVERTS}:
%\begin{itemize}
%\item option 1 computes the flow according to $\Delta H$ (only works for
%circular sections).
%This is the default option (and the older available in \tel),
%\item option 2 uses Bodhaine (1968) + Carlier (1976) formulae.
%The flowrate is calculated according to 5 types of flow calculation
%of discharges based on water levels equations.
%\end{itemize}

%For option 1, if the linear pressure loss is negligible,
%it could have different entry/exit sections.
%For option 2, in case the entrance and exit surfaces of the culvert are not
%equal, we assume that the smallest surface will limit the flow through
%the culvert.
%This surface is thus taken to calculate the discharge.
