\chapter{Tracer transport}

The \telemac{3D} software makes it possible to take into account the transport of
passive or active tracers (active tracers affect the hydrodynamics), being
either conservative or not.

This chapter discloses the tracer transport features.

The maximum number of tracers is set to 20 by default but it can be changed by
the user with the keyword \telkey{MAXIMUM NUMBER OF TRACERS}.
This avoids changing the previously hardcoded values (until release 7.0),
which required recompiling the whole package.

\section{General setup}

The number of tracers is defined by the keywords \telkey{NUMBER OF TRACERS}. If
that number is set to zero (default value), then the tracers will not be taken
into account by \telemac{3D}.
Additional modules which can be coupled with \telemac{3D}
(e.g. \waqtel or \gaia, see section \ref{sec:coupling})
can add extra tracers to the initial set defined in the \telemac{3D}
steering file.

In addition to the number of tracers, the user should enter the \telkey{NAMES
OF TRACERS}. A tracer name should be written with 32 characters
(16 for the name and 16 for the unit).

\begin{lstlisting}[language=TelemacCas]
NUMBER OF TRACERS : 2
NAMES OF TRACERS :
'TEMPERATURE     C              ';'SALINITY        G/L             '
\end{lstlisting}

Obviously, it is necessary to add the appropriate specifications
in the keywords \telkey{VARIABLES FOR 3D GRAPHIC PRINTOUTS} and
\telkey{VARIABLES FOR 2D GRAPHIC PRINTOUTS}.
The name of the variables is a letter TA followed by the number of tracer.
For example 'TA1,TA3' stand for first and third tracer.
It is possible to use the character * as wildcards (replace any character).
TA* stands for TA1 to TA9, and TA** stands for TA10 to TA99.


\subsection{Prescribing the initial conditions}

If the initial values of tracers are constant all over the domain, just insert,
into the steering file, the keyword \telkey{INITIAL VALUES OF TRACERS} with the
required value(s) separated with a semicolumn; if more than one.
The number of supplied values must be equal to the number of declared tracers.

In more complex cases, it is necessary to work directly in the
\telfile{USER\_CONDI3D\_TRAC} subroutine, in a similar way to that described
in the subsection \ref{sec:prescr_IC} 
dealing with the initial hydrodynamic conditions.

When resuming a computation, the initial condition of tracers corresponds to
the condition of the last time step which was stored into the restart file.
The tracer management sequence order during the previous computation needs to
be well known and the same sequence order shall be followed in the
computational suite in order to prevent on confusion.
If the restart file does not contain any information about the tracer,
\telemac{3D} will use the value as set by the keyword
\telkey{INITIAL VALUES OF TRACERS}.


\subsection{Prescribing the boundary conditions}

Tracer boundary conditions are prescribed according to the same principle
as the hydrodynamic boundary conditions (see section \ref{sec:prescr_BC}).

The boundary condition type will be given by the value of \telfile{LITBOR}
in the boundary conditions file.

In case of an entering liquid boundary with one or several prescribed tracer(s)
(\telfile{LITBOR} = 5), the tracer value can be given in various ways:

\begin{itemize}
\item If the value is constant along the boundary and in time, it is provided
in the steering file by the keyword \telkey{PRESCRIBED TRACERS VALUES}.
This is an array of real numbers for managing several boundaries and several
tracers (100 at most, this number can be changed with the keyword
\telkey{MAXIMUM NUMBER OF TRACERS}).
The writing convention is as follows: value of tracer 1 at
boundary 1, value of tracer 2 at boundary 1, \dots , value of tracer N at
boundary 1, value of tracer 1 at boundary 2, value of tracer 2 at boundary,
\dots , value of tracer N at boundary 2, etc. The boundary order is the same as
in the case of hydrodynamic boundary conditions.
The writing convention is the same as that used for the hydrodynamic boundary
conditions.
The values specified by the keyword cancel the values read from the boundary
conditions file,

\item If the value is constant in time but varies along the boundary,
it will be set directly by the \telfile{TBOR} variable
in the \telkey{BOUNDARY CONDITIONS FILE},

\item If the value is constant along the boundary but varies in time,
the user may either use the \telkey{LIQUID BOUNDARIES FILE}
or specify this with the \telfile{USER\_TR3} subroutine.
The latter will be programmed somewhat like the
\telfile{USER\_VIT3, USER\_Q3} and \telfile{USER\_SL3} subroutines.
Note that the liquid boundaries file will not be taken into account
if the keywords \telkey{PRESCRIBED ELEVATIONS} and \telkey{PRESCRIBED
FLOWRATES} do not appear in the steering file.
\end{itemize}

\begin{lstlisting}[language=bash]
T       SL(1)   TR(2,1)     TR(2,2)
s       m       C         C
0       0.47    24.7      38.0
1040400 0.57    28.0      36.7
\end{lstlisting}

In the above example of a liquid boundaries file, for the indices of tracers
values, the first value is the number of the liquid boundary and the second the
number of the tracer,

\begin{itemize}
\item If the value varies in both time and space, the user should then modify
the \telfile{USER\_BORD3D} subroutine, in the part regarding the tracer.
\end{itemize}

The keyword \telkey{TREATMENT OF FLUXES AT THE BOUNDARIES} enables,
during the convection step (with the SUPG, PSI and N schemes), to set a
priority among the tracer flux across the boundary and tracer value at that wall.
Option 2 ("Priority to fluxes") will then induce a change in the tracer
prescribed value, but will bring about a good assessment of the "mass" of tracer
passing across the boundary.
On the other hand, option 1 ("Priority to prescribed values", default value)
sets the tracer value without checking the fluxes.
There are as many values as the number of liquid boundaries, contrary to
what can be done in \telemac{2D} (one single value for every liquid boundary).

Finally, in the case of an input boundary, it is possible to specify a
concentration profile in the vertical using the keyword \telkey{TRACERS
VERTICAL PROFILES}.
There are as many values to be given as the product of the number of tracers
and the number of open boundaries.
The options are:

\begin{itemize}
\item 0: user programming (in \telfile{USER\_BORD3D}),

\item 1: constant profile (default),

\item 2: constant profile (tracer diluted) or Rouse profile (sediment),

\item 3: Rouse (normalised) and imposed concentration.
\end{itemize}

\section{Physical setup}

\subsection{Active tracers}

The active tracers affect the flow through the hydrostatic pressure gradient
term. As a rule, indeed, the pressure is written as:

\begin{align}
p=p_{h} +p_{d} =\rho g(Z_{s} -z)+\rho _{0} g\int _{z}^{Z_{s} }\frac{\Delta \rho
}{\rho _{0} }  dz+p_{d}
\end{align}

where $p_{h} $ and $p_{d} $ denote the hydrostatic pressure and the dynamic
pressure, respectively.

Thus, two elements should be defined, namely: $\rho_{0} $ and $\frac{\Delta
\rho }{\rho _{0} } $.

The term $\rho _{0} $ is defined by the keyword \telkey{AVERAGE WATER DENSITY}.
The default value is 1,025; it corresponds to sea (ocean) water.

The second term, which operates in the buoyancy source terms, directly depends
on the values of the active tracers and is defined by the keyword
\telkey{DENSITY LAW}.

The available values for that keyword \telkey{DENSITY LAW} are:

\begin{itemize}
\item 0: no interaction with the tracers, except if present sediment
(default value),

\item 1: variation of density according to temperature (and sediment if present),

\item 2: variation of density according to salinity (and sediment if present),

\item 3: variation of density according to temperature and salinity
(and sediment if present),

\item 4: variation as a function of the spatial expansion coefficients
(and effect of the sediment with its own behaviour if present),

\item 5: the sediment and other tracers are forced to be passive,

\item 6: Jackett et al. law (2006) with variation of density according to
         temperature and salinity using a 25 term formula.
\end{itemize}

With the 1-3 options, the variations are given by the law as defined in
\telemac{3D}. In such a case, the name of the salinity tracer (expressed in
kg/m${}^{3}$) shall necessarily begin with SALINIT and the name of the
temperature tracer (in ${}^\circ$C) shall begin with TEMPERATURE.\\

With option 4, the term $\frac{\Delta \rho }{\rho _{0} } $ is described by a
linear function of the $T_i$ tracer of the type:
\begin{align}
\frac{\Delta \rho }{\rho _{0} } =-\sum _{i}\beta _{i}  (T_{i} -T_{i}^{0} )_{i}
\end{align}
The $\beta_{i}$ coefficients (spatial expansion coefficients) are set by the
values of the keyword \telkey{BETA EXPANSION COEFFICIENT FOR TRACERS} (default
= 0.). They can be either positive (temperature) or negative (salinity,
suspended sediment). The values $T_{i}^{0}$ are defined by the values of the
keyword \telkey{STANDARD VALUES FOR TRACERS} (default = 0.).

The user shall enter the expansion coefficients, the standard values and the
tracer names in the same sequence order to ensure that each tracer will have
the correct parameters.

\subsection{Punctual source terms}

For each source, the user shall enter the tracer value at the sources by means
of the keyword \telkey{VALUE OF THE TRACERS AT THE SOURCES}. Thus, it is an
array of reals specifying the concentration of tracers at the source. The
writing convention is as follows: source value 1~of tracer 1; source value 1~of
tracer 2;~\dots , source value 1~of tracer~$n$; source value 2~of tracer
1; \dots ,~source value 2~of tracer $n$ etc. In case of time dependent
value, it is necessary to use the source file or a specific FORTRAN programming
(\telfile{USER\_T3D\_TRSCE} subroutine).


\subsection{General source terms}

If one wants to take the tracer generation or disappearance source terms into
account, then this has to be implemented within the \telfile{SOURCE\_TRAC}
subroutine.\\

In case of calculation with consideration of rain, it is possible to specify
the contribution related to the rain with the keyword
\telkey{VALUES OF TRACERS IN THE RAIN} (default value is 0.).
It is important to note that, in the case of evaporation,
no tracer is taken into account in the water loss,
which is incorrect if the tracer is the temperature.


\section{Numerical setup}

As with hydrodynamics, the advection schemes \telkey{SCHEME FOR ADVECTION OF
TRACERS}, option for these advection schemes
\telkey{SCHEME OPTION FOR ADVECTION OF TRACERS}
(refer to subsection \ref{sec:advstep}) and the diffusion schemes
\telkey{SCHEME FOR DIFFUSION OF TRACERS} (refer to subsection
\ref{sec:difstep}) can be modified.

Otherwise, the horizontal and vertical diffusion of the tracers can be set with
the keywords \telkey{COEFFICIENT FOR HORIZONTAL DIFFUSION OF TRACERS } and
\telkey{COEFFICIENT FOR VERTICAL DIFFUSION OF TRACERS}. Their default values
are $10^{-6}$~m${}^{2}$/s.
These two keywords are arrays since release 7.1, with one value per tracer,
separated by semicolons, so that different values can be given for different
tracers.

\telkey{TREATMENT ON TIDAL FLATS FOR TRACERS} keywords is for setting,
after coverage, the value of the variable which is masked
at the diffusion step for tracer for tidal flatss.
The available options for this keyword are:
\begin{itemize}
\item 0: that option corresponds to a setting to zero of the variable on the
element (default value),
\item 1: that option sets to its prior-to-masking value.
\end{itemize}
