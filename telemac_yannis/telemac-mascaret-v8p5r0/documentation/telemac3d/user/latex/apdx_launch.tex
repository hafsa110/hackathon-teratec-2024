\chapter{Launching the computation}

A computation is launched through the \telfile{telemac3d.py} command.
That command activates the execution of a script which is common to all the
computation modules in the \tel system.
Depending on the platform, some options may be unavailable.

The syntaxes in that command are as follows:

\begin{lstlisting}[language=bash]
telemac3d.py [cas]   [--options]
\end{lstlisting}

\begin{itemize}
\item cas: name of the steering file,

\item --ncsize=NCSIZE: specifies the number of processors forced in parallel
mode, default = the number defined in the steering file,

\item -c CONFIGNAME or --configname=CONFIGNAME: specifies the configuration
name, default is randomly found in the configuration file,

\item -f CONFIGFILE, --configfile=CONFIGFILE: specifies the configuration
file, default = systel.cfg,

\item -s, --sortiefile: specifies whether there is a sortie file, default is
no,

\item -t or --tmpdirectory: the temporary work directory is not destroyed on
completion of computation.
\end{itemize}

By default, the procedure runs the computation in an interactive mode and
displays the control listing on the monitor.

The operations performed by that script are as follows:

\begin{itemize}
\item Creation of a temporary directory
(\telfile{name\_cas\_YYYY-MM-DD\_HHhMMminSSs}),

\item Duplication of the dictionary and the input files into that directory,

\item Execution of the \damo software in order to determine the work file names,

\item Creation of the computation launching script,

\item Allocation of the files,

\item Compilation of the FORTRAN file and link editing (as required),

\item Launching of the computation,

\item Retrieval of the results files and destruction of the temporary directory.
\end{itemize}

The procedure takes place with slight differences according to the selected
options.

The detailed description of that procedure can be obtained through the help
command:

\begin{lstlisting}[language=bash]
telemac3d.py --help
telemac3d.py -h
\end{lstlisting}
