\chapter{Oil spill modelling}

During a hydrodynamic simulation, \telemac{3D} offers an opportunity to follow
the paths of an oil spill which is released into the fluid from a discharge
point.

The oil spill model introduced here combines an Eulerian and a Lagrangian
approach. The Lagrangian model simulates the transport of an oil spill near the
surface. The oil slick is represented by a large set of hydrocarbon particles.
Each particle is considered as a mixture of discrete non-interacting
hydrocarbon components. Particles are therefore represented by component
categories (soluble and unsoluble components), and the fate of each component
is tracked separately. Each particle has associated to it, amongst other
properties, an area, a mass, its barycentric coordinates within the element it
is located in, and the physico-chemical properties of each of its components.
The model accounts for the main processes that act on the spilled oil:
advection, effect of wind, diffusion, evaporation and dissolution. Though
generally considered to be a minor process, dissolution is important from the
point of view of toxicity. To simulate soluble oil component dissolution in
water, an Eulerian advection-diffusion model is used. The fraction of each
dissolved component is represented by a tracer whose mass directly depends on
the dissolved mass of oil particles. The hydrodynamic data required for either
Lagrangian and Eulerian transport approach are provided by the \telemac{3D}
hydrodynamic model. The oil spill theoretical background is explained in
\cite{JolyGoeury2013}.

The result of the oil spill modelling is provided in the form of a TECPLOT
formatted file which contains the various positions of the oil particles and a
\telemac{3D} result file (SERAFIN format) storing the oil dissolved components in
water column during the computation.


\section{Input files}

In addition to the minimum set of input files necessary to run a \telemac{3D}
case, an oil spill computation also needs an oil spill steering file.
Furthermore, to run an oil spill model the subroutine \telfile{OIL\_FLOT} needs
to be modified in the FORTRAN file.


\section{Steering file}

In addition to the necessary information for running the \telemac{3D}
hydrodynamic model the following essential information must be specified in the
\telemac{3D} steering file to run an oil spill propagation model:

\begin{itemize}
\item The use of the oil spill model must be declared: \telkey{OIL SPILL MODEL}
(= YES, default = NO),

\item The name of the oil spill steering file which contains the oil
characteristics: \telkey{OIL SPILL STEERING FILE} (\telkey{= name chosen by the
user}),

\item The number of oil releases during oil spill: \telkey{MAXIMUM NUMBER OF
DROGUES} (= number chosen by the user),

\item The frequency of the drogues printout period: \telkey{PRINTOUT PERIOD FOR
DROGUES} (= number chosen by the user),

\item The name of the tecplot oil file containing the oil displacement:
\telkey{ASCII DROGUES FILE} (= name chosen by the user, default = 1).
\end{itemize}

With the oil spill module, it is possible to take into account the transport of
soluble oil components in water (whose presence has no effect on the
hydrodynamics). These may or may not be diffused within the flow but their
characteristics have to be defined in the \telkey{OIL SPILL STEERING FILE}. If
these components are allowed to diffuse in the flow, they are then treated with
the tracer transport computations of \telemac{3D}. This implies that the
\telkey{NUMBER OF TRACERS} must be set to the number of the oil soluble
components. In addition the TRACER keywords, described in chapter 7, can be
specified.


\section{Oil spill steering file}

As seen previously, the \telkey{OIL SPILL STEERING FILE} name is given by the
user in the TELEMAC steering file. This file contains all the informations for
an oil spill calculations based on the composition considered by the user,
i.e.:

\begin{itemize}
\item The number of unsoluble components in oil,

\item The parameters of these components such as the mass fraction (\%) and
boiling point of each component (K),

\item The number of soluble components in oil,

\item The parameters of these components such as the mass fraction (\%),
boiling point of each component (K), solubility (kg.m${}^{-3}$) and the mass
transfer coefficient of the dissolution and volatilization phenomena
(m.s${}^{-1}$)

\item The oil density,

\item The oil viscosity (m${}^{2}$.s${}^{-1}$),

\item The volume of the spilled oil (m${}^{3}$),

\item The water surface temperature (K),

\item The spreading model chosen by the user:

\begin{itemize}
\item Fay's model,

\item Migr'Hycar model,

\item Constant area model.
\end{itemize}
\end{itemize}

\textbf{WARNING:}

\begin{WarningBlock}{Warning:}
\begin{itemize}
\item The parameters of soluble (or unsoluble) components need to be informed
only if the number of these components is not null

\item If the sum of all mass fraction components is not equal to 1, the run is
interrupted and the following error message is displayed:
\end{itemize}

\begin{lstlisting}[language=TelemacCas]
WARNING::THE SUM OF EACH COMPONENT MASS FRACTION IS NOT EQUAL TO 1.
PLEASE, MODIFY THE INPUT STEERING FILE
\end{lstlisting}
\end{WarningBlock}

An example of the oil spill steering file is given.

\begin{lstlisting}[language=bash]
NUMBER OF UNSOLUBLE COMPONENTS IN OIL
6
UNSOLUBLE COMPONENTS PARAMETERS (FRAC MASS, TEB)
5.1D-02       ,402.32D0
9.2D-02       ,428.37D0
3.16D-01      ,458.37D0
3.5156D-01    ,503.37D0
8.5D-02       ,543.37D0
9.4D-02       ,628.37D0
NUMBER OF SOLUBLE COMPONENTS IN OIL
4
SOLUBLE COMPONENTS PARAMETERS (FRAC MASS, TEB, SOL, KDISS, KVOL)
1.D-02   ,497.05D0,  0.018D0   , 1.25D-05 ,5.0D-05
3.2D-02  ,551.52D0,  0.00176D0 , 5.63D-06 ,1.51D-05
1.D-04   ,674.68D0,  2.0D-04   , 2.D-06   ,4.085D-07
2.D-05   ,728.15D0,  1.33D-06  , 1.33D-06 ,1.20D-07
OIL DENSITY
830.D0
OIL VISCOSITY
4.2D-06
OIL SPILL VOLUME
2.02D-05
WATER TEMPERATURE
292.05D0
SPREADING MODEL (1=FAY'S MODEL, 2=MIGR'HYCAR MODEL, 3=CONSTANT AREA)
2
\end{lstlisting}

If in the oil spill steering file, the SPREADING MODEL is set to 3, two lines
must be added to the previous example:

\begin{lstlisting}[language=bash]
CONSTANT AREA VALUE CHOSEN BY THE USER FOR EACH OIL PARTICLE
1 (example if the user wants area particle equal to 1 m2)
\end{lstlisting}


\section{The OIL\_FLOT subroutine}

After inserting the \telfile{OIL\_FLOT} subroutine in the FORTRAN file, it must
be modified it in order to indicate the release time step, together with the
coordinates of the release point. If the release point coordinates are outside
the domain, the run is interrupted and an error message is displayed. In
addition, if a particle leaves the domain during the simulation, it is of
course no longer monitored but its previous track remains in the results file
for consultation.

An example of modifications in the \telfile{OIL\_FLOT} subroutine is given.

The release time step in the first condition statement and the coordinates of
the release point must be changed:

\begin{lstlisting}[language=TelFortran]
...
IF(LT.EQ.10000)THEN
  NUM_GLO=0
  NUM_MAX=0
  NUM_LOC=0
  COORD_X=0.D0
  COORD_Y=0.D0
  NUM_MAX=INT(SQRT(REAL(NFLOT_MAX)))
  DO K=1,NUM_MAX
    DO J=1,NUM_MAX
      COORD_X=336000.D0+REAL(J)
      COORD_Y=371000.D0+REAL(K)
      NUM_GLO=NUM_GLO+1
      NFLOT_OIL=0
      CALL ADD_PARTICLEADD_PARTICLE(COORD_X,COORD_Y,0.D0,NUM_GLO,NFLOT_OIL,
&                       1,XFLOTXFLOT,YFLOTYFLOT,YFLOT,TAGFLOTAGFLO,
&                       SHPFLO,SHPFLO,ELTFLO,ELTFLO,MESH,1,
&                       0.D0,0.D0,0.D0,0.D0,0,0)
...
    END DO
  END DO
END IF
\end{lstlisting}


\section{Output files}

During an oil spill computation, the \telemac{3D} software produces at least two
output files:

\begin{itemize}
\item The \telkey{3D RESULT FILE},

\item The output \telkey{ASCII DROGUES FILE}.
\end{itemize}


\subsection{The 3d result file}

This is the file in which \telemac{3D} stores information during the computation.
It is normally in SERAFIN format. First of all, it contains information on the
mesh geometry, then the names of the stored variables. It then contains the
time for each time step and the values of the different variables for all mesh
points. For complementary information on the \telkey{3D RESULT FILE}, the
reader may refer to \ref{sec:3dres}.


\subsection{The output drogues file}

This is an ASCII file created by \telemac{3D} during the computation. It
stores drogue positions in TECPLOT format. To visualize the drogue positions
with Tecplot software, the user must:

\begin{itemize}
\item Use the File$>$Load Data File(s) command to load the \telkey{3D RESULT
FILE}

\item Use the File$>$Load Data File(s) command to load the Tecplot drogue file
\end{itemize}

See section \ref{sec:drogues} for more information as it is the same file as for drogues.
