
%\paragraph{Calculation of the buoyancy source terms in the transformed mesh:}

In the transformed mesh, we have:%

\begin{equation}
F_{x}=g\frac{\Delta\rho}{\rho_{o}}\left(  \frac{\partial \eta}{\partial
x}\right)  _{y,t}-g\left[  \frac{\partial}{\partial x}\left(  \int%
\nolimits_{z}^{\eta}\frac{\Delta\rho}{\rho_{o}}\,dz\right)  _{y,z^{\ast}%
,t}+\left(  \frac{\partial z}{\partial x}\right)  _{y,z^{\ast},t}\frac
{\Delta\rho}{\rho_{o}}\right]
\end{equation}


The variable $zs=z-\eta$ is introduced; $F_{x}$ is then simplified as:%

\begin{equation}
F_{x}=-g\left[  \frac{\partial}{\partial x}\left(  \int\nolimits_{zs}^{0}%
\frac{\Delta\rho}{\rho_{o}}\,dz\right)  _{y,z^{\ast},t}+\left(  \frac{\partial
zs}{\partial x}\right)  _{y,z^{\ast},t}\frac{\Delta\rho}{\rho_{o}}\right]
\end{equation}


At this stage of reasoning, it is worth pursuing the calculation of $F_{x}$ at
the discrete level because the expression formulated above can be further
simplified. The variation of this term will be calculated between two nodes of
the mesh denoted as \textquotedblleft inf\textquotedblright\ and
\textquotedblleft sup\textquotedblright, one located just above the other,
with the knowledge that on the surface $F_{x}$ is zero:%

\[
F_{x}^{\inf}-F_{x}^{\sup}=-g\left[  \frac{\partial}{\partial x}\left(
\int\nolimits_{zs^{\inf}}^{zs^{\sup}}\frac{\Delta\rho}{\rho_{o}}\,dz\right)
_{y,z^{\ast},t}\right]
\]
%

\begin{equation}
-g\left[  \left(  \frac{\partial zs^{\inf}}{\partial x}\right)  _{y,z^{\ast
},t}\frac{\Delta\rho^{\inf}}{\rho_{o}}-\left(  \frac{\partial zs^{\sup}%
}{\partial x}\right)  _{y,z^{\ast},t}\frac{\Delta\rho^{\sup}}{\rho_{o}%
}\right]
\end{equation}


With linear functions along $z$, this gives:%

\[
F_{x}^{\inf}-F_{x}^{\sup}=-g\frac{\partial}{\partial x}\left[  \frac{1}%
{2}\left(  zs^{\sup}-zs^{\inf}\right)  \left(  \frac{\Delta\rho^{\sup}}%
{\rho_{o}}+\frac{\Delta\rho^{\inf}}{\rho_{o}}\right)  \right]  _{_{y,z^{\ast
},t}}%
\]
%

\begin{equation}
-g\left[  \left(  \frac{\partial zs^{\inf}}{\partial x}\right)  _{y,z^{\ast
},t}\frac{\Delta\rho^{\inf}}{\rho_{o}}-\left(  \frac{\partial zs^{\sup}%
}{\partial x}\right)  _{y,z^{\ast},t}\frac{\Delta\rho^{\sup}}{\rho_{o}%
}\right]
\end{equation}
and we finally get:%

\[
F_{x}^{\inf}-F_{x}^{\sup}=\frac{g}{2}\left[  \left(  \frac{\Delta\rho^{\sup}%
}{\rho_{o}}-\frac{\Delta\rho^{\inf}}{\rho_{o}}\right)  \frac{\partial
}{\partial x}\left(  zs^{\sup}+zs^{\inf}\right)  _{y,z^{\ast},t}\right]
\]
%

\begin{equation}
-\frac{g}{2}\left[  \left(  zs^{\sup}-zs^{\inf}\right)  \frac{\partial
}{\partial x}\left(  \frac{\Delta\rho^{\sup}}{\rho_{o}}+\frac{\Delta\rho
^{\inf}}{\rho_{o}}\right)  _{y,z^{\ast},t}\right]
\end{equation}


The two derivatives along $x$ are discontinuous functions at the nodes of the
mesh, and are calculated as follows:

\begin{itemize}
\item For every node $M$, we can define the set $ELEM(M)$ of elements of the
mesh having the node $M$ among their vertices.

\item For each of the elements $i$ of $ELEM(M)$ we can calculate the $\left(
\frac{\partial f}{\partial x}\right)  _{i}$ derivative of $f$ at $M$ on the
side of the element $i$.
\end{itemize}

The derivative of $f$ at $M$ is then estimated by:%

\begin{equation}
\left(  \frac{\partial f}{\partial x}\right)  \left(  M\right)  =\frac
{\sum\limits_{i\;\in\;ELEM(M)}\int\limits_{\Omega}\left(  \frac{\partial
f}{\partial x}\right)  _{i}\Psi_{M}d\Omega}{\int\limits_{\Omega}\Psi
_{M}d\Omega}%
\end{equation}
where $\Psi_{M}\,$is the basis function associated to node $M$.