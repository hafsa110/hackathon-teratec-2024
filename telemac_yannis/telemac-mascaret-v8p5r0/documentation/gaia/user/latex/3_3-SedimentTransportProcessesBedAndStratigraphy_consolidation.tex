\pagebreak
%-------------------------------------------------------------------------------
\section{Consolidation processes}
%-------------------------------------------------------------------------------
For the current version of \textsc{Gaia}, consolidation processes are based on the semi-empirical formulation originally developed by Villaret and Walther~\cite{VillaretWalther2008}, which uses the iso-pycnal and first-order kinetics formulations. Consolidation of mud deposits is modeled using a layer discretization, where the first layer corresponds to the freshest deposit, while the lower layer is the most consolidated layer. Sediment deposition from the water column is added directly to the first layer.
A rate (or \textit{flux}) of consolidation is computed for each layer and for each class of cohesive sediment separately. The values of the computed fluxes depend on the availability of each class in the layer considered.

%The keyword {\ttfamily MUD CONSOLIDATION} (logical type, set to {\ttfamily = NO} by default) activates consolidation processes in \gaia{}. Two different models for consolidation are available with the keyword {\ttfamily CONSOLIDATION MODEL} (logical type, set to {\ttfamily = 1} by default):
Consolidation can be activated using the keyword \telkey{BED MODEL = 2} (integer type variable, set to {\ttfamily = 1} by default).

%\begin{itemize}
%\item Multilayer model ({\ttfamily = 1}): This model was originally developed by Villaret and Walther~\cite{} by mixing two
%approaches of iso-pycnal and first-order kinetics. In this model, the muddy bed is discretised into a fixed number of layers.
%Each layer $j$ is characterised by its mass concentration $C_j$ [kg/m$^3$], its mass per unit surface $M_s(j)$
%[kg/m$^2$], its thickness $ep_j$ [m] and a set of mass transfer coefficient $a_j$ (s$^{-1}$). This empirical model assumes that the vertical flux of
%sediment from layer $j$ to underneath layer $j+1$ is proportional to the mass of sediments $M_s(j)$ contained in the layer $j$.
%
%\item The Gibson/Thiebot's model ({\ttfamily = 2}): This is a 1DV sedimentation-consolidation multi-layer model, based on an original
%technique to solve the Gibson equation, developed by Thiebot et al.~\cite{thiebot08}. The advantage of
%this representation is that the flux of sedimentation and consolidation is calculated based on
%the Gibson theory. In this model, the concentration of different layers are fixed, the associated thicknesses are directly
%linked to the amount of sediment that they contain. The scheme of this model is similar to the multilayer model.
%However, instead of using the transfer coefficients which are
%arbitrary, this model is based on the Gibson's theory for the definition of the settling velocity
%of solid grains and the determination of mass fluxes
%\end{itemize}

\subsection{Associated keywords for consolidation models}
\begin{itemize}
\item Multilayer model (\telkey{BED MODEL = 2})
\begin{itemize}
\item {\ttfamily LAYER MASS TRANSFER} (real list, set to {\ttfamily = 5.D-05;4.5D-05;...} by default) provides the mass transfert coefficients of the multilayer consolidation model (in s$^{-1}$)
\end{itemize}
%\item For the Gibson/Thiebot's model ({\ttfamily = 2}), the following values are used in the closure relationship equation for the permeability:
%\begin{itemize}
%%\item {\ttfamily GEL CONCENTRATION} (real type, set to {\ttfamily = 310.D0} kg/m$^3$ by default) is the transition concentration between the sedimentation and consolidation schemes
%%\item {\ttfamily MAXIMUM CONCENTRATION} (real type, set to {\ttfamily = 364.D0} kg/m$^3$ by default) is the maximum concentration for the Gibson/Thiebot's model
%%\item {\ttfamily PERMEABILITY COEFFICIENT} (real type, set to {\ttfamily = 8.D0} by default) %CHECK IF OK FOR MODEL 2 AND UNITS?
%\end{itemize}
\end{itemize}

Further information about both models can be found in~\cite{Lan12}.

The parameters per layer of the consolidation (cohesive sediment concentration, critical erosion shear stress, and rate of mass transfer to the layer underneath) model are set using the following keywords respectively: {\ttfamily LAYERS MUD CONCENTRATION}, {\ttfamily LAYERS CRITICAL EROSION SHEAR STRESS OF THE MUD}, {\ttfamily LAYERS PARTHENIADES CONSTANT} and {\ttfamily LAYERS MASS TRANSFER}.

\subsubsection{Consolidation fluxes}
The transfer of mass of sediment from one layer ({\ttfamily ILAYER}) to the more consolidated layer ({\ttfamily ILAYER+1}) below is computed according to the following law: %~\ref{} (Regis, reference ?)
\begin{equation}
  \frac{dM(ILAYER)}{dt}=TRANS\_MASS(ILAYER )\times M(ILAYER)
\end{equation}
With $M$ mass of sediment in the layer (kg/m$^2$) and $TRANS\_MASS$ the rate of mass transfer (s$^{-1}$).
This proposed law to model consolidation could of course be adapted or changed by the user inside subroutine \texttt{bed1\_consolidation\_layer.f}
