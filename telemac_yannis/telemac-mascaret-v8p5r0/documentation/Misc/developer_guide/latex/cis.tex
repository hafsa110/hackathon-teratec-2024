%
%%%%%%%%%%%%%%%%%%%%%%%%%%%%%%%%%%%%%%%%%%%%%%%%%%%%%%%%%%%%%%%%%%%%%%%%
\chapter{CIS}
%%%%%%%%%%%%%%%%%%%%%%%%%%%%%%%%%%%%%%%%%%%%%%%%%%%%%%%%%%%%%%%%%%%%%%%%
%
%
%\section{Launch a test}
%
%
%To launch a test you must go on the CIS main page then click on your branch tab.
%This will lead you to the page on Fig~\ref{fig:cis-main}.
%
%\begin{figure}[H]
%    \centering
%    \includegraphics[scale=0.35]{graphics/cis-main.jpg}
%    \caption{CIS Branch Page}
%    \label{fig:cis-main}
%\end{figure}
%
%Click on the button in the red circle to launch the job. If you do not see that
%button check that you are logged in. This will lead you to the page on Fig~\ref{fig:cis-run-job}.
%
%\begin{figure}[H]
%    \centering
%    \includegraphics[scale=0.35]{graphics/cis-run-job.jpg}
%    \caption{CIS Job execution}
%    \label{fig:cis-run-job}
%\end{figure}
%
%To run the validation on your branch the following information are needed:
%\begin{itemize}
%\item The modules to run, if you want to run the simulation on only one module
%do not forget to add partel and gretel as well to be able to run in parallel,
%for exemple "partel.telemac2d.telemac3d.gretel" will run the validation on
%telemac2d and telemac3d.

%Writing "system" will run the validation on all the modules this is what you
%need to validate a development before integration.
%\item The level of complexity of test cases to run. The higher the number the
%longer the test cases simulation is. For validation of a development 0 is
%needed.
%\end{itemize}

If you want to run the validation of your developments locally, you need to type one of the following commands:
\begin{lstlisting}[language=bash]
validate_telemac.py --tags module
validate_telemac.py --ncsize=3
validate_telemac.py
\end{lstlisting}
The first one will launch the validation of a specify list of modules (for example
"telemac2d", "tomawac", "artemis").
The second one will launch the validation of the whole system but for the
parallel test cases it will replace the number of processors by 3.
The third one will run the validation on the whole system.
The list of authorised modules is:
\begin{itemize}
\item telemac2d
\item telemac3d
\item mascaret
\item courlis
\item nestor
\item tomawac
\item aed
\item waqtel
\item python2
\item python3
\item apistudy
\item api$\_$mascaret
\item coupling
\item postel3d
\item stbtel
\item khione
\item artemis
\item gaia
\item gotm
\item full$\_$valid
\item med
\item fv
\item api
\end{itemize}

%
%
%\section{Get the listing of the test}
%
%
%To see the output of the job you need to follow the step described in
%Fig~\ref{fig:cis-to-listing}. For step 2 you need to click on the Linux
%distribution you want want to get the listing from.  For information only the
%ubuntu configuration runs the test cases in parallel.
%
%\begin{figure}[H]
%    \centering
%    \includegraphics[scale=0.3]{graphics/cis-to-listing.jpg}
%    \caption{Process to get the listing of a job execution}
%    \label{fig:cis-to-listing}
%\end{figure}
%
%On step 4, clicking on the "raw" button will give you the complete listing as
%the "Console Output" button will only give you the tail of the output that you
%can see on Fig~\ref{fig:cis-listing}.
%
%\begin{figure}[H]
%    \centering
%    \includegraphics[scale=0.35]{graphics/cis-listing.png}
%    \caption{Listing of a job execution}
%    \label{fig:cis-listing}
%\end{figure}
%
