%
%%%%%%%%%%%%%%%%%%%%%%%%%%%%%%%%%%%%%%%%%%%%%%%%%%%%%%%%%%%%%%%%%%%%%%%%
\chapter{Useful stuff}
%%%%%%%%%%%%%%%%%%%%%%%%%%%%%%%%%%%%%%%%%%%%%%%%%%%%%%%%%%%%%%%%%%%%%%%%
%
\section{Little script to check part of the coding conventions}
%
In the main branch or after (V7.0), you can sue the command
\verb!compile_telemac.py --check! that will scan your source code and check for
a few points of the coding conventions. You should run this script before
submitting your development. The lines with issue are written in the
\verb!check_code.log! file

\section{Adding a new test case}
%
\label{testcase}
This section will guide you through the hard but needed action of adding a new
case do not frighten it is not that hard. First of all, you need to create a new
folder in the examples of the folder corresponding to the module the test case
is for. That folder must contain the following elements:
\begin{itemize}
\item All the \textbf{input} files you need to run the case, and just the input of
the files generated by a successful run should be in the GitLab repository.
See the table~\ref{tab:namingconv} below for the convention for the naming of
the files.
\item A reference file and/or data to do the validation.
\item The \verb!doc! folder which contains the documentation for the test case
in LaTeX format (See other test cases for example).
\item The Python script file starting with prefix vnv\_ to run the test case
  (See other test cases for example).
\end{itemize}
All the Selafin file must have the extension ".slf", the steering file the
extension ".cas".

\begin{table}[H]
\begin{center}
%
\caption{Table with contents ranging over several cells horizontally and vertically.}%
\label{tab:namingconv}
%
\begin{tabular*}{0.9\textwidth}{@{\extracolsep{\fill}}cccccc}
\toprule
\toprule
            & geometry, boundary & reference & results & restart & steering and others \\
\midrule
\telemac{2D} & geo & f2d & r2d & i2d & t2d \\
\telemac{3D} & geo & f3d & r3d & i3d & t3d \\
\tomawac    & geo & fom & r2d & ini & tom \\
\stbtel     & geo & xxx & r2d & ini & stb \\
\gaia       & geo & gai\_ref & gai & ini & gai \\
\postel     & geo & xxx & res & xxx & p3d \\
\waqtel     & geo & xxx & xxx & xxx & waq \\
\artemis    & geo & frt & r2d & ini & art \\
\end{tabular*}
%
\end{center}
\end{table}

NB: \waqtel results are tracers values included in \telemac{2D} or \telemac{3D}
results files, that is why the 3 columns corresponding to reference, results and
restart are not filled.
