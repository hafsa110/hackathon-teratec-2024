\chapter{The procedure at EDF R\&D}

The procedure \verb!H-D03-2011-01748-FR! ''Les actions de management de projet
applicables à EDF R\&D et leur modulation'' from the R\&D Quality system
division~\cite{projmanag} states the requirements needed at EDF
R\&D to submit, define, contractualize, report the state and conclude a R\&D
project, as said in the referential 1999 of the EDF project management and
following the EDF steering process. Especially, it gives the following
definition of a Software Quality Plan: ''The Quality plan is a document which
describes the standard for technical, administrative, financial, timetable
order as well as the actions to apply for organisation, steering and managing
off the project. It describes the timetable of the project, the product
expected and all the elements about handling problems.'' (translated from the
original in French). It also describes the processes and actions aimed at
maintaining the quality of the software and its continuous improvement.\\

The document~\cite{guidepql} was also used as a guideline to write this
Software Quality Plan.\\

The document~\cite{capelemnum} helps to classify numerical developments by
dividing them into four categories and outlining the steps to take to
capitalise on each one.\\

The document~\cite{Ocs}, if not specific to the \telemacsystem{}, gives good
guidelines on how to handle Numerical Software.

\section{Documents specific to the \telemacsystem}

The practical documents for the respect of the Software Quality Plan are the
following:
\begin{itemize}
\item All the documents referenced in the SQP given in appendix: the
  development plan, the organisation of the \telemacsystem{} activity and a
  nominative list of the people in charge of the \telemacsystem{}.
\end{itemize}
