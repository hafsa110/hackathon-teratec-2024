\chapter{Dry elements elimination}
when constructing models and during a simulation, (on the fluvial domain for
example, or as for a dam break simulation) it could be well-advised to
eliminate from the mesh the dry elements.\\
It can be realised by \stbtel from the result file of \telemac{2D} (or from
others simulation codes producing a result file with the information “level of
water”). It can be activated with the logical keyword \telkey{DRY ELEMENTS
ELIMINATION} (default value NO). \stbtel re-read the entire result file and
determines the nodes mesh that keep dry during the simulation. The user
dimensioned the level of water (in meter) under which we consider that a node
is dry. The user is helped with the keyword \telkey{DRY LIMIT} (default value
0.1). The dry elements are removed of the mesh and the boundaries are re-count
(with island creation if necessary).  By default, only the completely dry
elements are removed (the three triangle points keep dry). In order to remove
completely the dry zones on the domain, the user can specify the partially dry
elements elimination (at least of one dry point) with the keyword
\telkey{PARTIALLY DRY ELEMENTS ELIMINATION} (default value NO)\\
On normal mode, \stbtel stores in the output file only the last time step read
on the \telemac{2D} file. The user can ask for the storage of all the time step
with the keyword \telkey{STORAGE OF EVERY TIME STEP} ( default value NO).
