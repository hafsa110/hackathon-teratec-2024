\chapter{Sliding}
%

% - Purpose & Problem description:
%     These first two parts give reader short details about the test case,
%     the physical phenomena involved and specify how the numerical solution
%     will be validated
%
\section{Purpose}
%
The purpose of this test is to show the effect of the two sliding formulations
available in \gaia{}.

%
\section{Problem setup}
%
The 160 m long and 11 m wide flume is taken from the sandpit test case.
The bottom was scaled by factor 10 in order to get higher slopes (see Figure
~\ref{slide:bottom}). All boundaries are closed set and the initial 
water levels are set constant in order to prevent a flow.

One sediment class with a diameter of 0.1 mm is used. Normal sediment transport
is avoided by using Meyer-Peter and Müller formula with a MPM factor of zero.
Therefore the bottom evolution origins only from the sliding formulations.


\begin{figure}[!h]
\centering
\includegraphicsmaybe{[width=.96\textwidth]}{../img/sliding_bathy.png}
 \caption{Longitudinal section at initial topography.}\label{slide:bottom}
\end{figure}

\section{Numerical setup}
%
Numerical simulations were conducted on an unstructured, triangular finite
element mesh with $1600$ elements and $819$ nodes (Figure~\ref{slide:mesh}).
The simulation started with a constant water level of 2.55 m.
The initial water levels are set to 2.55 m.
The two sliding formulations  (\telkey{SEDIMENT SLIDE = 1 or 2}) are compared
using different slope effects
 (\telkey{FORMULA FOR SLOPE EFFECT = 1 to 3} and \telkey{FORMULA FOR DEVIATION
= 1 to 3}).
The angle of repose is set to \telkey{FRICTION ANGLE OF SEDIMENTS = 20 }.

\begin{figure} [!h]
\centering
\includegraphicsmaybe{[width=.96\textwidth]}{../img/sliding_mesh.png}
 \caption{Discretization of the flume.}\label{slide:mesh}
\end{figure}

The time step is set to $1$ s and the simulation runs for 1000 s in order to
reach steady conditions with bottom slopes equals $\leq 20^\circ$.


\section{Results}
%
Numerical results of the final bottom using the two sliding formulas
\textit{slope smoothing} or \textit{avalanching} are shown in Figure
~\ref{slide:results1}. The avalanching formula calculates no significant
differences in the cross sections as expected. The slope smoothing seems to be
influenced by the mesh and creates bottom differences in y-direction.

The comparison between both sliding formulas and the given angle of repose is
presented in Figure~\ref{slide:results2} using a longitudinal flume section.
It can be
seen that with the avalanching formula the angle of repose is reached. With the
bottom smoothing formula the final angle is lower than the prescribed angle of
repose.
Furthermore, as expected different slope and deviation formulas show no
influence for none of the sliding formulas.

Figure~\ref{slide:results3} and \ref{slide:results4} present the bottom
evolutions at different time steps for both sliding formulations and Koch and
Flokstra slope and deviation formulas. The bottom smoothing sliding formula
reaches the
steady state much later than the avalanching formula.

With the help of the avalanching formula a fast and grid independent solution
to avoid steep slopes can be achieved.

\begin{figure} [!h]
  \centering
\includegraphicsmaybe{[width=.96\textwidth]}{../img/final_evol.png}
\includegraphicsmaybe{[width=.96\textwidth]}{../img/final_evol2.png}
\caption{Simulated final bottom using slope and deviation effect of Koch and
Flokstra with \textit{slope smoothing} sliding formula 1 (top) and with
\textit{avalanching} sliding formula 2 (bottom).}
\label{slide:results1}
\end{figure}

\begin{figure} [!h]
\centering
\includegraphicsmaybe{[width=.96\textwidth]}{../img/sliding_bottom.png}
 \caption{Comparison of simulated bottom elevation at a longitudinal flume
section for all sliding and slope formulations.}\label{slide:results2}
\end{figure}

\begin{figure} [!h]
\centering
\includegraphicsmaybe{[width=.96\textwidth]}{../img/sliding1_bottom_time.png}
 \caption{Simulated bottom elevation at a longitudinal flume section for the
 sliding formula 1, slope 1 and different time steps}.\label{slide:results3}
\end{figure}

\begin{figure} [!h]
\centering
\includegraphicsmaybe{[width=.96\textwidth]}{../img/sliding2_bottom_time.png}
 \caption{Simulated bottom elevation at a longitudinal flume section for the
sliding formula 2 and slope 1 and different time steps}.\label{slide:results4}
\end{figure}

