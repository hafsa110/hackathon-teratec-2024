\chapter{Littoral}
%

% - Purpose & Problem description:
%     These first two parts give reader short details about the test case,
%     the physical phenomena involved and specify how the numerical solution will be validated
%
\section{Purpose}
%
This test case illustrates the setup of a three-way coupling problem waves,
currents and sediment transport.

%
\section{Description of the problem}
A wave, current and sediment transport simulation in a straight, uniform
stretch of coastline is considered. The beach is located at $y = 200$~m, the
sloping bed is imposed in subroutine \texttt{corfon}. The offshore depth is
$10$~m.
This is the classical test case of a rectilinear beach with sloping bed
The model allows to calculate the littoral transport.
!
This test case illustrates the effect of waves which is :
\begin{itemize}
\item to generate the current induced littoral current parallel to the beach
\item to increase the sand transport rate using the Bijker sand transport
  formula.
  \end{itemize}

\section{Physical parameters}

\subsection{Geometry and Mesh}
A domain of $200\times 1000$~m$^2$ is considered, with a regular mesh with
elements size of the order $\Delta x=20$~m and $\Delta y=5$~m
The beach is 1000 m long, 200 m wide
 The beach slope (Y=200m) is 5\% and defined in corfon.f
 The water depth along the open boundary (Y=0) is h=10m
We use a trianglular regular grid

\section{Initial and Boundary Conditions}
%
\subsection{Wave conditions}
Incoming waves (waves height , period and directions) are imposed offshore at
$y=0$, such that $H_s=1$~m, $T_p=8$~s. The Jonswap spectrum is used. The waves
direction is $30$~deg relative to the $y-$axis.
The mesh is as shown on Figure \ref{littoralmesh}
\begin{figure} [!h]
\centering
\includegraphicsmaybe{[width=0.85\textwidth]}{../img/fond.png}
 \caption{mesh of the case littoral}
\label{littoralmesh}
\end{figure}

\section{Numerical parameters}
%
$\Rightarrow $ Offshore (Y=0): Offshore wave imposed/no littoral current/no
set up

Tomawac:
The wave height is imposed on the offshore boundary (5 4 4) (Hs=1m), for a
wave period (Tp=8s).

Telemac2D:
The current and free surface are imposed to 0 along the offshore boundary
(5 5 5).

% - Results:
%     We comment in this part the numerical results against the reference ones,
%     giving understanding keys and making assumptions when necessary.
%
%
\section{Results}
%
Results (littoral current and transport rates) as well as wave set up/set down
are in good agreement with expectations from theoretical classical results
(Longuet Higgins).The model is able to reproduce the wave induced current, as
well as the effect of set down/set up as the waves dissipate in the breaking
zone. The sediment transport rate is located in the near shore breaking zone,
where the longshore current is generated.
Similar results for the littoral transport could be obtained by using an
integrated formula (e.g. CERC formula).

The results are presented Figures \ref{resultsT2Dg} (Velocity U)
\ref{resultsTOMg}(Wave heigth Hm0) and  \ref{resultsGAI} (Bed Shear stress)
\begin{figure} [!h]
\centering
\includegraphicsmaybe{[width=0.85\textwidth]}{../img/resultsT2D.png}
 \caption{Velocity along U of the case littoral}
\label{resultsT2Dg}
\end{figure}
\begin{figure} [!h]
\centering
\includegraphicsmaybe{[width=0.85\textwidth]}{../img/resultsTOM.png}
 \caption{Wave heigth Hm0 of the case littoral}
\label{resultsTOMg}
\end{figure}
\begin{figure} [!h]
\centering
\includegraphicsmaybe{[width=0.85\textwidth]}{../img/resultsGAI.png}
 \caption{Bed shear Stress of the case littoral}
\label{resultsGAI}
\end{figure}

\section{A new way to couple telemac2d and tomawac}
On the same domain we test a new way to couple telemac2d and tomawac with
different meshes, this module is called tel2tom \cite{breugem2019}. It
requires to calculate weigths of interpolation before, those weights are part
of the mesh. This is done using

{\it run\_telfile.py tel2tom t2dmesh tommesh --t2d-bnd Telemac Boundary File
  --tom-bnd Tomawac Boundary File}.

The user will find a notebook on this use in {\it
  \$HOMETEL/notebooks/pretel/tel2tom.ipynb}. In practice, in the VNV processus
instead of using {\it run\_telfile.py}, the function  {\it connect\_tel2tom}
is called in the python.

In a first time we are using tel2tom using the same mesh, and in a second time
using different meshes.

 \begin{figure} [!h]
 \centering
 \includegraphicsmaybe{[width=0.85\textwidth]}{../img/fondTEL2TOM.png}
 \includegraphicsmaybe{[width=0.85\textwidth]}{../img/fondTOM2TEL.png}
 \label{fondTEL2TOMg}
 \caption{mesh of telemac2d (top) and tomawac (bottom)}
 \end{figure}
  
\subsection{Results using the same mesh}
We can see that the results of Figure \ref{resultsTEL2TOMg} \ref{resultsTOM2TELg}
and \ref{resultsGAITEL2TOM} are the same of the ones of, respectively,
\ref{resultsT2Dg} \ref{resultsTOMg} and \ref{resultsGAI}. Seeing the figures does
not show that results are exactly the same, but very close. But if we use the
python scripts to compute the differences between both results, we see that
difference is 0.

\begin{figure} [!h]
\centering
\includegraphicsmaybe{[width=0.85\textwidth]}{../img/resultsTEL2TOM.png}
\caption{Velocity along U of the case littoral using tel2tom with the same
  mesh}
\label{resultsTEL2TOMg}
\end{figure}
\begin{figure} [!h]
\centering
\includegraphicsmaybe{[width=0.85\textwidth]}{../img/resultsTOM2TEL.png}
 \caption{Wave heigth Hm0 of the case littoral using tel2tom with the same mesh}
\label{resultsTOM2TELg}
\end{figure}
\begin{figure} [!h]
\centering
\includegraphicsmaybe{[width=0.85\textwidth]}{../img/resultsGAITEL2TOM.png}
\caption{Bed shear Stress of the case littoral using tel2tom with the same
  mesh}
\label{resultsGAITEL2TOM}
\end{figure}

\subsection{Results using different meshes}
We can see that the results of Figure \ref{resultsTEL2TOMdiffg}
\ref{resultsTOM2TELdiffg} and \ref{resultsGAITEL2TOMdiff} are closed to the
ones of, respectively, \ref{resultsTEL2TOMg} \ref{resultsTOM2TELg} and
\ref{resultsGAITEL2TOM}.

\begin{figure} [!h]
\centering
\includegraphicsmaybe{[width=0.85\textwidth]}{../img/resultsTEL2TOMdiff.png}
\caption{Velocity along U of the case littoral using tel2tom with different
  mesh}
\label{resultsTEL2TOMdiffg}
\end{figure}
\begin{figure} [!h]
\centering
\includegraphicsmaybe{[width=0.85\textwidth]}{../img/resultsTOM2TELdiff.png}
\caption{Wave heigth Hm0 of the case littoral using tel2tom with different
  mesh}
\label{resultsTOM2TELdiffg}
\end{figure}
\begin{figure} [!h]
\centering
\includegraphicsmaybe{[width=0.85\textwidth]}{../img/resultsGAITEL2TOMdiff.png}
\caption{Bed shear Stress of the case littoral using tel2tom with different
  mesh}
\label{resultsGAITEL2TOMdiff}
\end{figure}

% Here is an example of how to include the graph generated by validateTELEMAC.py
% They should be in test_case/img
%\begin{figure} [!h]
%\centering
%\includegraphics[scale=0.3]{../img/mygraph.png}
% \caption{mycaption}\label{mylabel}
%\end{figure}


