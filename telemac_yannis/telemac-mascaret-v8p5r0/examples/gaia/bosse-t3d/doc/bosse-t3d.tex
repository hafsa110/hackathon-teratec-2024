\chapter{bosse-t3d}
%

% - Purpose & Problem description:
%     These first two parts give reader short details about the test case,
%     the physical phenomena involved and specify how the numerical solution will be validated
%
\section{Purpose}
The simulation of the propagation of an initially symmetric bump in a three-dimensional flow field is proposed in this test case.
As for the case \textit{bosse-t2d}, this case assesses the capability of the model at propagating an isolated bedform subject to a unidirectional hydrodynamic forcing.

\section{Description}

\subsection{Geometry and mesh discretization}
%
The computational domain consists of a rectangular flume with dimensions $16$m long times $1$m wide, discretized with 1600 triangular elements, see figure \ref{fig:bosse-t3d:mesh}. The three-dimensional discretization is obtained by first dividing the two-dimensional domain with non-overlapping linear triangles and followed by extruding each triangle along the
vertical direction into linear prismatic columns that exactly fitted the bottom and the free surface. Then, each column
is partitioned into non-overlapping layers. For this test case, 15 superimposed layers are used along the vertical direction.

\begin{figure}[H]
 \centering
 \includegraphicsmaybe{[width=\textwidth]}{../img/bathy_mesh.png}
 \caption{Mesh discretization and bathymetry of the \textit{bosse-t3d} test case.}
 \label{fig:bosse-t3d:mesh}
\end{figure}
%
\subsection{Bathymetry}
%
As for the \textit{bosse-t2d} case, the bottom is flat with a finite amplitude perturbation of the bed level (see figure~\ref{fig:bosse-t3d:bottom}) given by equation:
\begin {equation}
z_b=\left\{
\begin{array}{ll}
\displaystyle
0.1\sin^2\left(\frac{\pi (x-2)}{8}\right), & \text{si 2m $\leq x \leq$ 10m}  \\
\displaystyle
 0, & \text{otherwise} \label{eq:topographie_initiale} \\
\end{array}
\right.
\end{equation}

\begin{figure}[H]
 \centering
 \includegraphicsmaybe{[width=\textwidth]}{../img/InitialBottom.png}
 \caption{1D profile of initial bed level.}
 \label{fig:bosse-t3d:bottom}
\end{figure}

\subsection{Initial conditions}
%
The hotstart file \texttt{bosse-t2d\_init} computed from a \telemac{2d} simulation provides the initial conditions for the velocity field and water-depth over the computational domain.
%
\subsection{Boundary conditions}
%
At the left boundary we set a discharge equal to $0.25$ m$^3$s$^{-1}$ and an equilibrium sediment discharge. At the right boundary, the water surface elevation is set to $0.6$ m and a free boundary condition is set for sediments. Lateral boundaries are considered as solid walls without friction.
%

% - Numerical parameters:
%     This part is used to specify the numerical parameters used
%     (adaptive time step, mass-lumping when necessary...)
%
\subsection{Physical and numerical parameters}
%

The bottom friction is described by the Stricker law, with a coefficient equal to $50~m^{1/3}s^{-1}$. The horizontal turbulence is modelled by a constant viscosity model with a horizontal diffusion coefficient equal to $\nu_h=10^{-1}~m^2s^{-1}$. The vertical turbulence is parameterized with a mixing-lenght model with a vertical diffusion coefficient equal to $\nu_v=10^{-6}~m^2s^{-1}$. 

The bed is composed of sediments with a constant median diameter $d_m=0.15~$ mm and a porosity equal to $0.375$. The sediment transport mechanism considered for this test case is bedload. The sediment transport capacity is computed with the (total) Engelund and Hansen parameterization formula.

The time step is set to $\Delta t=1~$s and the duration is equal to $14000~$s.
%
\section{Results}
Figure \ref{fig:bosse-t3d:solution} shows the evolution of the bedform at times $3000$s, $6000$s, $9000$s and $12000$s. The bump exhibits the characteristics behaviour of propagating sand-wave that evolves and start developing a shock wave, without decreasing of its amplitude.


\begin{figure}[H]
 \begin{minipage}[t]{0.5\textwidth}
  \centering
  \includegraphicsmaybe{[width=\textwidth]}{../img/bottom3.png}
 \end{minipage}%
 \begin{minipage}[t]{0.5\textwidth}
  \centering
  \includegraphicsmaybe{[width=\textwidth]}{../img/bottom6.png}
 \end{minipage}
 \\
  \begin{minipage}[t]{0.5\textwidth}
  \centering
  \includegraphicsmaybe{[width=\textwidth]}{../img/bottom9.png}
 \end{minipage}%
 \begin{minipage}[t]{0.5\textwidth}
  \centering
  \includegraphicsmaybe{[width=\textwidth]}{../img/bottom12.png}
 \end{minipage}
 \caption{Bed evolution at times $3000$s, $6000$s, $9000$s and $12000$s.}
 \label{fig:bosse-t3d:solution}
\end{figure}


%TO DO: Add error computation between analytical and numerical solution?
\section{Conclusion}
This test shows that \gaia{} is able to reproduce the propagation of an isolated bedform, subject to a unidirectional flow in a three-dimensional domain.
%
%
%
%


% Here is an example of how to include the graph generated by validateTELEMAC.py
% They should be in test_case/img
%\begin{figure} [!h]
%\centering
%\includegraphics[scale=0.3]{../img/mygraph.png}
% \caption{mycaption}\label{mylabel}
%\end{figure}


