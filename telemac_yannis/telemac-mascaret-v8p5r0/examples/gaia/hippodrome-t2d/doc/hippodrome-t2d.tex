\chapter{hippodrome-t2d}
%

% - Purpose & Problem description:
%     These first two parts give reader short details about the test case,
%     the physical phenomena involved and specify how the numerical solution will be validated
%
\section{Purpose}
A racetrack shape configuration has been adopted during the earlier developments of \gaia{} to assess its conservativeness properties, to test its ability at reproducing bed and layer thicknesses evolutions and to optimize the code implementation within the new module structure. The same configuration has been chosen to test different sediment transport processes in 2D, namely:
\begin{itemize}
\item \texttt{t2d\_1COs}: suspended sediment transport, cohesive sediment, 1 sediment class.
\item \texttt{t2d\_1NCOb}: bedload transport, non-cohesive sediment, 1 sediment class.
\item \texttt{t2d\_1NCOb\_vf}: idem previous case, finite volume.
\item \texttt{t2d\_1NCOs}: suspended sediment transport, non-cohesive sediment, 1 sediment class.
\item \texttt{t2d\_4NCOb}: bedload transport, non-cohesive sediment, 4 sediment classes.
\item \texttt{t2d\_4NCOb\_vf}: idem previous case, finite volume.
\item \texttt{t2d\_4NCOb\_strat\_vf}: bedload transport, non-cohesive sediment, 4 sediment classes, stratigraphy discretization with 3 layers.
\end{itemize}

To simplify the involved physical processes, the wind is considered as the only driving force of the flow. This test case can be useful to users who want to test their own developments on a simple configuration. The documentation presented here only refers to the case \texttt{t2d\_1NCOb}.


\section{Description}

\subsection{Geometry, initial bathymetry and mesh discretization}
%
The computational domain and finite element discretization are showed in Figure~\ref{fig:hippodrome-t2d:mesh}. 

\begin{figure}[H]
 \centering
 \includegraphicsmaybe{[width=\textwidth]}{../img/hippodrome-t2d_mesh.png}
 \caption{Mesh discretization of the \textit{hippodrome-t2d} test case.}
 \label{fig:hippodrome-t2d:mesh}
\end{figure}
%
\subsection{Bathymetry}
%
The bump and the lateral banks in the initial bathymetry favor the bed evolution on both longitudinal and lateral slopes, see Figure~\ref{fig:hippodrome-t2d:solution}. Lateral banks allow the formation of dry areas in the computational domain.

\subsection{Initial conditions}
%
A fluid-at-rest is imposed as initial condition with constant elevation equal to $5$m and zero velocity field.

\subsection{Boundary conditions}
%
No liquid boundaries are included in the numerical simulations. Only wall-type conditions are imposed at the internal and external domain boundaries, see Figure~\ref{fig:hippodrome-t2d:mesh}.

\subsection{Physical and numerical parameters}
%
This test case accounts for bedload transport of uniform, non-cohesive sediment of diameter $0.0002$m. Sediment fluxes are computed with the Soulsby-van Rijn total sediment transport capacity formula.

\section{Results}
Figure \ref{fig:hippodrome-t2d:solution} shows the evolution of the bottom from time $0$s to $10000$s for case \texttt{t2d\_1NCOb}. As expected, scouring processes next to the bump as well as sediment accretion inside the trench are observed.


\begin{figure}[H]
 \begin{minipage}[t]{0.5\textwidth}
  \centering
  \includegraphicsmaybe{[width=\textwidth]}{../img/hippodrome-gai_bathy.png}
 \end{minipage}%
 \begin{minipage}[t]{0.5\textwidth}
  \centering
  \includegraphicsmaybe{[width=\textwidth]}{../img/hippodrome-gai_bathy_tend.png}
 \end{minipage}
 \caption{Bed evolution at times $0$s and $10000$s.}
 \label{fig:hippodrome-t2d:solution}
\end{figure}


%TO DO: Add error computation between analytical and numerical solution?
\section{Conclusion}
This test provides a starting point to set-up 2D morphodynamic applications as well as a simplified configuration for quick code \textit{debugging} and testing.
%
%
%
%


% Here is an example of how to include the graph generated by validateTELEMAC.py
% They should be in test_case/img
%\begin{figure} [!h]
%\centering
%\includegraphics[scale=0.3]{../img/mygraph.png}
% \caption{mycaption}\label{mylabel}
%\end{figure}


