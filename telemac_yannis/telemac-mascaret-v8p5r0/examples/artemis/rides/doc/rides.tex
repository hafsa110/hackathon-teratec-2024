\chapter{Reflection on rippled bottom: rides}

\section{Summary}

The objective of this test case is to analyze the wave behavior related to a
rippled bathymetry. We take the opportunity to underline the interest of
taking into account the terms of slope and curvature in the Berkhoff equation
(extended diffraction-refraction equation, option {\it RAPIDLY VARYING
 TOPOGRAPHY}. Artemis is compared to experimental results \cite{davies1984}
and numerical results \cite{Michel1999}.

We consider the case of a channel with analytical bathymetry. This case was
defined by \cite{davies1984} to study the phenomenon of reflection for a swell
propagating on a rippled bottom. The rippled bottom takes the form of 10
periods of a sine function. The interest of this case, besides proposing an
experimental reference, is to be a discriminating case between predictions of
the Berkhoff equation alone and predictions of the extended equation.

Results without the slope and curvature terms are also presented. This
allows us to compare our values with the numerical results of
\cite{Michel1999}, and to show the interest of taking into account the slope
and curvature terms.

The results are satisfactory and show the good agreement between the numerical
prediction and the experimental results.


\section{Description of the test case}

The general configuration is described in Figure \ref{fig:rides_config}, which
presents the bathymetry and dimensions of the channel. The detailed bathymetry
can be found in \ref{rides_geom}.

\begin{figure}[h]
\begin{center}
  \includegraphicsmaybe{[width=0.7\textwidth]}{config_rides.png}
\end{center}
\caption{Configuration of the problem}
\label{fig:rides_config}
\end{figure}


Several wave frequencies are tested for validation. They are described in
\ref{rides_bc}. By default, the test case file sets the wave period to
T = 1.3 s. The user is free to modify this value. The test case is chosen as
simple as possible, so the calculation does not include
neither breaking nor bottom friction.

\subsection{Geometry}
\label{rides_geom}
The modeled channel is a 40 m long and 2 m wide rectangle.

The bathymetry is fixed at z= 0 from the channel entrance to x=L1=25m.
Then it takes the form of a sine function:
$$
\begin{array}{lr}
  \mbox {For } L_1<x<L_1+b & z = A \sin (\frac{2\pi}{\lambda_b}(x-L_1))
  \end{array}
$$
with
\begin{itemize}
\item A = 0.05m
\item $\lambda_b = 1m$
\item $L_1=25m$
\end{itemize}

The coast of the free surface is 0.313m.


\subsection{Mesh}

The mesh used is an adjusted mesh generated by Janet. It consists of 400 000
triangular elements. The nodes are regularly spaced by 2 cm on the x-axis and
y-axis.

\subsection{Boundary conditions}
\label{rides_bc}
The boundary conditions of the domain are presented in Figure \ref{fig:rides_bc}.

\begin{figure}[h]
\begin{center}
  \includegraphicsmaybe{[width=0.7\textwidth]}{bc_rides.png}
\end{center}
\caption{Boudary conditions}
\label{fig:rides_bc}
\end{figure}

Incidental swell:
\begin{itemize}
\item Period = variable [0.974 s; 2.37 s]
\item Wave height : 0.01 m
\item Propagation direction: 0$^\circ$
\item Exit direction: 0$^\circ$.
\item Phase (ALFAP): 0$^\circ$ on the whole liquid boundary
\end{itemize}
Wall:
\begin{itemize}
\item Reflection coefficient = 1.
\item Phase shift: 0$^\circ$
\item Angle of attack: 0$^\circ$
\end{itemize}
Free exit:
\begin{itemize}
\item Angle of attack of the outgoing waves: 0$^\circ$.
\end{itemize}

\section{Reference solution}
The experimental results proposed in \cite{davies1984} constitute data over a wide range of
frequency range. They are presented as reflection coefficients as a function of:
$$
\frac{2\lambda_b}{\lambda_{swell}}
$$
where $\lambda_{swell}$ is the wavelength of the swell function of the chosen period.
In \cite{Michel1999} we find curves related to this configuration. The results are presented
with a Berkhoff approach alone and also taking into account the effects of slope and
curvature effects.

Artemis is compared with these two references. The comparison criterion is the
reflection coefficient, which we define as follows:
$$
C_{Reflexion}=\frac{H_{max}-H_{incident}}{H_{incident}}
$$

with $H_{incident}$ the incident wave height and
$H_{max}$ the maximum wave height observed in the resulting wave field (incident
+reflected)

\section{Results}

The results obtained with Artemis are in good agreement with the experimental reference. 
In particular we can see that the peak of the reflection coefficient for $2\lambda_b /\lambda_swell = 1$ is much better predicted by the extended Berkhoff equation than by the Berkhoff equation alone.

Code-to-code comparison with REEF2000 gives an excellent agreement.

\begin{figure}[h]
\begin{center}
  \includegraphicsmaybe{[width=0.9\textwidth]}{resu_rides.png}
\end{center}
\caption{Reflection coefficients as a function of bathymetry (ripple wavelength).
Comparison between ARTEMIS (blue and violet dots) with
the experiment (white point) \cite{Michel1999} (black line)}
\label{fig:rides_resu}
\end{figure}

\section{Conclusions}

The experimental results are well reproduced in a qualitative way, and are satisfactory
quantitatively. The comparison with the REEF2000 code shows an excellent correlation.
This allows in particular to validate the option {\it RAPIDLY VARYING
 TOPOGRAPHY} option of the code.
