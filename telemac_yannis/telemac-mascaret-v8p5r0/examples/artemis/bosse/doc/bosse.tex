

\chapter{Transformation of a wave on a circular bump: bosse}

\section{Summary}

The objective of this test case is to analyze the predictions of Artemis in
the case of a bathymetry with a bump of significant height. This case
highlights the interest of taking into account the slope and curvature terms
in the Berkhoff equation (extended diffraction-refraction equation
refraction equation, option {\it RAPIDLY VARYING TOPOGRAPHY}). We compare
Artemis to experimental results \cite{Williams1980}.

We consider the case of a basin with an analytical bathymetry, defining a
circular bump. This case was defined by \cite{Williams1980}, then taken up by 
\cite{Chandrasekera1997} and \cite{Lee2004}. It allows to study the
reflection-diffraction phenomenon for a wave propagating over a highly
variable bottom, with effects in the 2 directions of the free surface plane.
The interest of this case, in addition to give a reference to validate Artemis,
is that it is a case that is quite discriminating between predictions of the
Berkhoff equation alone and predictions of the extended Berkhoff equation.
The results are quite satisfactory.

\section{Description of the test case}

The general configuration is described in Figure \ref{fig:bosse_config}, which
presents the bathymetry and dimensions of the dimensions of the channel.
The detailed bathymetry can be found in \ref{bosse_geom}

\begin{figure}[h]
\begin{center}
  \includegraphicsmaybe{[width=0.7\textwidth]}{config.png}
\end{center}
\caption{Configuration of the problem}
\label{fig:bosse_config}
\end{figure}

For the numerical model, we set R=1m. We use the geometric ratios proposed by
\cite{Williams1980}. In particular :
\begin{itemize}
\item b/H = 0.807
\item b/R = 0.4
\item kH = 3
\end{itemize}
where k is the wave number of the incident wave. 

The breaking is not taken into account, no friction on the bottom is taken into account either.

\section{Geometry}
\label{bosse_geom}
The modeled basin is a rectangle of 12 m long and 20 m wide.

The bathymetry is set at z= 0 everywhere in the basin, except at the hump. The
hump iscentered on the point $x_c=6m$ and $y_c=10m$:
$$
d= \sqrt{(x-x_c)^2+(y-y_c)^2}< R \hspace{1cm} \Rightarrow z =b(1-(\frac{d}{R})^2)
$$

\section{Mesh}

The mesh used is an adjusted mesh generated by Janet. It has 1 200 000
elements triangular elements. The nodes are regularly spaced by 2 cm on the
abscissa and ordinate.

\section{Boundary conditions}

The boundary conditions of the domain are presented in Figure
\ref{fig:bc_bosse}.

\begin{figure}[h]
\begin{center}
  \includegraphicsmaybe{[width=0.7\textwidth]}{bc_bosse.png}
\end{center}
\caption{Configuration of the problem}
\label{fig:bc_bosse}
\end{figure}

Incidental swell:
\begin{itemize}
\item Period 0.633s
\item Wave height: 0.01 m-
\item Propagation direction: 0$^\circ$.
\item Exit direction: 0$^\circ$.
\item Phase (ALFAP) : 0$^\circ$ on the whole liquid boundary
\end{itemize}
Wall:
\begin{itemize}
\item Reflection coefficient = 1.
\item Phase shift : 0$^\circ$
\item Angle of attack: 0$^\circ$
\end{itemize}
Free exit:
\begin{itemize}
\item Angle of attack of the outgoing waves: 0$^\circ$.
\end{itemize}


\section{Reference solution}
The experimental results proposed in \cite{Chandrasekera1997} are the
reference data. They are wave heights measured in some planes. They are
presented in the form of normalized wave height :
$$
\frac{H}{H_{incident}}
$$
with $H_{incident}$ the incident wave height.

We have chosen the plane noted "a" by the authors (see Figure
\ref{fig:plan_bosse}) because of
its character discriminant between the Berkhoff equation and the extended
Berkhoff equation.

\begin{figure}[h]
\begin{center}
  \includegraphicsmaybe{[width=0.7\textwidth]}{plan_bosse.png}
\end{center}
\caption{Definition of the plan ``a''}
\label{fig:plan_bosse}
\end{figure}


\section{Results}

The results obtained with Artemis in the " a " plane and the experimental r
eference are presented below in figure 4.



We give the results with and without taking into account the effects of slope
and curvature. We note that in the case studied, the use of the extended
Berkhoff equation allows a better estimation of the experimental results.
This is particularly true for the prediction of the
prediction of the extreme height.

\begin{figure}[h]
\begin{center}
  \includegraphicsmaybe{[width=0.9\textwidth]}{resu_bosse.png}
\end{center}
\caption{ Water height in plane a. Comparison between the experiment (green
  points) and 2 ARTEMIS calculations: in blue without taking into account
  slope and curvature effects, in red with}
\label{fig:resu_bosse}
\end{figure}

The comparison is satisfactory. Taking into account the second order b
athymetric effects has allowed to better estimate the
effects has allowed to better estimate the water level amplification.

\section{Conclusions}

This test case allows to validate Artemis on a wave tank test. The numerical
results are in  general in good agreement with the experimental ones,
when using the {\it RAPIDLY VARYING TOPOGRAPHY} option.
This  option allow to take into account the effects of slope and curvature in
the Berkhoff equation.



