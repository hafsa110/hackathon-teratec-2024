\chapter{Friction on sandy bottom.}
\section{Summary}
The aim of this test case is to analyze the behavior of Artemis when damping
swell heigh due to bottom friction forces.
The results are compared with experimental results obtained on a physical
model by Inman \& Bowen \cite{Inman1962}. They are used to verify that the
Artemis code correctly take into account bottom friction.

\section{description of the test case}
The case modeled is the one described in \cite{Inman1962}. Wave damping is
simulated using the bottom friction calculated by Artemis on the basis of
sediment characteristics (grain diameters $d_{50}$ and $d_{90}$). Artemis automatically
calculates the overall roughness which takes into account both skin friction
and shape friction.

This roughness, together with the type of flow (laminar or turbulent),
then determines the friction factor used in formulating the dissipation
coefficient. The friction formulation used here is the one of Putnam \&
Johnson \cite{Putman1949}. The results considering only skin roughness are
also presented.


\subsection{Formulas}
The general roughness formula used is as follows:
$$
k = k_p + k_f
$$
$k_p$ is the skin roughness, which depends on the size of grain and on
the type of flow:
\begin{equation}
\begin{array}{ll} 
k_p = 3 d_{90} & \mbox{ for } \Theta < 1 \\[6pt]
k_p = 3 \Theta d_{90}  & \mbox{ for } \Theta \ge  1
\end{array}
\end{equation}
$\Theta$ is the Shields mobility parameter, relative to the flow, which
expresses the ratio between the force resulting from the shear stress on
the grain and the immersed weight of the grain calculated as a function of
$d_{50}$ (Van Rijn \cite{VanRijn1993}).


$k_f$ represents the shape roughness, i.e. the roughness due to the formation of wrinkles, we use the formula proposed by Van Rijn \cite{VanRijn1993}:
$$
k_f = 20 \Gamma_r \Delta_r (\frac{\Delta_r}{\lambda_r})
$$
\begin{itemize}
\item $\Delta_r$ Wrinkles height
\item $\lambda_r$ Wrinkles wavelength
\item $\Gamma_r$ Wrinkle presence factor (equal to 1 for wrinkles alone, equal to 0.7 for wrinkles
  superposed on sand waves). Here, it is equal to 0.7.
  \end{itemize}

Once the roughness has been calculated, we can obtain the friction factor $f_w$,
whose formula depends on the type of flow. For rough turbulent flow, we use the
expression given by Van Rijn \cite{VanRijn1993}. Finally, we get the expression
for the shear stress due to the Putnam \& Johnson formula \cite{Putnam1949}.

\subsection{Geometry}
Domain: 20 m x 2 m

Water depth: 0.5 m
\subsection{Mesh}
Triangular elements: 1896

Taille des mailles: 0.13 m (longitudinal max)

Nodes: 1169

\begin{figure}[h]
\begin{center}
  \includegraphicsmaybe{[width=0.6\textwidth]}{../img/Mesh.png}
\end{center}
\caption{Mesh of the case}
\label{fig:friction_Mesh}
\end{figure}

\subsection{Sediments}
Caracteristical diameters:
\begin{itemize}
\item $D_{50}$ = 0.2 mm
\item $D_{90}$ = 0.3 mm
\end{itemize}

\subsection{Boundary conditions}
\begin{figure}[h]
\begin{center}
  \includegraphicsmaybe{[width=0.6\textwidth]}{boundary.png}
\end{center}
\caption{boundary conditions}
\label{fig:friction_boundary}
\end{figure}


Houle incidente :
\begin{itemize}
\item Period: 2 s
\item swell height: 0.175 m
\item Phase (ALFAP): 0 (on all the liquid boundary)
\item Propagation direction de propagation: $0^\circ$
\end{itemize}
Solid Wall:
\begin{itemize}
\item Reflection Coefficient: 1
\item Dephasing: 0
\item Attack angle: $0^\circ$
\end{itemize}
Liquid wall:
\begin{itemize}
  \item Reflexion coefficient: 0
\item Dephasing: 0
\item Attack angle: $0^\circ$
\end{itemize}
  
\subsection{Solver}
The direct solver (solver = 8) was used with
accuracy of $10^{-4}$. Calculation time is less than 1 second.

\section{Results}
The experimental results obtained in \cite{CERC1984} constitute the reference
values, distributed over a section at y=1m. 

The measurements were taken once ripple formation and sediment transport had reached a stable state.
ripple formation and sediment transport have reached a steady state. We compare the wave heights1
obtained numerically and experimentally. We also compare the
ripple characteristics.

The comparison (figure \ref{fig:friction_comparison}) between wave height calculated by artemis and
the ones obtained in  \cite{CERC1984} show good behavior. It highlights how well Artemis takes into
account by Artemis of the frictional stresses exerted by the seabed and the reduction in swell height
due to this friction.

In addition, Table \ref{tab:friction_ripples} shows the results obtained for ripple characteristics
with a friction factor
$f_w$ = 0.17 (calculated by Artemis). The correlation with the results of Inman \& Bowen [1] is
satisfactory. Artemis is therefore able to determine whether or not wrinkles are formed, and thus
to deduce their characteristics.

The height and wavelength of the ripples are much greater than the size of the sediment, which
explains the poor results: the height and wavelength of the ripples are much greater than the size
of the sediment, which explains the poor results obtained by considering only the skin roughness.

\begin{figure}[h]
\begin{center}
  \includegraphicsmaybe{[width=0.9\textwidth]}{../img/Wave_height.png}
\end{center}
\caption{Wave height}
\label{fig:friction_WH}
\end{figure}

\begin{figure}[h]
\begin{center}
  \includegraphicsmaybe{[width=0.9\textwidth]}{comparison.png}
\end{center}
\caption{Wave height Comparison with measure}
\label{fig:friction_comparison}
\end{figure}

\begin{figure}[h]
\begin{center}
  \includegraphicsmaybe{[width=0.9\textwidth]}{../img/profile.png}
\end{center}
\caption{Wave height of last calculation on the profile}
\label{fig:friction_profile}
\end{figure}

\begin{table}
  \begin{center}
\begin{tabular*}{0.5\linewidth}{@{\extracolsep{\fill}}ccc}
\toprule
\toprule
& $\Delta_r(cm)$&$\lambda_r$(cm) \\
Artemis&1.5&8.6\\
Berkhoff&1.5&10.8\\
\bottomrule
\bottomrule
\end{tabular*}
\caption{Heigth and wavelength of the ripples}
\label{tab:friction_ripples}
\end{center}
\end{table}
\section{Conclusions}
This test case compares the results produced by Artemis with experimental results. It
validates the inclusion of bottom friction and the calculation of ripple characteristics by Artemis.

In this case, Artemis calculates the friction factor from the size of the sediments.
sediments. However, it is possible to impose it if the user wishes or if the bottom
is not sandy. However, the values imposed are often taken from classic tables which
generally determined for stationary flows. Here, however, the flow is
oscillating because it is subject to wave action.
