\chapter{Wave field around a floating body}
\section{Summary}
The aim of this test case is to validate the functionality of imposing an
incident potential on a boundary. This feature appeared in version 6.2.
We compare Artemis with analytical results.

We consider the case of a circular domain with flat bathymetry. At the center
of the domain (x=0 and y=0), a floating body submitted to an incident swell
generates a disturbance of the wave field, whose potential $\Phi_P$  can be
expressed. The center of the simulation is not considered in the simulation.
The floating body is taken into account by imposing an incident potential on
the inner circle of the fluid domain. This incident potential can be calculated
analytically at any distance from the body, providing an analytical reference.
The outer circle of the domain is a free output. The results are highly
satisfactory.

\section{Description of the test case}
\subsection{Geometry}

\begin{figure}[h]
\begin{center}
  \includegraphicsmaybe{[width=0.7\textwidth]}{../img/Mesh.png}
\end{center}
\caption{Domain of the case}
\label{fig:kochin_Mesh}
\end{figure}
The general configuration is described on Figure \ref{fig:kochin_Mesh} which
shows the fluid meshed and its dimension. The bathymetry is flat and the
waterdepth is 100m. The inner ray is 50m and the outter ray is 283m
No breaking and no friction is taken into account. 

The mesh is made with 6912 triangular elements and the nodes are evenly spaced
by 5 degrees, with refinement according to the distance to the center.

\subsection{Boundary conditions}
Boundary conditions are given on Figure \ref{fig:kochin_bc}
\begin{figure}[h]
\begin{center}
  \includegraphicsmaybe{[width=\textwidth]}{boundary.png}
\end{center}
\caption{Boundary Conditions and orientation}
\label{fig:kochin_bc}
\end{figure}

Houle incidente:
\begin{itemize}
\item Définie par le potentiel et son gradient :
\begin{itemize}
\item PRBT: Incident potential, real part
\item PIBT: Incident potential, imaginary part 
\item DDXPRBT: x-derivative of the incident potential, real part
\item DDYPRBT: y-derivative of the incident potential, real part
\item DDXPIBT: x-derivative of the incident potential, imaginary part
\item DDYPIBT: y-derivative of the incident potential, imaginary part
\end{itemize}
\item output direction: $0^\circ$
\item Incident swell period: 8s
\end{itemize}

The expression for the incident potential is given by \cite{Babarit2011}

\begin{equation}
  Phi_P = \sqrt{\frac{k}{2\pi R}}.e^{ikR\frac{\pi}{4}}H(\Theta)
  \end{equation}
The z-dependency is given, in accordance with ARTEMIS assumptions, by:
$\frac{ch(k(z+d))}{ch(kz)}$

Where
\begin{itemize}
\item $R$ is the distance to the centre O (ray in meters).
\item $\Theta$ is the angle to x-axe.
\item $k$ is the wave number.
\item $d$ is the water depth
\end{itemize}
The function $H(\Theta)$ is called Kochin function. In the test case it is written under exponential form:
$$
H(\Theta) = Z(\Theta) e^{i\phi(\Theta)}
$$
Where $Z$ and $\Phi$ are real function read each $5^\circ$
Gradiants are calculated by finite difference on functions $Z$ and $\Phi$

Free output with an attack angle of $0^\circ$

\section{Reference solution}
$\Phi_P$ can be calculated at any distance of the centre. This is the reference
result.  ARTEMIS, for its part, propagates the incident potential entered at
R=50m to the free exit at R=283m. In this area, the result of the Berkhoff
equation can be compared with the analytical reference.


\section{Results}
The results obtained with ARTEMIS are shown in Figure\ref{fig:kochin_resu}, left-hand column.
In the right column, we present the result of the analytical solution. We compare directly
The imaginary and real part of the potential from which all other quantities (wave height,
free surface, phase ..) depend. 
The comparison is very satisfactory. An incident potential of any shape is taken into
account and its propagation is validated.

\begin{figure}[h]
\begin{center}
  \includegraphicsmaybe{[width=\textwidth]}{resuKochin.png}
\end{center}
\caption{Real and imaginary parts of the potential. Comparison between the analytical
  solution (right) and the ARTEMIS calculation (left) over the entire domain.}
\label{fig:kochin_resu}
\end{figure}


\section{Conclusions}
This test case validate that a user can enter an incident swell data other than the
"Wave height" and "Propagation direction" data. Any potential can be used. 

Radial swell propagation due to the presence of a floating body is also well modelled by
ARTEMIS, which is consistent with the analytical solution used for the test case.
