\chapter{Flamanville - Agitation in the intake channel : flam}

\section{Purpose}
This Artemis case describes the wave analysis in the intake channel of the
Flamanville power plant. It gives an example of the use of the code in a
concrete case: boundary conditions conditions read from a TOMAWAC file,
different reflection coefficients, parallel calculation


\section{ Description of the test case}

\subsection{Geometry}

The domain is largely limited by the solid walls of the channel. The only
liquid boundary is constituted by the incident swell at the entrance of the
channel. The bathymetry can be visualized on figure \ref{fig:flam_bathy}.
The initial coastline
of the free surface is 12.71 m.

\begin{figure}[h]
\begin{center}
  \includegraphicsmaybe{[width=0.7\textwidth]}{../img/Bathy.png}
\end{center}
\caption{Bathymetry.}
\label{fig:flam_bathy}
\end{figure}

\subsection{Mesh}

The mesh consists of 5258 triangle elements and 2781 nodes. The size
of a mesh is 7m. The mesh can be visualized on figure \ref{fig:flam_mesh}.

\begin{figure}[h]
\begin{center}
  \includegraphicsmaybe{[width=0.7\textwidth]}{../img/Mesh.png}
\end{center}
\caption{Mesh.}
\label{fig:flam_mesh}
\end{figure}

\subsection{Boundary conditions}

\begin{figure}[h]
\begin{center}
  \includegraphicsmaybe{[width=0.7\textwidth]}{flam_bc.png}
\end{center}
\caption{Boundary conditions}
\label{fig:flam_bc}
\end{figure}

The incident swell is partly read in a TOMAWAC file. The significant heights
and propagation directions are retrieved. A JONSWAP spectrum is then applied.
\begin{itemize}
\item Peak period: 13 s
\item Max and Min period: 26 s and 6 s.
\item Spectrum parameter: gamma = 3.
\item Phase (ALFAP): 0$^\circ$.
\end{itemize}
Wall 1:
\begin{itemize}
\item Reflection coefficient = 0.5
\item Phase shift : 0$^\circ$
\item Angle of attack : 0$^\circ$
\end{itemize}
  Wall 2 :
\begin{itemize}
\item Reflection coefficient = 1.
\item Phase shift : 0$^\circ$
\item Angle of attack : 0$^\circ$
\end{itemize}

\section{Results}

The requested graphical outputs are the significant wave height, the phase,
the bathymetry, breaking rate, phase velocity.

\begin{figure}[h]
\begin{center}
  \includegraphicsmaybe{[width=0.7\textwidth]}{../img/WaveHeight.png}
\end{center}
\caption{Wave Height.}
\label{fig:flam_waveheight}
\end{figure}

\begin{figure}[h]
\begin{center}
  \includegraphicsmaybe{[width=0.7\textwidth]}{../img/Breaking.png}
\end{center}
\caption{Breaking rate.}
\label{fig:flam_qb}
\end{figure}

\section{Conclusions}

This example of use gives a concrete case of the use of Artemis, with random swell
known in amplitude and direction. The reading of the boundary conditions, from a
TOMAWAC file, gives an interesting example to the user.
