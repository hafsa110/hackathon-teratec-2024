\chapter{Breach}
\section{Summary}
The aim of this test case is to compare the results produced by Artemis with
experimental results on a known application: the diffraction of a swell
entering a harbor through a gap twice its wavelength. The results are
compared with those given by the Shore Protection Manual of the Coastal
Engineering Research Center, USA (CERC, \cite{CERC84}, page 2-93).
\section{Description of the test case}
The modelled case represents a narrow entrance to a square port with a constant flat bottom.
\subsection{Geometry}
Port Entrance: 550m

Domain: 1830mx1830m

Water depth: 64 m

\subsection{Mesh}
Number of triangular elements: 14304

Nodes: 7313

Mesh per wavelenth: 14

\begin{figure}[h]
\begin{center}
  \includegraphicsmaybe{[width=0.6\textwidth]}{../img/Mesh.png}
\end{center}
\caption{Mesh of the case}
\label{fig:breach_Mesh}
\end{figure}

\subsection{Boundary conditions}
\begin{figure}[h]
\begin{center}
  \includegraphicsmaybe{[width=0.7\textwidth]}{boundary.png}
\end{center}
\caption{Boundary Conditions}
\label{fig:breach_bc}
\end{figure}

Incident swell:
\begin{itemize}
\item Period: 14 s
\item Wave height: 1 m
\item Phase (ALFAP): 0 on all the boundary
\item Propagation direction: $0^\circ$
\end{itemize}
Solid wall:
\begin{itemize}
\item Reflection coefficient: 1
\item Dephasing: 0
\item Attack angle: $0^\circ$
\end{itemize}
Paroi liquide :
\begin{itemize}
\item Attack angle: $0^\circ$
\end{itemize}
\subsection{Solver}
Two solver have been tested with a precision of $10^{-4}$
\begin{itemize}
\item Direct solver (solver=8) time of calculation less than 1s. 
\item Conjugate gradient on a normal equation (solver=3) time of calculation of 2s.
  \end{itemize}
\section{Results}
The experimental results obtained in \cite{CERC1984} constitute the reference values, distributed over a section shown in
section shown in Fig \ref{fig:breach_CERCResult}. A comparison is made between the wave heights obtained
numerically and experimentally.

The results obtained by Artemis (Solver 8) are shown figure \ref{fig:breach_results}, the ones obtained by the CERC is on Figure
\ref{fig:breach_CERCResult} and the comparison between both are shown in Figure \ref{fig:breach_section}. 
\begin{figure}[h]
\begin{center}
\includegraphicsmaybe{[width=0.45\textwidth]}{../img/WaveHeight.png}
\includegraphicsmaybe{[width=0.45\textwidth]}{../img/Free_Surface.png}
\end{center}
\caption{Breach: Wave height on left and free surface on right}
\label{fig:breach_results}
\end{figure}


\begin{figure}[h]
\begin{center}
\includegraphicsmaybe{[width=\textwidth]}{CERC.png}
\end{center}
\caption{Breach: }
\label{fig:breach_CERCResult}
\end{figure}

\begin{figure}[h]
\begin{center}
\includegraphicsmaybe{[width=\textwidth]}{../img/Section.png}
\end{center}
\caption{Breach: Comparison of wave heigth between measurements and Artemis }
\label{fig:breach_section}
\end{figure}

\section{Conclusions}
This test case compares results acquired with Artemis with experimental results.
It validates the wave diffraction model. The agreement between
numerical predictions and experiment is satisfactory.
