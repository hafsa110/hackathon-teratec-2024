\chapter{Flow along a beach (plage)}

\section{Purpose}

This example tests the $k-\omega$ model of \telemac{3D}.

\section{Description}

The configuration is a flat channel ($z$ = -0.43~m) with a kind of cavity and a
beach where the bathymetry is increasing from -0.43~m to 0~m in the cavity
(see Figure \ref{t3d:plage:fig:Bottom}).

\begin{figure}[!htbp]
 \centering
 \includegraphicsmaybe{[width=\textwidth]}{../img/Bottom.png}
 \caption{Bottom elevation.}
 \label{t3d:plage:fig:Bottom}
\end{figure}

\subsection{Initial and Boundary Conditions}

The computation is initialised with a constant elevation equal to 0.1~m
and no velocity.

The boundary conditions are:
\begin{itemize}
\item For the solid walls, a slip condition on channel banks is used for the
velocities,
\item On the bottom, a Strickler law with friction coefficient equal to
60~m$^{1/3}$/s is prescribed,
\item Upstream a flowrate equal to 0.155~m$^3$/s is prescribed,
\item Downstream the water level is equal to 0.1~m (= initial elevation).
\end{itemize}

\subsection{Mesh and numerical parameters}

The 2D mesh (Figure \ref{t3d:plage:fig:meshH})
is made of 8,796 triangular elements (4,561 nodes).
7 planes are regularly spaced on the vertical (Figure \ref{t3d:plage:fig:meshV}).

\begin{figure}[!htbp]
 \centering
 \includegraphicsmaybe{[width=\textwidth]}{../img/MeshH.png}
 \caption{Horizontal mesh.}
 \label{t3d:plage:fig:meshH}
\end{figure}

\begin{figure}[!htbp]
 \centering
 \includegraphicsmaybe{[width=\textwidth]}{../img/MeshV.png}
 \caption{Initial vertical mesh along $x$ = 15~m.}
 \label{t3d:plage:fig:meshV}
\end{figure}

The hydrostatic version of \telemac{3D} is used.

To solve the advection, the method of characteristics is used for velocities and
$k-\omega$ variables.

The time step is 0.2~s for a simulated period of 200~s.

\subsection{Physical parameters}

Turbulence is modelled with the $k-\omega$ model of \telemac{3D}.

\section{Results}

\begin{figure}[H]
  \centering
  \includegraphicsmaybe{[width=\textwidth]}{../img/FreeSurface.png}
  \caption{Free surface at final time step.}
  \label{t3d:plage:FreeSurf}
\end{figure}

A recirculation can be seen in the cavity representing the beach
(Figure \ref{t3d:plage:Velo}).

\begin{figure}[H]
  \centering
  \includegraphicsmaybe{[width=\textwidth]}{../img/Velocity.png}
  \caption{Velocity magnitude at the surface at final time step.}
  \label{t3d:plage:Velo}
\end{figure}

%\section{Conclusion}

%This example validates the $k-\omega$ turbulence model of \telemac{3D}.
