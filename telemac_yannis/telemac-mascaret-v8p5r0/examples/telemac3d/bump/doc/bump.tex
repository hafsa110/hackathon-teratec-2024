\chapter{Flow over a bump (bump)}

\section{Description}
\bigskip
This test calculates the fluvial regime in a horizontal straight channel
including a topographical singularity (a bump on the bottom). 
It allows to show that \telemac{3D} is able to correctly reproduce the hydrodynamic 
impact of changing bed slopes, vertical flow contractions and expansions.
Furthermore, this problem has an analytical solution in 2D. 
The solution computed by \telemac{3D} will be compared with this analytical 
solution. Therefore, it allows also testing the accuracy on the computation of the
free-surface, with respect to the bottom gradient. 

\bigskip
The geometry dimensions of the channel are 2 m wide and 20.96 m long. 
The elevation of the flat part of the channel is at -0.2~m 
(see Figure \ref{t3d:bump:fig:bathy}).
The maximum elevation of the bump in its middle is 0 m (at $x$ = 10.0~m) 
with the profile of the singularity (Figure \ref{t3d:bump:fig:profil}) 
which is defined by:
\begin{equation*}
z_{\rm{f}}(x) =max(-0.2, -0.0246875(x-10)^2)
\end{equation*}
The studied flow regime provides a transition to super-critical flow conditions 
located on the bump,
and the downstream water level imposes the presence of an hydraulic jump 
in the downstream reach of the channel.
Note that the turbulent viscosity is constant along the horizontal direction
and equal to $10^{-6}~\text{m}^2.\text{s}^{-1}$ (default value) and a mixing
length model is applied in the vertical direction (Prandtl formula).

\begin{figure}[!htbp]
 \centering
 \includegraphicsmaybe{[width=\textwidth]}{../img/bathy.png}
 \caption{Bathymetry.}
 \label{t3d:bump:fig:bathy}
\end{figure}
\begin{figure}[!htbp]
 \centering
 \includegraphicsmaybe{[width=0.9\textwidth]}{../img/profil.png}
 \caption{Bump profile.}
 \label{t3d:bump:fig:profil}
\end{figure}

\bigskip
As seen previously, this problem has an analytical solution in 2D. 
This solution is based on the Bernoulli equations.\\
For a given discharge per unit length $q$, imposed at the upstream
boundary and a water depth $h$ imposed at the downstream boundary,
the water line follows the Bernoulli equation written below:
\begin{equation*}
\frac{q^2}{2gh^3}+ (h + z_{\rm{f}}) = H_0
\end{equation*}
where $H_0$ is specific energy and $z_{\rm{f}}$ is bottom elevation.


\subsection{Initial and boundary conditions}

\bigskip
The initial free surface is 0.4~m with a fluid at rest.

\bigskip
The boundary conditions are:
\begin{itemize}
\item For the solid walls, a slip condition on channel banks is used for the velocity,
\item On the bottom, Strickler law with friction coefficient equal to
50~$\text{m}^{1/3}.\text{s}^{-1}$ is imposed,
\item Upstream a flowrate equal to 2~$\text{m}^{3}.\text{s}^{-1}$ is imposed,
\item Downstream the water level is equal to 0.4~m.
\end{itemize}

\subsection{Mesh and numerical parameters}
\bigskip
The mesh (Figures \ref{t3d:bump:fig:meshH} and \ref{t3d:bump:fig:meshV})  
is composed of 2,620 triangular elements (1,452 nodes) with 5 planes  
regularly spaced on the vertical, to form prism elements.
Note that the mesh is made up of squares whose sides measure 0.16~m cut into
triangles. 

\begin{figure}[!htbp]
 \centering
 \includegraphicsmaybe{[width=\textwidth]}{../img/Mesh.png}
 \caption{Horizontal mesh.}
 \label{t3d:bump:fig:meshH}
\end{figure}
\begin{figure}[!htbp]
 \centering
 \includegraphicsmaybe{[width=\textwidth]}{../img/MeshV.png}
 \caption{Vertical mesh  at initial state.}
 \label{t3d:bump:fig:meshV}
\end{figure}
\bigskip
The time step is 0.01~s for a simulated period of 50~s. 

\bigskip
This case is computed with the non-hydrostatic version.
To solve advection, the method of characteristics is used for the velocities (scheme 1).
The GMRES (Generalized Minimal Residual Method, scheme 7) is used to solve 
the Pressure Poisson Equation with an accuracy fixed at the default value
$10^{-8}$  (since version 8.1).
The implicitation coefficients for depth and velocities are equal to 0.6. 

\section{Results}

\bigskip
Qualitatively the variation of the velocity field is regular at the surface and
vertically in the critical flow area.
It follows well the shape of the bump (Figures \ref{t3d:bump:fig:veloH} 
and \ref{t3d:bump:fig:veloV}).
Generally, the results (Figure \ref{t3d:bump:fig:free_surface}) 
are in good agreement with the analytical
solution. This solution is described in details in [1].\\
Moreover, the position of the hydraulic jump is correctly computed.
However, a shift (difference of the free surface at the entrance) is
observed on the free surface, which seems to be due to the way to
calculate the free surface by \telemac{3D}.

%\section{Conclusion}

\bigskip
To conclude, this flow is well reproduced by \telemac{3D}.

\begin{figure}[!htbp]
 \centering
 \includegraphicsmaybe{[width=\textwidth]}{../img/veloH.png}
 \caption{Horizontal distribution of velocities at the surface.}
 \label{t3d:bump:fig:veloH}
\end{figure}
\begin{figure}[!htbp]
 \centering
 \includegraphicsmaybe{[width=\textwidth]}{../img/veloV.png}
 \caption{Vertical distribution of velocities.}
 \label{t3d:bump:fig:veloV}
\end{figure}

\begin{figure}[!htbp]
 \centering
 \includegraphicsmaybe{[width=\textwidth]}{../img/free_surface.png}
 \caption{Free surface profile comparison between analytical solution 
 and \telemac{3D} solution.}
 \label{t3d:bump:fig:free_surface}
\end{figure}

\section{Reference}

[1] Hydrodynamics of Free Surface Flows modelling with the finite
element method. Jean-Michel Hervouet (Wiley, 2007) pp 128-129.
