\chapter{Solitary wave propagation over an uneven bottom (uneven)}

\section{Purpose}

This test demonstrates the availability of \telemac{3D} to propagate
a wave over an uneven seabed.

\section{Description}

The configuration is a rectangular channel (600~m long and 6~m wide)
with a solitary wave that propagates over an uneven seabed.

A piecewise affine bathymetry only dependent on the abscissa coordinate
$x$ is defined (see Figure \ref{t3d:uneven:fig:Bottom}):
\begin{equation*}
\left\{
\begin{array}{rl}
  \textrm{if } x < 160, & z = -10,\\
  \textrm{if } 160 \le x < 260, & z = -10-0.05(160-x),\\
  \textrm{if } x \ge 260, & z = -5.
\end{array}
\right.
\end{equation*}

\begin{figure}[!htbp]
 \centering
 \includegraphicsmaybe{[width=\textwidth]}{../img/Bottom.png}
 \caption{Bottom elevation.}
 \label{t3d:uneven:fig:Bottom}
\end{figure}

\begin{figure}[!htbp]
 \centering
 \includegraphicsmaybe{[width=\textwidth]}{../img/Profil.png}
 \caption{Profile.}
 \label{t3d:uneven:fig:Profil}
\end{figure}

\subsection{Initial and Boundary Conditions}

An analytical solution of solitary wave is used as initial condition for
the water depth and the horizontal velocity components.

The boundary conditions are only closed boundaries:
\begin{itemize}
\item For the solid walls, a slip condition on channel banks is used for the
velocities,
\item No bottom friction.
\end{itemize}

\subsection{Mesh and numerical parameters}

The 2D mesh (Figure \ref{t3d:uneven:fig:meshH})
is made of 7,206 triangular elements (4,210 nodes).

3 planes are regularly spaced in the vertical direction
(see Figure \ref{t3d:uneven:fig:meshV}).

\begin{figure}[!htbp]
 \centering
 \includegraphicsmaybe{[width=\textwidth]}{../img/MeshH.png}
 \caption{Horizontal mesh.}
 \label{t3d:uneven:fig:meshH}
\end{figure}

\begin{figure}[!htbp]
 \centering
 \includegraphicsmaybe{[width=\textwidth]}{../img/MeshV.png}
 \caption{Initial vertical mesh along $y$ = 1~m.}
 \label{t3d:uneven:fig:meshV}
\end{figure}

The non-hydrostatic version of \telemac{3D} is used.

To solve the advection, the method of characteristics is used for velocities.

The time step is 0.25~s for a simulated period of 25~s.

\subsection{Physical parameters}

No diffusion is considered.

\section{Results}

\begin{figure}[H]
  \centering
  \includegraphicsmaybe{[width=\textwidth]}{../img/FreeSurface.png}
  \caption{Free surface at final time step.}
  \label{t3d:uneven:FreeSurf}
\end{figure}

\begin{figure}[H]
  \centering
  \includegraphicsmaybe{[width=\textwidth]}{../img/FreeSurfaceV.png}
  \caption{Vertical section of free surface at final time step.}
  \label{t3d:uneven:FreeSurfV}
\end{figure}

%\section{Conclusion}
