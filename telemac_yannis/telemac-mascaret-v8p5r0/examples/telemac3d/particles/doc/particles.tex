\chapter{Particles transport (particles)}

\section{Purpose}

This test demonstrates the ability of \telemac{3D} to track the
transport of particles which are released into the fluid from discharge
points.

\section{Description}

The configuration is a section of river (around 1,700~m long and 300~m wide)
with realistic bottom.
The geometric data include a groyne in the transversal direction and an island
(see Figure \ref{t3d:particles:fig:Bottom}).

\begin{figure}[!htbp]
 \centering
 \includegraphicsmaybe{[width=\textwidth]}{../img/Bottom.png}
 \caption{Bottom elevation.}
 \label{t3d:particles:fig:Bottom}
\end{figure}

\subsection{Initial and boundary conditions}

The initialisation of the computation is done from a 2D result file (water depth
and horizontal velocity components).

The boundary conditions are:
\begin{itemize}
\item For the solid walls, a slip condition on channel banks is used for the
velocities,
\item On the bottom, a Strickler law with friction coefficient equal to
55~m$^{1/3}$/s is prescribed,
\item Upstream a flowrate equal to 700~m$^3$/s is prescribed,
\item Downstream the water level is equal to 265~m.
\end{itemize}

Drogues are released every 10 time steps until the 600$^{\textrm{th}}$ time step
(= 3,000~s).
Thus a maximum of 61 drogues are released, each time at $x$ = -200~m.

\subsection{Mesh and numerical parameters}

The mesh (Figure \ref{t3d:particles:fig:meshH})
is made of 3,780 triangular elements (2,039 nodes).
Is is refined around the island and in front of the groyne.
10 planes regularly spaced on the vertical
(Figures \ref{t3d:particles:fig:meshVy} along $y$ axis and
 Figures \ref{t3d:particles:fig:meshVx} along $x$ axis).

\begin{figure}[!htbp]
 \centering
 \includegraphicsmaybe{[width=\textwidth]}{../img/MeshH.png}
 \caption{Horizontal mesh.}
 \label{t3d:particles:fig:meshH}
\end{figure}

\begin{figure}[!htbp]
 \centering
 \includegraphicsmaybe{[width=\textwidth]}{../img/MeshVx.png}
 \caption{Vertical mesh at initial time step along $x$ axis for $y$ = 500~m.}
 \label{t3d:particles:fig:meshVx}
\end{figure}

\begin{figure}[!htbp]
 \centering
 \includegraphicsmaybe{[width=\textwidth]}{../img/MeshVy.png}
 \caption{Vertical mesh at initial time step along $y$ axis for $x$ = 0~m.}
 \label{t3d:particles:fig:meshVy}
\end{figure}

The time step is 5~s for a simulated period of  2~h (= 7,200~s).

The non-hydrostatic version is used.

To solve the advection, the method of characteristics
is used for the velocities (scheme 1).
The GMRES is used for solving the propagation and PPE steps (option 7).

A maximum of 100 drogues are released.
%and control sections are calculated between two couples of nodes.

\subsection{Physical parameters}

Turbulence is modelled with a mixing length model over the vertical
and a horizontal constant viscosity equal to 10$^{-2}$~m$^2$/s.

\section{Results}

The flow establishes a steady flow where the free
surface is lighly higher before the groyne than after
(see Figure \ref{t3d:particles:results}).
The flow accelerates in front of the groyne due to the restriction of section.
A recirculation appears just after the groyne.

\begin{figure}[H]
  \centering
   \subfloat[][Free surface]{
  \includegraphicsmaybe{[width=\textwidth]}{../img/FreeSurface.png}}\\
  \subfloat[][Velocity]{
  \includegraphicsmaybe{[width=\textwidth]}{../img/Velocity.png}}
  \caption{Results.}\label{t3d:particles:results}
\end{figure}

%\section{Conclusion}
