\chapter{Flow in a channel with 2 bridge piers (pildepon)}

\section{Purpose}

This example demonstrates the ability of \telemac{3D} to represent
the impact of an obstacle on a channel flow.
It also demonstrates the capability to represent unsteady eddies in a
model with steady state boundary.

\section{Description}

The configuration is a 28.5~m long and 20~m wide prismatic channel with
trapezoidal cross-section contains bridge-like obstacles in one cross-section
made of two abutments and two circular 4~m diameter piles
(See Figure \ref{t3d:pildepon:fig:Bottom}).
The flow resulting from steady state boundary conditions is studied.
The deepest water depth is 4~m.

\begin{figure}[!htbp]
 \centering
 \includegraphicsmaybe{[width=0.9\textwidth]}{../img/Bottom.png}
 \caption{Bottom elevation.}
 \label{t3d:pildepon:fig:Bottom}
\end{figure}

\subsection{Initial and Boundary Conditions}

The computation is initialised with a constant elevation equal to 0~m
and no velocity.

The boundary conditions are:
\begin{itemize}
\item For the solid walls, a slip condition on channel banks is used for the
velocities,
\item On the bottom, a Strickler law with friction coefficient equal to
40~m$^{1/3}$/s is prescribed,
\item Upstream a flowrate equal to 62~m$^3$/s is prescribed,
linearly increasing from 0 to 62~m$^3$/s during the first 20~s,
\item Downstream the water level is equal to 0.~m (= initial elevation).
\end{itemize}

\subsection{Mesh and numerical parameters}

The 2D mesh (Figure \ref{t3d:pildepon:fig:meshH})
is made of 4,304 triangular elements (2,280 nodes).
6 planes are regularly spaced on the vertical (see Figure \ref{t3d:pildepon:fig:meshV}).\\

\begin{figure}[!htbp]
 \centering
 \includegraphicsmaybe{[width=0.9\textwidth]}{../img/MeshH.png}
 \caption{Horizontal mesh.}
 \label{t3d:pildepon:fig:meshH}
\end{figure}

\begin{figure}[!htbp]
 \centering
 \includegraphicsmaybe{[width=0.9\textwidth]}{../img/MeshV.png}
 \caption{Initial vertical mesh.}
 \label{t3d:pildepon:fig:meshV}
\end{figure}

2 computations are run: one with the hydrostatic hypothesis and one with
non-hydrostatic version.

To solve the advection, the method of characteristics is used for velocities.

GMRES is used for solving the propagation and diffusion of velocities (option 7).
The implicitation coefficients for depth and velocities are respectively equal
to 0.6 and 1.

The time step is 0.1~s for the hydrostatic version, 0.4~s for the
non-hydrostatic version.
The simulated period is 80~s for both cases.

\subsection{Physical parameters}

A mixing length model is used as vertical turbulence model combined with
constant horizontal viscosity for velocity equal to 0.005~m$^2$/s.

\section{Results}

The obstacles create a contraction of the streamlines, and Karman
vortices are observed behind the piers.
The Karman vortices produce an asymmetry of the velocity field.
This velocity field is unsteady behind the piers in the Karman vortices
(see top of Figures \ref{t3d:pildepon:VeloHydro} and \ref{t3d:pildepon:VeloNH},
where surface velocities are
shown for the hydrostatic and non-hydrostatic simulations respectively).
%At the bottom of the same figure a time profile of the depth-averaged
%vertical velocity is given.
%After a transition of about 150~s a periodic regime takes place.
%Streamlines for positions where $x$ > -0.5~m (behind the piles) are
%shown of figure 3.5.4 for the hydrostatic (a) and non-hydrostatic (b)
%simulations.
%The figures show that the Karman vortices are better represented by the
%non-hydrostatic simulation, indicating the necessity to solve such
%turbulence problems using the non-hydrostatic version of \telemac{3D}.

\begin{figure}[H]
  \centering
  \includegraphicsmaybe{[width=0.9\textwidth]}{../img/VelocityHHydro.png}
  \caption{Velocity magnitude at the surface at final time step for the hydrostatic case.}
  \label{t3d:pildepon:VeloHydro}
\end{figure}

\begin{figure}[H]
  \centering
  \includegraphicsmaybe{[width=0.9\textwidth]}{../img/VelocityHNH.png}
  \caption{Velocity magnitude at the surface at final time step for the non-hydrostatic case.}
  \label{t3d:pildepon:VeloNH}
\end{figure}

Figures \ref{t3d:pildepon:FreeSurfHydro} and \ref{t3d:pildepon:FreeSurfNH} show
the free surface at final time step (=~80~s) for the hydrostatic and non-hydrostatic
computations.

\begin{figure}[H]
  \centering
  \includegraphicsmaybe{[width=0.9\textwidth]}{../img/FreeSurfaceHydro.png}
  \caption{Free surface at final time step for the hydrostatic case.}
  \label{t3d:pildepon:FreeSurfHydro}
\end{figure}

\begin{figure}[H]
  \centering
  \includegraphicsmaybe{[width=0.9\textwidth]}{../img/FreeSurfaceNH.png}
  \caption{Free surface at final time step for the non-hydrostatic case.}
  \label{t3d:pildepon:FreeSurfNH}
\end{figure}

\section{Conclusion}

\telemac{3D} can be used to study the hydrodynamic impact of engineering
works (like bridge piers), and to analyse unsteady flow, such as the
Karman vortices.
