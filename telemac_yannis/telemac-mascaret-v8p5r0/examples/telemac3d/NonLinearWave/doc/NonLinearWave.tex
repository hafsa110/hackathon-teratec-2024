\chapter{Dingemans wave over a bar (NonLinearWave)}

\section{Purpose}

This test demonstrates the ability of \telemac{3D} to simulate the
evolution of a monochromatic linear wave over a bar.
This test case corresponds to a physical model and measurements
published by Dingemans (conditions C) \cite{Dingemans1994}.

\section{Description}

The configuration is a tank 32~m long and 0.3~m wide.
The evolution of the topography along the channel is presented in Figure
\ref{fig:nonlinear:bottom}. It represents a bar.
A wave is imposed at the entrance of the channel.
The goal is to simulate the evolution of this wave when propagating over
the bar.

%The simulation is made with and without hydrostatic hypothesis.

\begin{figure}[H]
 \centering
\includegraphicsmaybe{[width=0.85\textwidth]}{../img/FreeSurface_bottom.png}
\includegraphicsmaybe{[width=0.85\textwidth]}{../img/FreeSurface_long_profile.png}
 \caption{Free surface and bottom at the end of simulation.}\label{fig:nonlinear:bottom}
\end{figure}

\subsection{Initial and Boundary Conditions}

The computation is initialised with a constant elevation equal to 0~m
and no velocity.

The boundary conditions are only closed lateral boundaries:
\begin{itemize}
\item For the solid walls, a slip condition on channel banks is used for the
velocities,
\item No bottom friction,
\item Upstream a wave is imposed at the entrance (amplitude 0.04~m).
\end{itemize}

\subsection{Mesh and numerical parameters}

The 2D mesh is made of 7,680 triangular elements (5,124 nodes).

10 planes are regularly spaced in the vertical direction
(see Figure \ref{t3d:nonlinear:fig:meshV}).

\begin{figure}[!htbp]
 \centering
 \includegraphicsmaybe{[width=\textwidth]}{../img/MeshV.png}
 \caption{Initial vertical mesh along $y$ = 0.15~m.}
 \label{t3d:nonlinear:fig:meshV}
\end{figure}

The non-hydrostatic version of \telemac{3D} is used.

To solve the advection, the N-type MURD scheme (scheme \#4) is used for velocities.
The implicitation coefficients for depth and velocities are equal to 0.5 to be
accurate.

The time step is 0.0025~s for a simulated period of 33~s.

\subsection{Physical parameters}

No diffusion is considered.

\section{Results}

Figure \ref{fig:nonlinear:bottom} presents the general shape of the free surface at the end
of the simulation.\\
Figure \ref{fig:nonlinear:measure} presents a comparison of the \tel simulation and the
experimental results at various locations in the channel.
This comparison shows a very good agreement between results and
measurements which is comparable to the one obtained with 1D Boussinesq
models as published in \cite{Benoit2001}.

\begin{figure}[H]
 \centering
\includegraphicsmaybe{[width=0.49\textwidth]}{../img/pt0.png}
\includegraphicsmaybe{[width=0.49\textwidth]}{../img/pt1.png}
\includegraphicsmaybe{[width=0.49\textwidth]}{../img/pt2.png}
\includegraphicsmaybe{[width=0.49\textwidth]}{../img/pt3.png}
\includegraphicsmaybe{[width=0.49\textwidth]}{../img/pt4.png}
\includegraphicsmaybe{[width=0.49\textwidth]}{../img/pt5.png}
\caption{Comparison with measurements of Dingemans.}\label{fig:nonlinear:measure}
\end{figure}

\section{Conclusion}

\telemac{3D} simulates correctly the evolution of a wave on a bar.

%[1] DINGEMANS M.W., Comparison of computations with Boussinesq-like
%models and laboratory measurements.
%MAST-G8M note, H1684, Delft Hydraulics, 32 pp. 1994.

%[2] BENOIT M., Projet CLAROM-ECOMAC (FICHE CEP\&M M06101.99).
%Modélisation non-linéaire par les équations de Boussinesq de la
%propagation des vagues non-déferlantes en zone côtière.
%Rapport EDF-LNHE HP-75/01/069. 2001.
