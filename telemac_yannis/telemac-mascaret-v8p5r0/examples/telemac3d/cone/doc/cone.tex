\chapter{Advection of tracers with a rotating cone (cone)}

\section{Purpose}
This test shows the performance of the finite element advection schemes of
\telemac{3D} for passive scalar transport in a time dependent case.
It shows the advection of a tracer (or any other passive scalar)
in a square basin with flat frictionless bottom.
%and with closed boundaries.

%--------------------------------------------------------------------------------------------------
\section{Description}

\subsection{Geometry and mesh}

The domain is a 20.1~m long square.
The 2D mesh is made from a regular grid from which every square is cut in half,
with 8,978 elements and 4,624 nodes.
6 planes are regularly spaced on the vertical (see Figure \ref{t3d:cone:fig:meshV}).

\begin{figure}[h!]
\centering
\includegraphicsmaybe{[width=0.6\textwidth]}{../img/cone_mesh.png}
\caption{2D domain and mesh of the cone test case.}
\label{t3d:cone:mesh}
\end{figure}

\begin{figure}[!htbp]
 \centering
 \includegraphicsmaybe{[width=0.9\textwidth]}{../img/cone_meshV.png}
 \caption{Initial vertical mesh.}
 \label{t3d:cone:fig:meshV}
\end{figure}

\subsection{Initial condition}

The water depth is constant in time and in space, equal to 1~m.
The velocity field is constant in time as well and is divergence free:
\begin{equation*}
  \vec{u}=\left\{
         \begin{array}{l}
          u(x,y)=-(y-y_0) \\
          v(x,y)=(x-x_0)
         \end{array}\right.
\end{equation*}
With $x_0$ = 10.05~m and $y_0$ = 10.05~m.
The initial value for the tracer is given by the Gaussian function
off-centered 4.95~m to the right of $(x_0, y_0)$:
\begin{equation*}
c^0(x,y)=e^{-\frac{1}{2}[(x-15)^2+(y-10.2)^2]}
\end{equation*}

\subsection{Analytic solution}

The tracer is described by a Gaussian function and is submitted to a rotating
velocity field.
After one period, we expect that the tracer function has the same position and
the same values as the initial condition (i.e. maximum value equal to 1 at the center).
The analytical solution for the tracer $c$ is given by:
\begin{equation*}
c(x,y,t)=e^{-\frac{1}{2}[X^2+Y^2]}
\end{equation*}
with:
\begin{equation*}
\left\{
    \begin{array}{ll}
        X = x - x_0 - R \cos(\omega t) \\
        Y = y - y_0 - R \sin(\omega t)
    \end{array}
\right.
\end{equation*}
where $R$ = 4.95~m.

\subsection{Physical parameters}

In this case, the tracer advection equation is solved using fixed hydrodynamic conditions.
No bottom friction is imposed and the diffusivities of velocities and tracers are set to zero.
Angular velocity of the rotating cone $\omega$ is equal to 1 rad.s$^{-1}$ which gives a
rotation period equal to $T=2\pi$ (6.2831854820251465~s).

Diffusion for velocities or tracers are both set to 0.

\subsection{Numerical parameters}

The simulation time is set to one period of rotation.
The time step is chosen in order to do the whole period in 64 steps, so it is
equal to 0.098174771~s.

The hydrostatic version is used.

For tracers advection, nearly all the numerical schemes available in \telemac{3D}
are tested (except strong characteristics which are known to be non conservative).
For weak characteristics the number of Gauss points is set to the single
possible choice: 6.
For distributive schemes, like predictor-corrector (PC) schemes (scheme 4 and 5
with options 2 or 3) and locally implicit schemes (LIPS: scheme 4 and 5 with option 4),
the number of corrections is set to 5, which is usually sufficient to converge
to accurate results.
For the locally implicit schemes (scheme 4 and 5 with option 4), the number of
sub-steps is equal to 10.

The conjugate gradient is used to solve every linear system except when solving
the diffusion of tracers for the SUPG scheme (= 2) for which GMRES is used
with Krylov dimension equal to 10.
In this last case only diagonal preconditioning is used (2 = default option)
whereas the combo 34 (= 17 $\times$ 2) is possible for other advection schemes
(and conjugate gradient).

Accuracy for solving the tracers diffusion step is set to 10$^{-12}$.

%--------------------------------------------------------------------------------------------------
\section{Results}

\subsection{Comparison of schemes}

The final contour maps after one rotation of the cone are plotted for each
scheme in Figures \ref{t3d:cone:profiles1} and \ref{t3d:cone:profiles2}.

White areas mean the values are outside [0;1].
The shape of the Gaussian function is rather distorted with SUPG (2) and original
MURD schemes (PSI, N, Leo Postma, respectively 5, 4 and 3 with scheme option = 1).
Using corrections (Predictor-Correction or LIPS) improve the results with less
distorded shapes combined to a peak better computed (and closer to 1):
i.e. schemes 4 or 5 with scheme option 2, 3 or 4.
NERD schemes (13 and 14) give lightly better results than MURD schemes,
but are more diffusive than using corrections from MURD schemes.
Weak characteristics give the best results for the peak after one period
but some values are negative that is a strong drawback for tracers.
SUPG also computes many negative values in addition to a quite distorded shape.
%One dimensional profiles are also extracted from
%slice plane $(x,y=10,z)$ at $t=T/2$ and $t=T$ on figure \ref{t3d:cone:1dslice}.

%\newpage

\begin{figure}[H]
\begin{minipage}[t]{0.50\textwidth}
 \centering
 \includegraphicsmaybe{[width=0.95\textwidth]}{../img/figure_EX.png}
\end{minipage}
\begin{minipage}[t]{0.50\textwidth}
 \centering
 \includegraphicsmaybe{[width=0.95\textwidth]}{../img/figure_WCHAR.png}
\end{minipage}
\begin{minipage}[t]{0.50\textwidth}
 \centering
 \includegraphicsmaybe{[width=0.95\textwidth]}{../img/figure_NERD13.png}
\end{minipage}
\begin{minipage}[t]{0.50\textwidth}
 \centering
 \includegraphicsmaybe{[width=0.95\textwidth]}{../img/figure_NERD14.png}
\end{minipage}
\begin{minipage}[t]{0.50\textwidth}
 \centering
 \includegraphicsmaybe{[width=0.95\textwidth]}{../img/figure_LIPS.png}
\end{minipage}
\begin{minipage}[t]{0.50\textwidth}
 \centering
 \includegraphicsmaybe{[width=0.95\textwidth]}{../img/figure_SUPG.png}
\end{minipage}
  \caption{Cone test: contour maps of tracer after one period of rotation, for the advection schemes of \telemac{3D}.}
 \label{t3d:cone:profiles1}
\end{figure}

\begin{figure}[H]
\begin{minipage}[t]{0.50\textwidth}
 \centering
 \includegraphicsmaybe{[width=0.95\textwidth]}{../img/figure_PSIPC1.png}
\end{minipage}
\begin{minipage}[t]{0.50\textwidth}
 \centering
 \includegraphicsmaybe{[width=0.95\textwidth]}{../img/figure_NPC1.png}
\end{minipage}
\begin{minipage}[t]{0.50\textwidth}
 \centering
 \includegraphicsmaybe{[width=0.95\textwidth]}{../img/figure_PSIPC2.png}
\end{minipage}
\begin{minipage}[t]{0.50\textwidth}
 \centering
 \includegraphicsmaybe{[width=0.95\textwidth]}{../img/figure_NPC2.png}
\end{minipage}
\begin{minipage}[t]{0.50\textwidth}
 \centering
 \includegraphicsmaybe{[width=0.95\textwidth]}{../img/figure_PSI.png}
\end{minipage}
\begin{minipage}[t]{0.50\textwidth}
 \centering
 \includegraphicsmaybe{[width=0.95\textwidth]}{../img/figure_N.png}
\end{minipage}
\begin{minipage}[t]{0.50\textwidth}
 \centering
 \includegraphicsmaybe{[width=0.95\textwidth]}{../img/figure_LPO.png}
\end{minipage}
 \caption{Cone test: contour maps of tracer after one period of rotation, for the advection schemes of \telemac{3D}.}
 \label{t3d:cone:profiles2}
\end{figure}

\begin{figure}[H]
\begin{minipage}[t]{0.50\textwidth}
 \centering
 \includegraphicsmaybe{[width=\textwidth]}{../img/figure_1d_halfT.png}
\end{minipage}
\begin{minipage}[t]{0.50\textwidth}
 \centering
 \includegraphicsmaybe{[width=\textwidth]}{../img/figure_1d_T.png}
\end{minipage}
\caption{1D solution along slice plane $(x,y)$, $y=10$ at $t=T/2$ (left) and $t=T$ (right).}
\label{t3d:cone:1dslice}
\end{figure}

%\newpage

\subsection{Maximum principle}

The minimum value of the Gaussian function is measured after one rotation.
The maximum value is computed as well, in order to check the respect of the
maximum principle (or monotonicity).
Results are shown in Figures \ref{t3d:cone:minmax}.

\begin{figure}[H]
\centering
\includegraphicsmaybe{[width=0.9\textwidth]}{../img/t3d_cone_maxT.png}
\includegraphicsmaybe{[width=0.9\textwidth]}{../img/t3d_cone_minT.png}
\caption{Maximum and minimum values of tracer after one rotation of the cone.}
\label{t3d:cone:minmax}
\end{figure}

As already shown in the previous section, weak characteristics and SUPG compute
negative values which is a strong drawback when modelling tracers.

MURD family schemes (all other advection schemes) satisfy the maximum principle
as seen in Figures \ref{t3d:cone:minmax} (no value outside [0, 1]).

%\subsection{Accuracy}
%
% TO BE DONE
%
%In order to evaluate the behaviour of the scheme, the error norms $L^1, L^2, L^{\infty}$ are computed.
%Error norms are integrated over time to take into account the unsteady nature of the problem.
%The error norms integrated over time for one rotation of the cone are reported in figure \ref{t3d:cone:error_timeintegrals}.
%
%\begin{figure}[H]
%\centering
%\includegraphicsmaybe{[width=0.9\textwidth]}{../img/t3d_cone_errors_timeintegrals.png}
%\caption{Error norms integrated over time for one rotation of the cone.}
%\label{t3d:cone:error_timeintegrals}
%\end{figure}

\section{Conclusion}
\telemac{3D} is able to model passive scalar transport problems.
