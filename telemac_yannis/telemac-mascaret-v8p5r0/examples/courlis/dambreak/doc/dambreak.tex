\chapter{Dambreak}\label{chapter:dambreak}

\section{Purpose}

This test simulates the erosion after a dambreak.

\section{Description}

\subsection{Physical parameters}

The test case of dam failure induces the presence of shock. 
It allows, among other things, to verify the capacity of the numerical
scheme to manage regime changes and to capture shocks.
As a general rule, the method using a splitting resolution of the 
Saint-Venant-Exner system does not allow to ensure the stability of 
the solution for this type of test case.
Considering that this case involves regime changes, the river kernel 
does not allow to manage this type of configuration and therefore, 
it will not be tested.
The parameters are given in the table \ref{paramDambreak}.
The final time is T=1.4 s.

\begin{table}[h]
\begin{center}
\begin{tabular}{l r l l}
  \hline
  Channel width             & $l=$               &$0.25$              &m \\
  Channel length            & $L=$               &$6$                 &m \\
  \hline
  Gravity constant  & $g=$             &$9.8$               &m.s$^{-2}$ \\
  Water density             & $\rho_w=$          &$10^3$              &kg.m$^{-3}$ \\
  Sediment density          & $\rho_s=$          &$2.65\times10^3$    &kg.m$^{-3}$ \\
  Grain diameter            & $d=$               &$3.9\times10^{-3}$  &m \\
  \hline
  \hline
  Inflow            &$q=$                &$0$                 & m$^2$.s$^{-1}$ \\
  Water depth                &$h(x \leqslant 3)=$ &$0.35$              & m  \\
                               &$h(x>3)=$           &$0.01$              & m  \\
  \hline
  \hline
  Strickler coefficient & $K_h=$  & $60$  & m$^{1/3}$.s$^{-1}$ \\
                           & $K_s=$  & $88$  & m$^{1/3}$.s$^{-1}$ \\
  Skin friction Strickler      & $K_p=$  & $20$  & m$^{1/3}$.s$^{-1}$ \\
  \hline
\end{tabular}
\caption{\label{paramDambreak} Parameters for the test case of dam failure.}
\end{center}
\end{table}

\subsection{Numerical parameters}

The mascaret kernel is used to perform this simulation. The time step 
is variable respecting a Courant number of 0.8. The mesh size is 0.02 
m.

\section{Results}

Figure \ref{fig:dambreak:results} shows the bottom evolution at the
beginning and the end of the simulation. Despite some small
instabilities, the behavior of the solution corresponds to what 
is expected.

\begin{figure}[h]
 \centering
 \includegraphicsmaybe{[width=0.7\textwidth]}{../img/profile.png}
 \caption{Bottom elevation at the initial and final time step.}
 \label{fig:dambreak:results}
\end{figure}

\section{Conclusion}

\courlis is able to simulate a dambreak.



