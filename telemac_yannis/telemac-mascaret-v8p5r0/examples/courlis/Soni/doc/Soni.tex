\chapter{Soni}\label{chapter:Newton}

\section{Purpose}

The laboratory experiment of \cite{soni1981laboratory} 
reports the phenomenon of deposition
following a sudden increase in the supply of sediments in a channel.

\section{Description}

The experiment is carried out in a rectangular channel covered with
an erodible bed of uniform sandy sediment. It is carried out in two
stages. The first one consists in reaching a dynamic equilibrium 
state under a fixed constant flow. After having obtained a stable 
equilibrium slope, an addition of sediments superior to the transport 
capacity is injected at the entrance of the channel. This addition 
results in a deposit in the system and the measurements of the 
evolution of the funds begin. It is this second step that is modeled.

During the experiment, the presence of ripples and dunes was observed
and more particularly in the part where the deposit is not pronounced. 
The measurements presented are averaged. 

Figure \ref{soni:fig:initial} shows the initial longitudinal
profile of the flume.

\begin{figure}[h]
 \centering
 \includegraphicsmaybe{[width=0.7\textwidth]}{../img/profile_init.png}
 \caption{Initial longitudinal profile of the flume.}
 \label{soni:fig:initial}
\end{figure}

\section{Physical parameters}

\begin{table}[H] %create a central table
   \begin{center}
   \caption{Geometric, hydraulic and sedimentary characteristics of Newton's experiment} %legende
       \begin{tabular}{|c|c|} %beginning of the table
       \hline %horizontal line
   Length of the channel& $L=30$ m \\%the & symbolizes a horizontal line
   Width of the channel&$ B=0.2$ m \\
   Slope of the channel& $I=0.00427$ \\
   \hline
  Upstream discharge &$Q=0.00571$ m$^{3}$.s$^{-1}$\\
  Downstream water depth &$ H=0.072$ m \\
  Mean velocity &$V=0.49$ m.s$^{-1}$ \\  
    \hline
   Concentration initially injected& $C_s=4.88$ kg.m$^{-3}$ \\
   Concentration of the sand layer& $C=1650$ kg.m$^{-3}$ \\
   Median diameter of the sand& $d_{50}=0.32$ mm \\   
   Mass density of the sediment& $\rho_s = 2650$ kg.m$^{-3}$\\
   Settling velocity of the sand& $w_s=0.042$ m.s$^{-1}$\\
  \hline
        \end{tabular}
     \end{center}
\end{table}

The Strickler coefficient is determined by a calibration on the
initial water line: $K_h = 62$ m$^{1/3}$.s$^{-1}$. 
The bedload transport formula chosen is Meyer-Peter and Muller
formula.
The friction 
coefficient for the transport formula has been set to 62 
m$^{1/3}$.s$^{-1}$ and the skin friction coefficient to 62
m$^{1/3}$.s$^{-1}$.

\section{Numerical parameters}

The mesh size is refined at the beginning of the channel (5cm mesh). 
For the rest of the domain, the meshes are every 25cm.

The model is run with the 3 kernels. The time step is set to 0.5 s with 
Sarap, 0.15 s with Rezo and it is variable with mascaret, respecting 
a Courant number of 0.8.

\section{Results}

Figure \ref{soni:fig:sarap} shows the results with the Sarap kernel
in comparison with experimental data.

\begin{figure}[h]
 \centering
 \includegraphicsmaybe{[width=0.7\textwidth]}{../img/profile_T_sarap.png}
 \caption{Sarap results and diffenrent times.}
 \label{soni:fig:sarap}
\end{figure}

Figure \ref{soni:fig:rezo} shows the results with the Rezo kernel
in comparison with experimental data.

\begin{figure}[h]
 \centering
 \includegraphicsmaybe{[width=0.7\textwidth]}{../img/profile_T_rezodt.png}
 \caption{Rezo results and diffenrent times.}
 \label{soni:fig:rezo}
\end{figure}

Figure \ref{soni:fig:mascaret} shows the results with the Rezo kernel
in comparison with experimental data.

\begin{figure}[h]
 \centering
 \includegraphicsmaybe{[width=0.7\textwidth]}{../img/profile_T_mascaret.png}
 \caption{Mascaret results and diffenrent times.}
 \label{soni:fig:mascaret}
\end{figure}

\section{Conclusion}

Courlis is able to represent the deposition process 
of a laboratory channel 
experimental data, with every hydraulic kernels.
