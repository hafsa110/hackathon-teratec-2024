\chapter{Bedload formulae test}\label{chapter:Bedload formulae test}

\section{Purpose}

This test case tests the 4 bedload transport formulae in 6 tests each.

\section{Description}

\subsection{Physical parameters}

For each bedload formula a case number is associated
(Meyer-Peter and Muller \cite{meyer1948formulas}: 0, Lefort: 1,
Recking 2013 \cite{recking2013simple}: 2, Recking 2015: 3).

The base configuration is an inflow slope of 1\% with an imposed
upstream discharge of 100 m$^3$/s. Its sediment distribution, noted D$_1$ is
d$_{84}$=0.0573 m, d$_m$=0.03 m, d$_{16}$=0.00573 m,
d$_{50}$=0.27285714 m.
The influence of the slope is studied with cases with 2\% slope and
0.5\% slope.
Then the inflow vary from 20 m$^3$/s to 300 m$^3$/s.
A second distribution (D$_2$) of the sediment size is also tested:
d$_{84}$=0.09545455 m, d$_m$=0.05 m, d$_{16}$=0.00954545 m,
d$_{50}$=0.4545455 m.

Table \ref{tab:formulea:config} summarize all the tested configuration:

\begin{table}[h] %create a central table
\begin{center}
\caption{Configurations tested.\label{tab:formulea:config}} %legende
\begin{tabular}{|c|c|c|c|c|} %beginning of the table

\hline %horizontal line
Case number & Formula & Slope (\%) & Inflow (m$^3$/s) & Sediment distribution\\
\hline
000 & Meyer-Peter and Muller & 1 & 100 & D$_1$\\
\hline
011 & Meyer-Peter and Muller & 0.5 & 100 & D$_1$\\
\hline
012 & Meyer-Peter and Muller & 2 & 100 & D$_1$\\
\hline
021 & Meyer-Peter and Muller & 1 & 300 & D$_1$\\
\hline
022 & Meyer-Peter and Muller & 1 & 20 & D$_1$\\
\hline
031 & Meyer-Peter and Muller & 1 & 100 & D$_2$\\
\hline
100 & Lefort & 1 & 100 & D$_1$\\
\hline
111 & Lefort & 0.5 & 100 & D$_1$\\
\hline
112 & Lefort & 2 & 100 & D$_1$\\
\hline
121 & Lefort & 1 & 300 & D$_1$\\
\hline
122 & Lefort & 1 & 20 & D$_1$\\
\hline
131 & Lefort & 1 & 100 & D$_2$\\
\hline
200 & Recking 2013 & 1 & 100 & D$_1$\\
\hline
211 & Recking 2013 & 0.5 & 100 & D$_1$\\
\hline
212 & Recking 2013 & 2 & 100 & D$_1$\\
\hline
221 & Recking 2013 & 1 & 300 & D$_1$\\
\hline
222 & Recking 2013 & 1 & 20 & D$_1$\\
\hline
231 & Recking 2013 & 1 & 100 & D$_2$\\
\hline
300 & Recking 2015 & 1 & 100 & D$_1$\\
\hline
311 & Recking 2015 & 0.5 & 100 & D$_1$\\
\hline
312 & Recking 2015 & 2 & 100 & D$_1$\\
\hline
321 & Recking 2015 & 1 & 300 & D$_1$\\
\hline
322 & Recking 2015 & 1 & 20 & D$_1$\\
\hline
331 & Recking 2015 & 1 & 100 & D$_2$\\
\hline
\end{tabular}
\end{center}
\end{table}

\subsection{Numerical parameters}

The mascaret kernel used is Sarap, with a duration of 2 s and a time step of 1 s.

\section{Results}

Figure \ref{fig:bedload} compares the numerical solution given by the
code to a reference solution for each test case. It shows a very good
agreement between simulated and expected results.

\begin{figure}[h]
 \centering
 \includegraphicsmaybe{[width=0.7\textwidth]}{../img/flux_error.png}
 \caption{Comparison between simulated and expected bedload fluxes.}
 \label{fig:bedload}
\end{figure}

\section{Conclusion}

Courlis is able to reproduce the accurate bedload fluxes for different
transport formulae, slope conditions, input discharges and grain size
distribution.


