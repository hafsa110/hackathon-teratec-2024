\chapter{Van Rijn}\label{chapter:VR}

\section{Purpose}

This test case tests the non-cohesive sediment transport in suspension.

\section{Description}

This experiment, conducted by Van Rijn at the Delft Hydraulics 
Laboratory, consists in studying the phenomenon of progressive erosion
associated with the trapping of sediments in a pit in the presence of a
stationary and uniform flow with a dominant mode of transport which is 
suspension. The local excavations are too short for the water level to 
drop at the deepened part, it remains imposed by the downstream reach, 
preventing acceleration upstream. In the area of the excavation, the 
water level increases and the flow velocity decreases. The transport 
capacity of the flow decreases, the sediments coming from upstream are
trapped in the pit. Downstream, the head decreases and the velocity 
increases, the transport capacity increases and the flow must therefore
take material to restore the solid flow to its initial level
(Figure 1). Thus, progressive deposition in the pit and progressive 
erosion in the downstream portion of the channel are expected. 
Overall, the pit moves downstream and deforms. 

\section{Physical parameters}

The physical parameters of the model are:

\begin{table}[h!]
   \begin{center}
   \caption{Geometric, hydraulic and sedimentary characteristics}
       \begin{tabular}{|c|c|}
       \hline
   Channel length& $L=30$ m \\ 
   Channel width&$ B=0.5$ m \\
   Channel slope& $I=0.001517$ \\  
   Pit length& $L_f=3$ m \\
   Pit slope& $p_f=0.1$ \\
   Pit depth& $h_f=0.15$ m \\
       \hline
  Upstream inflow &$Q=0.09945$ m$^{3}$.s$^{-1}$ \\
  Downstream water depth &$ H=0.39$ m \\
  Strickler coefficient& $K=43$ m$^{1/3}$.s$^{-1}$\\
       \hline  
   Initial concentration& $C_s=0.21$ kg.m$^{-3}$ \\ 
   Soil sand concentration& $C=1590 $ kg.m$^{-3}$\\
   Mean diameter& $d_{50}=0.16$ mm \\
   Settling velocity&$w_s=0.013$ m.s$^{-1}$ \\
  \hline 
     \end{tabular}
     \end{center}
\end{table}

Figure \ref{fig:VR:init} shows the initial water depth and bottom 
elevation.

\begin{figure}[h]
 \centering
 \includegraphicsmaybe{[width=0.7\textwidth]}{../img/profile_init.png}
 \caption{Initial configuration.}
 \label{fig:VR:init}
\end{figure}

\section{Numerical parameters}

The longitudinal space step is 50 m. The mascaret kernel is used 
with a variable time step and a Courant number of 0.8.

\section{Results}

Figure \ref{fig:VR:final} shows the computed water depth and bottom
elevation with a comparison with experimental data.

\begin{figure}[h]
 \centering
 \includegraphicsmaybe{[width=0.7\textwidth]}{../img/profile_final.png}
 \caption{Final results.}
 \label{fig:VR:final}
\end{figure}

\section{Conclusion}

\courlis is able to model the non-cohesive suspended sediment
of the Van Rijn experiment.
