\chapter{Newton}\label{chapter:Newton}

\section{Purpose}

The \cite{newton1951experimental}'s laboratory experiment presents the phenomenon of 
progressive erosion in a channel with an erodible bottom:
in a rectangular channel, fed by an upstream solid flow, 
initially at equilibrium, the solid input is completely stopped. 
Measurements of the evolution of the bottom are made up to 24 hours.

\section{Description}

The experiment takes place in two steps. In a rectangular channel,
a state of dynamic equilibrium is obtained with the help of
a sediment supply. The channel is assumed to be in equilibrium
when the amount of sediment injected upstream is completely 
transported downstream but also when the slope of the bottom and 
that of the free surface become uniform. At the downstream end, 
a weir was placed to maintain a constant water level during the 
experiment. Once equilibrium is reached, the upstream supply of 
sediment is stopped, which leads to a change in the bottom. The 
solid concentration at the inlet becomes zero and thus lower than 
the equilibrium concentration. To reach the latter, the system 
erodes the sediments of the domain and tends to find a new equilibrium 
state. The measurements are made during this second step, it is this 
part of the experiment that is modeled.

Total transport with dominant bedload was observed. This observation 
may give an indication of the most suitable transport formula. 
In addition, the presence of dunes on the bed was identified during 
the experiment.

Figure \ref{newton:fig:initial} shows the initial longitudinal
profile of the flume.

\begin{figure}[h]
 \centering
 \includegraphicsmaybe{[width=0.7\textwidth]}{../img/profile_init.png}
 \caption{Initial longitudinal profile of the flume.}
 \label{newton:fig:initial}
\end{figure}

\section{Physical parameters}

\begin{table}[H] %create a central table
   \begin{center}
   \caption{Geometric, hydraulic and sedimentary characteristics of Newton's experiment} %legende
       \begin{tabular}{|c|c|} %beginning of the table
       \hline %horizontal line
   Length of the channel& $L=9.14$ m \\%the & symbolizes a horizontal line
   Width of the channel&$ B=0.3048$ m \\
   Slope of the channel& $I=0.00416$ \\
   \hline
  Upstream discharge &$Q=0.00566$ m$^{3}$.s$^{-1}$\\
  Downstream water depth &$ H=0.041$ m \\
  Mean velocity &$V=0.45$ m.s$^{-1}$ \\  
    \hline
   Concentration initially injected& $C_s=0.88$ kg.m$^{-3}$ \\
   Concentration of the sand layer& $C=1610$ kg.m$^{-3}$ \\
   Median diameter of the sand& $d_{50}=0.68$ mm \\   
   Mass density of the sediment& $\rho_s = 2650$ kg.m$^{-3}$\\
   Settling velocity of the sand& $w_s=0.09$ m.s$^{-1}$\\
  \hline
        \end{tabular}
     \end{center}
\end{table}

The Strickler coefficient is determined by a calibration on the
initial water line: $K_h = 69$ m$^{1/3}$.s$^{-1}$. 
The bedload transport formula chosen is Meyer-Peter and Muller
formula.
The friction 
coefficient for the transport formula has been set to 69 
m$^{1/3}$.s$^{-1}$ and the skin friction coefficient to 88
m$^{1/3}$.s$^{-1}$.

\section{Numerical parameters}

The mesh size is refined at the beginning of the channel (5cm mesh). 
For the rest of the domain, the meshes are every 25cm.

The model is run with the 3 kernels. The time step is set to 1 s with 
Sarap, 0.1 s with Rezo and it is variable with mascaret, respecting 
a Courant number of 0.8.

\section{Results}

Figure \ref{newton:fig:sarap} shows the results with the Sarap kernel
in comparison with experimental data.

\begin{figure}[h]
 \centering
 \includegraphicsmaybe{[width=0.7\textwidth]}{../img/profile_T_sarap.png}
 \caption{Sarap results and diffenrent times.}
 \label{newton:fig:sarap}
\end{figure}

Figure \ref{newton:fig:rezo} shows the results with the Rezo kernel
in comparison with experimental data.

\begin{figure}[h]
 \centering
 \includegraphicsmaybe{[width=0.7\textwidth]}{../img/profile_T_rezodt.png}
 \caption{Rezo results and diffenrent times.}
 \label{newton:fig:rezo}
\end{figure}

Figure \ref{newton:fig:mascaret} shows the results with the Rezo kernel
in comparison with experimental data.

\begin{figure}[h]
 \centering
 \includegraphicsmaybe{[width=0.7\textwidth]}{../img/profile_T_mascaret.png}
 \caption{Mascaret results and diffenrent times.}
 \label{newton:fig:mascaret}
\end{figure}

\section{Conclusion}

Courlis is able to represent the erosion of a laboratory channel 
experimental data, with every hydraulic kernels.
