% case name
\chapter{heat\_exchange}
%
% - Purpose & Description:
%     These first two parts give reader short details about the test case,
%     the physical phenomena involved, the geometry and specify how the numerical solution will be validated
%
\section{Purpose}
%
This case is an example of the use of the heat exchange with atmosphere module
of \waqtel coupled with \telemac{3D}.
%
\section{Description}
%
A square basin at rest is considered (length and width = 200~m)
with flat bathymetry and elevation at 0~m.
%
\section{Computational options}
%
\subsection{Mesh}
%
The triangular mesh is composed of 272 triangular elements and 159 nodes
(see Figure \ref{fig:heat_exchange:mesh}).

\begin{figure}[H]
 \centering
 \includegraphicsmaybe{[width=0.6\textwidth]}{../img/res_mesh.png}
\caption{Horizontal mesh}
 \label{fig:heat_exchange:mesh}
\end{figure}

To build the 3D mesh of prisms, 26 planes are regularly spaced over the vertical.
The vertical mesh between nodes of coordinates (-100 ; 0) to (100 ; 0) can be
seen on Figure \ref{fig:heat_exchange:mesh_section}.

\begin{figure}[H]
 \centering
 \includegraphicsmaybe{[width=0.8\textwidth]}{../img/res_mesh_section.png}
\caption{Vertical mesh}
 \label{fig:heat_exchange:mesh_section}
\end{figure}
%
\subsection{Physical parameters}
%
The heat exchange module is activated by setting \telkey{WATER QUALITY PROCESS}
= 11 in the \telemac{3D} \telkey{STEERING FILE}.\\

The tracer used is temperature.
%
The density is considered as a function of the water temperature:
\begin{equation}
  \rho(T) = \rho_0 \left(1 - \alpha (T-T_0)^2\right),
\end{equation}
where $\rho_0$ = 999.972~kg.m$^{-3}$, $T_0$ = 4~$^{\circ}$C
and $\alpha$ = 7.10$^{-6}$.\\

Several default values have been modified in the \waqtel \telkey{STEERING FILE}:
\begin{itemize}
\item \telkey{ATMOSPHERE-WATER EXCHANGE MODEL} = 2 (i.e. heat exchange model
with complete balance),
\item \telkey{SECCHI DEPTH} = 2. (default = 0.9),
\item \telkey{COEFFICIENT TO CALIBRATE THE ATMOSPHERE-WATER EXCHANGE MODEL} =
0.0018 (default = 0.0025).
\end{itemize}

Wind and rain are taken into account in the Navier-Stokes equations
by setting the 2 booleans \telkey{WIND} and \telkey{RAIN OR EVAPORATION} to YES
in addition to \telkey{WATER QUALITY PROCESS} = 11.
They had to be set to YES so that this example works correctly.

\telkey{COEFFICIENT OF WIND INFLUENCE VARYING WITH WIND SPEED} has been let to
its default value YES so that it was not needed to choose a constant value for
\telkey{COEFFICIENT OF WIND INFLUENCE}.

The origin coordinates of a point representative to the atmospheric conditions
of the area are given by means of the keywords
\telkey{LATITUDE OF ORIGIN POINT} = 43.4458~$^{\circ}$ N and
\telkey{LONGITUDE OF ORIGIN POINT} = 5.1139~$^{\circ}$ E.
These coordinates are needed when the solar radiation is computed by the heat
exchange model with complete balance.\\

Input meteo data are given through an ASCII file with the help of the keyword
\telkey{ASCII ATMOSPHERIC DATA FILE}.
For several sampled times, the following variables are needed.
Since release 8.2 and the module \telfile{METEO\_TELEMAC} which enables the use
of flexible ASCII file for meteo data but with the need to use
fixed names of variables in the first lines, the order is not mandatory anymore.

The mandatory meteo variables are:
\begin{itemize}
\item wind speed + wind direction (shortnames WINDS and WINDD) or wind
  components along $x$ and $y$ in m/s,
\item air temperature (shortname TAIR) in $^{\circ}$C,
\item atmospheric pressure (shortname PATM) in mbar or hPa,
\item relative humidity (shortname HREL) in \%, 
\item cloud coverage (shortname CLDC) in octas,
\item rain (shortname RAINC or RAINI depending if it is given as cumulated
  variable or an interpolated variable as other usual variables),
\end{itemize}

If the user does not provide one of this data, a constant value over the entire
period of simulation is taken.
These constant values can be changed by means of the following keywords in the
\telemac{3D} \telkey{STEERING FILE}:
\begin{itemize}
\item \telkey{WIND VELOCITY ALONG X} and \telkey{WIND VELOCITY ALONG Y},
\item \telkey{AIR TEMPERATURE},
\item \telkey{VALUE OF ATMOSPHERIC PRESSURE},
\item \telkey{RELATIVE HUMIDITY},
\item \telkey{CLOUD COVER},
\item \telkey{RAIN OR EVAPORATION IN MM PER DAY}.
\end{itemize}

Solar radiation (shortname RAY3) is not mandatory but can be given by the user
to be read rather than computing it by the module:
In that case, the user has to set \telkey{SOLAR RADIATION READ IN METEO FILE}
= YES (default = NO).

The keywords \telkey{ORIGINAL DATE OF TIME} and \telkey{ORIGINAL HOUR OF TIME}
are set and reference date and hours are filled in with \telfile{REFDATE}
at the beginning of the \telkey{ASCII ATMOSPHERIC DATA FILE} so that
the computation can start at a different time of this file (see \telemac{3D}
user manual).\\

The $k$-$\epsilon$ turbulence model is used for the vertical direction whereas
a constant viscosity is used for the horizontal directions with the default value
of 10$^{-6}$~m$^2$/s.
%
\subsection{Initial and Boundary Conditions}
%
The initial water depth is 5~m with a fluid at rest.\\
The initial temperature is 20~$^{\circ}$C in the whole domain.\\

There are only closed lateral boundaries with free slip condition and
Nikuradse law is used to model friction at the bottom with a coefficient
0.001~m.
%
\subsection{General parameters}
%
The time step is 2~min = 120~s for a simulated period of 30~days
(= 2,592,000~s).
%
\subsection{Numerical parameters}
%
The non-hydrostatic version of \telemac{3D} is used.
It is mandatory to model stratifications correctly.

To solve the advection steps, the method of characteristics is chosen for the
velocities + $k$-$\epsilon$ and the PSI-type MURD scheme for the temperature
for CPU time reasons.

To reach convergence when solving linear systems, the keywords
\telkey{MASS-LUMPING FOR DIFFUSION} has to be set to 1. (default = 0.) whereas
\telkey{MAXIMUM NUMBER OF ITERATIONS FOR PPE} has to be set to 200 (Poisson
Pressure Equation).

To accelerate solving, preconditioning 34 (= 2 $\times$ 17) is used for
every conjugate gradient solver for diffusion step (velocities, tracers and
$k$-$\epsilon$).

To enable not too long computations, the following parameters have been relaxed:
\telkey{ACCURACY FOR PPE} = 10$^{-4}$ and \telkey{ACCURACY FOR PROPAGATION} = 10$^{-6}$
(default = 10$^{-8}$ for both of them).
%
%     We comment in this part the numerical results against the reference ones,
%     giving understanding keys and making assumptions when necessary.
%
\section{Results}
%
Figure \ref{fig:heat_exchange:res_evol} shows the temperature evolution along
time at 3 locations (top, middle and bottom of the water column at the center of
the square, same $x$ and $y$).

\begin{figure} [H]
\centering
\includegraphicsmaybe{[width=0.9\textwidth]}{../img/res_temp_evol.png}
 \caption{Temperature evolution}
 \label{fig:heat_exchange:res_evol}
\end{figure}

When it is not mixed (it is mixed e.g. between days 3 to 5 or 16 to 20, 21 to 23),
we can see daily patterns with higher temperature at the surface during the day
and a quite light stratification.

This can be confirmed by plotting temperature values on a vertical segment
over time at the same $x$ and $y$ location,
see Figure \ref{fig:heat_exchange:res_temp_segment}.
The vertical segment is located in the middle of the square at (0 ; 0).

\begin{figure} [H]
\centering
\includegraphicsmaybe{[width=0.8\textwidth]}{../img/res_temp_segment.png}
\caption{Temperature evolution}
\label{fig:heat_exchange:res_temp_segment}
\end{figure}

Anyway, due to wind blowing at the surface, the flow in not homogeneous in the
whole domaine as it can be seen in a vertical slice in
Figure \ref{fig:heat_exchange:res_temp_section} at final time.

\begin{figure}[H]
\centering
\includegraphicsmaybe{[width=0.9\textwidth]}{../img/res_temp_section.png}
\caption{Vertical distribution of temperature at final time}
\label{fig:heat_exchange:res_temp_section}
\end{figure}
%
\section{Conclusion}
%
\waqtel is able to model heat exchange with atmosphere phenomena when coupled
with \telemac{3D}.
%
% Here is an example of how to include the graph generated by validate_telemac.py
% They should be in test_case/img
%\begin{figure} [!h]
%\centering
%\includegraphics[scale=0.3]{../img/mygraph.png}
% \caption{mycaption}\label{mylabel}
%\end{figure}
