\chapter{Streeter and Phelps experiment using Benson and Krause formula
(waq2d\_eutro\_streeter\_phelps)}

\section{Purpose}

This case is an example of the use of the EUTRO module of \waqtel coupled with
\telemac{2D} with Benson and Krause formula to compute oxygen saturation
concentration $C_s$.

\section{Description}

In 1925, Streeter and Phelps published a work on the dissolved oxygen "sag curve"
representative of the oxygen decrease with downstream distance in the Ohio River
due to the degradation of soluble organic biochemical oxygen demand (BOD).
They proposed a mathematical equation to describe this observation, which has
since become widely known as the Streeter-Phelps equation.

A channel is considered (30~000~m long, 400~m wide, flat horizontal bottom
= -10~m).\\

\section{Computational options}

\subsection{Mesh}

The triangular mesh is composed of 3,200 triangular elements and 1,809 nodes
(see Figure \ref{fig:waq2d_eutro_streeter_phelps:mesh}).

\begin{figure}[H]
 \centering
 \includegraphicsmaybe{[width=0.9\textwidth]}{../img/res_mesh.png}
\caption{Mesh}
 \label{fig:waq2d_eutro_streeter_phelps:mesh}
\end{figure}

\subsection{Physical parameters}

Horizontal constant viscosity for velocity: with default value
(10$^{-6}$~m$^2$/s)\\
No diffusion for tracers\\
Bottom friction: no friction on the bottom\\

The EUTRO module is activated by setting \telkey{WATER QUALITY PROCESS} = 5
in the \telemac{2D} \telkey{STEERING FILE}.\\

8 tracers are considered:
\begin{itemize}
\item phytoplanktonic biomass (PHY),
\item dissolved mineral phosphorus (PO$_4$),
\item degradable phosphorus non assimilated by phytoplankton (POR),
\item dissolved mineral nitrogen assimilated by phytoplankton (NO$_3$),
\item degradable nitrogen assimilated by phyto phytoplankton (NOR),
\item ammoniacal load (NH$_4$),
\item organic load (L),
\item dissolved oxygen (O$_2$).
\end{itemize}

Only the following water quality parameters have been changed
in the \waqtel \telkey{STEERING FILE} compared to the default values:
\begin{itemize}
\item \telkey{FORMULA FOR COMPUTING CS} = 3 (i.e. Benson and Krause),
\item \telkey{CONSTANT OF DEGRADATION OF ORGANIC LOAD K120} = 0.36~d$^{-1}$,
\item \telkey{BENTHIC DEMAND} = 0~gO$_2$/m$^2$/d,
\item \telkey{FORMULA FOR COMPUTING K2} = 0 (i.e. $k_2$ is constant),
\item \telkey{WATER TEMPERATURE} = 20$^\circ$C (which is the mean water
temperature).
\item \telkey{WATER SALINITY} = 0~g/l (which is the mean water salinity).
\end{itemize}

Water temperature is 20$^\circ$C. Salinity is 0~mg/l.

\subsection{Initial and Boundary Conditions}

The initial water elevation is 0~m (= 10~m water depth).\\
A flow rate of 200~m$^3$/s is prescribed upstream.\\
An elevation of 0~m is prescribed downstream.\\

The initial values, upstream and downstream conditions for tracers
[PHY],[PO4],[POR],[NO3],[NOR],[NH$_4$]
are homogeneous and nil.

The initial conditions are 0~mg/l for [L] and 9.0921~mgO$_2$/l for [O$_2$],
which is the saturation concentration.

The upstream boundary conditions are 10~mg/l for [L] and 9.0921~mgO$_2$/l for
[O$_2$].

The downstream boundary conditions are 0~mg/l for [L] and 9.0921~mgO$_2$/l for
[O$_2$].

\subsection{General parameters}

The time step is 10~s for a simulated period of 691,200~s = 8~days.
This is the duration needed to reach equilibrium state.

The first 24 hours are the initialisation with the discharge rate increasing
from zero to 200~m$^3$/s.

\subsection{Numerical parameters}

The PSI scheme is chosen to solve the advection for the last 2 tracers
(L and O$_2$) whereas for the other tracers no advection scheme nor diffusion
scheme are used. It means that only tracers L and O$_2$ are activated.
Water temperature and salinity are constant parameters of the simulation.
Moreover, the method of characteristics is chosen to solve the advection for
the velocities.

\section{Analytical solution}
\label{Analytical}
An analytical solution for the longitudinal distribution of the DO concentration
can be calculated in the case of steady hydrodynamic conditions:

$$ DO(x)=C_s-(C_s-DO_o) e^{-k_a x/u} 
- \frac{k_d L_0}{k_a - k_d}
(e^{-k_d x/u} - e^{-k_a x/u})$$

Where $L_0$ is the upstream concentration of the organic load $L$,
$u$ is the stream velocity (m/s), $x$ is the distance from the upstream boundary
(m), $k_a$ is the coefficient of reaeration (s$^{-1}$), $k_d$ is the kinetic
degradation of organic load (s$^{-1}$),
$C_s$ is the saturation concentration (mgO$_2$/l) and $DO_o$ is the DO upstream
concentration (mgO$_2$/l).

Other representative values of the DO sag curve can be evaluated:

\begin{itemize}
\item The minimum dissolved oxygen $DO_{\textrm{min}}$ concentration occurring
along the river stretch:
 
$$DO_{\textrm{min}} = C_s + \frac{k_d L_0}{k_a} e^{-k_d x /u }$$
 
\item The Critical distance $x_c$ where the minimum dissolved oxygen
concentration occurs:
 $$x_c = \frac{u}{k_a - k_d} \log \left[ \frac{k_a} {k_d}
 \left( 1-\frac{k_a - k_d}{k_d} \frac{DO - C_s} {L_0} \right)\right] $$
\end{itemize}

The results with the parameters chosen are $DO_{\textrm{min}}$ = 6.92~mg/l and
$x_c$ = 7~330.33~m.

\section{Results}

Figure \ref{fig:waq2d_eutro_streeter_phelps:res} shows the longitudinal profile
of the modelled dissolved oxygen concentration compared to the analytical
solution at the last time step of the simulation.

\begin{figure} [H]
\centering
\includegraphicsmaybe{[width=\textwidth]}{../img/res_O2.png}
 \caption{Dissolved oxygen}
 \label{fig:waq2d_eutro_streeter_phelps:res}
\end{figure}

\section{Conclusion}

The submodule EUTRO of \waqtel can be coupled with \telemac{2D} to model
unsteady flows and is validated against the analytical solution of the
Streeter-Phelps dissolved oxygen sag curve.
