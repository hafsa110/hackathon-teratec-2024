\chapter{Flow in a channel with a soft boundary (soft)}

\section{Description}

This example demonstrates the use of the soft boundary in \telemac{2D}.
A soft boundary is a boundary where the imposed level is modified according to
the velocity at the boundary.
If the flow is into the model, the level is lowered and if the flow is out of
the model, the level is raised (see the manual for more details).
This is useful to eliminate unwanted jetting flow at the boundary, which can
occur when the imposed level varies across the width of a liquid boundary.

The test is for a channel with imposed level at the left and imposed outward
flow of 1~m/s at the right.
The channel is 10,000~m and 1,000~m wide.
The imposed level varies across the width of the channel from -0.05~m to 0.05~m.
This level is set in the \telfile{USER\_BORD} subroutine.
The model is run for 6 hours.
The model mesh is shown in Figure \ref{t2d:soft:fig:mesh}.

\begin{figure}[H]
 \centering
 \includegraphicsmaybe{[width=\textwidth]}{../img/Mesh.png}
 \caption{Model mesh.}
 \label{t2d:soft:fig:mesh}
\end{figure}

The model is run with a soft boundary applied at the left elevation boundary.
Option 1 is used with a coefficient of 0.1.

The keywords are given below:

\telkey{OPTION FOR SOFT BOUNDARIES} : 0; 1

\telkey{COEFFICIENT FOR SOFT BOUNDARIES} = 0.; 0.1

The model is also run with default values for these keywords, so that no soft
boundary is applied.

\section{Results}

Figure \ref{t2d:def:fig:inflow} shows the flow into the model for the end of the
run, for the case with no soft boundary.
Figure \ref{t2d:soft:fig:inflow} shows the same figure for the case with the
soft boundary.

It can be seen that the flow at the boundary is very uneven with flow into and
out of the channel.
The case with the soft boundary has much more uniform flow.


\begin{figure}[H]
\centering
\includegraphicsmaybe{[width=0.9\textwidth]}{../img/Inflow-Def.png}
\caption{Flow into model. Default.}
\label{t2d:def:fig:inflow}
\end{figure}

\begin{figure}[H]
\centering
\includegraphicsmaybe{[width=0.9\textwidth]}{../img/Inflow-Soft.png}
\caption{Flow into model. Soft boundary.}
\label{t2d:soft:fig:inflow}
\end{figure}

\section{Conclusions}

This example shows that \telemac{2D} is able to implement a soft boundary.
