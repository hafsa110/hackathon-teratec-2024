\chapter{Oil spill artificial river experiments (riv\_art)}

\section{Purpose}

An artificial river test campaign was conducted by Veolia Environnement
Recherche et Innovation in order to observe the capacity of oil spill pollutant
to dissolve PAHs (Polycyclic Aromatic Hydrocarbons).
The UBA (Umweltbundesamt German Federal Agency for the Environment) has on its
site 16 identical systems of artificial rivers with each 100 m in circumference.
Among these rivers called FSA (acronym for Fliess und
StillgewässersimulationsAnlage: simulator rivers and lakes), eight are located
outdoors.
A water flow is generated in these rivers with a screw pump.
A system for continuous measurement of physical parameters river is installed
for each river, and there is one weather station.
For the purpose of the experiment set up, two rivers were linked together to
increase the installation length and sinuosity (Figure \ref{fig:riv_art}).

\begin{figure}[htbp]
  \begin{center}
    \includegraphicsmaybe{[width=0.95\textwidth,height=0.15\textheight]}{./riv_art_sketch.png}
  \end{center}
  \caption{Artificial river sketch.}
  \label{fig:riv_art}
\end{figure}

The release of the hydrocarbon is achieved through a ring on water surface,
the pollutant is injected inside it.
Then, the ring is removed to allow the pollutant transport
(Figure \ref{fig:obs_oil_spill_in_riv_art}).
To observe the evolution of the concentration of dissolved PAHs, a fluorescence
probe is used.
Every morning a sample called "white" is made to know the initial concentration
of PAHs already present in the channel.
When the signal (\%) is approximately on the peak, a water sample is taken
during 30 seconds using an automatic device located with the probe.
The samples are then sent to the CEDRE for the analysis of dissolved
concentrations of PAHs in samples.
For each sample, there is therefore a concentration of total PAHs (ng/L) and a
probe signal (\%).

\begin{figure}[htbp]
  \begin{center}
    \includegraphicsmaybe{[width=0.45\textwidth,height=0.21\textheight]}{./image_ra.png}
    \includegraphicsmaybe{[width=0.4\textwidth,height=0.21\textheight]}{./obs_riv_art.png} \\
\begin{flushleft} \hspace{4.1cm} (a) \hspace{6cm} (b) \end{flushleft} 
  \end{center}
  \caption{Oil spill in artificial river without and with obstacles respectively (a) and (b).}
  \label{fig:obs_oil_spill_in_riv_art}
\end{figure}

\section{Description}

The finite element mesh consists of 23,234 nodes and about 43,000 triangles of
average size 0.08~m (Figure \ref{fig:riv_art_mesh}).
The flow velocity and surface elevation are imposed respectively on inflow and
outflow boundary conditions.
For shoreline nodes, solid wall conditions are considered.

\begin{figure}[h]
\begin{center}
  \includegraphicsmaybe{[width=1\textwidth]}{../img/Mesh.pdf}
\end{center}
\caption{Mesh.}
\label{fig:riv_art_mesh}
\end{figure}

In the simulation, a kerosene spill which occurs in the first artificial river
curve is considered.
A volume of $2\times 10^{-5}$ m$^3$ has been spilled into the channel.
The flow velocity is imposed to 0.1~m/s on inflow boundary and obstacles are in
the channel (Figure \ref{fig:obs_oil_spill_in_riv_art} b).

\section{Result}

In the simulation result, the particles represent the oil surface slick whereas
the eulerian tracer represents the dissolved petroleum in the water column.
The numerical and experiment concentrations are shown in Figure
\ref{fig:riv_art_simu}.

\begin{figure}[h]
\begin{center}
  \includegraphicsmaybe{[width=1\textwidth]}{../img/conc_over_time.pdf}
\end{center}
\caption{Kerosene concentration in the water column.}
\label{fig:riv_art_simu}
\end{figure}

The dissolved hydrocarbons concentration in the water column has the same order
of magnitude and compares well with experiments.
However, there is a delay (about 160~s) between the model results and the
experimental expected values.
This lag can be explained by the outdoor conditions which cannot be modelled,
such as gusts of wind.
