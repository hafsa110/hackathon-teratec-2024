\chapter{riogrande}

\section{Description}

%This example shows that \telemac{2D} is able to model river flow
%for a realistic river.

The configuration is a part of river (around 6~km long) with schematic bathymetry
defined by sections depending on ordinate $y$
(affine function with respect to $y$, see Figure \ref{t2d:riogrande:bottom}).

\begin{figure}[H]
 \centering
  \includegraphicsmaybe{[width=0.3\textwidth]}{../img/Bottom.png}
\caption{Bottom elevation.}
\label{t2d:riogrande:bottom}
\end{figure}

\subsection{Initial and boundary conditions}

The computation is initialised with a constant elevation equal to 13~m,
a constant velocity along $x$ equal to $\frac{0.25}{0.6\times1.1}$.

The boundary conditions are:
\begin{itemize}
\item For the solid walls, a slip condition on channel banks is used for the
velocities,
\item On the bottom, a Chézy law with friction coefficient equal to
55~m$^{1/2}$/s is prescribed,
\item Upstream a flowrate is prescribed,
linearly increasing from 0 to final flowrate during the simulation.
A velocity profile proportional to the square root of depth (option 4)
is used to prevent from potential dry segments along the liquid boundary,
\item Downstream the water level is also linearly increasing from 0 to final
surface level during the simulation.
\end{itemize}

\subsection{Mesh and numerical parameters}

The mesh (see Figure \ref{t2d:riogrande:meshbnd}) is made of 8,268
triangular elements (4,989 nodes).
%the types of boundary conditions used for boundaries can be seen in Figure.

\begin{figure}[!htbp]
\begin{minipage}[t]{0.5\textwidth}
 \centering
  \includegraphicsmaybe{[width=0.5\textwidth]}{../img/Mesh.png}
\end{minipage}
\begin{minipage}[t]{0.5\textwidth}
 \centering
  \includegraphicsmaybe{[width=0.5\textwidth]}{../img/BC.png}
\end{minipage}
\caption{Mesh and boundary conditions types.}
\label{t2d:riogrande:meshbnd}
\end{figure}

Zooms around inlet and outlet of the mesh can be seen in Figure
\ref{t2d:riogrande:meshinout}.

\begin{figure}[!htbp]
\begin{minipage}[t]{0.5\textwidth}
 \centering
  \includegraphicsmaybe{[width=0.8\textwidth]}{../img/Meshin.png}
\end{minipage}
\begin{minipage}[t]{0.5\textwidth}
 \centering
  \includegraphicsmaybe{[width=0.8\textwidth]}{../img/Meshout.png}
\end{minipage}
\caption{Zoom of the mesh at the inlet and the outlet.}
\label{t2d:riogrande:meshinout}
\end{figure}

The time step is 5~s for a simulated period of 25,000~s
(a little bit less than 7~h).

To solve the advection, the characteristics is used for the velocities.
The conjugate gradient
is used for solving the propagation step (option 1) and
the implicitation coefficients
for depth and velocities are respectively equal to 1. and 0.55 (= default value).

\subsection{Physical parameters}

A constant horizontal viscosity for velocity equal to 0.0001~m$^2$/s is chosen.

\section{Results}

Figure \ref{t2d:riogrande:FSVelo} shows the free surface elevation and velocity
magnitude at the end of the computation.

\begin{figure}[H]
\begin{minipage}[t]{0.5\textwidth}
 \centering
  \includegraphicsmaybe{[width=0.7\textwidth]}{../img/FreeSurface.png}
\end{minipage}
\begin{minipage}[t]{0.5\textwidth}
 \centering
  \includegraphicsmaybe{[width=0.7\textwidth]}{../img/Velocity.png}
\end{minipage}
\caption{Free surface and velocity magnitude at final time step.}
\label{t2d:riogrande:FSVelo}
\end{figure}

\section{Conclusion}

\telemac{2D} can be used to model river flows.
