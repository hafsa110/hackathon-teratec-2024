\chapter{Flow in a channel submitted to wind (wind)}

\section{Purpose}

This test case presents the hydrodynamics study resulting from wind set-up in a 
closed rectangular basin or channel.
It allows to show that \telemac{2D} is able to correctly simulate the effect 
of meteorological conditions such as surface layer motion generated by the wind blowing
at the water surface provided a depth integration of this process is adequate.
This test case allows also to demonstrate that \telemac{2D} produces the expected
one-dimensional solution even though the grid of triangles is irregular 
(various sizes and orientations of triangles).

\section{Description}

\subsection{Analytical solution}

The wind blowing on whole basin produces a surface current in the direction of 
the wind and a bottom current in the opposite direction. The total discharge in each 
cross-section is null and the wind shear stress is balanced by the slope of the 
induced free surface. The solution produced by \telemac{2D} is compared 
with the analytical solution to this problem. 
The analytical solution for this problem is given by the equation:
\begin{equation}
H(x)=\sqrt{H_0^2 +a_{wind}\frac{2\rho_{air}}{g\rho_{eau}}||\vec{Wind}||^{2} x}
\end{equation}
where $||\vec{Wind}||^{2}$ is the norm of wind velocity vector. $H_0$ is 
depth water on the wind entrance side (west side in this case) 
and $H_L$ is depth water at the end basin within $L$ distance (east side in this case).
In addition, $H_0$ is the solution of the following equation: 
\begin{equation}
F(x)=\left( a_{wind}\frac{2\rho_{air}}{g\rho_{eau}}||\vec{Wind}||^{2} 
L\right)^{3/2}-x^3-a_{wind}\frac{3\rho_{air}}{g\rho_{eau}}||\vec{Wind}||^{2}
L\cdot x\cdot H_{initial}=0
\end{equation}
With $H_{initial}$ is the initial water depth equal here to 2~m.
In this test case, the analytical values of water depths in basin are 
$H_0 \approx 1.56431~\text{m}$ and consequently $H_L \approx 2.37947~\text{m}$.

\subsection{Geometry and mesh}

This test case models the hydrodynamics behaviours due to wind blowing 
in a closed rectangular basin. The geometry dimensions of basin are 100~m wide 
and  500~m long. The basin has a flat bottom and the water depth is equal to 2~m depth. 
The mesh is irregular in the basin. The mesh is shown in Figure \ref{t2d:wind:fig:mesh}. 
It is composed of 551 triangular elements (319 nodes) and the size of triangles
ranges between 14~m and 24~m. The triangular elements types are linear triangles 
(P1, 3 values per element) for water depth and for velocities.

\begin{figure}[!htbp]
 \centering
 \includegraphicsmaybe{[width=0.9\textwidth]}{../img/Mesh.png}
 \caption{Mesh.}
 \label{t2d:wind:fig:mesh}
\end{figure}

\subsection{Initial conditions}

The initial water depth is 2~m with null velocity.

\subsection{Boundary conditions}

The boundary conditions are:
\begin{itemize}
\item For the solid walls, a slip condition in the basin is used for the velocity.
\item No bottom friction.
\item The wind stress at the surface is imposed.
\end{itemize}
The wind conditions allowing to compute the wind stress are :
\begin{itemize}
\item The wind velocity is equal to 5~m.s$^{-1}$ 
(West wind; the wind is coming from the West)
\item The coefficient of wind influence is
 $a_{wind} \frac{\rho_{air}}{\rho_{water}} = 1.2615 \cdot 10^{-3}$ 
 with $\rho_{air}$, $\rho_{water}$ which are respectively the air density 
 and the water density and with $a_{wind}$ an addimentional coefficient.
\end{itemize}

\subsection{Physical parameters}

The turbulent viscosity is constant with velocity 
diffusivity equal to 0~m$^2$.s$^{-1}$.

\subsection{Numerical parameters}

The time step is 10~s for a period of 100~s which is an added at initial computation 
of 500~s. The simulation duration is then 600~s. The resolution accuracy for the 
velocity is taken at $10^{-8}$.
Note that a pre-computation is carried out with the precedent conditions until 500~s.
 For the pre-computation, the wind is applied progressively during 
the 4 first time step (40~s). 
The computation is after continued until 10 time steps. 
The results are so observed after 60 time step (600~s). 
Note that for numerical resolution, conjugate gradient on a normal equation 
is used for solving the propagation step (option 3). To solve advection, 
the characteristics scheme (scheme 1), and the conservative scheme (scheme 5) 
is used respectively for the velocities and for the depth. To finish, the implicitation 
coefficients for depth and velocities are equal to 0.5.\\

\section{Results}

The obtained \telemac{2D} solution reproduces well the behaviour of a balance 
between the wind stress and the surface slope occurs. 
When you consider the one-dimensional solution (independent of the $y$-axis, 
Figure \ref{t2d:wind:fig:freeSbasin}) as shown in Figure~\ref{t2d:wind:fig:freeS1d}, 
the water surface elevation difference between 
the two extremities of this 500~m long basin is 81.5116 . 10$^{-2}$~m 
whereas it should be 81.5164 . 10$^{-2}$~m according to the analytical 
solution, i.e. an error of 0.006~\%. %0.005882515\%

\begin{figure}[H]
 \centering
 \includegraphicsmaybe{[width=0.9\textwidth]}{../img/vector.png}
 \caption{Wind velocity vector.}
 \label{t2d:wind:fig:windvelo}
\end{figure}
\begin{figure}[H]
 \centering
 \includegraphicsmaybe{[width=0.9\textwidth]}{../img/free_surface2D.png}
 \caption{Free surface along basin.}
 \label{t2d:wind:fig:freeSbasin}
\end{figure}

\begin{figure}[H]
 \centering
 \includegraphicsmaybe{[width=0.9\textwidth]}{../img/free_surface1D.png}
 \caption{One-dimensional free surface.}
 \label{t2d:wind:fig:freeS1d}
\end{figure}

\section{Conclusion}
\telemac{2D} is able to compute wind generated flows on the basis of the
empirical wind shear stress formulation  (presented in the \telemac{2D} User Manual). 
The solution computed by \telemac{2D} in this one-dimensional test case is well
independent of the computational grid characteristics although the grid meshes
are very irregular.
