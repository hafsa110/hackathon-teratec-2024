\chapter{Particle transport (flotteurs)}

\section{Description}

This example checks that \telemac{2D} is able to model particles transport with
drogues.

The configuration is a section of river (around 1,700~m long and 300~m wide)
with realistic bottom.
The geometric data include a groyne in the transversal direction and an island
(see Figure \ref{t2d:flotteurs:fig:meshH}).

\subsection{Initial and boundary conditions}

There are 2 computations:
\begin{itemize}
\item The first computation (with t2d\_flotteurs\_v1p0.cas steering file)
writes the r2d\_flotteurs\_v1p0.slf file
and is initialised with a constant elevation equal to 265~m and no velocity,
\item The second computation (with t2d\_flotteurs\_v2p0.cas steering file)
is a continued computation with drogues from the result file of the
first computation.
\end{itemize}

The boundary conditions are:
\begin{itemize}
\item For the solid walls, a slip condition on channel banks is used for the
velocities,
\item On the bottom, a Strickler law with friction coefficient equal to
55~m$^{1/3}$/s is prescribed,
\item Upstream a flowrate equal to 500~m$^3$/s is prescribed
(linearly increasing from 0 to 500~m$^3$/s during the first half hour)
for the first computation and always to 700~m$^3$/s for the second computation.
The velocity profile is given in the boundary conditions file along this liquid
boundary for the first computation and is a constant normal profile for the
second one,
\item Downstream the water level is equal to 265~m.
\end{itemize}

Drogues are released every 10 time steps until the 600$^{\textrm{th}}$ time step
(= 3,000~s).
Thus a maximum of 61 drogues are released, each time at $x$ = -200~m.

\subsection{Mesh and numerical parameters}

The mesh (Figure \ref{t2d:flotteurs:fig:meshH})
is made of 3,780 triangular elements (2,039 nodes).
Is is refined around the island and in front of the groyne.

\begin{figure}[!htbp]
 \centering
 \includegraphicsmaybe{[width=\textwidth]}{../img/Mesh.png}
 \caption{Horizontal mesh.}
 \label{t2d:flotteurs:fig:meshH}
\end{figure}

The time step is 5~s for a simulated period of 15~h (= 54,000~s) for the first
computation and 2~h (= 7,200~s) for the second computation.

To solve the advection, the method of characteristics
is used for the velocities (scheme 1).
The GMRES is used for solving the propagation step (option 7).

For the second computation only, a maximum of 100 drogues are released
and control sections are calculated between two couples of nodes.

\subsection{Physical parameters}

Turbulence is modelled by a constant viscosity equal to 10$^{-2}$~m$^2$/s.

\section{Results}

In the first computation, the flow establishes a steady flow where the free
surface is lighly higher before the groyne than after
(see Figure \ref{t2d:flotteurs:results1}).
The flow accelerates in front of the groyne due to the restriction of section.
A recirculation appears just after the groyne.

\begin{figure}[H]
  \centering
   \subfloat[][Free surface]{
  \includegraphicsmaybe{[width=0.9\textwidth]}{../img/FreeSurface1.png}}\\
  \subfloat[][Velocity]{
  \includegraphicsmaybe{[width=0.9\textwidth]}{../img/Velocity1.png}}
  \caption{Results.}\label{t2d:flotteurs:results1}
\end{figure}

The same remarks can be done for the second computation
(see Figure \ref{t2d:flotteurs:results2}).
The velocities are bigger in the second computation than in the first
due to the bigger inlet flowrate (700~m$^3$/s compared to 500~m$^3$/s).

\begin{figure}[H]
  \centering
   \subfloat[][Free surface]{
  \includegraphicsmaybe{[width=0.9\textwidth]}{../img/FreeSurface2.png}}\\
  \subfloat[][Velocity]{
  \includegraphicsmaybe{[width=0.9\textwidth]}{../img/Velocity2.png}}
  \caption{Results.}\label{t2d:flotteurs:results2}
\end{figure}
