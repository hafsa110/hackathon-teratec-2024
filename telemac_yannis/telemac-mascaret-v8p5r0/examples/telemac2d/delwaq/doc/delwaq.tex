\chapter{One way chaining with DELWAQ (delwaq)}

\section{Description}

This test demonstrates the ability of \telemac{2D} to be chained
with DELWAQ, the water quality software from Deltares.
This is a one way-chaining by files.

A 20~m wide and 28.5~m long prismatic channel with trapezoidal cross-section
contains bridge-like obstacles in one cross-section made of two abutments and two
circular 4~m diameter piles (see Figure \ref{t2d:delwaq:fig:Bottom}).

\begin{figure}[!htbp]
 \centering
 \includegraphicsmaybe{[width=0.9\textwidth]}{../img/Bottom.png}
 \caption{Bottom elevation.}
 \label{t2d:delwaq:fig:Bottom}
\end{figure}

The flow resulting from steady state boundary conditions is studied.
The deepest water depth is 4~m.
The hydrodynamic part is similar to the pildepon test case.
The tracer used is salinity.

\subsection{Initial and boundary conditions}

The computation is initialised with a constant elevation equal to 0~m,
no velocity and a uniform salinity at 0.

The boundary conditions are:
\begin{itemize}
\item For the solid walls, a slip condition on channel banks is used for the
velocities,
\item On the bottom, a Strickler law with friction coefficient equal to
40~m$^{1/3}$/s is prescribed,
\item Upstream a flowrate equal to 62~m$^3$/s is prescribed,
linearly increasing from 0 to 62~m$^3$/s during the first 10~s.
Salinity of 1~g/L is prescribed only along only 3~m of the liquid boundary
(See Figure \ref{t2d:delwaq:Salinityt1} for salinity after 1~s),
\item Downstream the water level is equal to 0~m.
\end{itemize}

\begin{figure}[H]
  \centering
  \includegraphicsmaybe{[width=0.9\textwidth]}{../img/Salinity_t1.png}
  \caption{Salinity at the surface after 2~s.}
  \label{t2d:delwaq:Salinityt1}
\end{figure}

\subsection{Mesh and numerical parameters}

The 2D mesh (Figure \ref{t2d:delwaq:fig:mesh})
is made of 4,304 triangular elements (2,280 nodes).

\begin{figure}[!htbp]
 \centering
 \includegraphicsmaybe{[width=0.9\textwidth]}{../img/Mesh.png}
 \caption{Horizontal mesh.}
 \label{t2d:delwaq:fig:mesh}
\end{figure}

The time step is 0.2~s for a simulated period of 80~s.

To solve the advection, the characteristics (scheme 1) and N scheme (scheme 4)
are respectively used for the velocities and tracers.
The conjugate gradient
is used for solving the propagation step (option 1) and
the implicitation coefficients
for depth and velocities are respectively equal to 0.6 and 1.

\subsection{Physical parameters}

A constant horizontal viscosity for velocity equal to 0.005~m$^2$/s is chosen.
No diffusion is done for tracer salinity.

\section{Results}

Figure \ref{t2d:delwaq:FreeSurf} shows the free surface elevation at the end of
the computation.

\begin{figure}[H]
  \centering
  \includegraphicsmaybe{[width=0.9\textwidth]}{../img/FreeSurface.png}
  \caption{Free surface at final time step.}
  \label{t2d:delwaq:FreeSurf}
\end{figure}

Figure \ref{t2d:delwaq:Velo} shows the magnitude of velocity at the end of the
computation.

\begin{figure}[H]
  \centering
  \includegraphicsmaybe{[width=0.9\textwidth]}{../img/Velocity_tf.png}
  \caption{Velocity magnitude at the surface at final time step.}
  \label{t2d:delwaq:Velo}
\end{figure}

If running DELWAQ with the DELWAQ result files written by \telemac{2D}, the
velocity results are similar with both codes.

Figure \ref{t2d:delwaq:Salinity} shows the salinity at the end of the
computation.

\begin{figure}[H]
  \centering
  \includegraphicsmaybe{[width=0.9\textwidth]}{../img/Salinity.png}
  \caption{Salinity at the surface at final time step.}
  \label{t2d:delwaq:Salinity}
\end{figure}

\section{Conclusion}

\telemac{2D} can be used to chain with DELWAQ.
