\chapter{Parameter estimation (estimation)}

\section{Description}

This example shows how to use parameter estimation with \telemac{2D}.

The configuration is a straight channel 500~m long and 100~m wide
with a flat bottom.

\subsection{Initial and boundary conditions}

The computation is initialised with a constant elevation equal to 5~m
and horizontal velocity along $x$ equal to 1~m/s.

The boundary conditions are:
\begin{itemize}
\item For the solid walls, a slip condition on channel banks is used for the
velocities,
\item On the bottom, a Strickler law is prescribed,
\item Upstream a flowrate equal to 50~m$^3$/s is prescribed,
linearly increasing from 1 to 50~m$^3$/s during the first 20~s,
\item Downstream the water level is suggested to be equal to 0.5~m.
\end{itemize}

\subsection{Mesh and numerical parameters}

The mesh is made of 551 triangular elements (319 nodes),
see Figure \ref{t2d:estimation:fig:mesh}.

\begin{figure}[!htbp]
 \centering
 \includegraphicsmaybe{[width=\textwidth]}{../img/Mesh.png}
 \caption{Horizontal mesh.}
 \label{t2d:estimation:fig:mesh}
\end{figure}

The time step is 2~s.
%The simulated period is 1,000~s.

To solve the advection, the NERD scheme (= 13) is used for the velocities.
The GMRES method
is used for solving the propagation step (option 1) and
the implicitation coefficients
for depth and velocities are let to default values.
To use the parameter estimation mode, primitive equations are mandatory
(\telkey{TREATMENT OF THE LINEAR SYSTEM} = 1) which is not the default option
(= wave equation with \telkey{TREATMENT OF THE LINEAR SYSTEM} = 2).

Please read the \telemac{2D} user manual to know how to use specific keywords
for parameter estimation feature.

\subsection{Physical parameters}

No diffusion is chosen for this computation (constant horizontal viscosity for
velocity equal to 0.~m$^2$/s).

\section{Conclusions}

This example shows how parameter estimation can be used with \telemac{2D}.
