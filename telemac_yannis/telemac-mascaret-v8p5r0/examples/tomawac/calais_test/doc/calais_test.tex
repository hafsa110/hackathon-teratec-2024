\chapter{Calais}
%
% - Purpose & Problem description:
%     These first two parts give reader short details about the test case,
%     the physical phenomena involved and specify how the numerical solution will be validated
%
\section{Purpose}
%
This test case has been exhibited from an old version. It has got mainly one
interest, to compare results from version 5P8 to the new version.  

%
\section{Description of the problem}
We simulate waves around the harbour of Calais coming from offshore. 
At the boundary, the significative heigth is 4m the frequency peak 0.1 and the
direction of 135 degree.

We discretize with 23 frequencies and 24 directions. The mesh is shown on
picture \ref{figcalaismesh}

\begin{figure} [!h]
\centering
\includegraphicsmaybe{[width=0.85\textwidth]}{../img/mesh.png}
 \caption{Mesh of the neighborood of Calais.}
\label{figcalaismesh}
\end{figure}

\section{Results and Comments}

If one run the validation with an old version of the code, one will see a
difference of 11 degree on direction in results, 11 Watt on Power and 0.5 m
on wave heigth. But as we can see on figure  \ref{figcalaishm0}, 
\ref{figcalaishm0v6p0}, \ref{figcalaishm0v5p8},this is not the case when
observing global results. Anyway those differences come from changes made in
the characteristics, so now we take as the reference the results computed with
version 7.2. 

\begin{figure} [!h]
\centering
\includegraphicsmaybe{[width=0.85\textwidth]}{../img/hm0.png}
 \caption{Wave height for last version.}
\label{figcalaishm0}
\end{figure}
\begin{figure} [!h]
\centering
\includegraphicsmaybe{[width=0.85\textwidth]}{../img/hm0v6p0.png}
 \caption{Wave height for V6p0 version.}
\label{figcalaishm0v6p0}
\end{figure}
\begin{figure} [!h]
\centering
\includegraphicsmaybe{[width=0.85\textwidth]}{../img/hm0v5p8.png}
 \caption{Wave height for V5p8 version.}
\label{figcalaishm0v5p8}
\end{figure}
\begin{figure} [!h]
\centering
\includegraphicsmaybe{[width=0.85\textwidth]}{../img/direction.png}
 \caption{Direction for last version.}
\label{figcalaisdirection}
\end{figure}
\begin{figure} [!h]
\centering
\includegraphicsmaybe{[width=0.85\textwidth]}{../img/directionv6p0.png}
 \caption{Direction for V6P0  version.}
\label{figcalaisdirectionV6P0}
\end{figure}
\begin{figure} [!h]
\centering
\includegraphicsmaybe{[width=0.85\textwidth]}{../img/power.png}
 \caption{Power for last version.}
\label{figcalaispower}
\end{figure}
\begin{figure} [!h]
\centering
\includegraphicsmaybe{[width=0.85\textwidth]}{../img/powerv5p8.png}
 \caption{Power for V5P8 version.}
\label{figcalaispowerv5P8}
\end{figure}
