\chapter{Coupling Wind}
\section{Purpose}
This case has been created to ensure that in case of a coupling between
\telemac2d or \telemac3d and \tomawac, the wind transmitted as a coupling
variable is well taken into account.

\section{Description of the cases}
  
Several case of coupling are tested. The first one is inheritaded from case
littoral but to simplify we erased the coupling with (gaia) (steering file
{\it t2d\_littoral.cas}). From this case, we also test the tel2tom technic. We
remind here that this technic allows us to couple telemac2d and tomawac on
different meshes and different domain. First we test the technic on the same mesh
(steering file {\it t2d\_same.cas}) and then on different meshes (steering file
{\it t2d\_different.cas}). In the case of different mesh tomawac is calculated on a
coarser mesh but a little bit bigger.

In a second time, we are testing the coupling with wind in 3 dimension.
Note that the wind is still in 2D since \tomawac uses a 2D wind. The first case is
a 2D coupling in 3 dimension (steering file {\it t3d\_littoral.cas}) and the second
case is a real 3D coupling (steering file {\it t3d\_3Dcoupling.cas})

In all case, we build the reference with a wind exchanged by file (written by
\telemac and read by \tomawac). Note that the reference can not be constructed
anymore, since after the validation, the coupling with the use of a wind file
in \tomawac is forbidden. Then we verify that the wind exchanged by coupling
leads to a result equal to the result of the reference. Note that in the case of
different mesh, the wind written on the mesh of \tomawac is written by another
telemac2d steering file ({\it t2d\_diffwind.cas}). In that case the boundary
conditions are not the same as for littoral as it would necessitate a dedicated
user fortran for that mesh and it is not the point of this test. 

On a practical point of vue, since a wind read by \tomawac is interpolated on time
step between 2 read, we had to construct our results with a coupling period of 1,
so that the wind transmitted by coupling has the same actualisation. 

\section{Results}
We present the results obtained for hm0 in all case, but the most important result
is that there is no difference if the wind is exchanged by file or by coupling.
This validates the developpements done for all the coupling.

\begin{figure} [!h]
\centering
\includegraphicsmaybe{[width=0.85\textwidth]}{../img/hm0_littoral2d.png}
 \caption{Hm0 in classical 2D coupling.}
\label{hm0littoral2d}
\end{figure}

\begin{figure} [!h]
\centering
\includegraphicsmaybe{[width=0.85\textwidth]}{../img/hm0_same.png}
 \caption{Hm0 with tel2tom and same mesh.}
\label{hm0same}
\end{figure}

\begin{figure} [!h]
\centering
\includegraphicsmaybe{[width=0.85\textwidth]}{../img/hm0_different.png}
 \caption{Hm0 with tel2tom and different mesh.}
\label{hm0different}
\end{figure}

\begin{figure} [!h]
\centering
\includegraphicsmaybe{[width=0.85\textwidth]}{../img/hm0_3dlittoral.png}
 \caption{Hm0 in the case 3D littoral.}
\label{hm03dlittoral}
\end{figure}

\begin{figure} [!h]
\centering
\includegraphicsmaybe{[width=0.85\textwidth]}{../img/hm0_3dcoupling.png}
 \caption{Hm0 in the case 3D coupling.}
\label{hm03dcoupling}
\end{figure}


\section{Coupling \telemac{2D} and \tomawac with a wind and a hotstart}
We give here an example of a coupling between \telemac{2D} and \tomawac with a constant
wind in space varying linearly in time. The aim is to ensure that the wind is the same
on both side at every time step. After a hot start we also ensure that it is still the case
(see Figure \ref{windcouplinghotstart}).
A first run is done during 500s, then we do a hot start and simulate during 500s. The solution
is then compared to a solution we call reference solution, i.e. a simulation done during 1000s
without any hotstart (see Figure \ref{hm0couplinghotstart}).   
\begin{figure} [!h]
\centering
\includegraphicsmaybe{[width=0.85\textwidth]}{../img/wind.png}
 \caption{Wind in \tomawac before and after hotstart compared to the wind in the reference \telemac{2D} result file}
\label{windcouplinghotstart}
\end{figure}

\begin{figure} [!h]
\centering
\includegraphicsmaybe{[width=0.85\textwidth]}{../img/waveheight.png}
\caption{Comparison of the  wave height between the hot start calculation and the reference
  calculated during 1000s}
\label{hm0couplinghotstart}
\end{figure}
