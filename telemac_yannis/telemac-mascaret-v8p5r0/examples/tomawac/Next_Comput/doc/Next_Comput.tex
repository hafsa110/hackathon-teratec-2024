\chapter{Next\_Comput}

\section{Purpose}
%
The aim of this case is to test a computation that follows another one. We
start from a calculated state at a time and we simulate some new time step.
%
\section{Description of the problem}
%
The simulation is the same as the shoal test case so one will read that test
case for physical description. We start here from a shoal simulation until time
600s. The result of this previous simulation is inside the file indicated by
the keyword {\it PREVIOUS COMPUTATION FILE} ({\it FICHIER DU CALCUL PRECEDENT}
in french). This file has been created by adding the keyword  {\it GLOBAL
  RESULT FILE} ({\it FICHIER DES RESULTATS GLOBAUX} in french) in the shoal
steering case.
We simulate for 600 other seconds. We can compare the result to the one
obtained by shoal test case with 1200 seconds of simulation.
\begin{figure} [!h]
\centering
\includegraphicsmaybe{[width=0.85\textwidth]}{../img/mesh.png}
 \caption{Mesh of the domain.}
\label{meshnextcomp}
\end{figure}

\section{Results}
After 600 s, the difference with the full shoal simulation is null. The
reference file is the result of shoal with 12000s.
\begin{figure} [!h]
\centering
\includegraphicsmaybe{[width=0.85\textwidth]}{../img/results.png}
%\includegraphics{bathy.png}
 \caption{Wave heigth hm0.}
\label{resnextcomp}
\end{figure}

\section{Hot start while reading a tide}
A second test is made to verify that when reading a tidal file with
water depth and currents varying in time, those variables are still
interpolated after the hot start see Figures \ref{interpcurrent} and
\ref{interpwaterdepth}.
\begin{figure} [!h]
\centering
\includegraphicsmaybe{[width=0.85\textwidth]}{../img/normcurr.png}
 \caption{Current norm interpolated after the hot start}
\label{interpcurrent}
\end{figure}

\begin{figure} [!h]
\centering
\includegraphicsmaybe{[width=0.85\textwidth]}{../img/waterdepth.png}
 \caption{Water depth interpolated after the hot start}
\label{interpwaterdepth}
\end{figure}

\section{Hot start while reading a wind varying in time}
A third test is made to verify that when reading a wind varying in time in a
file the wind is still interpolated between two points read see Figure
\ref{interpwind}, we can see the effect on the wave height on Figure
\ref{hotstartwindwaveheight} where the solution after a hot start is the same
as the one of a calculation for the full time.
\begin{figure} [!h]
\centering
\includegraphicsmaybe{[width=0.85\textwidth]}{../img/wind.png}
 \caption{Wind norm interpolated after the hot start}
\label{interpwind}
\end{figure}

\begin{figure} [!h]
\centering
\includegraphicsmaybe{[width=0.85\textwidth]}{../img/waveheight.png}
\caption{Wave height compared between a hot start and a reference calculated
  from the real start.}
\label{hotstartwindwaveheight}
\end{figure}
